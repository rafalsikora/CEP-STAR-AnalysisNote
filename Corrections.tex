%%===========================================================%%
%%                                                           %%
%%                       CORRECTIONS                         %%
%%                                                           %%
%%===========================================================%%


\chapter{Corrections}\label{chap:corrections}

In this chapter we describe methodology of correcting the data to the hadron-level cross sections. We describe technical details of the procedure (Sec.~\ref{sec:correctionProcedure}), as well as present derivation of relevant acceptances and efficiencies (Sec.~\ref{sec:acceptanceAndEff}). At the end of the chapter we show the closure tests performed on MC to validate data correction procedure together with the entire set of efficiencies.

%%%%%%%

\section{Method of corrections application}\label{sec:correctionProcedure}

Differential cross section in every bin of measured quantity of interest $q$ have been calculated according to Eq.~\eqref{eq:crossSec}:

\begin{equation}\label{eq:crossSec} 
  \frac{d\sigma}{dq} = \frac{1}{\Delta q} \times \frac{1}{\varepsilon} \times \frac{N^{\mathit{w}}-N^{\mathit{w}}_\textrm{bkgd}}{\mathit{L}_{\textrm{int}}^{\textrm{eff}}},
\end{equation}%
%
where $N^{w}$ is weighted number of measured events in given bin, $N^{w}_{\text{bkgd}}$ is weighted number of background events in given bin, $\Delta q$ is width of the bin, $\varepsilon$ is joint efficiency of 

%remembed about accounting for RP trigger eff!!!
\begin{equation}\label{eq:effectiveLumi}
	\mathit{L}_{\textrm{int}}^{\textrm{eff}} = \sum\limits_{\textrm{run}}\mathit{L}_{\textrm{int}}^{\textrm{run}} \times \epsilon_{\textrm{veto}}(L^{\textrm{run}})
\end{equation}
% \left(\epsilon_{\textrm{veto}}^{\textrm{online}} \oplus \epsilon_{\textrm{veto}}^{\textrm{offline}}(L_{\textrm{run}}) \right)

\begin{equation}
	\varepsilon = \epsilon_{\textrm{\tiny ET/IT}} \times \epsilon_{\textrm{vrtx}}(q) \times \epsilon_{\ref{enum:CutZVx}} \times \epsilon_{\ref{enum:CutDeltaZVx}} \times \epsilon_{\ref{enum:CutMissingPt}} \times \epsilon_{\textrm{\tiny PID}}(q)
\end{equation}

\begin{equation}
	N^{\mathit{w}} = \sum\limits_{\textrm{event}}\mathit{w}_{\textrm{event}}
\end{equation}



\begin{equation}\label{eq:weight}
	\mathit{w} = \left[\prod\limits_{\textrm{sign}} \epsilon_{\textrm{\tiny TOF}}(\textrm{sign}, \textrm{PID}, p_{\text{T}},z_{vx},\eta)  \times \prod\limits_{\textrm{sign}} \epsilon_{\textrm{\tiny TPC}}(\textrm{sign}, \textrm{PID}, p_{\text{T}},z_{vx},\eta) \times \prod\limits_{\textrm{side}}\epsilon_{\textrm{\tiny RP}}^{\textrm{side}}(p_{x},p_{y}) \right]^{-1},
\end{equation}
\[\textrm{sign}\in\{+,-\},~~\textrm{side}\in\{E,W\}\]
% () 

Figure~\ref{fig:weights} shows the distribution of weights for all three types of analyzed particle pairs. These correction factors are large,  however they are products of many other factors which are in general close to 1.

%---------------------------
\begin{figure}[h!]
\centering%
\parbox{0.4725\textwidth}{%
  \centering%
  \includegraphics[width=\linewidth]{graphics/corrections/Weights.pdf}\label{fig:weights} 
}%
\quad%
\parbox{0.4725\textwidth}{%
    \caption[Distribution of weights assigned to selected CEP events.]{Distribution of weights assigned to selected CEP events in $\pi^{+}\pi^{-}$~(black circle), $K^{+}K^{-}$~(red square) and $p\bar{p}$~(blue triangle) channel.}\label{fig:weigths}% 
}
\end{figure}
%---------------------------




\section{Acceptances and efficiencies}\label{sec:acceptanceAndEff}

In this section we present calculation of all efficiencies except TPC track reconstruction and TOF hit reconstruction and matching efficiency, which were discussed and presented in Ref.~\cite{supplementaryNote}.

\subsection{Trigger efficiency}\label{sec:triggerEff} 


\subsubsection{Online TOF efficiency}\label{sec:tofOnlineEff}
The efficiency of the part of the trigger related to TOF subsystem $\mbox{\LARGE$\varepsilon$}_{\text{TOF}}^{\text{trig}}$ was calculated using the zero-bias data. Events with exactly 1 TOF vertex (cut~\ref{enum:CutPrimVx}), exactly 2 primary good quality TOF-matched TPC tracks (cuts~\ref{enum:TpcTofMatched}, \ref{enum:TpcQualityCuts}) and maximally 3 TOF clusters (cut~\ref{enum:CutTofClusters}) were used to calculate this efficiency defined as a probability of having at least two TOF hits on the trigger level in selected set of events:

\begin{equation}
 \mbox{\LARGE$\varepsilon$}_{\text{TOF}}^{\text{trig}} = \frac{\text{\#events with $\geq2$ L0 TOF multiplicity and 1 TOF vtx and 2 TOF trks and $\leq3$ TOF cltrs}}{\text{\#events with 1 TOF vtx and 2 TOF trks and $\leq3$ TOF cltrs}}
\end{equation}

%---------------------------
\begin{figure}[ht!]
\centering%
\parbox{0.4725\textwidth}{%
  \centering%
  \includegraphics[width=\linewidth]{graphics/corrections/TofTrigEff_Final.pdf}\label{fig:tofTrigEff} 
}%
\quad%
\parbox{0.4725\textwidth}{%
    \caption[TOF trigger efficiency as a function of lower $p_{T}$ of two TOF-matched tracks.]{TOF trigger efficiency as a function of lower $p_{T}$ of two TOF-matched tracks. Black points denote efficiency of requirement of at least 2 online TOF hits, while open blue circles represent efficiency with added upper multiplicity limit (10). These efficiencies are identical therefore the same efficiency (98.7\%, solid red line) is used as a correction factor for all analyzed events.}\label{fig:tofTrigEff2}% 
}
\end{figure}
%---------------------------

It has been verified that described efficiency does not depend on the transverse momentum of two tracks. In Fig.~\ref{fig:tofTrigEff2} we show dependence of the TOF trigger efficiency in one dimension, as a function of lower $p_{T}$ of two TOF-matched tracks. Data points are consistent with constant value, herefore a single number for this efficiency is used, equal to 98.7\%. It is important to account for the fact that at some point of data-taking (see Tab.~\ref{tab:triggers}) an upper limit ($<10$) on L0 TOF multiplicity was imposed. Study shows that with the cut on offline TOF cluster multiplicity (cut~\ref{enum:CutTofClusters}) the efficiency of the trigger with and without upper limit of online TOF multiplicity remains unchanged, so the same value of the efficiency is used for the entire dataset.



\subsubsection{Online veto (BBC-small and ZDC veto)}\label{sec:onlineVetoEff}

Vetoeing signal in BBC-small and ZDC detectors on both sides of STAR was implemented in the logic of RP\_CPT2 trigger. Common correction of the online and offline vetoes which is used in the correction procedure explained in Sec.~\ref{sec:correctionProcedure} is presented in. Sec.~\ref{sec:onlineAndOfflineVetoEff}. However, to help quantifiy effect of just the online vetoes in BBC-small and ZDCs we show the Fig.~\ref{fig:onlineVetoEff} with the efficiency of the joint BBC-small and ZDC veto as a function of the instantaneous luminosity calculated from the zero-bias data. Details of the way the efficiency was calculated as well as description of the data in the Figure is the same as explained in Sec.~\ref{sec:onlineAndOfflineVetoEff}.
%---------------------------
\begin{figure}[ht!]
\centering%
\includegraphics[width=0.65\linewidth,page=1]{graphics/corrections/OnlineVetoEffVsInstLumi_graph.pdf}%
\caption{Overall efficiency of the online BBC-small and ZDC veto as a function of instantaneous luminosity.}\label{fig:onlineVetoEff}%
\end{figure}
%---------------------------

\subsubsection{RP triggering efficiency}\label{sec:rpTrigEff}

Based on studies presented in Sec.~10.4 of Ref.~\cite{supplementaryNote} we concluded that the RP triggering efficiency is nearly 100~\%, which is in agreement wtih the Geant4 MC simulation of RP system. Therefore we use in the correction procedure
\begin{equation}
\mbox{\LARGE$\varepsilon$}\left(\TRE\land\TRW\right) =  \mbox{\LARGE$\varepsilon$}\left(\TRW\right) \times \mbox{\LARGE$\varepsilon$}\left(\TRE\right) = 1.
\end{equation}


% Wydajnosc samego trygerowania
% \begin{equation}
% \mbox{\LARGE$\varepsilon$}\left(\TRE\land\TRW\right) = \mbox{\LARGE$\varepsilon$}\left(\TRE\right) \times \mbox{\LARGE$\varepsilon$}\left(\TRW\right),
% \end{equation}
% \begin{equation}
% \mbox{\LARGE$\varepsilon$}\left(\TRE\right) = \mbox{\LARGE$\varepsilon$}\left(\TRE_{1}\vee\TRE_{2}\right),~~~~\mbox{\LARGE$\varepsilon$}\left(\TRW\right) = \mbox{\LARGE$\varepsilon$}\left(\TRW_{1}\vee\TRW_{2}\right),
% \end{equation}
% dobrze bedzie policzyc już z danych, najlepiej elastycznych (ale można też z CD) patrzac jak czesto stacje z dobrymi sladami maja sygnal trygerowy.
% 
% \begin{equation}\begin{split}
% \mbox{\LARGE$\varepsilon$}\left(\TRW\right) = \mbox{\LARGE$\varepsilon$}\left(\TRW\left|\RPE\&\RPW\&~!\TRNE\&~!\TRNW\right.\right) = \mbox{\LARGE$\varepsilon$}\left(\TRW_{1}\vee\TRW_{2}\left|\RPE\&\RPW\&~!\TRNE\&~!\TRNW\right.\right)=\\
% =\mbox{\LARGE$\varepsilon$}\left(\TRW_{1}\left|\RPE\&\RPW\&~!\TRNE\&~!\TRNW\right.\right)+\mbox{\LARGE$\varepsilon$}\left(\TRW_{2}\left|\RPE\&\RPW\&~!\TRNE\&~!\TRNW\right.\right)+\\-\mbox{\LARGE$\varepsilon$}\left(\TRW_{1}\land\TRW_{2}\left|\RPE\&\RPW\&~!\TRNE\&~!\TRNW\right.\right)~~~~~\text{(analogicznie po stronie EAST)}
% \end{split}\end{equation}


\subsubsection{Up and Down RP combination veto (due to dead material)}\label{sec:rpDeadMat}

Probability that secondaries induced by proton with successfully reconstructed and selected RP track generate a trigger signal in the other RP branch on the same side was calculated using the embedded MC (see Sec.~6.3 in Ref.~\cite{supplementaryNote} for details of RP simulation in Geant4). Forward protons from CEP process provided by GenEx~\cite{GenEx} were simultaneously generated from the interaction point spatially distributed the same as in the data. MC samples for all runs with RP\_CPT2 triggers were produced to account for non-constant positions of the RP detectors throughout the run 15. Number of simulated events for each run was proportional to number of RP\_CPT2 triggers in given run. The angular divergence of the beams was also simulated.

The discussed probability, $\mathcal{P}_{\text{DM~veto}}^{\text{side}}$, was calculated as a probability that a MC-trigger signal is present in the branch on given side other than east and west branches where primary forward protons are expected from their initial momenta, under condition that these east and west branches detect a MC-trigger signal and there is no veto due to simultaneous ET\&IT trigger bits in the overlayed data (no pile-up veto). By MC-trigger we understand the trigger signal reconstructed solely from the simulated data (not from the data embedded into).

Technically the $\mathcal{P}_{\text{DM~veto}}^{\text{side}}$ was obtained in the following procedure:
\begin{enumerate}
  \item No simultaneous ET\&IT trigger bits were allowed in the data of an event that simulated signal was embedded into.
	\item It was verified if there are MC-trigger signals in east and west branches that the primary forward protons were expected to reach based on their $p_{y}$ ($p_{y}>0$ - branch UP, $p_{y}<0$ - branch DOWN). These events formed $set~A$.
	\item Events with the MC-trigger signal in RP branch other than the branch with MC-trigger signal expected from proton $p_{y}$ on given side formed $set~B$.
	\item The probability was determined by the ratio of histograms from $set~B$ and $set~A$:
	\begin{equation}\label{eq:rpDeadMatProb}
	\begin{split}
 \mathcal{P}_{\text{DM~veto}}^{\text{side}}(p_{x}, p_{y}, z_{\text{vtx}}) = \mbox{\LARGE$\varepsilon$}^{\text{side}}\left(\Vdm\left|!\Vpu\land\TRE\land\TRW\right.\right) = ~~~~~~~~~~~~~~~~~~~~~~~~~~~~~~~~~~~~~~\\~~~~~~~~~~~~~~~~~~~~~~~~~~~~~~~ =%
 \frac{(p_{x}, p_{y}, z_{\text{vtx}})~\text{histogram for protons from}~set~B}{(p_{x}, p_{y}, z_{\text{vtx}})~\text{histogram for protons from}~set~A}
 \end{split}
  \end{equation}
	
\end{enumerate}



It should be noted that the momentum components $(p_{x}, p_{y})$ were taken from the proton with accounted effect of the beam divergence (after the original initial momentum smearing). Sample probability of a dead-material-induced veto is shown in Fig.~\ref{fig:sampleRpDeadMatVeto}, with all the remaining results contained in Appendix~\ref{appendix:rpEff}.

The efficiency of the discussed veto which is finally used to correct the data is opposite of the veto probability, namely
\begin{equation}
 \mbox{\LARGE$\epsilon$}_{\text{DM~veto}}^{\text{side}} = 1 - \mathcal{P}_{\text{DM~veto}}^{\text{side}}.
\end{equation}

%---------------------------
\begin{figure}[ht!]
\centering%
\parbox{0.4725\textwidth}{%
  \centering%
  \includegraphics[width=\linewidth,page=10]{graphics/corrections/mcDeadMatProbPxPy.pdf}\label{fig:sampleDeadMatVetoProb}
}%
\quad%
\parbox{0.4725\textwidth}{%
    \caption[Sample probability of ET\&IT trigger veto due to forward proton interaction with dead material.]{Sample probability of ET\&IT trigger veto due to forward proton interaction with dead material. Results were obtained from forward proton MC simulation embedded into zero-bias data.}\label{fig:sampleRpDeadMatVeto}%
}
\end{figure}
%---------------------------






\subsection{Reconstruction and selection efficiency}\label{sec:cutsEff}
\subsubsection{TPC \texorpdfstring{$z$}{z}-vertex cut~(\ref{enum:CutZVx})}

Removing from analysis specific range of $z$-positions of primary vertices effectively reduces accepted luminosity with respect to that delivered by the collider. This loss of luminosity has to be accounted when calculating the cross sections, which is the goal of presented analysis. Assuming that the distribution of $z_{\text{vtx}}$ of all primary interactions that take place when east and west beams overlap follows a normal distribution ($\mathcal{N}$), then the formula describing efficiency of the cut~\ref{enum:CutZVx} has the following form:
\begin{equation}\label{eq:zVtxCutEff}
 \mbox{\LARGE$\epsilon$}_{z_{\text{vtx}}} = \int\limits_{z_{\text{vtx}}^{\text{min}}}^{z_{\text{vtx}}^{\text{max}}} \mathcal{N}(z_{\text{vtx}}; \mu, \sigma)~\text{d}z_{\text{vtx}} = \frac{1}{2}\left[ \text{Erf}\left(\frac{z_{\text{vtx}}^{\text{max}} - \mu}{\sqrt{2}\sigma}\right) - \text{Erf}\left(\frac{z_{\text{vtx}}^{\text{min}} - \mu}{\sqrt{2}\sigma}\right) \right],
\end{equation}
where $z_{\text{vtx}}^{\text{min}}$ and $z_{\text{vtx}}^{\text{max}}$ are respectively minimum and maximum value of the longitudinal position of the vertex accepted in analysis (in our case these are -80~cm and 80~cm), and parameters $\mu$ and $\sigma$ are respectively the mean and standard deviation of the normal distribution:
\[\mu = \langle z_{\text{vtx}}\rangle,~~~~~~~~~~~~~~~\sigma = \sigma(z_{\text{vtx}}).\]

Real parameters of $z_{\text{vtx}}$ distribution were studied separately for every single fill of the collider. It is motivated by the fact that each fill of the machine is nearly independent from the previous thus the shape and position of $z_{\text{vtx}}$ may vary on fill by fill basis. We neglect possible changes within the fill (e.g. widening of the distribution due to intrabeam scattering etc.) arguing that this effect if expected to be smaller than the systematic uncertainties related to determination of position and width of $z_{\text{vtx}}$ distribution.

For every fill of RHIC the distribution of $z_{\text{vtx}}$ of single TOF vertices was prepared, as shown in Fig.~\ref{fig:sampleZVtxHist}. This distribution was fitted with the Gaussian function in a range $z_{\text{vtx}}\in[-120~\text{cm}, 120~\text{cm}]$. The output parameters of all fits were plotted as a function of the fill number, as shown in Fig.~\ref{fig:zVxMeanAndSigmaVsFillNumber}. The efficiency used in the correction procedure was calculated independetly for each fill using presented values of $\langle z_{\text{vtx}}\rangle$ and $\sigma(z_{\text{vtx}})$. Typical numerical value of this efficiency equals $\sim88\%$.

%---------------------------
\begin{figure}[ht!]% 
\centering%
\begin{minipage}{.4725\textwidth}%
  \centering%
  \includegraphics[width=\linewidth]{graphics/corrections/sampleZVtxHist.pdf}%
  \caption[Sample distribution of $z_{\text{vtx}}$ with fitted normal distribution.]{Sample distribution of $z_{\text{vtx}}$ of single TOF vertices together with the fit of normal distribution (dashed blue) extended outside the range of the fit (solid red). Hashed red area represents part of the distribution rejected by cut~\ref{enum:CutZVx} with the cut value marked with dashed red vertical lines and arrows.}\label{fig:sampleZVtxHist}
\end{minipage}% 
\quad\quad%
\begin{minipage}{.4725\textwidth}%
  \centering
  \includegraphics[width=\linewidth]{graphics/corrections/zVxMeanAndSigmaVsFillNumber.pdf}%
  \caption[Mean and width of the $z_{\text{vtx}}$ distribution as a function of the RHIC fill number.]{Mean (top panel) and width (bottom panel) parameters of the normal distribution obtained from a fit of normal distribution to $z_{\text{vtx}}$ distribution in a range $[-120~\text{cm}, 120~\text{cm}]$ for each RHIC fill as a function of the fill number.\newline}\label{fig:zVxMeanAndSigmaVsFillNumber}
\end{minipage}%
\end{figure}%
%---------------------------


\subsubsection{TPC-RP \texorpdfstring{$z$}{z}-vertex matching~(\ref{enum:CutDeltaZVx})}

This cut was safely set to $3.5\sigma_{\Delta z_{\text{vtx}}}$, therefore the correction for the introduced inefficiency is negligible.

\subsubsection{Primary vertices limit~(\ref{enum:CutPrimVx}), BBC-large veto~(\ref{enum:CutBbcLarge}), TOF clusters limit~(\ref{enum:CutTofClusters}) and Up and Down RP combination veto (due to pile-up)}\label{sec:onlineAndOfflineVetoEff}

Combined efficiency of the online veto in BBC-small and ZDC (Sec.~\ref{sec:onlineVetoEff}) and offline cuts (vetoes) on extra TPC-TOF vertices, extra TOF clusters, signal in BBC-large and simultaneous signal in Up and Down RPs, was calculated using the zero-bias data. For each run a fraction of events (for colliding bunches) was calculated in which all mentioned cuts would be satisfied in case of the CEP $\pi^{+}\pi^{-}$/$K^{+}K^{-}$/$p\bar{p}$ event (event would not be vetoed). One can tranform this prescription to simple formula below:

\begin{equation}
 \mbox{\LARGE$\epsilon$}^{\text{veto}}_{b_{E}b_{W}}=\dfrac{\splitdfrac{~~~~~\text{\#events in the run without TOF vertices, without signal in BBC-S, BBC-L, ZDC,}}{\text{RP branches other than} ~b_{E},~b_{W},~\text{and with no more than 1 reconstructed TOF cluster}}}{\text{\#events in the run}}
\end{equation}

In Fig.~\ref{fig:onlineAndOfflineVetoEff} this efficiency is presented as a function instantaneous luminosity delivered by the machine, for each combination of east and west RP branches. Result for each combination is nearly identical as the effect of ET\&IT trigger veto in RPs is not dominant, as well as trigger in all branches had similar acceptance. The data points were fitted with the exponential function (of the form containted in the figure) which reflects the fact that this efficiency should behave similar to the probability of lack of any interaction in the bunch crossing given by the Poisson distribution:
\begin{equation}\label{eq:poisson}
 \text{Pois}(0;\mu) = \frac{\mu^{0}}{0!} \times e^{-\mu} = e^{-\mu}.
\end{equation}
Comparison of the $\mu$ in Eq.~\eqref{eq:poisson} with the fit parameters in Fig.~\ref{fig:onlineAndOfflineVetoEff} leads to approximate determination of the average interaction probability per bunch crossing equal $0.2-0.9$. The result of the fit, $ \mbox{\LARGE$\epsilon$}^{\text{veto}}_{b_{E}b_{W}}(\mathcal{L})$, is finally used to correct measured data as described in Sec.~\ref{sec:correctionProcedure}. 

Comparison of efficiencies in Fig.~\ref{fig:onlineAndOfflineVetoEff} with similar efficiency in Fig.~\ref{fig:onlineVetoEff} demonstrates that offline selection has much smaller impact on the loss of signal events than online selection. It has to be underlined that online vetoes were necessary to set trigger purity to satisfactory level, as well as reduce prescale of the trigger.
% %---------------------------
% \begin{figure}[h]
% \centering
% \parbox{0.4725\textwidth}{
%   \centering
%   \begin{subfigure}[b]{\linewidth}
%                 \subcaptionbox{}{\includegraphics[width=\linewidth,page=1]{graphics/corrections/FullVetoEffVsInstLumi.pdf}}
%   \end{subfigure}\\
%   \begin{subfigure}[b]{\linewidth}\addtocounter{subfigure}{1}
%                 \subcaptionbox{}{\includegraphics[width=\linewidth,page=3]{graphics/corrections/FullVetoEffVsInstLumi.pdf}}
%   \end{subfigure}
% }%
% \quad\quad%
% \parbox{0.4725\textwidth}{
%   \centering
%   \begin{subfigure}[b]{\linewidth}\addtocounter{subfigure}{-2}
%                 \subcaptionbox{}{\includegraphics[width=\linewidth,page=2]{graphics/corrections/FullVetoEffVsInstLumi.pdf}}
%   \end{subfigure}\\
%   \begin{subfigure}[b]{\linewidth}\addtocounter{subfigure}{1}
%                 \subcaptionbox{}{\includegraphics[width=\linewidth,page=4]{graphics/corrections/FullVetoEffVsInstLumi.pdf}}
%   \end{subfigure}
% }%
% \caption[Overall efficiency of online and offlince cuts as a function of instantaneous luminosity.]{Overall efficiency of the online BBC-small, ZDC and ET\&IT trigger veto, primary vertices limit~(\ref{enum:CutPrimVx}), BBC-large veto~(\ref{enum:CutBbcLarge}) and TOF clusters limit~(\ref{enum:CutTofClusters}) as a function of instantaneous luminosity for all possible combinations of east and west RP branches. Red and blue points represent runs lasting for less and more than 20 minutes, respectively. Black dotted lines reprent fits of exponential functions to blue points.}\label{fig:onlineAndOfflineVetoEff}%
% \end{figure}
% %---------------------------
\begin{figure}[h]
  \centering
  \begin{tabular}{@{}p{0.49\linewidth}@{\quad}p{0.49\linewidth}@{}}
    \subfigimg[width=\linewidth,page=1]{~~~~~~a)}{graphics/corrections/FullVetoEffVsInstLumi.pdf} &
    \subfigimg[width=\linewidth,page=2]{~~~~~~b)}{graphics/corrections/FullVetoEffVsInstLumi.pdf} \\
    \subfigimg[width=\linewidth,page=3]{~~~~~~d)}{graphics/corrections/FullVetoEffVsInstLumi.pdf} &
    \subfigimg[width=\linewidth,page=4]{~~~~~~e)}{graphics/corrections/FullVetoEffVsInstLumi.pdf}
  \end{tabular}
\caption[Overall efficiency of online and offlince cuts as a function of instantaneous luminosity.]{Overall efficiency of the online BBC-small, ZDC and ET\&IT trigger veto, primary vertices limit~(\ref{enum:CutPrimVx}), BBC-large veto~(\ref{enum:CutBbcLarge}) and TOF clusters limit~(\ref{enum:CutTofClusters}) as a function of instantaneous luminosity for all possible combinations of east and west RP branches. Red and blue points represent runs lasting for less and more than 20 minutes, respectively. Black dotted lines reprent fits of exponential functions to blue points.}\label{fig:onlineAndOfflineVetoEff}%
\end{figure}



\subsubsection{TOF clusters limit~(\ref{enum:CutTofClusters})}\label{sec:tofClusterLimitEff}

Correction described in previous section (Sec.~\ref{sec:onlineAndOfflineVetoEff}) covers inefficiency related to the pile-up interactions. It is also possible that a higher multiplicity of TOF clusters is generated in the exclusive $h^{+}h^{-}$ signal event. Therefore, the upper limit on the number of TOF clusters equal to 3 introduces some inefficiency. To study this effect a no-pile-up signal MC was analyzed (pure STARsim without embedding into zero-bias data).

In Fig.~\ref{fig:NTofClustersStarsim} the multiplicity of TOF clusters is shown for the signal MC. One can clearly notice that with cut on $N^{\text{TOF}}_{\text{cltrs}}\leq 3$ some part of the signal is truncated. The level of truncation is presented in Fig.~\ref{fig:NTofClustersEff} as a function of the invariant of the $\pi^{+}\pi^{-}$. No significant mass dependence is observed hence we use a single efficiency correction number equal to 95.7\%, which results from the fit of a constant to efficiency points.



%---------------------------
\begin{figure}[ht!]% 
	\centering%
	\begin{minipage}{.4725\textwidth}%
		\centering%
		\includegraphics[width=\linewidth]{graphics/corrections/NTofClusters2.pdf}%
		\caption[Multiplicity of $N^{\text{TOF}}_{\text{cltrs}}$ in simulated CEP $\pi^{+}\pi^{-}$ events.]{Multiplicity of $N^{\text{TOF}}_{\text{cltrs}}$ in CEP $\pi^{+}\pi^{-}$ events passed through STAR detector simulation without embedding into zero-bias data. Red dashed line and arrow mark the limit on number of TOF clusters accepted in analysis.}\label{fig:NTofClustersStarsim}
	\end{minipage}% 
	\quad\quad%
	\begin{minipage}{.4725\textwidth}%
		\centering%
		\includegraphics[width=\linewidth]{graphics/corrections/NTofClustersCutEff.pdf}%
		\caption[Efficiency of upper limit of $N^{\text{TOF}}_{\text{cltrs}}$.]{Efficiency of upper limit of $N^{\text{TOF}}_{\text{cltrs}}$ as a function of the invariant mass of the $\pi^{+}\pi^{-}$, based on the STAR simulation of exclusive $\pi^{+}\pi^{-}$ without embedding into zero-bias data. Red line shows the fit of a constant to all efficiency points. }\label{fig:NTofClustersEff}
	\end{minipage}%
\end{figure}%
%---------------------------



\subsubsection{Missing \texorpdfstring{$p_{\text{T}}$}{pT} cut~(\ref{enum:CutMissingPt})}\label{sec:ptMissCutEff}

The resolution of the total transverse momentum is in general determined by the angular divergence of the proton beams. Based on known parameters of the beams in run 15 (see Ref.~\cite{run15Overview}) expected divergence for a single beam amounts 180~$\mu$rad. In forward proton track reconstruction the direction of the incoming proton is assumed to be the same in each event (aligned with nominal $z$-axis), hence effectively the transverse components of reconstructed forward protons momenta are smeared by this amount. One can check the comparison of the collinerities of elastically scattered protons in the data and embedded MC (Fig.~10.19 of Ref.~\cite{supplementaryNote}) to see the satisfactory agreement between the two with assumed MC angular divergence equal to 180~$\mu$rad. For 100~GeV beam such divergence results in smearing of the tranverse momentum components for (elastically) scattered protons equal $100~\text{GeV}\times 180\cdot 10^{-6} = 18~\text{MeV}$. Using this value to calculate the resolution of total $p_{x}$($p_{y}$) momentum of two forward protons we get $\sqrt{2}\times 18~\text{MeV} = 25.5~\text{MeV}$. One can compare it with the output of fits to distributions of  $p_{x}^{\text{miss}}$($p_{y}^{\text{miss}}$) and find that indeed the angular beam divergence dominates the resolution of the total transverse momentum in CEP events. 

There is, however, another ingredient to the total transverse momentum resolution, namely the momentum resolution of the central tracks, whose significance rises up together with increasing track $p_{\text{T}}$. Most of CEP events is characterized by low invariant mass of the central tracks pair ($\lesssim 1.5$~GeV) which is inextricably linked with low momentum of tracks ($\lesssim 0.7$~GeV), therefore an effect of $p_{\text{T}}^{\text{miss}}$ widening is barely visible in the missing momentum distribution integrated over mass (e.g. Fig.~\ref{fig:MissingPxPy}). Nevertheless one can calculate the efficiency of the exclusivity cut as a function of the central tracks momenta to directly see this effect.

Physics model of the CEP process from GenEx was used in embedded signal MC, therefore there was no sufficient statistics (too low population of high-$p_{\text{T}}$ tracks) to calculate the efficiency of $p_{\text{T}}^{\text{miss}}$ cut as a function of central tracks momenta. In such case the efficiency of this cut was calculated using simplified MC method described below.

A simple CEP event generator was used to produce large sample of CEP events (kinematics of protons and pions) in the fiducial phase space of the measurement. For each event the hadronic level momenta of positive and negative central tracks were smeared according to resolution function obtained from the embedded MC (6~MeV if $p_{\text{T}}<0.3$~GeV, $2.4~\text{MeV} + 1.2\%\times p_{\text{T}}$ if $p_{\text{T}}>0.3$~GeV). Next, the $p_{x}^{\text{miss}}$ and $p_{y}^{\text{miss}}$ were calculated from these smeared pions momenta and hadronic level forward protons momenta. Both $p_{x}^{\text{miss}}$ and $p_{y}^{\text{miss}}$ were added the Gaussian component corresponding to the experimental resolution extracted in Figs.~\ref{fig:MissingPxPy}a and~\ref{fig:MissingPxPy}b (means and standard deviations printed in the plots were used). At the end, $p_{\text{T}}^{\text{miss}}$ was calculated with the use of Eq.~\eqref{eq:missingPt}. The efficiency of the exclusivity cut was defined as a following ratio:
\begin{equation}\label{eq:ptMissCutEff}
 \mbox{\LARGE$\epsilon$}_{p_{\text{T}}^{\text{miss}}}(p^{\text{max}}, p^{\text{min}}) = \frac{N_{p_{\text{T}}^{\text{miss}}}(p^{\text{max}}, p^{\text{min}})}{N_{\text{all}}(p^{\text{max}}, p^{\text{min}})},
\end{equation}
where $N_{\text{all}}$ is a number of all generated CEP events and $N_{p_{\text{T}}^{\text{miss}}} $ is number of events passing exclusivity cut \ref{enum:CutMissingPt}. The result is shown in Fig.~\ref{fig:ptMissCutEff_MC}.

The cut efficiency for very low tracks momenta (low pair invariant mass) obtained from simple MC (96.8\%) agrees perfectly with the efficiency calculated using embedded MC sample (96.4\%) for the same momenta range $p^{max}, p^{min}<0.4$~GeV, which validates presented efficiency. Another validation comes from the comparisons with the efficiency estimated using data-driven method. The same definition of efficiency was used (Eq.~\eqref{eq:ptMissCutEff}), but $N_{p_{\text{T}}^{\text{miss}}}$ and $N_{\text{all}}$ were estimated from the $p_{\text{T}}^{\text{miss}}$ distributions after all selection cuts except~\ref{enum:CutMissingPt}, in bins of $(p^{\text{max}}, p^{\text{min}})$. The result for the data is shown in Fig.~\ref{fig:ptMissCutEff_DATA}. The efficiency for $p^{max}, p^{min}<0.4$~GeV ($(96.8\pm0.1)$\%) perfectly agrees with the simulation. Efficiencies for other selected bins also agree between data and simulation.

Clearly, the deterioration of the central track momentum resolution starts to play significant role in efficiency of $p_{\text{T}}^{\text{miss}}$ cut at about 1~GeV track momentum.


%---------------------------
\begin{figure}[h]
	\centering
	\parbox{0.4725\textwidth}{
		\centering
		\begin{subfigure}[b]{\linewidth}{
				\subcaptionbox{\label{fig:ptMissCutEff_MC}}{\includegraphics[width=\linewidth,page=1]{graphics/corrections/ptmisscuteff.pdf}\vspace*{-7pt}}}
		\end{subfigure}
	}
	\quad
	\parbox{0.4725\textwidth}{
		\centering
		\begin{subfigure}[b]{\linewidth}{
				\subcaptionbox{\label{fig:ptMissCutEff_DATA}}{\includegraphics[width=\linewidth,page=1]{graphics/corrections/ptMissEff.pdf}\vspace*{-7pt}}}
		\end{subfigure}
	}% 
    \caption[Efficiency of $p_{\text{T}}^{\text{miss}}$ cut as a function of central tracks' momenta.]{Efficiency of $p_{\text{T}}^{\text{miss}}$ cut as a function of higher ($x$-axis) and lower ($y$-axis) momentum of the central tracks, calculated with MC method described in the text of this Section~(\subref{fig:ptMissCutEff_MC}) and with the data-driven method~(\subref{fig:ptMissCutEff_DATA}). Efficiency in a few points of ($p^{\text{min}}$,~$p^{\text{max}}$) space is printed on top of each of the plots. Lower right corners have no entries/low statistics due to kinematic constraints in the fiducial phase space (this region is not polpulated with events in the data). Binning for the data is significantly wider than for the simulation due to limited data statistics.}\label{fig:ptMissCutEff}%
\end{figure}
%---------------------------


\subsubsection{Particle identification~(\ref{enum:CutPid})}\label{sec:pidEff}

The identification efficiency/misidentification probability was studied in two dimensions - as a function of particles' momenta. Because of that the shape of momentum distribution for PID efficiency study was not required to match the data and was defined flat, spanning between $0.2$~GeV and 3~GeV, to provide low statistical uncertainties over full momentum space. We considered particles' momenta as the best quantities to study pair identification because $n\sigma_{X}$ and $m^{2}_{\text{TOF}}$ and their resolutions depend nearly solely on magnitude of momentum.

\begin{figure}[t!]
  \centering
  \begin{tabular}{@{}p{0.315\linewidth}@{\quad}p{0.315\linewidth}@{\quad}p{0.315\linewidth}@{}}
    \subfigimg[width=\linewidth,page=1]{~~~~~~~~~~~~~~~~~~~~~~~a)}{graphics/corrections/EffVsPt.pdf} &
    \subfigimg[width=\linewidth,page=2]{~~~~~~~~~~~~~~~~~~~~~~~b)}{graphics/corrections/EffVsPt.pdf} &
    \subfigimg[width=\linewidth,page=3]{~~~~~~~~~~~~~~~~~~~~~~~c)}{graphics/corrections/EffVsPt.pdf} \\
    \subfigimg[width=\linewidth,page=4]{~~~~~~~~~~~~~~~~~~~~~~~d)}{graphics/corrections/EffVsPt.pdf} &
    \subfigimg[width=\linewidth,page=5]{~~~~~~~~~~~~~~~~~~~~~~~e)}{graphics/corrections/EffVsPt.pdf} &
    \subfigimg[width=\linewidth,page=6]{~~~~~~~~~~~~~~~~~~~~~~~f)}{graphics/corrections/EffVsPt.pdf} \\
    \subfigimg[width=\linewidth,page=7]{~~~~~~~~~~~~~~~~~~~~~~~g)}{graphics/corrections/EffVsPt.pdf} &
    \subfigimg[width=\linewidth,page=8]{~~~~~~~~~~~~~~~~~~~~~~~h)}{graphics/corrections/EffVsPt.pdf} &
    \subfigimg[width=\linewidth,page=9]{~~~~~~~~~~~~~~~~~~~~~~~i)}{graphics/corrections/EffVsPt.pdf}    
  \end{tabular}
  \caption[Pair identification efficiency and misidentification probability as a function of tracks' $p_{\text{T}}$.]{Pair identification efficiency (diagonal) and misidentification probability (off-diagonal) as a function of tracks' $p_{\text{T}}$ for $\pi^{+}\pi^{-}$, $K^{+}K^{-}$ and $p\bar{p}$ pairs. The results were obtained from the dedicated MC simulation described in the text. Blue lines and arrows mark the cut value on lower of the track $p_{\text{T}}$'s for kaons and protons.}\label{fig:pidEffVsPt}
\end{figure}

The identification efficiency $\epsilon^{X}$ and misidentification probability $\lambda^{X\rightarrow Y}$ were calculated for each species using the pair selection cuts~\ref{enum:CutPidNoPtLimit} described in more detail in Sec.~\ref{subsec:pidCuts}, following the definitions provided in Eqs.~\eqref{eq:pidEff} and~\eqref{eq:pMisidProb}:\vspace*{-10pt}

%
\begin{tabulary}{\textwidth}{LL}
\begin{equation}\label{eq:pidEff}
 \mbox{\LARGE$\epsilon$}^{X}(p^{\text{max}}, p^{\text{min}}) = \frac{N^{\text{reco}X}_{\text{true}X}(p^{\text{max}}, p^{\text{min}})}{N_{\text{true}X}(p^{\text{max}}, p^{\text{min}})},
\end{equation}~~~~~~~~~~~~~~~~~~~~~
&
\begin{equation}\label{eq:pMisidProb}
 \mbox{\LARGE$\lambda$}^{X\rightarrow Y}(p^{\text{max}}, p^{\text{min}}) = \frac{N^{\text{reco}Y}_{\text{true}X}(p^{\text{max}}, p^{\text{min}})}{N_{\text{true}X}(p^{\text{max}}, p^{\text{min}})},
\end{equation}~~~~~~~~~~~~~~~~~~~~~
\end{tabulary}
where $N_{\text{true}X}$ is number of $XX$ pairs, $N^{\text{reco}X}_{\text{true}X}$ is number of $XX$ pairs correctly recognized as $XX$, $N^{\text{reco}Y}_{\text{true}X}$ is number of $XX$ pairs misidentified as $YY$ and $p^{\text{max}}$ and $p^{\text{min}}$ are respectively higher and lower momentum of particles in the pair. The results as a function of unsmeared true-level $(p^{\text{max}}, p^{\text{min}})$ are presented in Fig.~\ref{fig:pidEffVsP} (Appendix~\ref{appendix:pidEff}). Here in Fig.~\ref{fig:pidEffVsPt} we show the result presented as a function of higher and lower transverse momentum of particles in the pair $(p_{\text{T}}^{\text{max}}, p_{\text{T}}^{\text{min}})$. This result clearly indicated necessity of adding a cut on the lower transverse momentum of particle in the pair for $K^{+}K^{-}$ and $p\bar{p}$, as above certain $p_{\text{T}}^{\text{min}}$ the identification efficiency drops to a very low level ($<10\%$). The cuts (\ref{enum:CutPidPtLimits}) are $p_{\text{T}}^{\text{min}}>0.7$~GeV for kaons and $p_{\text{T}}^{\text{min}}>1.1$~GeV for protons, indicated with dashed blue lines and arrows.

%---------------------------









\subsection{RP track acceptance and reconstruction efficiency}\label{sec:rpAccAndEff}

To calculate RP acceptance and track reconstruction efficiency the embedded MC technique was used. The same sample was used as that described in Sec.~\ref{sec:rpDeadMat}, used for calculation of the dead-material related trigger veto.

The joint RP acceptance and track reconstruction efficiency for a given STAR side, $\mbox{\LARGE$\epsilon$}_{\text{RP}}^{\text{side}}$, was calculated as a probability that a single good quality RP track (satisfying cuts~\ref{enum:RpQualityCuts}-\ref{enum:RpLocalAngles}) matched with true-level primary forward proton is reconstructed on given side in the branch expected based on sign of $p_{y}$ of the proton, under condition that there is a trigger signal in that branch and there is no trigger signal in the other branch on the same side.

Technically the $\mbox{\LARGE$\epsilon$}_{\text{RP}}^{\text{side}}$ was obtained in the following procedure:
\begin{enumerate}
	\item It was verified if there is a trigger signal in the branch that the primary forward proton is expected to reach based on its $p_{y}$ ($p_{y}>0$ - branch UP, $p_{y}<0$ - branch DOWN). Additionally required lack of trigger signal in the other branch on the same side. These events formed $set~A$.
	\item The nominal RP track selection algorithm was used to find a single good quality track (cuts~\ref{enum:RpQualityCuts}-\ref{enum:RpLocalAngles}) on given side. If exactly one such track was found, it was additionally checked if it is matched with true-level primary proton. These events formed $set~B$.
	\item The efficiency was determined by the ratio of histograms from $set~B$ and $set~A$:
	\begin{equation}\label{eq:rpEffDef}
 \mbox{\LARGE$\epsilon$}_{\text{RP}}^{\text{side}}(p_{x}, p_{y}, z_{\text{vtx}}) = \mbox{\LARGE$\varepsilon$}\left(\RPSIDE\left|\TRSIDE\land~!\TRNSIDE\right.\right) = %
 \frac{(p_{x}, p_{y}, z_{\text{vtx}})~\text{histogram for protons from}~set~B}{(p_{x}, p_{y}, z_{\text{vtx}})~\text{histogram for protons from}~set~A}
  \end{equation}
	
\end{enumerate}

It should be noted that the momentum components $(p_{x}, p_{y})$ were taken from the proton with accounted effect of the beam divergence (after the original initial momentum smearing).

%---------------------------
\begin{figure}[ht!]%
\centering%
\begin{minipage}{.4725\textwidth}%
  \centering%
  \includegraphics[width=\linewidth,page=10]{graphics/corrections/mcFullEffPxPy.pdf}%
  \caption[Sample RP track reconstruction efficiency in a single $z$-vertex bin.]{Sample RP track reconstruction efficiency in a single $z$-vertex bin on the east STAR side. The efficiency was calculated using forward proton MC simulation embedded into zero-bias data. Green envelopes mark the fiducial region of the measurement, while dashed yellow lines mark the part of the fiducial region with a data-driven efficiency correction needed, as explained in Sec.~10.3.1 of Ref.~\cite{supplementaryNote}.}\label{fig:rpEffSample}
\end{minipage}%
\quad\quad%
\begin{minipage}{.4725\textwidth}%
  \centering%
% % %   \includegraphics[width=\linewidth]{graphics/corrections/SimultaneousEastWestRpTrackLoss.pdf}%
% % %   \caption{Probability of simultaneous inefficiency of the RP track reconstruction calculated from the CEP MC embedded into zero-bias data.\newline\newline\newline\newline\newline}\label{fig:SimultaneousEastWestRpTrackLoss}
  \includegraphics[width=\linewidth]{graphics/corrections/RpTotalEffEastWestCorrelationVsMandelstamTSum.pdf}%
  \caption{Correlation between joint RP track reconstruction and trigger efficiency (no dead material veto) in east and west RPs calculated from the CEP MC embedded into zero-bias data. The result for protons within the fiducial region and with vertex position $|z_{\text{vtx}}|<80$~cm in four different combinations of east and west branches are presented as a function of the total squared four-momentum transfer.}\label{fig:SimultaneousEastWestRpTrackLoss}
\end{minipage}%
\end{figure}%
%---------------------------


The efficiency calculated in the way described above is, by design, the acceptance, reconstruction and selection efficiency for a single proton on one side of the IP. In CEP event there are two independent forward protons which may be simultaneously not reconstructed or rejected by the selection algorithm due to e.g. elastic pile-up interaction providing additional good quality proton tracks on both sides of IP. One could, in principle, calculate 5-dimensional efficiency for both forward protons (in variables $p_{x}^{\text{E}}, p_{y}^{\text{E}}, p_{x}^{\text{W}}, p_{y}^{\text{W}}$ and $z_{\text{vtx}}$) which would ultimately account for the simultaneous east and west inefficiency, however this would require orders of magnitude larger statistics of MC to provide reasonably low statistical uncertainty of the efficiency. Instead, on top of the 3-dimensional reconstruction and selection efficiencies for east and west RPs we calculate (from embedded MC) correlation between these efficiencies, as defined in Eq.~\eqref{eq:rpEffCorrelation} of appendix~\ref{appendix:totalRpEffFormulation}. It is shown in 
Fig.~\ref{fig:SimultaneousEastWestRpTrackLoss} for all four comibnations of east and west RP branches. A need to account for this correlation was found based on the closure tests performed on the MC sample.

% probability that the proton tracks are simultaneously not reconstructed or selected on the east and west side, despite the trigger signal solely in expected RP branches (no trigger veto) and true-level $(p_{x}, p_{y})$ of both forward protons contained in the fiducial region. This probability, denoted as
% \begin{equation}
% \mathcal{P}^{\text{E\&W}}_{\text{loss}} = \mbox{\LARGE$\varepsilon$}\left(!\RPE\land~!\RPW\Big|\TRE\land\TRW\land~!\V\Big.\right),
% \end{equation}
%  is presented in Fig.~\ref{fig:SimultaneousEastWestRpTrackLoss} for all four comibnations of east and west RP branches.




\newpage
\subsection{TPC vertex reconstruction efficiency}\label{sec:tpcVxRecoEff}

%---------------------------
\begin{wrapfigure}{o}{0.365\textwidth}\vspace*{-9pt}
  \centering
  \includegraphics[width=0.365\textwidth]{graphics/corrections/deltaZ0Definition2.pdf}
  \caption[Sketch with definition of $\Delta z_{0}$.]
   {Sketch with definition of the longitudinal separation of two tracks (helices) $\Delta z_{0}$.}
   \label{fig:deltaZ0Sketch}\vspace*{-9pt} 
\end{wrapfigure}
%---------------------------

The definition of vertex reconstruction efficiency ($\mbox{\LARGE$\epsilon$}_{\text{vtx}}$) established in this analysis is the probability that two global tracks, both associated with true-level primary particles from the kinematic region of the measurement, both satisfying kinematic and quality criteria (cuts~\ref{enum:TpcKinematicCuts} and ~\ref{enum:TpcQualityCuts}) and both matched with hits in TOF, form a vertex listed in the collection of reconstructed primary vertices and DCA(R) and DCA(z) of both global tracks calculated w.r.t. this vertex is contained within the limits of cut~\ref{enum:TpcDcaCuts}.

We calculated this efficiency as a function of the longitudinal separation between two tracks (global helices) $\Delta z_{0}$. Illustration of this quantity is given in Fig.~\ref{fig:deltaZ0Sketch}. We consider this a natural quantity to present the vertexing efficiency - the closer to each other the helices are on the beamline, the more probable it is that two tracks will form a common primary vertex. It is in accordance with the way the vertexing algorithm works.


The vertexing efficiency was calculated from the data in the following way:
\begin{enumerate}
\item Data from RP\_CPT2 trigger were used. Events were selected with nearly the same cuts as in nominal CEP analysis (Sec.~\ref{sec:listOfCuts}). The requirement of exactly one primary vertex with exactly two primary TOF tracks was dropped. Instead, analysis utilized only global TOF tracks - exactly two global TOF tracks were required (cut~\ref{enum:TpcTofMatched} without primary track requirement), passing also cuts~\ref{enum:TpcOppoSign}-\ref{enum:TpcQualityCuts}. In this case the position of the vertex was reconstructed as\vspace*{-10pt}
\begin{equation}
 z_{\text{vtx}} = \frac{z_{0}^{+}+z_{0}^{-}}{2},\vspace*{-5pt}
\end{equation}
where $z_{0}^{+}$ and $z_{0}^{-}$ are longitudinal impact parameters ($z$-coordinates of points of closest approach to the beamline) of positive and negative charge particle tracks, respectively. The vertex position was normally required to satisfy cut~\ref{enum:CutZVx}. Events after full selection, classified as exclusive $\pi^{+}\pi^{-}$ candidates, formed $set~A$.

\item The two global TOF tracks were checked if they have associated primary tracks, and the two primary tracks are assigned to the same primary vertex. If yes, the tracks were additionally subjected to cut~\ref{enum:TpcDcaCuts}. Events passing described selection formed $set~B$.

	\item The efficiency was determined by the ratio of histograms from $set~B$ and $set~A$:
	\begin{equation}\label{eq:vertexingEffDef}
 \mbox{\LARGE$\epsilon$}_{\text{vtx}}(|\Delta z_{0}|) = \frac{|\Delta z_{0}|~\text{histogram for events from}~set~B}{|\Delta z_{0}|~\text{histogram for events from}~set~A}.
  \end{equation}
\end{enumerate}

Distribution of $|\Delta z_{0}|$ between two CEP global track candidates after full selection ($set A$) is presented in Fig.~\ref{fig:deltaZ0}. The vertexing efficiency obtained with described method is shown in Fig.~\ref{fig:vertexingEff}. Solid green points represent efficiency calculated with the non-exclusive background preserved, while open black points represent efficiency with this background subtracted% (using method described in Sec.~\ref{sec:bkgdSubtraction})%
. Since the vertexing efficiency does not depend on the physics process and background is purely of physics origin the black and green points should overlap. Such picture emerges from presented comparison. We have calculated the same efficiency using CEP MC embedded into zero-bias data, the result is shown in Fig.~\ref{fig:vertexingEff} with red points. There is very good agreement between vertexing efficiency in the data and embedded MC for $|\Delta z_{0}|<1$~cm, where most ($\sim80\%$) of the signal is present. The differences in high-$|\Delta z_{0}|$ tail are understood as a result of the imperfect description of the pointing resolution (here: the transverse resolution) of TPC tracks in STARsim. Although the pointing resolution is adjusted to gain more accurate description of the data by MC simulation as described in Chapter~8 of Ref.~\cite{supplementaryNote}, this does not help with the vertexing which is performed at the level of raw data (MC) processing to MuDst. Another reason could be different $p_{\text{T}}$ (thus also $d_{0}$) spectrum of CEP tracks in the data and MC (GenEx).

%---------------------------
\begin{figure}[ht!]%
\centering%
\begin{minipage}{.4725\textwidth}%
  \centering%
  \includegraphics[width=\linewidth]{graphics/corrections/DeltaZ0WithBackground.pdf}%
  \caption[Distribution of $\Delta z_{0}$ together with fraction of non-exclusive background.]{Distribution of $\Delta z_{0}$ between two CEP-candidate global tracks (green and black points) together with fraction of non-exclusive background in black distribution as a function of $\Delta z_{0}$ (violet points).}\label{fig:deltaZ0}
\end{minipage}% 
\quad\quad%
\begin{minipage}{.4725\textwidth}%
  \centering
  \includegraphics[width=\linewidth]{graphics/corrections/VertexingEfficiency.pdf}%
  \caption[Vertexing efficiency.]{Vertexing efficiency calculated from the data (open black and full green points) and from embedded MC (red points) as a function of $\Delta z_{0}$ between two CEP-candidate global tracks.}\label{fig:vertexingEff}
\end{minipage}%
\end{figure}%
%---------------------------

Based on the width of $|\Delta z_{0}|$ distribution, the background content as a function of $|\Delta z_{0}|$, and the value of $\mbox{\LARGE$\epsilon$}_{\text{vtx}}$ as a function of $|\Delta z_{0}|$, we decided to accept in analysis only tracks which satisfy $|\Delta z_{0}|<2$~cm. This assures that the vertexing efficiency does not drop below $\approx 30\%$, as well as it coincides with the primary tracks requirement of $|\text{DCA}(z)|<1$~cm. In the correction procedure we nominally use the vertexing efficiency represented by open black points in Fig.~\ref{fig:vertexingEff} (to correct the MC e.g. in closure tests we use red points instead).

The cut on maximum $|\Delta z_{0}|$ (2~cm) rejects some small fraction of the signal events. An additional correction connected with the limit on $|\Delta z_{0}|$ was calculated from the data as a function of the average $p_{\text{T}}$ of two tracks in a pair. This quantity was chosen because the distribution of $\Delta z_{0}$ gets narrower with increasing $p_{\text{T}}$ of the tracks (with increasing invariant mass of the pair), and the largest contribution to the width of $\Delta z_{0}$ comes from the track of lower $p_{\text{T}}$. The efficiency was calculated in the similar way to the vertexing efficiency, with a modification of the $set~B$ to events with the global tracks from $set~A$ satisfying a limit $|\Delta z_{0}|<2~\text{cm}$ (called $set~B'$):
	\begin{equation}\label{eq:deltaZ0EffDef}
\mbox{\LARGE$\epsilon$}_{|\Delta z_{0}|} = \frac{|\Delta z_{0}|~\text{histogram for events from}~set~B'}{|\Delta z_{0}|~\text{histogram for events from}~set~A}.
\end{equation}

Figure~\ref{fig:DeltaZ0Eff} presents the efficiency of $|\Delta z_{0}|<2~\text{cm}$ as a function of the lower $p_{\text{T}}$ of the central tracks. One can see that the MC simulation (with helices adjusted to match the pointing resolution in data) reasonably well describes the data-driven efficiency.

%---------------------------
\begin{figure}[ht!]
	\centering%
	\parbox{0.4725\textwidth}{%
		\centering%
		\includegraphics[width=\linewidth]{graphics/corrections/DeltaZ0Efficiency_2.pdf}
	}%
	\quad%
	\parbox{0.4725\textwidth}{%
		\caption[Efficiency of limit $|\Delta z_{0}|<2~\text{cm}$.]{Efficiency of cut on $|\Delta z_{0}|$ calculated from the data (open black and full green points) and from embedded MC (red points) as a function of the lower $p_{T}$ of the two global tracks in CEP event candidates.}\label{fig:DeltaZ0Eff}%  
	}
\end{figure}
%---------------------------
\newpage

\section{Particle energy loss}\label{sec:energyLoss}

Energy loss correction as a function of reconstructed particle $p_{\text{T}}$ in bins of $z$-position of reconstructed vertex has been calculated and presented in Chapter~5 of Ref.~\cite{supplementaryNote} for all analyzed particle species and both positive and negative charges. The correction was applied independetly for each particle in the following procedure:

\begin{enumerate}
	\item After central particles were identified (cut~\ref{enum:CutPid}) an absolute value of the particle transverse momentum correction ($-\Delta p_{\text{T}} = p_{\text{T}}^{\text{meas}}-p_{\text{T}}^{\text{true}}$) was read from the histogram corresponding to reconstructed $z_{\text{vtx}}$ and to assigned particle ID (Appendix~C of Ref.~\cite{supplementaryNote}).
	\item The momentum correction factor $f_{p}^{\text{corr}}$ was calculated:
	\begin{equation}\label{eq:pCorrFactor}
 f_{p}^{\text{corr}} = \frac{p_{\text{T}}^{\text{meas}} + \Delta p_{\text{T}}}{p_{\text{T}}^{\text{meas}}}.
  \end{equation}
  \item New, corrected momentum $\vec{p}^{corr}$ was assigned to the particle:
  \begin{equation}
   \vec{p}^{~\text{corr}} = f_{p}^{\text{corr}} \cdot \vec{p}^{~\text{meas}}.
  \end{equation}
  In this way all three components of particle momentum are corrected so that the pseudorapidity of a particle remains unchanged.
\end{enumerate}
This new momentum was further used in determination of total transverse momentum of all reconstructed particles, $p_{\text{T}}^{\text{miss}}$, as well as in applying TPC and TOF efficiency corrections and preparing histograms (cross sections) of physics quantities (e.g. invariant mass, rapidity of a pair of central tracks).



\section{Fake tracks and migrations into and out of fiducial region}\label{sec:migrations}

Finite resolutions play non-negligible role in presented analysis. Smearing of reconstructed quantities with respect to their true value at the hadron level makes some tracks from outside of the fiducial phase space migrating/leaking into it, as well as some tracks from inside the fiducial region escaping from it.

Correction related to migrations and fake tracks was calculated using quantities described in the listing below. It applies to both TPC and RP tracks.

\storestyleof{itemize}
\begin{listliketab}
    \begin{tabular}{Llp{0.86\linewidth}}
        \textbullet & \textbf{$\bm{N_{\textrm{reco}}}$} & - number of true level-matched tracks reconstructed inside fiducial region,\\
        \textbullet & \textbf{$\bm{N_{\textrm{migr.}}^{\textrm{in}}}$} & - number of true level-matched tracks from outside of fiducial region (at the hadron level), reconstructed inside fiducial regions,\\
        \textbullet & \textbf{$\bm{N_{\textrm{migr.}}^{\textrm{out}}}$} & - number of true level-matched tracks from inside of fiducial region (at the hadron level), reconstructed outside fiducial region,\\
        \textbullet & \textbf{$\bm{N_{\textrm{fakes}}}$} & - number of tracks not matched with true level primary but consistent with it, reconstructed inside fiducial region.\\
    \end{tabular}
\end{listliketab}

All quantities above are represented as a function of reconstructed values (TPC tracks: $p_{T}$ and $\eta$, RP tracks: $p_{x}$ and $p_{y}$), except $N_{\textrm{migr.}}^{\textrm{out}}$ for which true (hadron) level values are used. Definitions of fraction of migrations into and out of fiducial region, as well as fraction of fake tracks inside the fiducial region, are provided in Eqs.~\eqref{eq:migrIn}-\eqref{eq:fakes} in the sane order:

\begin{tabulary}{\textwidth}{LLL}
\begin{equation}\label{eq:migrIn}%\hspace*{-10pt}
	f_{\text{migr.}}^{\text{in}} = \frac{N_{\textrm{migr.}}^{\textrm{in}}}{N_{\textrm{reco}}},
\end{equation}~~~~~~~~~~~~~~~~~~~~~~~~~~~~&
\begin{equation}\label{eq:migrOut}%\hspace*{-10pt}
	f_{\text{migr.}}^{\text{out}} = \frac{N_{\textrm{migr.}}^{\textrm{out}}}{N_{\textrm{reco}}},
\end{equation}~~~~~~~~~~~~~~~~~~~~~~~~~&
\begin{equation}\label{eq:fakes}
f_{\text{fakes}} = \frac{N_{\textrm{fakes}}}{N_{\textrm{reco}}}.
\end{equation}~~~~~~~~~~~~~~~~~~~~~~~
\end{tabulary}

Definition of number of fake tracks requires some elaboration. The necessary condition to claim a track being fake is that it is not matched with a true level primary particle. However, such tracks should typically have momentum significantly different from that of unmatched primary, therefore total momentum cut~\ref{enum:CutMissingPt} would not be satisfied and thus these tracks should not be subtracted. For this reason, an additional condition to assign a track as fake is that its reconstructed momentum is consistent with that of primary particle. In case of TPC tracks we use cut on the distance $\delta$ between true and reconstructed $(\eta,\phi)$, $\delta < 0.15$, introduced in Sec.~3.2.2 of Ref.~\cite{supplementaryNote}. For the RP tracks we use cut on the difference between true and reconstructed transverse momentum not greater than~70~MeV. Such value, somewhat arbitrary, is driven by the value used in cut~\ref{enum:CutMissingPt}. We do not consider any systematic effect connected with the fake RP track definition as this effect is believied to be much smaller than most of other systematic uncertainties. In Figs.~\ref{fig:DeltaPhiVsDeltaEta} and~\ref{fig:DeltaPyVsDeltaPx} we present distributions of quantities for matched and not matched tracks with magenta circles drawn on top marking the conditions used in fake track definitions.


%--------------------------- 
\begin{figure}%[ht!] 
\centering
\parbox{0.4725\textwidth}{
  \centering
  \begin{subfigure}[b]{\linewidth}{
                \subcaptionbox{\label{fig:DeltaPhiVsDeltaEta_Matched}}{\includegraphics[width=\linewidth,page=1]{graphics/corrections/DeltaPhiVsDeltaEta.pdf}\vspace*{-7pt}}}
  \end{subfigure}
}
\quad
\parbox{0.4725\textwidth}{
  \centering
  \begin{subfigure}[b]{\linewidth}{
                \subcaptionbox{\label{fig:DeltaPhiVsDeltaEta_NotMatched}}{\includegraphics[width=\linewidth,page=2]{graphics/corrections/DeltaPhiVsDeltaEta.pdf}\vspace*{-7pt}}}
  \end{subfigure}
}% 
\caption[Difference between reconstructed and true parameters of the TPC tracks.]{Difference between reconstructed and true parameters of the TPC tracks. Magenta circle of radius 0.15 drawn on top of distributions represent cut on maximum difference between true and reconstructed $\eta$ and $\varphi$ of tracks used for determination of fake tracks fraction, as explained in the text. The same $z$-axis (color) scale has been set up for both plots to better visualize relative content of true and fake tracks.} \label{fig:DeltaPhiVsDeltaEta}
\end{figure}


%--------------------------- 
\begin{figure}%[ht!] 
\centering
\parbox{0.4725\textwidth}{
  \centering
  \begin{subfigure}[b]{\linewidth}{
                \subcaptionbox{\label{fig:DeltaPyVsDeltaPx_Matched}}{\includegraphics[width=\linewidth,page=1]{graphics/corrections/DeltaPyVsDeltaPx.pdf}\vspace*{-7pt}}}
  \end{subfigure}
}
\quad
\parbox{0.4725\textwidth}{
  \centering
  \begin{subfigure}[b]{\linewidth}{
                \subcaptionbox{\label{fig:DeltaPyVsDeltaPx_NotMatched}}{\includegraphics[width=\linewidth,page=2]{graphics/corrections/DeltaPyVsDeltaPx.pdf}\vspace*{-7pt}}}
  \end{subfigure}
}% 
\caption[Difference between reconstructed and true momentum components of RP tracks.]{Difference between reconstructed and true momentum components of RP tracks matched (\subref{fig:DeltaPyVsDeltaPx_Matched}) and not matched (\subref{fig:DeltaPyVsDeltaPx_NotMatched}) with true level primary forward protons. Magenta circle of radius 70~MeV drawn on top of distributions represent cut on maximum difference between true and reconstructed transverse momentum used for determination of fake tracks fraction, as explained in the text. The same $z$-axis (color) scale has been set up for both plots to better visualize relative content of true and fake tracks.} \label{fig:DeltaPyVsDeltaPx}
\end{figure}


Fraction of tracks, for both TPC and RP detectors, migrating from/to fiducial region and being fake, is shown in Figs.~\ref{fig:migrationsAndFakes_TPC} and~\ref{fig:migrationsAndFakes_RP}. Clearly, all mentioned effects are dominant for RP tracks.



Multiplicative correction factor (per track) that takes into account all migrations and fakes can be formed as
\begin{equation}
 C_{\text{m,f}} = \frac{N_{\textrm{reco}} + N_{\textrm{migr.}}^{\textrm{out}} - N_{\textrm{migr.}}^{\textrm{in}} - N_{\textrm{fakes}}}{N_{\textrm{reco}}} = 1 + f_{\text{migr.}}^{\text{out}} - f_{\text{migr.}}^{\text{in}} - f_{\text{fakes}}.
\end{equation}
This correction is graphically represented in Fig.~\ref{fig:migrationsAndFakesCorr} for both TPC (\ref{fig:migrationPtEta}) and RP (\ref{fig:migrationPxPy}) tracks. One can easily notice that most important role plays correction related to migrations at the edge of fiducial region defining accepted forward scattered protons, reaching as much as 25\% correction for tracks at the corners of allowed $(p_{x}, p_{y})$. This is a direct consequence of sizeable angular divergence of proton beams.


%--------------------------- 
\begin{figure}\vspace{-30pt}%[ht!] 
\centering
\parbox{0.315\textwidth}{
  \centering
  \begin{subfigure}[b]{\linewidth}{
                \subcaptionbox{\label{fig:MigrationsInto_PtEta}}{\includegraphics[width=1.05\linewidth]{graphics/corrections/MigrationsInto_PtEta.pdf}\vspace*{-5pt}}}
  \end{subfigure}
}
\quad
\parbox{0.315\textwidth}{
  \centering
  \begin{subfigure}[b]{\linewidth}{
                \subcaptionbox{\label{fig:MigrationsOutside_PtEta}}{\includegraphics[width=1.05\linewidth]{graphics/corrections/MigrationsOutside_PtEta.pdf}\vspace*{-5pt}}}
  \end{subfigure}
}
\quad
\parbox{0.315\textwidth}{
  \centering
  \begin{subfigure}[b]{\linewidth}{
                \subcaptionbox{\label{fig:Fakes_PtEta}}{\includegraphics[width=1.05\linewidth]{graphics/corrections/Fakes_PtEta.pdf}\vspace*{-5pt}}}
  \end{subfigure}
}\vspace*{-5pt}% 
\caption[Fractions of migrating and fake TPC tracks matched with TOF.]{Fraction of TOF-matched TPC tracks migrating into the fiducial central tracks phase space from outside of it (\subref{fig:MigrationsInto_PtEta}), migrating outside the fiducial phase space from inside of it (\subref{fig:MigrationsOutside_PtEta}), and fraction of fake TOF matched TPC tracks reconstructed in the fiducial phase space (\subref{fig:Fakes_PtEta}). In case of reconstructed ($p_{T}$, $\eta$) escaping the fiducial phase space (\subref{fig:MigrationsOutside_PtEta}) the true value of $p_{T}$ and $\eta$ is used; otherwise reconstructed values are plotted.}
\label{fig:migrationsAndFakes_TPC}%
\end{figure}


%--------------------------- 
\begin{figure}%[ht!] 
\centering
\parbox{0.315\textwidth}{
  \centering
  \begin{subfigure}[b]{\linewidth}{
                \subcaptionbox{\label{fig:MigrationsInto_PxPy}}{\includegraphics[width=1.05\linewidth]{graphics/corrections/MigrationsInto_PxPy.pdf}}}
  \end{subfigure}
}
\quad
\parbox{0.315\textwidth}{
  \centering
  \begin{subfigure}[b]{\linewidth}{
                \subcaptionbox{\label{fig:MigrationsOutside_PxPy}}{\includegraphics[width=1.05\linewidth]{graphics/corrections/MigrationsOutside_PxPy.pdf}}}
  \end{subfigure}
}
\quad
\parbox{0.315\textwidth}{
  \centering
  \begin{subfigure}[b]{\linewidth}{
                \subcaptionbox{\label{fig:Fakes_PxPy}}{\includegraphics[width=1.05\linewidth]{graphics/corrections/Fakes_PxPy.pdf}}}
  \end{subfigure}
}% 
\caption[Fractions of migrating and fake RP tracks.]{Fraction of RP tracks migrating into the fiducial forward proton momentum area from outside of it (\subref{fig:MigrationsInto_PxPy}), migrating outside the fiducial area from inside of it (\subref{fig:MigrationsOutside_PxPy}), and fraction of fake RP tracks reconstructed in the fiducial area (\subref{fig:Fakes_PxPy}). In case of reconstructed tracks momentum escaping the fiducial area (\subref{fig:MigrationsOutside_PxPy}) the true value of $p_{x}$ and $p_{y}$ is used; otherwise reconstructed momentum components are plotted.} \label{fig:migrationsAndFakes_RP}
\end{figure}





%--------------------------- 
\begin{figure}%[ht!] 
\centering
\parbox{0.4725\textwidth}{
  \centering
  \begin{subfigure}[b]{\linewidth}{
                \subcaptionbox{\label{fig:migrationPtEta}}{\includegraphics[width=\linewidth]{graphics/corrections/MigrationsAndFakes_PtEta.pdf}\vspace*{-7pt}}}
  \end{subfigure}
}
\quad
\parbox{0.4725\textwidth}{
  \centering
  \begin{subfigure}[b]{\linewidth}{
                \subcaptionbox{\label{fig:migrationPxPy}}{\includegraphics[width=\linewidth]{graphics/corrections/MigrationsAndFakes_PxPy.pdf}\vspace*{-7pt}}}
  \end{subfigure}
}% 
\caption[Joint corrections for migrations and fake tracks for TPC and RP tracks.]{Joint correction factors for migrations and fake tracks for TOF-matched TPC tracks (\ref{fig:migrationPtEta}) and RP tracks (\ref{fig:migrationPxPy}).} \label{fig:migrationsAndFakesCorr}
\end{figure}



% \section{Background subtraction}\label{sec:bkgdSubtraction}
% \section{Unfolding}\label{sec:unfolding}

\section{Detector resolutions}\label{sec:resolutions}

In this section we present studies of detector resolutions of quantities relevant in the physics analysis. These resolutions were used to establish proper binning of the measured differential cross sections.

% %---------------------------
% \begin{figure}[h]
% \centering
% \parbox{0.4725\textwidth}{
%   \centering
%   \begin{subfigure}[b]{\linewidth}
%                 \subcaptionbox{\label{fig:ResponseMatrix_InvMass}}{\includegraphics[width=1.05\linewidth]{graphics/corrections/ResponseMatrix_InvMass.pdf}\vspace*{-70pt}}\vspace{50pt}
%   \end{subfigure}\\
%   \begin{subfigure}[b]{\linewidth}\addtocounter{subfigure}{1}
%                 \subcaptionbox{\label{fig:ResponseMatrix_MandelstamTSum}}{\includegraphics[width=1.05\linewidth]{graphics/corrections/ResponseMatrix_MandelstamTSum.pdf}\vspace*{-70pt}}\vspace{50pt}
%   \end{subfigure}\\
%   \begin{subfigure}[b]{\linewidth}\addtocounter{subfigure}{1}
%                 \subcaptionbox{\label{fig:ResponseMatrix_CosThetaCS}}{\includegraphics[width=1.05\linewidth]{graphics/corrections/ResponseMatrix_CosThetaCS.pdf}\vspace*{-70pt}}\vspace{50pt}
%   \end{subfigure}
% }%
% \quad\quad%
% \parbox{0.4725\textwidth}{
%   \centering
%   \begin{subfigure}[b]{\linewidth}\addtocounter{subfigure}{-4}
%                 \subcaptionbox{\label{fig:ResponseMatrix_PairRapidity}}{\includegraphics[width=1.05\linewidth]{graphics/corrections/ResponseMatrix_PairRapidity.pdf}\vspace*{-70pt}}\vspace{50pt}
%   \end{subfigure}\\
%   \begin{subfigure}[b]{\linewidth}\addtocounter{subfigure}{1}
%                 \subcaptionbox{\label{fig:ResponseMatrix_DeltaPhi}}{\includegraphics[width=1.05\linewidth]{graphics/corrections/ResponseMatrix_DeltaPhi.pdf}\vspace*{-70pt}}\vspace{50pt}
%   \end{subfigure}\\
%   \begin{subfigure}[b]{\linewidth}\addtocounter{subfigure}{1}
%                 \subcaptionbox{\label{fig:ResponseMatrix_PhiCS}}{\includegraphics[width=1.05\linewidth]{graphics/corrections/ResponseMatrix_PhiCS.pdf}\vspace*{-70pt}}\vspace{50pt}
%   \end{subfigure}
% }%
% \caption[Response matrices.]{Response matrices. Probabilities in each bin of the true-level value ($y$-axis) sum up to 1.}\label{fig:responseMatrices}
% \end{figure}
% %---------------------------



%---------------------------
\begin{figure}[h]
\centering
\parbox{0.4725\textwidth}{
  \centering
  \begin{subfigure}[b]{\linewidth}
                \subcaptionbox{\label{fig:RecoMinusTrueVsReco_InvMass}}{\includegraphics[width=1.05\linewidth,page=1]{graphics/corrections/RecoMinusTrueVsReco_InvMass.pdf}}
  \end{subfigure}\\
  \begin{subfigure}[b]{\linewidth}\addtocounter{subfigure}{1}
                \subcaptionbox{\label{fig:ResponseMatrix_InvMass}}{\includegraphics[width=1.05\linewidth,page=1]{graphics/corrections/ResponseMatrix_InvMass.pdf}}
  \end{subfigure} 
}%
\quad\quad%
\parbox{0.4725\textwidth}{
  \centering
  \begin{subfigure}[b]{\linewidth}\addtocounter{subfigure}{-2}\vspace*{-12pt}
                \subcaptionbox{\label{fig:RecoMinusTrueVsTrue_InvMass}}{\includegraphics[width=1.05\linewidth,page=1]{graphics/corrections/RecoMinusTrueVsTrue_InvMass.pdf}}
  \end{subfigure}\\
    \begin{minipage}[t][1.042\linewidth][t]{\linewidth}\vspace{10pt}
    \caption[Resolution of the invariant mass measurement from embedded $\pi^{+}\pi^{-}$ MC.]{Resolution of the invariant mass measurement from embedded $\pi^{+}\pi^{-}$ MC. Figure~\subref{fig:RecoMinusTrueVsReco_InvMass} and~\subref{fig:RecoMinusTrueVsTrue_InvMass} show the difference between reconstructed and hadron level invariant mass of $\pi^{+}\pi^{-}$ ($y$-axis) as a function of the reconstructed and hadron level invariant mass ($x$-axis), respectively. Probabilities in each $x$-axis bin were normalized to sum up to 1. On top of the two-dimensional distribution (in color) a candle plot is drawn. For each $x$-axis bin a circle indicate average difference, a box represents 68\% coverage of the distribution of difference symmetrically around median ($1\sigma$), while vertical lines extending beyond the box represent 99.3\% coverage of the distribution of difference symmetrically around median ($2.7\sigma$). A dashed black line is drawn across the plot to mark the difference between reconstructed and hadron level value equal to 0. Figure~\subref{fig:ResponseMatrix_InvMass} shows the response matrix - correlation between reconstructed ($x$-axis) and hadron level ($y$-axis) invariant mass. Probabilities in each bin of the true-level value were normalized to sum up to 1.}\label{fig:Response_InvMass}%
  \end{minipage}
}%
\end{figure}
%---------------------------


%---------------------------
\begin{figure}[h]
\centering
\parbox{0.4725\textwidth}{
  \centering
  \begin{subfigure}[b]{\linewidth}
                \subcaptionbox{\label{fig:RecoMinusTrueVsReco_PairRapidity}}{\includegraphics[width=1.05\linewidth,page=1]{graphics/corrections/RecoMinusTrueVsReco_PairRapidity.pdf}}
  \end{subfigure}\\
  \begin{subfigure}[b]{\linewidth}\addtocounter{subfigure}{1}
                \subcaptionbox{\label{fig:ResponseMatrix_PairRapidity}}{\includegraphics[width=1.05\linewidth,page=1]{graphics/corrections/ResponseMatrix_PairRapidity.pdf}}
  \end{subfigure} 
}%
\quad\quad%
\parbox{0.4725\textwidth}{
  \centering
  \begin{subfigure}[b]{\linewidth}\addtocounter{subfigure}{-2}\vspace*{-12pt}
                \subcaptionbox{\label{fig:RecoMinusTrueVsTrue_PairRapidity}}{\includegraphics[width=1.05\linewidth,page=1]{graphics/corrections/RecoMinusTrueVsTrue_PairRapidity.pdf}}
  \end{subfigure}\\
    \begin{minipage}[t][1.042\linewidth][t]{\linewidth}\vspace{10pt}
    \caption[Resolution of the rapidity measurement from embedded $\pi^{+}\pi^{-}$ MC.]{Resolution of the rapidity measurement from embedded $\pi^{+}\pi^{-}$ MC. Figure~\subref{fig:RecoMinusTrueVsReco_PairRapidity} and~\subref{fig:RecoMinusTrueVsTrue_PairRapidity} show the difference between reconstructed and hadron level rapidity of $\pi^{+}\pi^{-}$ ($y$-axis) as a function of the reconstructed and hadron level rapidity ($x$-axis), respectively. Probabilities in each $x$-axis bin were normalized to sum up to 1. On top of the two-dimensional distribution (in color) a candle plot is drawn. For each $x$-axis bin a circle indicate average difference, a box represents 68\% coverage of the distribution of difference symmetrically around median ($1\sigma$), while vertical lines extending beyond the box represent 99.3\% coverage of the distribution of difference symmetrically around median ($2.7\sigma$). A dashed black line is drawn across the plot to mark the difference between reconstructed and hadron level value equal to 0. Figure~\subref{fig:ResponseMatrix_PairRapidity} shows the response matrix - correlation between reconstructed ($x$-axis) and hadron level ($y$-axis) rapidity. Probabilities in each bin of the true-level value were normalized to sum up to 1.}\label{fig:Response_PairRapidity}%
  \end{minipage}
}%
\end{figure}
%---------------------------


%---------------------------
\begin{figure}[h]
\centering
\parbox{0.4725\textwidth}{
  \centering
  \begin{subfigure}[b]{\linewidth}
                \subcaptionbox{\label{fig:RecoMinusTrueVsReco_DeltaPhi}}{\includegraphics[width=1.05\linewidth,page=1]{graphics/corrections/RecoMinusTrueVsReco_DeltaPhi.pdf}}
  \end{subfigure}\\
  \begin{subfigure}[b]{\linewidth}\addtocounter{subfigure}{1}
                \subcaptionbox{\label{fig:ResponseMatrix_DeltaPhi}}{\includegraphics[width=1.05\linewidth,page=1]{graphics/corrections/ResponseMatrix_DeltaPhi.pdf}}
  \end{subfigure} 
}%
\quad\quad%
\parbox{0.4725\textwidth}{
  \centering
  \begin{subfigure}[b]{\linewidth}\addtocounter{subfigure}{-2}\vspace*{-12pt}
                \subcaptionbox{\label{fig:RecoMinusTrueVsTrue_DeltaPhi}}{\includegraphics[width=1.05\linewidth,page=1]{graphics/corrections/RecoMinusTrueVsTrue_DeltaPhi.pdf}}
  \end{subfigure}\\
    \begin{minipage}[t][1.042\linewidth][t]{\linewidth}\vspace{10pt}
    \caption[Resolution of the $\Delta\varphi$ measurement from embedded $\pi^{+}\pi^{-}$ MC.]{Resolution of the $\Delta\varphi$ measurement from embedded $\pi^{+}\pi^{-}$ MC. Figure~\subref{fig:RecoMinusTrueVsReco_DeltaPhi} and~\subref{fig:RecoMinusTrueVsTrue_DeltaPhi} show the difference between reconstructed and hadron level $\Delta\varphi$ between forward scattered protons ($y$-axis) as a function of the reconstructed and hadron level $\Delta\varphi$ ($x$-axis), respectively. Probabilities in each $x$-axis bin were normalized to sum up to 1. On top of the two-dimensional distribution (in color) a candle plot is drawn. For each $x$-axis bin a circle indicate average difference, a box represents 68\% coverage of the distribution of difference symmetrically around median ($1\sigma$), while vertical lines extending beyond the box represent 99.3\% coverage of the distribution of difference symmetrically around median ($2.7\sigma$). A dashed black line is drawn across the plot to mark the difference between reconstructed and hadron level value equal to 0. Figure~\subref{fig:ResponseMatrix_DeltaPhi} shows the response matrix - correlation between reconstructed ($x$-axis) and hadron level ($y$-axis) $\Delta\varphi$. Probabilities in each bin of the true-level value were normalized to sum up to 1.}\label{fig:Response_DeltaPhi}%
  \end{minipage}
}%
\end{figure}
%---------------------------


%---------------------------
\begin{figure}[h]
\centering
\parbox{0.4725\textwidth}{
  \centering
  \begin{subfigure}[b]{\linewidth}
                \subcaptionbox{\label{fig:RecoMinusTrueVsReco_MandelstamTSum}}{\includegraphics[width=1.05\linewidth,page=1]{graphics/corrections/RecoMinusTrueVsReco_MandelstamTSum.pdf}}
  \end{subfigure}\\
  \begin{subfigure}[b]{\linewidth}\addtocounter{subfigure}{1}
                \subcaptionbox{\label{fig:ResponseMatrix_MandelstamTSum}}{\includegraphics[width=1.05\linewidth,page=1]{graphics/corrections/ResponseMatrix_MandelstamTSum.pdf}}
  \end{subfigure} 
}%
\quad\quad%
\parbox{0.4725\textwidth}{
  \centering
  \begin{subfigure}[b]{\linewidth}\addtocounter{subfigure}{-2}\vspace*{-12pt}
                \subcaptionbox{\label{fig:RecoMinusTrueVsTrue_MandelstamTSum}}{\includegraphics[width=1.05\linewidth,page=1]{graphics/corrections/RecoMinusTrueVsTrue_MandelstamTSum.pdf}}
  \end{subfigure}\\
    \begin{minipage}[t][1.042\linewidth][t]{\linewidth}\vspace{10pt}
    \caption[Resolution of the $|t_{1}+t_{2}|$ measurement from embedded $\pi^{+}\pi^{-}$ MC.]{Resolution of the $|t_{1}+t_{2}|$ measurement from embedded $\pi^{+}\pi^{-}$ MC. Figure~\subref{fig:RecoMinusTrueVsReco_MandelstamTSum} and~\subref{fig:RecoMinusTrueVsTrue_MandelstamTSum} show the difference between reconstructed and hadron level $|t_{1}+t_{2}|$ ($y$-axis) as a function of the reconstructed and hadron level $|t_{1}+t_{2}|$ ($x$-axis), respectively. Probabilities in each $x$-axis bin were normalized to sum up to 1. On top of the two-dimensional distribution (in color) a candle plot is drawn. For each $x$-axis bin a circle indicate average difference, a box represents 68\% coverage of the distribution of difference symmetrically around median ($1\sigma$), while vertical lines extending beyond the box represent 99.3\% coverage of the distribution of difference symmetrically around median ($2.7\sigma$). A dashed black line is drawn across the plot to mark the difference between reconstructed and hadron level value equal to 0. Figure~\subref{fig:ResponseMatrix_MandelstamTSum} shows the response matrix - correlation between reconstructed ($x$-axis) and hadron level ($y$-axis) $|t_{1}+t_{2}|$. Probabilities in each bin of the true-level value were normalized to sum up to 1.}\label{fig:Response_MandelstamTSum}%
  \end{minipage}
}%
\end{figure}
%---------------------------




%---------------------------
\begin{figure}[h]
\centering
\parbox{0.4725\textwidth}{
  \centering
  \begin{subfigure}[b]{\linewidth}
                \subcaptionbox{\label{fig:RecoMinusTrueVsReco_CosThetaCS}}{\includegraphics[width=1.05\linewidth,page=1]{graphics/corrections/RecoMinusTrueVsReco_CosThetaCS.pdf}}
  \end{subfigure}\\
  \begin{subfigure}[b]{\linewidth}\addtocounter{subfigure}{1}
                \subcaptionbox{\label{fig:ResponseMatrix_CosThetaCS}}{\includegraphics[width=1.05\linewidth,page=1]{graphics/corrections/ResponseMatrix_CosThetaCS.pdf}}
  \end{subfigure} 
}%
\quad\quad%
\parbox{0.4725\textwidth}{
  \centering
  \begin{subfigure}[b]{\linewidth}\addtocounter{subfigure}{-2}\vspace*{-12pt}
                \subcaptionbox{\label{fig:RecoMinusTrueVsTrue_CosThetaCS}}{\includegraphics[width=1.05\linewidth,page=1]{graphics/corrections/RecoMinusTrueVsTrue_CosThetaCS.pdf}}
  \end{subfigure}\\
    \begin{minipage}[t][1.042\linewidth][t]{\linewidth}\vspace{10pt}
    \caption[Resolution of the $\cos\theta^{\text{CS}}$ measurement from embedded $\pi^{+}\pi^{-}$ MC.]{Resolution of the $\cos\theta^{\text{CS}}$ measurement from embedded $\pi^{+}\pi^{-}$ MC. Figure~\subref{fig:RecoMinusTrueVsReco_CosThetaCS} and~\subref{fig:RecoMinusTrueVsTrue_CosThetaCS} show the difference between reconstructed and hadron level $\cos\theta^{\text{CS}}$ of $\pi^{+}\pi^{-}$ ($y$-axis) as a function of the reconstructed and hadron level $\cos\theta^{\text{CS}}$ ($x$-axis), respectively. Probabilities in each $x$-axis bin were normalized to sum up to 1. On top of the two-dimensional distribution (in color) a candle plot is drawn. For each $x$-axis bin a circle indicate average difference, a box represents 68\% coverage of the distribution of difference symmetrically around median ($1\sigma$), while vertical lines extending beyond the box represent 99.3\% coverage of the distribution of difference symmetrically around median ($2.7\sigma$). A dashed black line is drawn across the plot to mark the difference between reconstructed and hadron level value equal to 0. Figure~\subref{fig:ResponseMatrix_CosThetaCS} shows the response matrix - correlation between reconstructed ($x$-axis) and hadron level ($y$-axis) $\cos\theta^{\text{CS}}$. Probabilities in each bin of the true-level value were normalized to sum up to 1.}\label{fig:Response_CosThetaCS}%
  \end{minipage}
}%
\end{figure}
%---------------------------


%---------------------------
\begin{figure}[h]
\centering
\parbox{0.4725\textwidth}{
  \centering
  \begin{subfigure}[b]{\linewidth}
                \subcaptionbox{\label{fig:RecoMinusTrueVsReco_PhiCS}}{\includegraphics[width=1.05\linewidth,page=1]{graphics/corrections/RecoMinusTrueVsReco_PhiCS.pdf}}
  \end{subfigure}\\
  \begin{subfigure}[b]{\linewidth}\addtocounter{subfigure}{1}
                \subcaptionbox{\label{fig:ResponseMatrix_PhiCS}}{\includegraphics[width=1.05\linewidth,page=1]{graphics/corrections/ResponseMatrix_PhiCS.pdf}}
  \end{subfigure} 
}%
\quad\quad%
\parbox{0.4725\textwidth}{
  \centering
  \begin{subfigure}[b]{\linewidth}\addtocounter{subfigure}{-2}\vspace*{-12pt}
                \subcaptionbox{\label{fig:RecoMinusTrueVsTrue_PhiCS}}{\includegraphics[width=1.05\linewidth,page=1]{graphics/corrections/RecoMinusTrueVsTrue_PhiCS.pdf}}
  \end{subfigure}\\
    \begin{minipage}[t][1.042\linewidth][t]{\linewidth}\vspace{10pt}
    \caption[Resolution of the $\phi^{\text{CS}}$ measurement from embedded $\pi^{+}\pi^{-}$ MC.]{Resolution of the $\phi^{\text{CS}}$ measurement from embedded $\pi^{+}\pi^{-}$ MC. Figure~\subref{fig:RecoMinusTrueVsReco_PhiCS} and~\subref{fig:RecoMinusTrueVsTrue_PhiCS} show the difference between reconstructed and hadron level $\phi^{\text{CS}}$ of $\pi^{+}\pi^{-}$ ($y$-axis) as a function of the reconstructed and hadron level $\phi^{\text{CS}}$ ($x$-axis), respectively. Probabilities in each $x$-axis bin were normalized to sum up to 1. On top of the two-dimensional distribution (in color) a candle plot is drawn. For each $x$-axis bin a circle indicate average difference, a box represents 68\% coverage of the distribution of difference symmetrically around median ($1\sigma$), while vertical lines extending beyond the box represent 99.3\% coverage of the distribution of difference symmetrically around median ($2.7\sigma$). A dashed black line is drawn across the plot to mark the difference between reconstructed and hadron level value equal to 0. Figure~\subref{fig:ResponseMatrix_PhiCS} shows the response matrix - correlation between reconstructed ($x$-axis) and hadron level ($y$-axis) $\phi^{\text{CS}}$. Probabilities in each bin of the true-level value were normalized to sum up to 1.}\label{fig:Response_PhiCS}% 
  \end{minipage}
}%
\end{figure}
%---------------------------



\section{Closure tests}\label{sec:closureTests}
\subsection{Data pull}

Sanity of the data after applied corrections was verified by preparing a pull histogram of the number of reconstructed and selected CEP events per unit of integrated luminosity (in other words - an integrated cross section). The pull quantity was calculated per run, defined as 
\begin{equation}\label{eq:pull}
  \text{Pull}_{i}=\left\{
  \begin{array}{@{}ll@{}}
    \frac{\sigma_{i}-\langle \sigma\rangle}{u^{-}(\sigma_{i})}, & \text{if}\ \sigma_{i}-\langle \sigma\rangle \geq 0, \\[8pt]
    \frac{\sigma_{i}-\langle \sigma\rangle}{u^{+}(\sigma_{i})}, & \text{if}\ \sigma_{i}-\langle \sigma\rangle < 0,
  \end{array}\right.
\end{equation}
where $\sigma_{i}$ is an integrated cross section calculated for $i$-th run with statistical uncertainty equal to $u^{-}(\sigma_{i})$ or $u^{+}(\sigma_{i})$ (lower or upper 68\% confidence level interval), and $\langle \sigma\rangle$ is an expectation value of integrated cross section. For each run the integrated cross section was calculated as
\begin{equation}\label{eq:crossSecPull}
 \sigma_{i} = \frac{N_{i}}{L_{i}}.
\end{equation}
where $N_{i}$ is a number of CEP events in $i$-th run and $L_{i}$ is a luminosity collected during $i$-th run. The expectation value $\langle \sigma\rangle$ was calculated using an estimator from Eq.~\eqref{eq:estimatorNL}:

\begin{equation}\label{eq:estimatorNL}
 \langle \sigma\rangle = \frac{\sum\limits_{i}\sigma_{i} \times u^{-2}(\sigma_{i})}{\sum\limits_{i}u^{-2}(\sigma_{i})}
\end{equation}
which is simply an average integrated cross section calculated with full dataset, weighted by the uncertainties (by reciprocal of variance).

Final pull histogram is shown in Fig.~\ref{fig:pull} for raw event counts (open circles) and efficiency-corrected events weighted according to prescription from Eq.~\eqref{eq:weight} (filled circles). Parameters of the Gaussian fits to data points show some deviation of the mean value from 0, and significant improvement of the width of distribution after appliance of corrections, which should be consistent with 1, as observed.


%---------------------------
\begin{figure}[ht!]
\centering%
\parbox{0.4725\textwidth}{%
  \centering%
  \includegraphics[width=\linewidth]{graphics/corrections/PullC.pdf}
}%
\quad%
\parbox{0.4725\textwidth}{%
    \caption[Pull histogram of number of events per unit of integrated luminosity.]{Pull histogram of number of events per unit of integrated luminosity. Open circles fitted with dashed blue Gaussian mark uncorrected data, whereas black dots fitted with red curve denote corrected (weigthed) data. Parameters obtained from the fit are provided in the plot.}\label{fig:pull}%  
}
\end{figure}
%---------------------------

% 
% \subsection{Run-by-run stability}
% 
% % Additional component of the systematic uncertainty on the integrated luminosity accounts for a consistency between integrated cross-section measured within reference fill \#18915, and measured with the entire dataset. The integrated cross-section measured using solely data from fill \#18915 is equal to
% % \begin{equation}
% %  \sigma_{\text{fid}}^{\text{CEP}} = 55.3\pm2.4~\text{mb},
% % \end{equation}
% % while using entire dataset we obtain
% % \begin{equation}
% %  \sigma_{\text{fid}}^{\text{CEP}} = 51.3\pm0.4~\text{mb}.
% % \end{equation}
% % The absolute and relative difference between the two is, respectively,
% % \begin{equation}
% %  \Delta\sigma_{\text{fid}}^{\text{CEP}} = 4.0\pm2.4~\text{mb},~~~~~~~~~\Delta\sigma_{\text{fid}}^{\text{CEP}}/\sigma_{\text{fid}}^{\text{CEP}} = \left(7.8\pm4.7\right)~\%.
% % \end{equation}
% % 
% % Finally, the total systematic uncertainty of the integrated luminosity is a quadratic sum of generic uncertainty equal to 4\%, and uncertainty arising from the consistentsy between cross-section measured using data from fill \#18915 and entire dataset, equal to 7.8\%:
% % 
% % \begin{equation}
% % \Delta\mathcal{L}/\mathcal{L} = 4.0\%\oplus7.8\% = 8.8\%. 
% % \end{equation}
% 
% 
% \begin{figure}[h]
% \centering
% \includegraphics[width=.48\textwidth,page=2]{graphics/systematics/sigmaVsRunNumber.pdf}~~~~%
% \includegraphics[width=.48\textwidth,page=1]{graphics/systematics/sigmaVsRunNumber.pdf}% 
% \caption[Luminosity uncertainty systematics.]{Integrated fiducial CEP cross-section for runs from fill \#18915~(left) and for all runs~(right).}
% \label{fig:lumiSyst}
% \end{figure}

\subsection{Monte Carlo closure tests}\label{subsec:mcClosure}

To verify that corrections are properly applied and reconstructed+corrected distributions reproduce these at the true-level, the closure tests are performed. Monte Carlo events are subjected to the same analysis flow as the data, and the corrected observables are compared with the MC input.

In the closure tests presented below, distributions at the true level are assumed to be known with infinite statistical precision (count uncertainties are assumed to be equal to 0). This enables statistical comparison of the reconstructed and corrected distributions with those at the true level.

\subsubsection{TPC and TOF efficiency correction validation}\label{subsec:closureTestTpcTof}

%---------------------------
\begin{figure}[h]
\centering
\parbox{0.4725\textwidth}{
  \centering
  \begin{subfigure}[b]{\linewidth}
                \subcaptionbox{\label{fig:tpcTofSingleTrkClosureTrackPt}}{\includegraphics[width=1.05\linewidth,page=1]{graphics/corrections/EffCorrConsistencyCheck_TrackPtTrueCutSet1.pdf}\vspace*{-15pt}}
  \end{subfigure} 
}%
\quad\quad%
\parbox{0.4725\textwidth}{
  \centering
  \begin{subfigure}[b]{\linewidth}\addtocounter{subfigure}{-2}
                \subcaptionbox{\label{fig:tpcTofSingleTrkClosureTrackEta}}{\includegraphics[width=1.05\linewidth,page=1]{graphics/corrections/EffCorrConsistencyCheck_TrackEtaTrueCutSet1.pdf}\vspace*{-15pt}}
  \end{subfigure}
}%
\caption[Results of TPC and TOF single-track efficiency correction closure test on embedded signal MC sample.]{Results of TPC and TOF single-track efficiency correction closure test on embedded signal MC sample. Distributions of the true-level track $p_{\text{T}}$~\subref{fig:tpcTofSingleTrkClosureTrackPt} and true-level $\eta$~\subref{fig:tpcTofSingleTrkClosureTrackEta} corrected for single-particle TPC efficiency (red points) and single-particle TPC and TOF efficiency (blue points) are compared with the true-level distributions of corresponding observables (black points). }\label{fig:closureTestTpcTofSingleTrk}%
\end{figure}

%---------------------------
\begin{figure}[h]
\centering
\parbox{0.4725\textwidth}{
  \centering
  \begin{subfigure}[b]{\linewidth}
                \subcaptionbox{\label{fig:tpcTofClosureInvMass}}{\includegraphics[width=1.05\linewidth,page=1]{graphics/corrections/EffCorrConsistencyCheck_PairInvMassTrueCutSet1.pdf}\vspace*{-15pt}}
  \end{subfigure}\\
  \begin{subfigure}[b]{\linewidth}\addtocounter{subfigure}{1}
                \subcaptionbox{\label{fig:tpcTofClosureMandelstamTSum}}{\includegraphics[width=1.05\linewidth,page=1]{graphics/corrections/EffCorrConsistencyCheck_MandelstamTSumTrueCutSet1.pdf}\vspace*{-15pt}}
  \end{subfigure} 
}%
\quad\quad%
\parbox{0.4725\textwidth}{
  \centering
  \begin{subfigure}[b]{\linewidth}\addtocounter{subfigure}{-2}
                \subcaptionbox{\label{fig:tpcTofClosureRapidity}}{\includegraphics[width=1.05\linewidth,page=1]{graphics/corrections/EffCorrConsistencyCheck_PairRapidityTrueCutSet1.pdf}\vspace*{-15pt}}
  \end{subfigure}\\
  \begin{subfigure}[b]{\linewidth}\addtocounter{subfigure}{1}
                \subcaptionbox{\label{fig:tpcTofClosureDeltaPhi}}{\includegraphics[width=1.05\linewidth,page=1]{graphics/corrections/EffCorrConsistencyCheck_DeltaPhiDegreesTrueCutSet1.pdf}\vspace*{-15pt}}
  \end{subfigure}
}%
\caption[Results of TPC and TOF efficiency correction closure test on embedded signal MC sample.]{Results of TPC and TOF efficiency correction closure test on embedded signal MC sample. Distributions of the true-level pion pair invariant mass~\subref{fig:tpcTofClosureInvMass} and rapidity~\subref{fig:tpcTofClosureRapidity}, as well as forward proton total squared four-momentum transfer~\subref{fig:tpcTofClosureMandelstamTSum} and azimuthal separation~\subref{fig:tpcTofClosureDeltaPhi} for events with both central state pion TPC tracks reconstructed (red points) and additionally matched with TOF (blue points), are compared with the true level distributions of corresponding observables (black points). All events were required to be contained within the fiducial region of the measurement.}\label{fig:closureTestTpcTof}%
\end{figure}






\subsubsection{RP efficiency correction validation}\label{subsec:closureTestRp}
The joint RP track reconstruction and selection efficiency and the dead-material-induced veto efficiency is tested here.
%---------------------------
\begin{figure}[h]
\centering
\parbox{0.4725\textwidth}{
  \centering
  \begin{subfigure}[b]{\linewidth}
                \subcaptionbox{\label{fig:EffValidation_RP_SingleTrack_px_East}}{\includegraphics[width=1.05\linewidth,page=1]{graphics/corrections/EffValidation_RP_SingleTrack_px_East.pdf}\vspace*{-15pt}}\\
  \end{subfigure}\\ 
  \begin{subfigure}[b]{\linewidth}\addtocounter{subfigure}{1}
                \subcaptionbox{\label{fig:EffValidation_RP_SingleTrack_px_West}}{\includegraphics[width=1.05\linewidth,page=1]{graphics/corrections/EffValidation_RP_SingleTrack_px_West.pdf}\vspace*{-15pt}}\\
  \end{subfigure}
}%
\quad\quad%
\parbox{0.4725\textwidth}{
  \centering
  \begin{subfigure}[b]{\linewidth}\addtocounter{subfigure}{-2}
                \subcaptionbox{\label{fig:EffValidation_RP_SingleTrack_py_East}}{\includegraphics[width=1.05\linewidth,page=1]{graphics/corrections/EffValidation_RP_SingleTrack_py_East.pdf}\vspace*{-15pt}}\\
  \end{subfigure}\\ 
  \begin{subfigure}[b]{\linewidth}\addtocounter{subfigure}{1}
                \subcaptionbox{\label{fig:EffValidation_RP_SingleTrack_py_West}}{\includegraphics[width=1.05\linewidth,page=1]{graphics/corrections/EffValidation_RP_SingleTrack_py_West.pdf}\vspace*{-15pt}}\\
  \end{subfigure}
}%
\caption[Results of RP single-track efficiency correction closure test on embedded signal MC sample.]{Results of RP single-track efficiency correction closure test on embedded signal MC sample. Distributions of the true-level track $p_{x}$ (left) and $p_{y}$ (right) for East (top) and West (bottom) RPs corrected for single-particle RP efficiency (red points) are compared with the true-level distributions of corresponding observables (black points). The true-level quantities here are taken after the angular beam divergence being applied.}\label{fig:closureTestTpcTofSingleTrk}%
\end{figure}

%---------------------------
\begin{figure}[h]
\centering
\parbox{0.4725\textwidth}{
  \centering
  \begin{subfigure}[b]{\linewidth}
                \subcaptionbox{\label{fig:EffValidation_RP_px_East}}{\includegraphics[width=0.96\linewidth,page=1]{graphics/corrections/EffValidation_RP_px_East.pdf}\vspace*{-15pt}}\\
  \end{subfigure}\\ 
  \begin{subfigure}[b]{\linewidth}\addtocounter{subfigure}{1}
                \subcaptionbox{\label{fig:EffValidation_RP_px_West}}{\includegraphics[width=0.96\linewidth,page=1]{graphics/corrections/EffValidation_RP_px_West.pdf}\vspace*{-15pt}}\\
  \end{subfigure}\\ 
  \begin{subfigure}[b]{\linewidth}\addtocounter{subfigure}{1}
                \subcaptionbox{\label{fig:EffValidation_RP_DeltaPhi}}{\includegraphics[width=0.96\linewidth,page=1]{graphics/corrections/EffValidation_RP_DeltaPhi.pdf}\vspace*{-15pt}}
  \end{subfigure}
}%
\quad\quad%
\parbox{0.4725\textwidth}{
  \centering
  \begin{subfigure}[b]{\linewidth}\addtocounter{subfigure}{-4}
                \subcaptionbox{\label{fig:EffValidation_RP_py_East}}{\includegraphics[width=0.96\linewidth,page=1]{graphics/corrections/EffValidation_RP_py_East.pdf}\vspace*{-15pt}}\\
  \end{subfigure}\\ 
  \begin{subfigure}[b]{\linewidth}\addtocounter{subfigure}{1}
                \subcaptionbox{\label{fig:EffValidation_RP_py_West}}{\includegraphics[width=0.96\linewidth,page=1]{graphics/corrections/EffValidation_RP_py_West.pdf}\vspace*{-15pt}}\\
  \end{subfigure}\\ 
  \begin{subfigure}[b]{\linewidth}\addtocounter{subfigure}{1}
                \subcaptionbox{\label{fig:EffValidation_RP_MandelstamTSum}}{\includegraphics[width=0.96\linewidth,page=1]{graphics/corrections/EffValidation_RP_MandelstamTSum.pdf}\vspace*{-15pt}}
  \end{subfigure}
}%
\caption[Results of RP efficiency correction closure test on embedded signal MC sample.]{Results of RP efficiency correction closure test on embedded signal MC sample. Distributions of the true-level track $p_{x}$, $p_{y}$, as well as $\Delta\varphi$ and $|t_{1}+t_{2}|$ corrected for full two-proton RP efficiency (red points) are compared with the true-level distributions of corresponding observables (black points). The true-level quantities here are taken after the angular beam divergence being applied.}\label{fig:closureTestTpcTofSingleTrk}%
\end{figure}




%---------------------------
\begin{figure}[h]
\centering
\parbox{0.4725\textwidth}{
  \centering 
  \begin{subfigure}[b]{\linewidth}\addtocounter{subfigure}{1}
                \subcaptionbox{\label{fig:EffValidation_RP_DeltaPhi_NoMatchingRequired}}{\includegraphics[width=1.05\linewidth,page=1]{graphics/corrections/EffValidation_RP_DeltaPhi_NoMatchingRequired.pdf}\vspace*{-15pt}}
  \end{subfigure}
}%
\quad\quad%
\parbox{0.4725\textwidth}{
  \centering
  \begin{subfigure}[b]{\linewidth}\addtocounter{subfigure}{1}
                \subcaptionbox{\label{fig:EffValidation_RP_MandelstamTSum_NoMatchingRequired}}{\includegraphics[width=1.05\linewidth,page=1]{graphics/corrections/EffValidation_RP_MandelstamTSum_NoMatchingRequired.pdf}\vspace*{-15pt}}
  \end{subfigure}
}%
\caption[Results of RP efficiency, migrations and fakes correction closure test on embedded signal MC sample.]{Results of RP efficiency,  migrations and fakes correction closure test on embedded signal MC sample. Distributions of the reconstructed and corrected $\Delta\varphi$ and $|t_{1}+t_{2}|$ corrected for full two-proton RP efficiency (red points) are compared with the true generator-level distributions of corresponding observables (black circles). By definition, both forward protons were required to be contained within the fiducial region of the measurement, either at the generator level (black) or reconstructed level (red).}\label{fig:closureTestTpcTofSingleTrk}% 
\end{figure}




\subsubsection{Total correction validation}\label{subsec:closureTestFull} 

%---------------------------
\begin{figure}[h]
\centering
\parbox{0.4725\textwidth}{
  \centering
  \begin{subfigure}[b]{\linewidth}
                \subcaptionbox{\label{fig:fullClosureInvMass}}{\includegraphics[width=1.05\linewidth,page=1]{graphics/corrections/FullCorrectionValidation_NoMatchingRequired_EffUsingRecoValues_InvMass.pdf}\vspace*{-15pt}}
  \end{subfigure}\\
  \begin{subfigure}[b]{\linewidth}\addtocounter{subfigure}{1}
                \subcaptionbox{\label{fig:fullClosureMandelstamTSum}}{\includegraphics[width=1.05\linewidth,page=1]{graphics/corrections/FullCorrectionValidation_NoMatchingRequired_EffUsingRecoValues_MandelstamTSum.pdf}\vspace*{-15pt}}
  \end{subfigure} 
}%
\quad\quad%
\parbox{0.4725\textwidth}{
  \centering
  \begin{subfigure}[b]{\linewidth}\addtocounter{subfigure}{-2}
                \subcaptionbox{\label{fig:fullClosureRapidity}}{\includegraphics[width=1.05\linewidth,page=1]{graphics/corrections/FullCorrectionValidation_NoMatchingRequired_EffUsingRecoValues_Rapidity.pdf}\vspace*{-15pt}}
  \end{subfigure}\\
  \begin{subfigure}[b]{\linewidth}\addtocounter{subfigure}{1}
                \subcaptionbox{\label{fig:fullClosureDeltaPhi}}{\includegraphics[width=1.05\linewidth,page=1]{graphics/corrections/FullCorrectionValidation_NoMatchingRequired_EffUsingRecoValues_DeltaPhi.pdf}\vspace*{-15pt}}
  \end{subfigure}
}%
\caption[Results of full closure test on embedded signal MC sample.]{Results of full closure test on embedded signal MC sample. Reconstructed and corrected distributions of the pion pair invariant mass~\subref{fig:fullClosureInvMass} and rapidity~\subref{fig:fullClosureRapidity}, as well as forward proton total squared four-momentum transfer~\subref{fig:fullClosureMandelstamTSum} and azimuthal separation~\subref{fig:fullClosureDeltaPhi} after CEP event selection, are compared with the true generator-level distributions of corresponding observables. By definition, all events were required to be contained within the fiducial region of the measurement, either at the generator level (black) or reconstructed level (red).}\label{fig:closureTestFull}%
\end{figure}


\section{Geometrical correction}\label{sec:geomCorr}


One of the goals of the analysis is to obtain information about the production of resonances in CEP process. Invariant mass distributions in the fiducial region of the measurement can not be directly used to extract and quantitatively describe possible resonances without extrapolation to the full kinematic region of the central pions pair, given by $p_\mathrm{T}\rightarrow 0$ and $|\eta|\rightarrow\infty$  (full solid angle in the central system rest frame). Extrapolation to unmeasured region is always model dependent. Here we present corrections to the full phase-space obtained with various assumptions (models). To limit size of the corrections, the measurement is restricted to $|y(\pi^+\pi^-)|<0.4$. In the correction calculation factorization of the central system and forward protons phase space is assumed. For the forward protons phase space a uniform distributions of the azimuthal angles are assumed while polar angles are generated according to exponential $t$ distribution with $t$-slope of 6~GeV$^{-2}$.

The measurement is extrapolated from the part of the fiducial region given by Eq.~\eqref{eq:RpFiducial} covering $0.05 \leq -t_1 , -t_2 \leq 0.16$ GeV$^2$ to such defined Lorentz invariant $t$ interval and full azimuthal angle of forward protons.
The measurement is further restricted to two regions of $\Delta\varphi<45^\circ$ and $\Delta\varphi>135^\circ$ which reduces size of acceptance corrections and related systematic uncertainties.

% %---------------------------
\begin{figure}[h]
  \centering
  \begin{tabular}{@{}p{0.49\linewidth}@{\quad}p{0.49\linewidth}@{}} 
    \subfigimg[width=\linewidth,page=1]{~~~~~~~~~~a)}{graphics/corrections/GeomAcc.pdf} &
    \subfigimg[width=\linewidth,page=2]{~~~~~~~~~~b)}{graphics/corrections/GeomAcc.pdf}
  \end{tabular}
\caption[Geometrical acceptance for $\pi^{+}\pi^{-}$.]{Geometrical acceptance for the $\pi^{+}\pi^{-}$ being in the fiducial region of the measurement ($|\eta(\pi^{\pm})|<0.7$, $p_{\text{T}}(\pi^{\pm})>0.2$~GeV) if forward protons separated in azimuth by (left) $\Delta\varphi<45^{\circ}$ and (right)  $\Delta\varphi>135^{\circ}$. Dashed red lines limit the $m(\pi^{+}\pi^{-})$ region in which the fit is perfomed, as described in Sec.~\ref{sec:InvMassFit}.}\label{fig:centralStateGeomCorr}% 
\end{figure}


% %---------------------------
\begin{figure}[h!]
  \centering
  \begin{tabular}{@{}p{0.49\linewidth}@{\quad}p{0.49\linewidth}@{}} 
    \subfigimg[width=\linewidth,page=1]{~~~~~~~~~~a)}{graphics/corrections/deltaPhiAcceptance.pdf} &
    \subfigimg[width=\linewidth,page=2]{~~~~~~~~~~b)}{graphics/corrections/deltaPhiAcceptance.pdf}
  \end{tabular}\vspace*{-15pt}
\caption[Geometrical acceptance for forward protons.]{Geometrical acceptance for the forward protons being both in the fiducial region given by Eq.~\eqref{eq:RpFiducial} if separated in azimuth by (left) $\Delta\varphi<45^{\circ}$ and (right)  $\Delta\varphi>135^{\circ}$. Phase-space MC assuming flat distribution of  azimuthal angle of the forward protons $\varphi$ has been used for determination of this acceptance. Solid magenta lines limit the region selected for the measurement, which provides reasonably high geometrical acceptance.}\label{fig:forwardProtonsGeomCorr}%
\end{figure}


% %---------------------------
\begin{figure}[h]
  \centering
  \begin{tabular}{@{}p{0.49\linewidth}@{\quad}p{0.49\linewidth}@{}} 
    \subfigimg[width=\linewidth,page=1]{~~~~~~~~~~a)}{graphics/corrections/deltaPhiAcceptance_Ratio_DiMeToPhaseSpace.pdf} &
    \subfigimg[width=\linewidth,page=2]{~~~~~~~~~~b)}{graphics/corrections/deltaPhiAcceptance_Ratio_DiMeToPhaseSpace.pdf} \\
    \subfigimg[width=\linewidth,page=1]{~~~~~~~~~~c)}{graphics/corrections/deltaPhiAcceptance_Ratio_GenExToPhaseSpace.pdf} &
    \subfigimg[width=\linewidth,page=2]{~~~~~~~~~~d)}{graphics/corrections/deltaPhiAcceptance_Ratio_GenExToPhaseSpace.pdf} \\
  \end{tabular}\vspace*{-15pt}
\caption[Geometrical acceptance comparison between DiMe, GenEx and phase-space MC.]{Geometrical acceptance comparison between DiMe and phase-space MC (top) and GenEx and phase-space MC (bottom) for (left) $\Delta\varphi<45^{\circ}$ and (right) $\Delta\varphi>135^{\circ}$.}\label{fig:forwardProtonsGeomCorrRatio}%
\end{figure}
