%%===========================================================%%
%%                                                           %%
%%                     EVENT SELECTION                       %%
%%                                                           %%
%%===========================================================%%


\newcommand{\itemm}{\item\hspace*{-5pt}.\hspace*{-1pt}~}

\chapter{Event selection}\label{chap:eventSelection}

Complete list of analysis cuts used for signal extraction is presented in Sec.~\ref{sec:listOfCuts}. Detailed description of each cut can be found in Sec.~\ref{sec:descriptionOfCuts}. [For PDF readers: you can directly move to description of given cut by clicking on corresponding bold cut number \textbf{CX} at the start of line in the list of cuts.]

\section[List of cuts]{List of cuts\footnote{Some cuts (e.g.~\ref{enum:CutTpcTrks}) are decomposed to constituent sub-cuts. Cut is formed by the logical AND of all its sub-cuts. Events must pass all cuts to be identified as a signal.}}\label{sec:listOfCuts}
\begin{enumerate}[label=\textbf{\hyperref[sec:C\arabic*]{C\arabic*}},ref=C\arabic*]
 \itemm Exactly 1 primary vertex with TPC track(s) matched with hits in TOF.\label{enum:CutPrimVx}
 \itemm TPC vertex from~\ref{enum:CutPrimVx} is placed within $|z_{\text{vx}}|<80$~cm.\label{enum:CutZVx}
 \itemm Exactly 2 opposite-sign primary TPC tracks~(\ref{enum:TpcOppoSign}) of good quality~(\ref{enum:TpcQualityCuts}) matched with hits in TOF~(\ref{enum:TpcTofMatched}) and reconstructed within kinematic region of high TPC acceptance~(\ref{enum:TpcKinematicCuts}), with associated global tracks characterized by small distance of closest approach (DCA) to the primary vertex~(\ref{enum:TpcDcaCuts}) and high proximity to each other at the beamline~(\ref{enum:TpcDeltaZ0Cut}).\label{enum:CutTpcTrks}
    \begin{enumerate}[label=\textbf{\theenumi.\arabic*},ref=\theenumi.\arabic*]
      \itemm Exactly 2 TOF-matched (match flag $>0$) primary tracks and no additional primary tracks matched with BEMC clusters,\label{enum:TpcTofMatched}
      \itemm Tracks are of opposite signs,\label{enum:TpcOppoSign}
      \itemm Both tracks are contained within the kinematic range:\label{enum:TpcKinematicCuts}\hspace*{13pt}
      $|\eta|<0.7$,~~~~$p_{T}>0.2~\text{GeV}$,
      \itemm Associated global tracks satisfy quality criteria:\label{enum:TpcQualityCuts}\hspace*{44pt}
      $N_{\text{hits}}^{\text{fit}}\geq25$,~~~$N_{\text{hits}}^{\text{dE/dx}}\geq15$,~~~$|d_{0}|<1.5$~cm,
      \itemm Associated global tracks match well to the prim. vertex:\label{enum:TpcDcaCuts}\hspace*{3.5pt}
      $\text{DCA}(R)<1.5$~cm,~~~~$|\text{DCA}(z)|<1$~cm,
      \itemm Associated global tracks are close at the beamline:\label{enum:TpcDeltaZ0Cut}\hspace*{29pt}
      $|\Delta z_{0}|<2$~cm.
    \end{enumerate}
 \itemm Exactly 1 RP track on each side of STAR central detector~(\ref{enum:RpOneTrkPerSide}) of good quality~(\ref{enum:RpQualityCuts}), with local angles consistent with the IP being the track origin~(\ref{enum:RpLocalAngles}), lying within fiducial region of high geometrical acceptance~(\ref{enum:RpFiducial}).\label{enum:CutRpTrks}
      \begin{enumerate}[label=\textbf{\theenumi.\arabic*},ref=\theenumi.\arabic*]
      \itemm RP tracks contain only track-points with at least 3 (out of 4) planes used in reconstruction,\label{enum:RpQualityCuts}
      \itemm Local angles ($\theta_{x}^{\text{RP}}$, $\theta_{y}^{\text{RP}}$) consistent with expectation for protons originating from the IP\label{enum:RpLocalAngles}%
      \[-2~\text{mrad}<\theta_{x}^{\text{RP}}-x^{\text{RP}}/|z^{\text{RP}}|<4~\text{mrad},~~~~~-2~\text{mrad}<\theta_{y}^{\text{RP}}-y^{\text{RP}}/|z^{\text{RP}}|<2~\text{mrad},\]
      \itemm Exactly 1 track passing cuts \ref{enum:RpQualityCuts}-\ref{enum:RpLocalAngles} per side,\label{enum:RpOneTrkPerSide}
      \itemm Tracks passing cut~\ref{enum:RpOneTrkPerSide} lie within the fiducial $(p_{x},p_{y})$ region defined as\vspace*{-7pt}\label{enum:RpFiducial}:\\
      \[0.2<|p_{y}|<0.4,~~~-0.2<p_{x},~~~(p_{x}+0.3)^{2}+p_{y}^{2}<0.5^{2}~~~(\text{all in GeV}).\]
    \end{enumerate}
 \itemm Vertex $z$-positions measured in TPC and reconstructed from the difference of proton detection time in west and east RPs are consistent with each other within the resolution (at $3.5\sigma_{\Delta z_{\text{vtx}}}$ level):
 \[|\Delta z_{\text{vtx}}| = |z_{\text{vx}}^{\text{TPC}}-z_{\text{vx}}^{\text{RP}}|<36~\text{cm}.\vspace{-17pt}\]\label{enum:CutDeltaZVx}
 \itemm No signal in any tile of BBC-large (east or west) with $\text{ADC}>\text{ADC}_{\text{thr}}$ and $100<\text{TDC}<2400$, where $\text{ADC}_{\text{thr}}$ is specific for each channel (see Tab.~\ref{tab:bbcLargeThresholds}).\label{enum:CutBbcLarge}%
 %
 \itemm Maximally 3 reconstructed TOF clusters $N^{\text{TOF}}_{\text{clstrs}}\leq 3$.\label{enum:CutTofClusters}%
 %
 \itemm Particle/pair identification (PID):\label{enum:CutPid}
 \begin{enumerate}[label=\textbf{\theenumi.\arabic*},ref=\theenumi.\arabic*]
      \itemm Identification of particle pairs based on $dE/dx$ ($\chi^{2}$) and $m^{2}_{\text{TOF}}$ (def. in Sec.~\ref{sec:C8} and App.~\ref{appendix:squaredMass}):\label{enum:CutPidNoPtLimit}\\[3pt]
        \textbf{~~if~~~}\hspace*{4.5pt}$\chi^{2}(\pi\pi)>9$\textbf{~~and~~}$\chi^{2}(KK)>9$\textbf{~~and~~}$\chi^{2}(pp)<9$\textbf{~~and~~}$m^{2}_{\text{TOF}}>0.6~\text{GeV}~~\rightarrow~~p\bar{p}$\\[5pt]%
        %
        \textbf{elif~~}$\chi^{2}(\pi\pi)>9$\textbf{~~and~~}$\chi^{2}(KK)<9$\textbf{~~and~~}$\chi^{2}(pp)>9$\textbf{~~and~~}$m^{2}_{\text{TOF}}>0.15~\text{GeV}~~\rightarrow~~K^{+}K^{-}$\\[5pt]%
        %
        \textbf{elif~} $\chi^{2}(\pi\pi)<12~~\rightarrow~~\pi^{+}\pi^{-}$.%~~~~~~~~~~~~~~~~~~~~~~~~~~~~~~~\textbf{otherwise~~} event rejected.
      \itemm Restricting fiducial cuts on $K^{+}K^{-}$ and $p\bar{p}$ (to reduce misidentifications and assure high PID eff.):\label{enum:CutPidPtLimits}\\[2pt]
      \textbf{~if~} $K^{+}K^{-}$:~~~~~~$p_{T}>0.3~\text{GeV}$,~~~~$min(p_{T}^{+},p_{T}^{-})<0.7~\text{GeV}$\\[2pt]%
      \textbf{~if~} $p\bar{p}$:~~~~~~~~~~~~\hspace*{1.7pt}$p_{T}>0.4~\text{GeV}$,~~~~$min(p_{T}^{+},p_{T}^{-})<1.1~\text{GeV}$%
\end{enumerate}
\itemm Missing (total) momentum of TPC tracks and RP tracks $p_{T}^{\text{miss}}<75~\text{MeV}$.\label{enum:CutMissingPt}%
 %
 
\end{enumerate}
%
%
%
%
%
\section{Description of cuts}\label{sec:descriptionOfCuts}%
%
\subsection{(\ref{enum:CutPrimVx},\ref{enum:CutZVx})~Primary vertex and its \texorpdfstring{$z$}{z}-position}\label{sec:C1}\label{sec:C2}
As it was designed in the trigger logic, we aim to perform CEP analysis in a clean, pile-up-free environment, therefore we cut on primary vertex multiplicity~(Fig.~\ref{fig:NumberOfPrimaryVertices}) to reject events with more than one interaction per bunch crossing. We required exactly one primary vertex containing TPC tracks matched with hits in TOF (matching of the track with hit in TOF is identified with the TOF match flag being different from 0). Later in the text we refer to such events as a single ``TOF vertex`` events.

%---------------------------
\begin{figure}[ht!]%
\centering%
\begin{minipage}{.4725\textwidth}%
  \centering%
  \includegraphics[width=\linewidth]{graphics/eventSelection/NumberOfPrimaryVertices.pdf}%
  \caption{Primary vertex multiplicity. Red arrow marks bin with events with exactly one primary vertex (with track(s) matched with hit in TOF), which are used in physics analysis.}\label{fig:NumberOfPrimaryVertices}
\end{minipage}%
\quad\quad%
\begin{minipage}{.4725\textwidth}%
  \centering
  \includegraphics[width=\linewidth]{graphics/eventSelection/zVertex_oneTof.pdf}%
  \caption{\texorpdfstring{$z$}{z}-position of the primary vertex in single TOF vertex events (passing cut~\ref{enum:CutPrimVx}). Red dashed line indicate range of longitudinal vertex position accepted in analysis.}\label{fig:zVertexTpc}
\end{minipage}%
\end{figure}%
%---------------------------


The single TOF vertex was required to be placed within a range $(-80~\text{cm},~80~\text{cm})$ along the $z$-axis~(Fig.~\ref{fig:zVertexTpc}). Events with vertices away from the nominal IP have low acceptance both for the central tracks and the forward protons (comparing to events with vertices close to nominal IP), therefore we reject them as their inclusion to analysis would naturally introduce large systematic uncertainties. See Sec.~3.2.3 in Ref.~\cite{supplementaryNote}.

%---------------------------
\begin{figure}[h]
\centering%
\parbox{0.5325\textwidth}{%
  \centering%
  \includegraphics[width=\linewidth]{graphics/backgrounds/dataVsMc/Ratio_Linear_ZVtx.pdf} 
}%
\quad%
\parbox{0.4125\textwidth}{%
    \caption[Comparison of $z_{\text{vtx}}$ distribution between data and embedded MC.]{Comparison of $z_{\text{vtx}}$ distribution between data and embedded MC after full selection. Data are represented by black points, while stacked MC predictions are drawn as histograms of different colors. Histogram from each MC process has been normalized according to prescription in Sec.~\ref{sec:bkgdSignalNorm}. Vertical error bars represent statistical uncertainties, horizontal bars represent bin sizes. Comparison is shown for only $z$-vertex range corresponding to offline selection cut~\ref{enum:CutZVx} due to such limited vertex range used in MC generation for increase of generation efficiency.}\label{fig:Ratio_Linear_ZVtx} %  
}
\end{figure}
%---------------------------


In Fig~\ref{fig:Ratio_Linear_ZVtx} we show comparison of the $z$-position of single TOF primary vertex measured in the TPC, between data and MC generated e.g. to study of detector effects present in analysis. The ratio of distributions which is compatible with unity indicates proper position, width and shape of distribution assumed at MC generation (gaussian with mean at 0 and width of 50~cm).




\subsection{(\ref{enum:CutTpcTrks})~TPC tracks}\label{sec:C3}

The TPC track selection starts from the selection of events with exactly two primary tracks matched with hit in TOF~(Fig.~\ref{fig:NumberOfTofTracksInSingleTofVertex}). Matching with TOF guarantee that analyzed tracks originate from the triggered bunch crossing (ensures that tracks are ''in-time``). It is in accordance with the trigger logic which required at least 2 L0 TOF hits, as well as it enables more accurate particle identification with merged time-of-flight and $dE/dx$ method, comparing to sole usage of $dE/dx$. Primary tracks not matched with hit in TOF, whose average multiplicity in single TOF vertex is $\sim$8, are hardly distinguished between real and fake (off-time) tracks, which is an additional reason for not analyzing events with only one TOF-matched primary TPC track (the other track might be unmatched due to TOF inefficiency).

%---------------------------
\begin{figure}[t!]%
\centering%
\begin{minipage}{.4725\textwidth}%
  \centering%
  \includegraphics[width=\linewidth]{graphics/eventSelection/TpcTracks/NumberOfTofTracksInSingleTofVertex.pdf}%
  \caption[Multiplicty of primary TPC tracks matched with hit in TOF for single TOF vertex events]{Multiplicty of primary TPC tracks matched with hit in TOF for single TOF vertex events. Red arrow marks bin with events with exactly two primary tracks matched with hit in TOF, which are used in physics analysis.\newline}\label{fig:NumberOfTofTracksInSingleTofVertex}
\end{minipage}%
\quad\quad%
\begin{minipage}{.4725\textwidth}%
  \centering
  \includegraphics[width=\linewidth]{graphics/eventSelection/TpcTracks/Rmin.pdf}%
  \caption[Distribution of a distance in $\eta-\phi$ space between the BEMC cluster closest to primary TPC track ($R_{\text{min}}$)]{Distribution of a distance in $\eta-\phi$ space between the BEMC cluster closest to primary TPC track matched (filled circle) or not matched (opened circle) with hit in TOF, for single TOF vertex events. Red dashed line indicate matching threshold $R^{\text{match}}_{\text{max}} = 0.05$.}\label{fig:Rmin} %which are expected to reack BEMC
\end{minipage}%
\end{figure}%
%---------------------------

Primary TPC tracks from the single TOF vertex which are matched with TOF are allowed to be also matched with BEMC clusters. Matching with BEMC cluster is claimed if the distance in $\eta-\phi$ space between the BEMC cluster position $(\eta_{\text{clus}},~\phi_{\text{clus}})$ and projected position of the track in BEMC $(\eta_{\text{proj}},~\phi_{\text{proj}})$, defined as
\begin{equation}\label{eq:etaPhiR}
 R=\sqrt{(\eta_{\text{clus}}-\eta_{\text{proj}})^{2} + (\phi_{\text{clus}}-\phi_{\text{proj}})^{2}},
\end{equation}
is less than $R^{\text{match}}_{\text{max}} = 0.05$. Distribution of the distance between the primary TPC track and the closest BEMC cluster is shown in~Fig.~\ref{fig:Rmin}.

However, if there are any primary TPC tracks matched with BEMC cluster and not matched with TOF in the single TOF vertex with two TOF-matched tracks, an event is rejected. Such configuration implies higher-than-2 multiplicity of the real tracks in the vertex, hence an event is unlikely a Central Exclusive Production of two particles.

We apply cuts on the quantities reflecting quality of reconstructed TPC tracks similar to these typically used at STAR. We cut on number of hits used in TPC track reconstruction $N_{\text{hits}}^{\text{fit}} \geq 25$ and number of hits used in specific energy loss reconstruction $N_{\text{hits}}^{\text{dE/dx}} \geq 15$ in order to achieve good momentum and $dE/dx$ resolution. We show distributions of aforementioned quantities together with spectrum of fraction of number of hits potentially generated by the track and finally used in the reconstruction $N_{\text{hits}}^{\text{fit}}/N_{\text{hits}}^{\text{poss}}$ in Fig.~\ref{fig:NHits}. One can see, that embedded MC simulation describes measured data well.

We also require that helices of global tracks associated with selected primary TOF tracks point well to the primary vertex ($\text{DCA}(R)<1.5$~cm and $|\text{DCA}(z)|<1$~cm), as well as the longitudinal separation of helices at the beamline (Fig.~\ref{fig:deltaZ0Sketch}) is small and coincides with cut on $|\text{DCA}(z)|$ ($|\Delta z_{0}|<2$~cm). Distributions of these quantities together with comparison against embedded MC are shown in Fig.~8.5 of Ref.~\cite{supplementaryNote}. In that reference one can also read how appropriate adjustment was derived needed to achieve satisfactory agreement of the $d_{0}$, $\text{DCA}(R)$, $\text{DCA}(z)$ and $|\Delta z_{0}|$ in data and embedded MC.

Figure~\ref{fig:TrackEtaPhi} shows comparison of the track pseudorapidity and azimuthal angle between data and embedded MC. These distributions are quite well described by MC. Large modulation in the $\phi$ distribution (enhancement at $\phi=\pm\pi/2$) is connected with the RP acceptance mostly at $\varphi=\pm\pi/2$ - central particles pair is always back-to-back in azumith with respect to pair of forward scattered protons, therefore pairs produced in ''up`` or ''down`` direction are preferred.

%---------------------------
\begin{figure}[hb]
\centering
\parbox{0.4725\textwidth}{
  \centering
  \begin{subfigure}[b]{\linewidth}
                \subcaptionbox{\label{fig:NHitsFit}}{\includegraphics[width=\linewidth]{graphics/eventSelection/TpcTracks/Ratio_Linear_NHitsFit.pdf}}
  \end{subfigure}\\
  \begin{subfigure}[b]{\linewidth}\addtocounter{subfigure}{1}
                \subcaptionbox{\label{fig:NHitsFit_to_NHitsPos}}{\includegraphics[width=\linewidth]{graphics/eventSelection/TpcTracks/Ratio_Linear_NHitsFit_to_NHitsPoss.pdf}}
  \end{subfigure}
}%
\quad\quad%
\parbox{0.4725\textwidth}{
  \centering
  \begin{subfigure}[b]{\linewidth}\addtocounter{subfigure}{-2}\vspace*{-23pt}
                \subcaptionbox{\label{fig:NHits_dEdx}}{\includegraphics[width=\linewidth]{graphics/eventSelection/TpcTracks/Ratio_Linear_NHitsDEdx.pdf}}
  \end{subfigure}\\
  \begin{minipage}[t][1.042\linewidth][t]{\linewidth}\vspace{10pt}
    \caption[Comparison of distribution of $N_{\text{hits}}^{\text{fit}}$,~$N_{\text{hits}}^{\text{dE/dx}}$ and $N_{\text{hits}}^{\text{fit}}/N_{\text{hits}}^{\text{poss}}$ in the data and embedded MC.]
    {Comparison of distribution of the number of hits used in TPC track reconstruction $N_{\text{hits}}^{\text{fit}}$ (\subref{fig:NHitsFit}), number of hits used in specific energy loss reconstruction $N_{\text{hits}}^{\text{dE/dx}}$ (\subref{fig:NHits_dEdx}) and fraction of number of hits potentially generated by the track and finally used in the reconstruction $N_{\text{hits}}^{\text{fit}}/N_{\text{hits}}^{\text{poss}}$ (\subref{fig:NHitsFit_to_NHitsPos}) in the data (black points) and embedded MC (stacked color histograms). Normalizations of the signal and backgrounds were established according to description in Sec.~\ref{sec:bkgdSignalNorm}. Predictions for MCs other than GenEx (yellow) were replaced by predictions for GenEx scaled to have the same integrals as replaced histograms - this was driven by the fact that only GenEx was embedded into zero-bias TPC data, which is required to describe the three presented quantities. Vertical error bars represent statistical uncertainties, horizontal bars represent bin sizes. Red dashed line and red arrow indicate the range of each quantity which is accepted in analysis (if cut on this quantity is applied).}\label{fig:NHits}
  \end{minipage}
}%

\end{figure}
%---------------------------






%---------------------------
\begin{figure}[ht!]
\centering
\parbox{0.4725\textwidth}{
  \centering
  \begin{subfigure}[b]{\linewidth}{
                \subcaptionbox{\label{fig:TrackEta}}{\includegraphics[width=\linewidth]{graphics/eventSelection/TpcTracks/Ratio_Linear_Eta.pdf}}}
  \end{subfigure}
}%
\quad\quad%
\parbox{0.4725\textwidth}{%
  \centering
  \begin{subfigure}[b]{\linewidth}{
                \subcaptionbox{\label{fig:TrackPhi}}{\includegraphics[width=\linewidth]{graphics/eventSelection/TpcTracks/Ratio_Linear_Phi.pdf}}}
  \end{subfigure}
}%1
\caption[Comparison of distribution of track $\eta$ and $\phi$ in the data and embedded MC.]
{Comparison of the track pseudorapidity $\eta$ (\subref{fig:TrackEta}) and the track azimuthal angle $\phi$ (\subref{fig:TrackPhi}) in the data (black points) and embedded MC (stacked color histograms). Normalizations of the signal and backgrounds were established according to description in Sec.~\ref{sec:bkgdSignalNorm}. Vertical error bars represent statistical uncertainties, horizontal bars represent bin sizes. Red dashed line and red arrow indicate the range of each quantity which is accepted in analysis (if cut on this quantity is applied).}\label{fig:TrackEtaPhi}
\end{figure}
%---------------------------









\subsection{(\ref{enum:CutRpTrks})~RP tracks}\label{sec:C4}

In presented physics analysis highest-level forward objects were used - the RP tracks. They are obtained through the reconstruction starting from the signals in single channels of silicon strip detectors housed inside RPs. General description of the reconstruction procedure has been given in Sec.~6.2 of the supplementary analysis note~\cite{supplementaryNote} and reference therein. A bit more detail description is available in elastic proton-proton scattering measurement note~\cite{elasticNote}.


Roman Pot data was analyzed offline as follows. First, all RP tracks which contain track points that had been formed of less than 3 hits of out 4 maximally possible (1 hit per silicon plane), were rejected. This is natural consequence of very high single plane efficiency $>99.5\%$, and prevents including to analysis tracks with track points formed from unmatched pairs of clusters in both $x$- and $y$-coordinate (e.g. from electronics noise).

%---------------------------
\begin{figure}[b!]
\centering
\parbox{0.4725\textwidth}{
  \centering
  \begin{subfigure}[b]{\linewidth}
                \subcaptionbox{\label{fig:localAngle2D_X}}{\includegraphics[width=\linewidth,page=1]{graphics/eventSelection/RpTracks/RpTrackCuts_2.pdf}\vspace*{-10pt}}
  \end{subfigure}\\
  \begin{subfigure}[b]{\linewidth}\addtocounter{subfigure}{1}
                \subcaptionbox{\label{fig:localAngle1D_X}}{\includegraphics[width=\linewidth,page=2]{graphics/eventSelection/RpTracks/RpTrackCuts_2.pdf}}
  \end{subfigure}
}%
\quad\quad%
\parbox{0.4725\textwidth}{
  \centering
  \begin{subfigure}[b]{\linewidth}\addtocounter{subfigure}{-2}
                \subcaptionbox{\label{fig:localAngle2D_Y}}{\includegraphics[width=\linewidth,page=3]{graphics/eventSelection/RpTracks/RpTrackCuts_2.pdf}\vspace*{-10pt}}
  \end{subfigure}\\
  \begin{subfigure}[b]{\linewidth}\addtocounter{subfigure}{1}
                \subcaptionbox{\label{fig:localAngle1D_Y}}{\includegraphics[width=\linewidth,page=4]{graphics/eventSelection/RpTracks/RpTrackCuts_2.pdf}}
  \end{subfigure}
}%
\caption[Local angle vs. position of RP tracks matched with true level primary protons.]{Typical correlation between local angle ($y$-axis) and position ($x$-axis) of RP tracks matched with true level primary protons for $x$- (\subref{fig:localAngle2D_X}) and $y$-coordinate (\subref{fig:localAngle2D_Y}), here shown for branch WU. The same events are contained in \subref{fig:localAngle1D_X} and \subref{fig:localAngle1D_Y} for $x$- and $y$-coordinate respectively, where difference between reconstructed local angle and local angle expected from the elastic track is histogrammed. Red lines and arrows visualize cuts imposed on RP tracks for final selection (cuts~\ref{enum:RpLocalAngles}).}\label{fig:localAngleRp}%
\end{figure}
%---------------------------

Next, preselected tracks were verified for consistency of their local angles with hypothesis of their origin being at the STAR IR. Using Geant4 simulation of RP system (see Sec.~6.3. of Ref.~\cite{supplementaryNote}) the impact of apertures limiting RP acceptance for the forward scattered protons generated at STAR IR was tested. The result is shown in Fig.~\ref{fig:localAngleRp}, where density maps of reconstructed RP track local angle $\theta^{\text{RP}}$ and corresponding track coordinate in RP station are drawn (we show it only for branch WU as the picture is the same in the remaing branches). Only RP tracks matched with generated primary forward protons were used to fill the histograms. Clear bands of primary proton tracks can be distinguished in the top plots (Figs.~\ref{fig:localAngle2D_X} and~\ref{fig:localAngle2D_Y}), with some very small number of tracks significantly scattered on the beampipe/DX/detector material. One-dimensional representation of the correlation between local angle and position can be obtained by constructing quantities
\begin{equation}
 \widetilde{\Delta}\theta_{x}^{RP} = \theta_{x}^{\text{RP}}-x^{\text{RP}}/|z^{\text{RP}}|,
\end{equation}
\begin{equation}
 \widetilde{\Delta}\theta_{y}^{RP} = \theta_{y}^{\text{RP}}-y^{\text{RP}}/|z^{\text{RP}}|,
\end{equation}%
which reflect deviation of reconstructed local angle from expectation for forward proton of the beam momentum, and whose distributions are presented in Fig.~\ref{fig:localAngle1D_X} and Fig.~\ref{fig:localAngle1D_Y}, respectively (black histograms). On these one-dimensional histograms we clearly see peaks from the true primary tracks. We considered optimal to restrict accepted $\widetilde{\Delta}\theta_{x}^{RP}$ from -2~mrad to 4~mrad, and $\widetilde{\Delta}\theta_{y}^{RP}$ from -2~mrad to 2~mrad. The upper cut on $\widetilde{\Delta}\theta_{x}^{RP}$ equal to  4~mrad may look too inclusive, but intention was to preserve tracks of protons with very large $\xi$, whose local angle highly deviates from that of elastically scattered protons (DX magnets bends more protons with lower momentum) and which might have been underpopulated in MC (GenEx predictions were used). It is also worth to comment on the blue histograms in Figs.~\ref{fig:localAngle1D_X} and~\ref{fig:localAngle1D_Y}, which represent local tracks (formed of single track points). These tracks are reconstructed assuming their momentum is equal to the beam momentum (angle at vertex equal to angle at RP station), therefore $\widetilde{\Delta}\theta_{x}^{RP}$ and $\widetilde{\Delta}\theta_{y}^{RP}$ is 0 by definition.


Once the set of cuts above was applied we required that on each side of STAR there was exactly one selected RP track. We did not allow higher number of tracks on one side because of no clear way to discriminate real tracks of primary protons.

In addition to cuts above, we restricted our measurement to the fiducial region defined as
\begin{equation}\label{eq:RpFiducial}
0.2<|p_{y}|<0.4,~~~-0.2<p_{x},~~~(p_{x}+0.3)^{2}+p_{y}^{2}<0.5^{2}~~~(\text{all in GeV}), 
\end{equation}
therefore both RP tracks were required to be contained within above envelope in $(p_{x}, p_{y})$ space. This fiducial area is drawn with black solid line on top of the $(p_{x}, p_{y})$ distribution of all measured CEP candidates (Fig.~\ref{fig:rp_hits}). It was chosen to compromize signal statistics and systematic uncertainties of the RP-related efficiencies (see e.g. Sec.~10.3 of Ref.~\cite{supplementaryNote}).

\begin{figure}[b!]
\centering
\includegraphics[width=.465\textwidth]{graphics/eventSelection/RpTracks/PxPyExclusiveAllMerged.pdf}
%\hfill
\includegraphics[width=.523\textwidth]{graphics/eventSelection/RpTracks/Paper_MandelstamT.pdf}
%
\caption{(left) Merged distributions of diffractively scattered protons momenta $p_y$ vs. $p_x$ in exclusive $h^{+}h^{-}$ events reconstructed with the East and West RP stations, together with the kinematic region used in the measurement marked with the black line. (right) Distributions of measured four momenta transfers at the proton vertices for exclusive $h^{+}h^{-}$ events with all particles in the fiducial phase space are shown for East and West stations with yellow and blue color, respectively.}
\label{fig:rp_hits}
\end{figure}


In the remaining part of the section we show comparisons of the track points position distributions between the data and embedded MC. In Fig.~\ref{fig:hitMap_DataVsMC} we present side-by-side comparisons of two-dimensional hit maps from the data and embedded MC. Figures~\ref{fig:xRp} and~\ref{fig:yRp} show the same comparisons, but between their $x$- and $y$-projections for each RP separately. One can see, that simulation generally describes data well, both in terms of shapes (which is mainly sensitive to detector alignment and geometry/apertures) and track points normalizations in various RPs (which is mainly sensitive to reconstruction efficiency).


%---------------------------
\begin{figure}[h]
\centering
\parbox{0.4725\textwidth}{
  \centering
  \begin{subfigure}[b]{\linewidth}
                \subcaptionbox{\label{fig:E1_HitMap}}{\includegraphics[width=\linewidth]{graphics/eventSelection/RpTracks/E1_HitMap.pdf}}
  \end{subfigure}\\[10pt]
  \begin{subfigure}[b]{\linewidth}\addtocounter{subfigure}{1}
                \subcaptionbox{\label{fig:W1_HitMap}}{\includegraphics[width=\linewidth]{graphics/eventSelection/RpTracks/W1_HitMap.pdf}}
  \end{subfigure}
}%
\quad\quad%
\parbox{0.4725\textwidth}{
  \centering
  \begin{subfigure}[b]{\linewidth}\addtocounter{subfigure}{-2}
                \subcaptionbox{\label{fig:E2_HitMap}}{\includegraphics[width=\linewidth]{graphics/eventSelection/RpTracks/E2_HitMap.pdf}}
  \end{subfigure}\\[10pt]
  \begin{subfigure}[b]{\linewidth}\addtocounter{subfigure}{1}
                \subcaptionbox{\label{fig:W2_HitMap}}{\includegraphics[width=\linewidth]{graphics/eventSelection/RpTracks/W2_HitMap.pdf}}
  \end{subfigure}
}%
\caption[Comparison of two-dimensional track point density map in the data and embedded MC.]
    {Comparison of two-dimensional track point density map in the data (left panel in subfigures) and stacked embedded MC (right panel in subfigures). Each subfigure corresponds to single RP station with position of track points measured in upper and lower RP visible at positive and negative $y$, respectively. Normalizations of the signal and backgrounds were established according to description in Sec.~\ref{sec:bkgdSignalNorm}.}\label{fig:hitMap_DataVsMC}%
\end{figure}
%---------------------------


%---------------------------
\begin{figure}[h]
\centering
\parbox{0.31\textwidth}{
  \centering
  \begin{subfigure}[b]{\linewidth}
                \subcaptionbox{\label{fig:Ratio_Linear_x_E1U}}{\includegraphics[width=\linewidth]{graphics/eventSelection/RpTracks/Ratio_Linear_x_E1U.pdf}\vspace*{-10pt}}
  \end{subfigure}\\[5pt]
  \begin{subfigure}[b]{\linewidth}%\addtocounter{subfigure}{1}
                \subcaptionbox{\label{fig:Ratio_Linear_x_W2U}}{\includegraphics[width=\linewidth]{graphics/eventSelection/RpTracks/Ratio_Linear_x_W2U.pdf}\vspace*{-10pt}}
  \end{subfigure}\\[5pt]
  \begin{subfigure}[b]{\linewidth}%\addtocounter{subfigure}{1}
                \subcaptionbox{\label{fig:Ratio_Linear_x_W1D}}{\includegraphics[width=\linewidth]{graphics/eventSelection/RpTracks/Ratio_Linear_x_W1D.pdf}\vspace*{-10pt}}
  \end{subfigure}
}%
\quad%
\parbox{0.31\textwidth}{
  \centering
  \begin{subfigure}[b]{\linewidth}%\addtocounter{subfigure}{-2}
                \subcaptionbox{\label{fig:Ratio_Linear_x_E2U}}{\includegraphics[width=\linewidth]{graphics/eventSelection/RpTracks/Ratio_Linear_x_E2U.pdf}\vspace*{-10pt}}
  \end{subfigure}\\[5pt]
  \begin{subfigure}[b]{\linewidth}%\addtocounter{subfigure}{1}
                \subcaptionbox{\label{fig:Ratio_Linear_x_E1D}}{\includegraphics[width=\linewidth]{graphics/eventSelection/RpTracks/Ratio_Linear_x_E1D.pdf}\vspace*{-10pt}}
  \end{subfigure}\\[5pt]
  \begin{subfigure}[b]{\linewidth}%\addtocounter{subfigure}{1}
                \subcaptionbox{\label{fig:Ratio_Linear_x_W2D}}{\includegraphics[width=\linewidth]{graphics/eventSelection/RpTracks/Ratio_Linear_x_W2D.pdf}\vspace*{-10pt}}
  \end{subfigure}
}%
\quad%
\parbox{0.31\textwidth}{
  \centering\vspace*{-22pt}
  \begin{subfigure}[b]{\linewidth}%\addtocounter{subfigure}{-2}
                \subcaptionbox{\label{fig:Ratio_Linear_x_W1U}}{\includegraphics[width=\linewidth]{graphics/eventSelection/RpTracks/Ratio_Linear_x_W1U.pdf}\vspace*{-10pt}}
  \end{subfigure}\\[5pt]
  \begin{subfigure}[b]{\linewidth}%\addtocounter{subfigure}{1}
                \subcaptionbox{\label{fig:Ratio_Linear_x_E2D}}{\includegraphics[width=\linewidth]{graphics/eventSelection/RpTracks/Ratio_Linear_x_E2D.pdf}\vspace*{-10pt}}
  \end{subfigure}
    \begin{minipage}[t][1.042\linewidth][t]{\linewidth}\end{minipage}
}
\caption[Comparison of $x$-position of track point between the data and stacked embedded MC.]{Comparison of $x$-position of track point between the data (black points) and stacked embedded MC (color histograms). Each subfigure corresponds to single RP station, whose name is printed in the right part subfigure. Vertical error bars represent statistical uncertainties, horizontal bars represent bin sizes. Normalizations of the signal and backgrounds were established according to description in Sec.~\ref{sec:bkgdSignalNorm}.}\label{fig:xRp}% 
\end{figure}
%---------------------------




%---------------------------
\begin{figure}[h]
\centering
\parbox{0.31\textwidth}{
  \centering
  \begin{subfigure}[b]{\linewidth}
                \subcaptionbox{\label{fig:Ratio_Linear_y_E1U}}{\includegraphics[width=\linewidth]{graphics/eventSelection/RpTracks/Ratio_Linear_y_E1U.pdf}\vspace*{-10pt}}
  \end{subfigure}\\[5pt]
  \begin{subfigure}[b]{\linewidth}%\addtocounter{subfigure}{1}
                \subcaptionbox{\label{fig:Ratio_Linear_y_W2U}}{\includegraphics[width=\linewidth]{graphics/eventSelection/RpTracks/Ratio_Linear_y_W2U.pdf}\vspace*{-10pt}}
  \end{subfigure}\\[5pt]
  \begin{subfigure}[b]{\linewidth}%\addtocounter{subfigure}{1}
                \subcaptionbox{\label{fig:Ratio_Linear_y_W1D}}{\includegraphics[width=\linewidth]{graphics/eventSelection/RpTracks/Ratio_Linear_y_W1D.pdf}\vspace*{-10pt}}
  \end{subfigure}
}%
\quad%
\parbox{0.31\textwidth}{
  \centering
  \begin{subfigure}[b]{\linewidth}%\addtocounter{subfigure}{-2}
                \subcaptionbox{\label{fig:Ratio_Linear_y_E2U}}{\includegraphics[width=\linewidth]{graphics/eventSelection/RpTracks/Ratio_Linear_y_E2U.pdf}\vspace*{-10pt}}
  \end{subfigure}\\[5pt]
  \begin{subfigure}[b]{\linewidth}%\addtocounter{subfigure}{1}
                \subcaptionbox{\label{fig:Ratio_Linear_y_E1D}}{\includegraphics[width=\linewidth]{graphics/eventSelection/RpTracks/Ratio_Linear_y_E1D.pdf}\vspace*{-10pt}}
  \end{subfigure}\\
  \begin{subfigure}[b]{\linewidth}%\addtocounter{subfigure}{1}
                \subcaptionbox{\label{fig:Ratio_Linear_y_W2D}}{\includegraphics[width=\linewidth]{graphics/eventSelection/RpTracks/Ratio_Linear_y_W2D.pdf}\vspace*{-10pt}}
  \end{subfigure}
}%
\quad%
\parbox{0.31\textwidth}{
  \centering\vspace*{-22pt}
  \begin{subfigure}[b]{\linewidth}%\addtocounter{subfigure}{-2}
                \subcaptionbox{\label{fig:Ratio_Linear_y_W1U}}{\includegraphics[width=\linewidth]{graphics/eventSelection/RpTracks/Ratio_Linear_y_W1U.pdf}\vspace*{-10pt}}
  \end{subfigure}\\[5pt]
  \begin{subfigure}[b]{\linewidth}%\addtocounter{subfigure}{1}
                \subcaptionbox{\label{fig:Ratio_Linear_y_E2D}}{\includegraphics[width=\linewidth]{graphics/eventSelection/RpTracks/Ratio_Linear_y_E2D.pdf}\vspace*{-10pt}}
  \end{subfigure}
    \begin{minipage}[t][1.042\linewidth][t]{\linewidth}\end{minipage}
}
\caption[Comparison of $y$-position of track point between the data and stacked embedded MC.]{Comparison of $y$-position of track point between the data (black points) and stacked embedded MC (color histograms). Each subfigure corresponds to single RP station, whose name is printed in the middle of subfigure. Vertical error bars represent statistical uncertainties, horizontal bars represent bin sizes. Normalizations of the signal and backgrounds were established according to description in Sec.~\ref{sec:bkgdSignalNorm}.}\label{fig:yRp}% 
\end{figure}
%---------------------------






%%%%%%%%%%%%%%%%%%%%%%%%%%%%%%%%%%%%%%%%%%%%%%%%%%%%%%%%%%%%%%%%%%%%%%%%%%%%%%%%%%%%%%%%%%%%%%%%%%%%%%%%%%%%%%%%%%%%%%%%%%%%%%%%%%%%
\subsection{(\ref{enum:CutDeltaZVx})~TPC-RP \texorpdfstring{$z$}{z}-vertex matching}\label{sec:C5}

In CEP tracks in the central detector and tracks in Roman Pots originate from the same interaction vertex. Measurement of the time of detection of forward protons in RPs gives access to reconstruction of the position of the vertex
\begin{equation}
z_{\text{vtx}}^{\text{RP}} = c\cdot\frac{t^{\text{RP}}_{\text{W}} - t^{\text{RP}}_{\text{E}}}{2}
\end{equation}
independently from TPC, which allows their comparison and rejection of the background if the two values disagree. Time of detection of proton in RP is provided in StMuRpsTrack object - it is an average of all TAC values from PMTs in RPs used to form a track, corrected for the slewing effect and adjusted to have the best correlation with the $z$-position of the vertex measured in TPC, translated to unit of time (all these steps are done at the level of raw data reconstruction). In Fig.~\ref{fig:zVertexRpTpc} the comparisons of the $z_{\text{vtx}}^{\text{RP}}$ and $z_{\text{vtx}}^{\text{TPC}}$ are shown with some preselection cuts applied. A clear signal from the Central Diffraction (and thus CEP) process is manifesting in high correlation of the two values (diagonal in Fig.~\ref{fig:zVertexRpVsTpc}) or significant and relatively narrow peak centered at 0 for the difference of two values (Fig.~\ref{fig:zVertexRpMinusZVertexTpc}). %
%---------------------------
\begin{figure}[ht!]
\centering
\parbox{0.4\textwidth}{
  \centering
  \begin{subfigure}[b]{\linewidth}{
                \subcaptionbox{\label{fig:zVertexRpVsTpc}}{\includegraphics[width=\linewidth]{graphics/eventSelection/zVertexRpVsTpc.pdf}}}
  \end{subfigure}
}
\quad
\parbox{0.545\textwidth}{
  \centering
  \begin{subfigure}[b]{\linewidth}{
                \subcaptionbox{\label{fig:zVertexRpMinusZVertexTpc}}{\includegraphics[width=\linewidth]{graphics/eventSelection/zVertexRpMinusZVertexTpc.pdf}}}
  \end{subfigure}
}%
\caption[Correlation and difference of $z$-vertex position measured in Roman Pots and TPC.]{Correlation (Fig.~\ref{fig:zVertexRpVsTpc}) and difference (Fig.~\ref{fig:zVertexRpMinusZVertexTpc}) of $z$-vertex position measured in Roman Pots and TPC in RP\_CPT2 triggers, after preselection described in the plots.}\label{fig:zVertexRpTpc}
\end{figure}%
%---------------------------
The sum of two Gaussian distributions was fitted to data in Fig.~\ref{fig:zVertexRpMinusZVertexTpc} yielding good description of the distribution of $\Delta z_{\text{vtx}}$ with the width parameters equal $10.3$~cm (CD signal) and $73.9$~cm (pile-up). The first parameter reflects the time resolution of RPs (the $z_{\text{vtx}}^{\text{RP}}$ measurement), as the TPC resolution is much better ($\sim 1$~cm). Value of the second parameter, consistent with $\sqrt{2}\sigma_{z_{\text{vtx}}}\approx\sqrt{2}\cdot52~\text{cm}\approx 73.5$~cm, confirms that the wide distribution under the narrow signal peak is uncorrelated background, in other words forward protons originating from a different vertex than the central tracks. To reject this background without significant loss of the signal, we introduce $3.5\sigma_{\Delta z_{\text{vtx}}}$ cut on $\Delta z_{\text{vtx}}$.


% %---------------------------
% \begin{figure}[ht!]
% % \begin{wrapfigure}{l}{0.475\textwidth}%[ht!]
% \centering%
% \includegraphics[width=0.475\linewidth,page=1]{graphics/eventSelection/DeltaZVx.pdf}%
% % % \includegraphics[width=\linewidth,page=1]{graphics/eventSelection/DeltaZVx.pdf}%
% \caption{Delta z-vx.}\label{fig:DeltaZVx}%
% \end{figure}
% % \end{wrapfigure}
% %---------------------------
%%%%%%%%%%%%%%%%%%%%%%%%%%%%%%%%%%%%%%%%%%%%%%%%%%%%%%%%%%%%%%%%%%%%%%%%%%%%%%%%%%%%%%%%%%%%%%%%%%%%%%%%%%%%%%%%%%%%%%%%%%%%%%%%%%%%




%%%%%%%%%%%%%%%%%%%%%%%%%%%%%%%%%%%%%%%%%%%%%%%%%%%%%%%%%%%%%%%%%%%%%%%%%%%%%%%%%%%%%%%%%%%%%%%%%%%%%%%%%%%%%%%%%%%%%%%%%%%%%%%%%%%%
\subsection{(\ref{enum:CutBbcLarge})~BBC-large signal veto}\label{sec:C6}

At the trigger level a veto on signal in small BBC detectors was used. During offline analysis we found that the non-exclusive background can be reduced if an additional veto on signal in large BBC detectors is added. It is connected with the fact that vast majority of selected RP\_CPT2 triggers were from the central diffraction process to which CEP belongs. Many of central diffraction events have particles produced in the rapidity region outside the TPC and TOF acceptance, some hitting large BBC tiles. Presence of signal in large BBC is therefore a signature of background or a pile-up interaction.

The response of large BBC tiles is different from that of small BBC tiles, as shown in sample plots in Fig.~\ref{fig:sampleBbcResponse} (similar distributions for all channels can be found in Appendix~\ref{appendix:bbc}). Typically in small BBC tiles a peak visible in ADC distribution around $100-150$ (Figs.~\ref{fig:sampleBbcSmallAdcVsTac},\ref{fig:sampleBbcSmallAdc}), a signature of good separation of the electronics noise and signal from the ionizing particle. No such feature is observed in corresponding distribution for large BBC tile (Figs.~\ref{fig:sampleBbcLargeAdcVsTac},\ref{fig:sampleBbcLargeAdc}), which can be explained by the difference in geometry (in size) of small and large tiles. In large BBC tiles the path that scintillation light must travel to reach PMT is much longer in comparison to smal BBC tiles (multiple reflections on the main tile surface due to small thickness of the tile) therefore it is highly attenuated and extended in time. This is possible reason of lack of signal peak in the ADC distribution in large BBC tile spectrum (Fig.~\ref{fig:sampleBbcLargeAdc}), as well as the late-TAC (TAC$<\sim600$, ADC$<100$) tail in the ADC vs. TAC spectrum (slewing effect, Fig.~\ref{fig:sampleBbcLargeAdcVsTac}). Nevertheless, the above features of BBC-large response does not disqualify this detector from being used as a veto detector, as in this case lower efficiency of the detector only reduce the background rejection power.


%---------------------------
\begin{figure}%[h]
\centering
\parbox{0.4725\textwidth}{
  \centering
  \begin{subfigure}[b]{\linewidth}
                \subcaptionbox{\label{fig:sampleBbcSmallAdcVsTac}}{\includegraphics[width=\linewidth,page=1]{graphics/eventSelection/bbc/Bbc_ADCvsTAC_collidingBunches.pdf}}
  \end{subfigure}\\
  \begin{subfigure}[b]{\linewidth}\addtocounter{subfigure}{1}
                \subcaptionbox{\label{fig:sampleBbcSmallAdc}}{\includegraphics[width=\linewidth,page=1]{graphics/eventSelection/bbc/Bbc_ADC.pdf}}
  \end{subfigure}
}%
\quad\quad%
\parbox{0.4725\textwidth}{
  \centering
  \begin{subfigure}[b]{\linewidth}\addtocounter{subfigure}{-2}
                \subcaptionbox{\label{fig:sampleBbcLargeAdcVsTac}}{\includegraphics[width=\linewidth,page=17]{graphics/eventSelection/bbc/Bbc_ADCvsTAC_collidingBunches.pdf}}
  \end{subfigure}\\
  \begin{subfigure}[b]{\linewidth}\addtocounter{subfigure}{1}
                \subcaptionbox{\label{fig:sampleBbcLargeAdc}}{\includegraphics[width=\linewidth,page=17]{graphics/eventSelection/bbc/Bbc_ADC.pdf}}
  \end{subfigure}
}%
\caption[Sample BBC-small and BBC-large response in zero-bias triggers.]{Sample BBC-small (left column) and BBC-large (right column) response in zero-bias data. Top row shows TAC vs. ADC distributions, bottom row shows projection of the corresponding two-dimensional ditribution on $x$-axis (ADC) in the TAC range quoted in the legend, for both abort gaps and colliding bunches. Red lines and arrows indicate thresholds for a signal in presented channels.}\label{fig:sampleBbcResponse}
\end{figure}
%---------------------------


Each channel of the BBC-large has different response to signal from ionizing particle, as well as different level of noise. We decided to set up a signal threshold for each channel based on a study of the noise in abort gaps (in zero-bias data). This noise, in principle, should be solely the electronics noise. We checked for each channel the probability to detect a signal with ADC above certain threshold and with TAC contained within 100 and 2400 (the same window is deafult for small BBC). The result is shown in Fig.~\ref{fig:bbcLargeThresholds}. Next, we established final ADC thresholds in each BBC-large channel by requiring that the noise in BBC-large would cause a veto in maximally $3.5\%$ of events. Such number was chosen because it was consistent with an average ADC threshold of 40, found optimal in terms of selection efficiency and sample purity (see Appendix~\ref{appendix:workingPoint}), as well as it was acceptably low. To transform it to $\text{ADC}_{thr}$ we first assumed that the noise is uncorrelated between the channels. With this assumption one can connect the probability of the veto in whole BBC-large detector (east and west) caused by noise $\mathcal{P}_{\text{veto}}^{\text{noise}}$ with the probability of the signal induced by noise in single BBC-large channel $\mathcal{P}_{i,\text{sig}}^{\text{noise}}$:
\begin{equation}\label{eq:bbcNoise1}
 \mathcal{P}_{\text{veto}}^{\text{noise}} = 1-\mathcal{P}_{!\text{veto}}^{\text{noise}} = 1-\left( 1-\mathcal{P}_{i,\text{sig}}^{\text{noise}} \right)^{N^{\text{BBC}}_{\text{ch}}}.
\end{equation}
In the equation above $N^{\text{BBC}}_{\text{ch}}$ denotes number of active channels in BBC-large. From plots contained in Appendix~\ref{appendix:bbc} one can read that there were 14 active channels in BBC-large. 2 dead channels were found on the west side (40 and 42). By transforming Eq.~\ref{eq:bbcNoise1} to the form presented below we can calculate the threshold probability for a single BBC-large channel:
\begin{equation}\label{eq:bbcNoise2}
 \mathcal{P}_{i,\text{sig}}^{\text{noise}} = 1-\sqrt[N^{\text{BBC}}_{\text{ch}}]{1-\mathcal{P}_{\text{veto}}^{\text{noise}}} = 1-\sqrt[14]{1-0.035} \approx 0.0025.
\end{equation}
In the last step we translated this number to ADC threshold for each channel of BBC-large. For this purpose we used Fig.~\ref{fig:bbcLargeThresholds}. The $x$-axis projection of the crossing point of each color line with the $y$-axis value of 0.0025 defines $\text{ADC}_{thr}$ for each particular channel. These numbers are listed in Tab.~\ref{tab:bbcLargeThresholds}. The event was dropped from analysis if any of the BBC-large channels registered signal of strength $\text{ADC}_{i}>\text{ADC}_{i,thr}$ and $100<\text{TAC}_{i}<2400$.



\begin{table}%[h]
	\begin{minipage}{0.65\linewidth}
		\centering
		\includegraphics[width=\linewidth]{graphics/eventSelection/bbc/BbbLargeThreshold.pdf}
		\captionof{figure}[Probability of false BBC-large signal (noise-induced).]{Percentage of events in abort gaps from zero-bias triggers with the ADC counts larger than the ADC threshold given in the $x$-axis, for each BBC-large channel. Measured points with statistical uncertainties are connected with a smooth line of corresponding color for better visualization.}
		\label{fig:bbcLargeThresholds}
	\end{minipage}\hfill
	\begin{minipage}{0.3\linewidth}
		\centering
		\begin{tabular}{c|c||c|c}
			\multicolumn{2}{c||}{East} & \multicolumn{2}{c}{West} \\ \hline
			$i$  & $\text{ADC}_{\text{thr}}$ & $i$  & $\text{ADC}_{\text{thr}}$ \\ \hline
			16 & 27 & 40 & (dead) \\
			17 & 30 & 41 & 31 \\
			18 & 26 & 42 & (dead) \\
			19 & 37 & 43 & 14 \\
			20 & 25 & 44 & 29 \\
			21 & 55 & 45 & 30 \\
			22 & 43 & 46 & 33 \\
			23 & 27 & 47 & 22 \\
		\end{tabular}
		\caption[Offline ADC thresholds in BBC-large.]{Offline ADC thresholds in BBC-large.\newline\newline\newline\newline\newline\newline\newline\newline}
		\label{tab:bbcLargeThresholds}
	\end{minipage}

\end{table}

Observation of high purification of CEP sample with described BBC-large veto in the data from run 15 was helpful to improve the CEP trigger for run 17. The improved trigger called RP\_CPT2noBBCL was similar to RP\_CPT2 with an addition of BBC-large veto using ADC threshold of 50.

%%%%%%%%%%%%%%%%%%%%%%%%%%%%%%%%%%%%%%%%%%%%%%%%%%%%%%%%%%%%%%%%%%%%%%%%%%%%%%%%%%%%%%%%%%%%%%%%%%%%%%%%%%%%%%%%%%%%%%%%%%%%%%%%%%%%



%%%%%%%%%%%%%%%%%%%%%%%%%%%%%%%%%%%%%%%%%%%%%%%%%%%%%%%%%%%%%%%%%%%%%%%%%%%%%%%%%%%%%%%%%%%%%%%%%%%%%%%%%%%%%%%%%%%%%%%%%%%%%%%%%%%%
\subsection{(\ref{enum:CutTofClusters})~TOF clusters limit}\label{sec:C7}

The TOF is mainly used to distinguish real TPC tracks from the fakes, as well as it helps to identify particles. However, we also used it to reject non-CEP events in which the TPC tracks were not reconstructed or were not successfully matched to TOF hit. For this we introduced a concept of a TOF cluster - a group of offline TOF hits close in space and time. We expect that such cluster of hits is induced by the single primary particle, eventually associated with the secondaries (e.g. delta rays).

We define a TOF cluster as a group of reconstructed TOF hits with the ($\phi$, $\eta$) space distance $R$ to neighbouring hit (defined similarly to Eq.~\eqref{eq:etaPhiR} not larger than 0.1 and with the time distance to the same hit $\Delta t$ not larger than 1.5~ns. In other words, TOF clusters are formed by the offline hits that form at least one pair with the other hit in the cluster satisfying
\begin{equation}
 R<0.1,~~~~~~\Delta t<1.5~\text{ns}.
\end{equation}
Per event no more than 1 additional TOF cluster was allowed, thus in total the number of reconstructed TOF clusters $N^{\text{TOF}}_{\text{clstrs}}$ could not exceed 3.

%---------------------------
\begin{figure}[ht!]
% \begin{wrapfigure}{l}{0.475\textwidth}%[ht!]
\centering%
\includegraphics[width=0.475\linewidth,page=1]{graphics/eventSelection/NTofClusters.pdf}%
% % \includegraphics[width=\linewidth,page=1]{graphics/eventSelection/NTofClusters.pdf}% 
\caption{NTofClusters.}\label{fig:NTofClusters}%
\end{figure}
% \end{wrapfigure}
%---------------------------

\subsection{(\ref{enum:CutPid})~Particle identification}\label{subsec:pidCuts}\label{sec:C8}

Particles were identified using combined information from the TPC ($dE/dx$) and TOF (time of hit detection in the TOF subsystem). Merging informations from two sources led to reduction of misidentifications, as well as gave access to higher kaon and proton momentum range where $dE/dx$ of different species overlap.

Compatibility of track $dE/dx$ with that expected from particle of type $X$ was determined using the quantity $n\sigma_{X}$ widely used at STAR, defined as
\protect \begin{equation}\label{eq:nSigmaDef} n\sigma_{X} =  \ln{\left[(dE/dx)^\text{measured} / (dE/dx)_{X}^\text{theory}\right]}~~/~~\sigma_{dE/dx}, \end{equation}
%
where $(dE/dx)^\text{measured}$ is the ionization energy loss of the TPC track, $(dE/dx)_{X}^\text{theory}$ is the Bethe-Bloch~\cite{Bichsel} expectation for the given particle type ($X=\pi$, $K$, $p$) at reconstructed track momentum, and $\sigma_{dE/dx}$ is the statistical uncertainty of $\ln{(dE/dx)^\text{measured}}$. Quantity $n\sigma_{X}$ is in fact a pull: $(dE/dx)^\text{measured}$ is (in first order) an average over $\text{Landau}\otimes\text{normal}$-distributed $dE/dx$ of single TPC hits forming the track, hence the $(dE/dx)^\text{measured}$ is distributed log-normally and $\ln{(dE/dx)^\text{measured}}$ - normally. From $n\sigma_{X}$ of the two tracks the $\chi^{2}$ statistic for a $XX$ pair hypothesis was calculated:
%
\begin{equation}\label{eq:chiSqDef}\chi^{2}_{dE/dx}(XX) = \left(n\sigma_{X}^{\text{trk1}}\right)^{2} + \left(n\sigma_{X}^{\text{trk2}}\right)^{2}.\end{equation}
%
Sometimes we also quote $n\sigma^{\text{pair}}$ quantity (which is no longer a Gaussian pull) connected with $\chi^{2}$ through relation
%
\begin{equation}\label{eq:nSigmaPairDef}n\sigma^{\text{pair}}_{X} = \sqrt{\chi^{2}_{dE/dx}(XX)} = \sqrt{\left(n\sigma_{X}^{\text{trk1}}\right)^{2} + \left(n\sigma_{X}^{\text{trk2}}\right)^{2}}.\end{equation}
%
The time of detection of particle in the TOF system was used to reconstruct its squared mass $m^{2}_{\text{TOF}}$. For this purpose the time of primary interaction is typically used (''start time``), reconstructed by detecting fragments of dissociated beam particles in VPD detectors on both sides of the interaction point\footnote{Time measured from protons in the RP detectors cannot be used because RP readout runs on independent clock from that used by VPD and TOF.}. However, it is not accessible in CEP as the initial protons survive the interaction intact. We therefore assumed that both central tracks are of the same type which is natural expectation for CEP events. With this assumption the time difference between TOF hits and measured tracks' momenta and lengths of helical paths between the primary vertex and TOF then allow to calculate $m^{2}_{\text{TOF}}$. The derivation of formula used to obtain $m^{2}_{\text{TOF}}$ is presented in Appendix~\ref{appendix:squaredMass}.

Particle identification involved a few steps. First, the $pp$ hypothesis was verified:
\begin{equation}\label{eq:pidPPbar}\lefteqn{\overbrace{\phantom{\chi^{2}_{dE/dx}(pp)<9\;\;\; \& \;\;\; m^{2}_{\text{TOF}} > 0.6~\mbox{GeV}^{2}}}^{\text{likely}~pp}}\chi^{2}_{dE/dx}(pp)<9\;\;\; \& \;\;\; \underbrace{m^{2}_{\text{TOF}} > 0.6~\mbox{GeV}^{2}\;\;\; \& \;\;\; \chi^{2}_{dE/dx}(\pi\pi)>9\;\;\; \& \;\;\; \chi^{2}_{dE/dx}(KK)>9}_{\text{unlikely}~\pi\pi~\text{or}~KK}.\end{equation}
If any of above was not satisfied, the pair was checked for compatibility with $KK$ hypothesis:
%
\begin{equation}\label{eq:pidKK}%
\lefteqn{\overbrace{\phantom{\chi^{2}_{dE/dx}(KK)<9\;\;\; \& \;\;\; m^{2}_{\text{TOF}} > 0.15~\mbox{GeV}^{2}}}^{\text{likely}~KK}}\chi^{2}_{dE/dx}(KK)<9\;\;\; \& \;\;\; \underbrace{m^{2}_{\text{TOF}} > 0.15~\mbox{GeV}^{2}\;\;\; \& \;\;\; \chi^{2}_{dE/dx}(\pi\pi)>9}_{\text{unlikely}~\pi\pi}\;\;\; \& \;\;\; \underbrace{\chi^{2}_{dE/dx}(pp)>9}_{\text{unlikely}~pp}.
\end{equation}
%
In case the pair was neither recognized as $p\bar{p}$ or $K^{+}K^{-}$, it was assumed to be a $\pi^{+}\pi^{-}$ pair if the $dE/dx$ of positive and negative charge track was consistent with pion hypothesis at $3\sigma$ level:
\begin{equation}\label{eq:pidPiPi}|n\sigma_{\pi}^{\text{trk1}}|<3\;\;\; \& \;\;\; |n\sigma_{\pi}^{\text{trk2}}|<3.\end{equation}




\begin{figure}[h]
\centering
\parbox{0.4725\textwidth}{
  \centering
  \begin{subfigure}[b]{\linewidth}
                \subcaptionbox{\label{fig:SqRootNSigma2D_a}}{\includegraphics[width=1.05\linewidth,page=1]{graphics/eventSelection/pid/PidSelector_SqRootNSigma2D.pdf}}
  \end{subfigure}\\
  \begin{subfigure}[b]{\linewidth}\addtocounter{subfigure}{1}
                \subcaptionbox{\label{fig:SqRootNSigma2D_c}}{\includegraphics[width=1.05\linewidth,page=3]{graphics/eventSelection/pid/PidSelector_SqRootNSigma2D.pdf}}
  \end{subfigure}
}%
\quad\quad%
\parbox{0.4725\textwidth}{
  \centering
  \begin{subfigure}[b]{\linewidth}\addtocounter{subfigure}{-2}\vspace*{-13pt}
                \subcaptionbox{\label{fig:SqRootNSigma2D_b}}{\includegraphics[width=1.05\linewidth,page=2]{graphics/eventSelection/pid/PidSelector_SqRootNSigma2D.pdf}}
  \end{subfigure}\\
  \begin{minipage}[t][1.042\linewidth][t]{\linewidth}\vspace{10pt}
    \caption[$n\sigma^{\text{pair}}_{X}$ vs. $n\sigma^{\text{pair}}_{Y}$.]{Two-dimensional distributions of $n\sigma^{\text{pair}}_{\pi}$ vs.~$n\sigma^{\text{pair}}_{K}$ (\subref{fig:SqRootNSigma2D_a}), $n\sigma^{\text{pair}}_{\pi}$ vs.~$n\sigma^{\text{pair}}_{p}$  (\subref{fig:SqRootNSigma2D_b}) and $n\sigma^{\text{pair}}_{K}$ vs.~$n\sigma^{\text{pair}}_{p}$  (\subref{fig:SqRootNSigma2D_c}) for exclusive event candidates after full event selection except PID cuts (except cuts~\ref{enum:CutPid}). Dashed lines indicate the value of $n\sigma^{\text{pair}}$ which is used in pair identification~\ref{enum:CutPid} ($n\sigma^{\text{pair}}_{X}=3$ which is equivalent to $\chi^{2}(XX)=9$).}\label{fig:SqRootNSigma2D}
  \end{minipage}
}%

\end{figure}
%--------------------------- 


In Fig.~\ref{fig:SqRootNSigma2D} we present two-dimensional distributions of $n\sigma^{\text{pair}}$ variables which help better undestand the behaviour and aim of $n\sigma^{\text{pair}}$ ($\chi^{2}$) cuts in Eqs.~\eqref{eq:pidPPbar}, \eqref{eq:pidKK}. Regions of enriched population of specific pair species are appropriately labeled. Similar connections between $n\sigma^{\text{pair}}$ and $m^{2}_{\text{TOF}}$ are shown in Fig.~\ref{fig:mSqVsNSigmaPair}.
 

 
\begin{figure}[ht!]
  \centering
  \begin{tabular}{@{}p{0.47\linewidth}@{\quad\quad}p{0.47\linewidth}@{}}
    \subfigimg[width=\linewidth,page=1]{~~~~~~~~~~~a)}{graphics/eventSelection/pid/PidSelector_SqMassTofVsSqRootNSigma_pion.pdf} &
    \subfigimg[width=\linewidth,page=1]{~~~~~~~~~~~c)}{graphics/eventSelection/pid/SqMassTofVsSqRootNSigma_pion.pdf} \\[-10pt]
    \subfigimg[width=\linewidth,page=1]{~~~~~~~~~~~d)}{graphics/eventSelection/pid/PidSelector_SqMassTofVsSqRootNSigma_kaon.pdf} &
    \subfigimg[width=\linewidth,page=1]{~~~~~~~~~~~f)}{graphics/eventSelection/pid/SqMassTofVsSqRootNSigma_kaon.pdf} \\[-10pt]
    \subfigimg[width=\linewidth,page=1]{~~~~~~~~~~~g)}{graphics/eventSelection/pid/PidSelector_SqMassTofVsSqRootNSigma_proton.pdf} &
    \subfigimg[width=\linewidth,page=1]{~~~~~~~~~~~i)}{graphics/eventSelection/pid/SqMassTofVsSqRootNSigma_proton.pdf}    
  \end{tabular}\vspace*{-5pt}
  \caption[$n\sigma^{\text{pair}}_{X}$ vs. $m^{2}_{\text{TOF}}$.]{Two-dimensional distributions of $n\sigma^{\text{pair}}_{\pi}$ (top row), $n\sigma^{\text{pair}}_{K}$ (middle row) and $n\sigma^{\text{pair}}_{p}$ (bottom row) vs. $m^{2}_{\text{TOF}}$. The left column contains all clean BBC-large events with single TOF vertex and two opposite sign TOF-matched tracks (passing cuts~\ref{enum:CutPrimVx}, \ref{enum:TpcTofMatched}, \ref{enum:TpcOppoSign} and~\ref{enum:CutBbcLarge}), which provides excellent statistics to see the signatures or pairs of specific ID. The right column cantains exclusive event candidates after full event selection except PID cuts (except cuts~\ref{enum:CutPid}). Dashed red line and arrow indicate the cut imposed on plotted quantities which are used to select exclusive pairs of given particle species (keep in mind that these are not the only cuts).}\label{fig:mSqVsNSigmaPair}
\end{figure}





\begin{figure}[ht!]
  \centering
  \begin{tabular}{@{}p{0.49\linewidth}@{\quad}p{0.49\linewidth}@{}}
    \subfigimg[width=\linewidth,page=1]{~~~~~~~~~~~~~~~~~~~~~~~~~~~~~~~~~~~~~~~~~~~~~~~~~~~~~~~~~~~~~~a)}{graphics/eventSelection/pid/Chi2NSigma_pion.pdf} &
    \subfigimg[width=\linewidth,page=1]{~~~~~~~~~~~~~~~~~~~~~~~~~~~~~~~~~~~~~~~~~~~~~~~~~~~~~~~~~~~~~~c)}{graphics/eventSelection/pid/SqMassTof_pion.pdf} \\
    \subfigimg[width=\linewidth,page=1]{~~~~~~~~~~~~~~~~~~~~~~~~~~~~~~~~~~~~~~~~~~~~~~~~~~~~~~~~~~~~~~d)}{graphics/eventSelection/pid/Chi2NSigma_kaon.pdf} &
    \subfigimg[width=\linewidth,page=1]{~~~~~~~~~~~~~~~~~~~~~~~~~~~~~~~~~~~~~~~~~~~~~~~~~~~~~~~~~~~~~~f)}{graphics/eventSelection/pid/SqMassTof_kaon.pdf} \\
    \subfigimg[width=\linewidth,page=1]{~~~~~~~~~~~~~~~~~~~~~~~~~~~~~~~~~~~~~~~~~~~~~~~~~~~~~~~~~~~~~~g)}{graphics/eventSelection/pid/Chi2NSigma_proton.pdf} &
    \subfigimg[width=\linewidth,page=1]{~~~~~~~~~~~~~~~~~~~~~~~~~~~~~~~~~~~~~~~~~~~~~~~~~~~~~~~~~~~~~~i)}{graphics/eventSelection/pid/SqMassTof_proton.pdf}    
  \end{tabular}
  \caption[$\chi^{2}_{dE/dx}$ and $m^{2}_{\text{TOF}}$ for exclusive $\pi^+\pi^-$, $K^+K^-$ and $p\bar{p}$ candidates.]{Raw distributions of $\chi^{2}_{dE/dx}$ (left column) and $m^{2}_{\text{TOF}}$ (right column) for exclusive $\pi^+\pi^-$ (top row), $K^+K^-$ (middle row) and $p\bar{p}$ (bottom row) candidates after full event selection. Data are shown as black points, while stacked predictions for signal and backgrounds are shown as color histograms. Dashed red line and arrow indicate the value of cut imposed on plotted quantity to select exclusive pairs of given particle species. Last bins in each subfigure are overflows representing an integral of the tail of distribution. Presented distributions were obtained after all the cuts were applied, except the cut on presented quantity in the last step in PID algorithm used to select pairs of given species. Non-exclusive background was determined with a method described in Sec.~\ref{sec:nonExclBkgdDetermination}, while predictions for exclusive contributions were obtained as described in Sec.~\ref{sec:exclBkgdDetermination}.}\label{fig:pid_plots}
\end{figure}

%---------------------------------------------------------------------------------------------------------------

\subsection{(\ref{enum:CutMissingPt})~Exclusivity cut (missing \texorpdfstring{$p_{\text{T}}$}{pT} cut)}\label{subsec:ptMiss}\label{sec:C9}%

The most important cut which is used in this analysis to select events of exclusively produced pairs of particles is the missing transverse momentum, or the total transverse momentum cut. It benefits from detection and reconstruction of the forward proton in RP detectors - a rare capability among high energy physics experiments which STAR provides. The observable $p_{\text{T}}^{\text{miss}}$ used to select exclusive event is defined as
\begin{equation}\label{eq:missingPt}
 p_{\text{T}}^{\text{miss}} = \Big( \vec{p}_{p'}^{\hspace*{2pt}\text{E}} + \vec{p}_{h^{+}} + \vec{p}_{h^{-}} + \vec{p}_{p'}^{\hspace*{2pt}\text{W}} \Big)_{\text{T}} = \sqrt{\Big(p_{x}^{\text{miss}}\Big)^{2} + \Big(p_{y}^{\text{miss}}\Big)^{2}},
\end{equation}
with the other total momentum components defined analogously:\\
\begin{tabulary}{\textwidth}{LL}
\begin{equation}\label{eq:missingPx}
 p_{x}^{\text{miss}} = \Big( \vec{p}_{p'}^{\hspace*{2pt}\text{E}} + \vec{p}_{h^{+}} + \vec{p}_{h^{-}} + \vec{p}_{p'}^{\hspace*{2pt}\text{W}} \Big)_{x},
\end{equation}~~~~~~~~~~~~~~~~~~~~~
&
\begin{equation}\label{eq:missingPy}
 p_{y}^{\text{miss}} = \Big( \vec{p}_{p'}^{\hspace*{2pt}\text{E}} + \vec{p}_{h^{+}} + \vec{p}_{h^{-}} + \vec{p}_{p'}^{\hspace*{2pt}\text{W}} \Big)_{y}.
\end{equation}~~~~~~~~~~~~~~~~~~~~~
\end{tabulary}


Figure~\ref{fig:PxPyCentralTrksVsProtons} visualize the (anti-)correlation between the momentum components of the forward system (sum of two forward protons momenta) and the central system (sum of two central tracks momenta). The enhanced band at anti-diagonal restricted by dashed lines contains events balanced in momentum, a signature of exclusivity. Events outside this band are the non exclusive backgrounds, in most cases Central Diffraction events with some particles undected (due to detector inefficiency or produced outside acceptance). Slight horizontal enhancement in all distributions around $[\vec{p}^{\hspace*{2pt}\text{W}}_{p'}+\vec{p}^{\hspace*{2pt}\text{E}}_{p'}]_{x} = [\vec{p}^{\hspace*{2pt}\text{W}}_{p'}+\vec{p}^{\hspace*{2pt}\text{E}}_{p'}]_{x} =0$ is a signature of the elastic proton-proton scattering background with some non-elastic pile-up interaction which mimics the CEP event. All these backgrounds are reasonably low after the exclusivity cut, as described in Sec.~\ref{sec:nonExclBkgd}.

The momentum balance is shown one-dimensionally in Fig.~\ref{fig:MissingPxPy}, with the sum of $x$- and $y$-components of momentum shown repectively in the left and right column for each analyzed particle species. The sum of signal and background (both assumed to be described a Gaussian) was fitted to $p_{x}^{\text{miss}}$ and $p_{y}^{\text{miss}}$ distributions. Results of the fit are given in each subfigure. One can notice that the widths of Gaussian functions representing the exclusive signal are consistent among species and amount $\sigma_{p_{x}^{\text{miss}}}=27.4$~MeV for the $x$-component of total momentum, and $\sigma_{p_{y}^{\text{miss}}}=28.1$~MeV for the $y$-component of total momentum, taking the values of the lowest statistical uncertainty - for $\pi^{+}\pi^{-}$. These values are measures of the total momentum resolution respectively for $p_{x}^{\text{miss}}$ and $p_{y}^{\text{miss}}$. Having these number it is possible to form an elliptical on the missing momentum:

\begin{equation}\label{eq:ptMissEllipse}%
\left(\frac{p_{x}^{\text{miss}}}{\sigma_{p_{x}^{\text{miss}}}}\right)^{2} + \left(\frac{p_{y}^{\text{miss}}}{\sigma_{p_{y}^{\text{miss}}}}\right)^{2} < n_{\text{cut}}^{2}
\end{equation}
%
where $n_{\text{cut}}$ is the parameter denoting radius of limiting ellipsis in units of standard deviations of distributions of total momentum components (resolutions). Since these resolutions are nearly identical ($\sigma_{p_{x}^{\text{miss}}} = \sigma_{p_{y}^{\text{miss}}} = \sigma_{p_{x,y}^{\text{miss}}}$) such cut can be reduced (multiplying Ineq.~\ref{eq:ptMissEllipse} by $\sigma_{p_{x,y}^{\text{miss}}}^{2}$) to one-dimensional cut on a single quantity:

\begin{equation}%
\left(p_{x}^{\text{miss}}\right)^{2} + \left(p_{y}^{\text{miss}}\right)^{2} < \Big(n_{\text{cut}}\cdot\sigma_{p_{x,y}^{\text{miss}}}\Big)^{2}~~~~~~~\xrightarrow[~]{\sqrt{~}}~~~~~~~p_{\text{T}}^{\text{miss}} < n_{\text{cut}}\cdot\sigma_{p_{x,y}^{\text{miss}}}~~~~
\end{equation}%
%
In current analysis the $n_{\text{cut}}$ was set to 2.5, which translates to threshold value $2.5\times 30~\text{MeV} = 75$~MeV. Such value was found optimal considering study described in Appendix~\ref{appendix:workingPoint}.

In Fig.~\ref{fig:MissingPt} the missing transverse momentum distributions are presented for the three studied CEP channels. In Sec.~\ref{sec:bkgdSignalNorm} a demonstration of various background contributions is given for $\pi^{+}\pi^{-}$, explaining all features of the distribution.



\begin{figure}[ht!]\vspace*{-20pt}
  \centering
  \begin{tabular}{@{}p{0.47\linewidth}@{\quad\quad}p{0.47\linewidth}@{}}
    \subfigimg[width=\linewidth,page=1]{~~~~~~~~~~~~~~~~a)}{graphics/eventSelection/exclusivity/PxCentralTrksVsProtons_pion.pdf} &
    \subfigimg[width=\linewidth,page=1]{~~~~~~~~~~~~~~~~~~~~~~c)}{graphics/eventSelection/exclusivity/PyCentralTrksVsProtons_pion.pdf} \\[-10pt]
    \subfigimg[width=\linewidth,page=1]{~~~~~~~~~~~~~~~~d)}{graphics/eventSelection/exclusivity/PxCentralTrksVsProtons_kaon.pdf} &
    \subfigimg[width=\linewidth,page=1]{~~~~~~~~~~~~~~~~~~~~~~f)}{graphics/eventSelection/exclusivity/PyCentralTrksVsProtons_kaon.pdf} \\[-10pt]
    \subfigimg[width=\linewidth,page=1]{~~~~~~~~~~~~~~~~g)}{graphics/eventSelection/exclusivity/PxCentralTrksVsProtons_proton.pdf} &
    \subfigimg[width=\linewidth,page=1]{~~~~~~~~~~~~~~~~~~~~~~i)}{graphics/eventSelection/exclusivity/PyCentralTrksVsProtons_proton.pdf}    
  \end{tabular}\vspace*{-5pt}
    \caption[Two-dimensional distributions of sum of forward protons momenta and sum of central tracks momenta for exclusive $\pi^+\pi^-$ (top row), $K^+K^-$ (middle row) and $p\bar{p}$ (bottom row) event candidates.]{Two-dimensional distributions of sum of forward protons momenta ($x$-axis) and sum of central tracks momenta ($y$-axis) for exclusive $\pi^+\pi^-$ (top row), $K^+K^-$ (middle row) and $p\bar{p}$ (bottom row) event candidates after full event selection, except the exclusivity cut~\ref{enum:CutMissingPt}. Left and right column shows correlation of respectively $x$- and $y$-component of tracks' momenta. Anti-diagonal representing perfect momentum balance of the central and forward system is limited with dashed lines extending by $\pm2.5\sigma$  ($\sigma\approx 30$~MeV) around the anti-diagonal. Three distinct horizontal regions in plots on the right hand side correspond to different forward proton configurations: elastic-like (protons in branches EU\&WD or ED\&WU, $\left|[\vec{p}^{\hspace*{2pt}\text{W}}_{p'}+\vec{p}^{\hspace*{2pt}\text{E}}_{p'}]_{y}\right| < 0.2$~GeV) and anti-elastic configuration (protons in branches ED\&WD or EU\&WU, $\left|[\vec{p}^{\hspace*{2pt}\text{W}}_{p'}+\vec{p}^{\hspace*{2pt}\text{E}}_{p'}]_{y}\right| > 0.4$~GeV).}\label{fig:PxPyCentralTrksVsProtons}
\end{figure}




\begin{figure}[ht!]
  \centering
  \begin{tabular}{@{}p{0.49\linewidth}@{\quad\quad}p{0.49\linewidth}@{}}
    \subfigimg[width=\linewidth,page=1]{~~~~~~~~~~~~~~~~~~~~~~~~~~~~~~~~~~~~~~~~~~~~~~~~~~~~~~~~~~~~~a)}{graphics/eventSelection/exclusivity/MissingPx_pion.pdf} &
    \subfigimg[width=\linewidth,page=1]{~~~~~~~~~~~~~~~~~~~~~~~~~~~~~~~~~~~~~~~~~~~~~~~~~~~~~~~~~~~~~b)}{graphics/eventSelection/exclusivity/MissingPy_pion.pdf} \\
    \subfigimg[width=\linewidth,page=1]{~~~~~~~~~~~~~~~~~~~~~~~~~~~~~~~~~~~~~~~~~~~~~~~~~~~~~~~~~~~~~c)}{graphics/eventSelection/exclusivity/MissingPx_kaon.pdf} &
    \subfigimg[width=\linewidth,page=1]{~~~~~~~~~~~~~~~~~~~~~~~~~~~~~~~~~~~~~~~~~~~~~~~~~~~~~~~~~~~~~d)}{graphics/eventSelection/exclusivity/MissingPy_kaon.pdf} \\
    \subfigimg[width=\linewidth,page=1]{~~~~~~~~~~~~~~~~~~~~~~~~~~~~~~~~~~~~~~~~~~~~~~~~~~~~~~~~~~~~~e)}{graphics/eventSelection/exclusivity/MissingPx_proton.pdf} &
    \subfigimg[width=\linewidth,page=1]{~~~~~~~~~~~~~~~~~~~~~~~~~~~~~~~~~~~~~~~~~~~~~~~~~~~~~~~~~~~~~f)}{graphics/eventSelection/exclusivity/MissingPy_proton.pdf}    
  \end{tabular}\vspace*{-5pt}
    \caption[Raw distributions of $p_{x}^{\text{miss}}$ and $p_{y}^{\text{miss}}$ for exclusive $\pi^+\pi^-$, $K^+K^-$ and $p\bar{p}$candidates.]{%
    Raw distributions of $p_{x}^{\text{miss}}$ (left column) and $p_{y}^{\text{miss}}$ (right column) for exclusive $\pi^+\pi^-$ (top row), $K^+K^-$ (middle row) and $p\bar{p}$ (bottom row) candidates after full event selection, except exclusivity cut~\ref{enum:CutMissingPt}. Solid red line represents the fit of sum of two Gaussian functions representing the exclusive event signal (orange) and non-exclusive background (violet). Parameters of the total momentum resolution for signal events obtained from the fit (given in the plots) roughly agree between all species.
    }\label{fig:MissingPxPy}
\end{figure}



\begin{figure}[h]
\centering
\parbox{0.4725\textwidth}{
  \centering
  \begin{subfigure}[b]{\linewidth}
                \subcaptionbox{\label{fig:MissingPt_pion}}{\includegraphics[width=1.05\linewidth,page=1]{graphics/eventSelection/exclusivity/Paper_MissingPt_pion.pdf}}
  \end{subfigure}\\
  \begin{subfigure}[b]{\linewidth}\addtocounter{subfigure}{1}
                \subcaptionbox{\label{fig:MissingPt_kaon}}{\includegraphics[width=1.05\linewidth,page=1]{graphics/eventSelection/exclusivity/Paper_MissingPt_proton.pdf}}
  \end{subfigure} 
}%
\quad\quad%
\parbox{0.4725\textwidth}{
  \centering
  \begin{subfigure}[b]{\linewidth}\addtocounter{subfigure}{-2}\vspace*{17pt}
                \subcaptionbox{\label{fig:MissingPt_proton}}{\includegraphics[width=1.05\linewidth,page=1]{graphics/eventSelection/exclusivity/Paper_MissingPt_kaon.pdf}}
  \end{subfigure}\\
  \begin{minipage}[t][1.042\linewidth][t]{\linewidth}\vspace{20pt}
    \caption[Uncorrected distributions of the CEP event candidates for missing transverse momentum $p_\mathrm{T}^\mathrm{\scriptscriptstyle miss}$ for $\pi^+\pi^-$ (top), $K^+K^-$ (middle) and $p\bar{p}$ (bottom) pairs.]{Uncorrected distributions of the CEP event candidates for missing transverse momentum $p_\mathrm{T}^\mathrm{\scriptscriptstyle miss}$ for $\pi^+\pi^-$ (top), $K^+K^-$ (middle) and $p\bar{p}$ (bottom) pairs. Distributions for opposite-sign and same-sign particle pairs are shown as black and red symbols, respectively. Estimated non-exclusive background contribution in the vicinity of signal region (low $p_{T}^{\text{miss}}$) has been drawn with magenta color. The vertical error bars represent statistical uncertainties. The horizontal bars represent bin sizes. Distribution for $\pi^+\pi^-$ channel with MC predictions for both signal and background can be found in Fig.~\ref{fig:Ratio_MissingPt_OppositeAndSameSign}.}\label{fig:MissingPt}
  \end{minipage}
}%
\end{figure}
%---------------------------

 



% \section{Signal per integrated luminosity}
% 
% \section{Cut flow}\label{sec:cutFlow}
% 
% %---------------------------
% \begin{figure}[ht!]
% \centering%
% \includegraphics[width=0.85\linewidth,page=1]{graphics/eventSelection/CutFlow.pdf}%
% \caption{Cut flow.}\label{fig:CutFlow}%
% \end{figure}
% %---------------------------
