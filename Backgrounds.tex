%%===========================================================%%
%%                                                           %%
%%                        BACKGROUNDS                        %%
%%                                                           %%
%%===========================================================%%

\chapter{Backgrounds}\label{chap:backgrounds}

\section{Sources of background}
\subsection{Non-exclusive background}\label{sec:nonExclBkgd}

The main background present in the final exclusive $\pi^{+}\pi^{-}/K^{+}K^{-}/p\bar{p}$ sample is the non-exclusive background. There are several classes of events which mimic topology of $h^{+}h^{-}$ CEP: two forward protons, two opposite sign central tracks and rapidity gaps. Below we list the most probable cases:
  \begin{itemize}
  \item Single physics processes:
  \begin{itemize}
  \item Central Diffraction (Fig.~\ref{fig:bkgdSources_cd}) - this process differs from CEP of $h^{+}h^{-}$ only by the number of particles produced in the mid-rapidity; protons originate from the same vertex as the central tracks, hence correlation of reconstructed vertex position from RPs and TPC is still observed.
  \end{itemize}
  \item Coincidences (pile-up):
  \begin{itemize}
  \item inelastic + elastic interaction (Fig.~\ref{fig:bkgdSources_mb_el}) - there may be overlap of protons from elastic scattering interaction and activity in the central detector from another (inelastic) interaction; it should be supressed by the rapidity gap veto in BBC-small (online) and BBC-large (offline); easy to identify through protons collinearity and lack of correlation of $z$-vertex from RPs and TPC.
  \end{itemize}\vspace*{-17pt}
  
  
  \hspace*{-7pt}$\left.
\begin{tabular}{p{.9\textwidth}}
  \begin{itemize}
  \item Single Diffraction + beam halo - there may be overlap of proton from SD on one side and beam halo proton on the opposite side, and activity in the central detector from diffractive state; it should be supressed by the rapidity gap vetos and (low) beam halo rate;
  \item  2$\times$beam halo + inelastic interaction.
  \end{itemize}
  \end{tabular}
\right\}_{\rotatebox{90}{~~~~negligibly low\hspace*{-25pt}}}$
  
 \end{itemize}

 These backgrounds are graphically presented in Fig.~\ref{fig:bkgdSources}.

%---------------------------
\begin{figure}[h]
\centering%
\parbox{0.315\textwidth}{%
  \centering%
  \begin{subfigure}[b]{0.9\linewidth}{
                \subcaptionbox{\label{fig:bkgdSources_cep}}{\includegraphics[width=\linewidth]{graphics/backgrounds/cep.pdf}}}
  \end{subfigure}
}%
\quad%
\parbox{0.315\textwidth}{%
  \centering%
  \begin{subfigure}[b]{0.9\linewidth}{
                \subcaptionbox{\label{fig:bkgdSources_cd}}{\includegraphics[width=\linewidth]{graphics/backgrounds/cd.pdf}}}
  \end{subfigure}
}%
\quad%
\parbox{0.315\textwidth}{%
  \centering%
  \begin{subfigure}[b]{0.9\linewidth}{
                \subcaptionbox{\label{fig:bkgdSources_mb_el}}{\includegraphics[width=\linewidth]{graphics/backgrounds/mb_el.pdf}}}
  \end{subfigure} 
}%
\caption[Sketches of main processes with CEP event topology.]{Sketches of main processes exhibiting $h^{+}h^{-}$ CEP event topology: the exclusive $h^{+}h^{-}$ signal itself (\subref{fig:bkgdSources_cep}), central diffraction event with some particles not detected (\subref{fig:bkgdSources_cd}) and elastic proton-proton scattering event with pile-up inelastic interaction in the central region (\subref{fig:bkgdSources_mb_el}). Particles represented by arrows are: forward scattered protons (blue), detected mid-rapidity particles (green) and undetected particles (dashed gray). Black dots mark primary interaction vertices.}\label{fig:bkgdSources}
\end{figure}
%---------------------------
 
 

% %---------------------------
% \begin{figure}[h]
% \centering%
% \includegraphics[width=0.65\linewidth,page=1]{graphics/backgrounds/Raw_MissingPtPid.pdf}%
% \caption{Missing pT.}\label{fig:missingPtBkgd}%
% \end{figure}
% %---------------------------
\newpage
\subsection{Exclusive background (particle misidentification)}\label{sec:exclBkgd}

Another source of background which is connected with finite particle identification power is the exclusive background from the particle species other than species under study.

%---------------------------
\begin{figure}[h]
\centering%
\parbox{0.4725\textwidth}{%
  \centering%
  \includegraphics[width=\linewidth]{graphics/backgrounds/pid-crop2.pdf}\label{fig:misidentificationGraph}
}%
\quad%
\parbox{0.4725\textwidth}{%
    \caption[Graph illustrating the misidentification problem.]{Graph illustrating the misidentification problem - the origin of exclusive background in selected samples. Gray arrows represent event rejection due to failed PID selection (\ref{enum:CutPid}). Magenta arrows indicate non-exclusive backgrounds described in Sec.~\ref{sec:nonExclBkgd}. Solid black arrows represent successful identification, whereas dashed black lines show misidentification paths.}
}%

\end{figure}
%---------------------------


\begin{subequations}\label{eq:misidentificationEqs}
\begin{equation}
  N^{\pi\pi}_{R}~~=~~\begingroup\color{gray}\underbrace{\color{black}\epsilon^{\pi\pi}\cdot N^{\pi\pi}_{T}}_{\textrm{true pion pairs}}\endgroup~~ + ~~\begingroup\color{gray}\underbrace{\color{black}\lambda^{ KK\rightarrow \pi\pi}\cdot N^{KK}_{T}}_{\substack{\textrm{kaon pairs reconstructed} \\ \textrm{as pion pairs}}}\endgroup~~ + ~~\begingroup\color{gray}\underbrace{\color{black}\lambda^{p\bar{p} \rightarrow \pi\pi} \cdot N^{p\bar{p}}_{T}}_{\substack{\textrm{proton pairs reconstructed} \\ \textrm{as pion pairs}}}\endgroup~~ + ~~\textcolor{magenta}{N^{\pi\pi}_{bkgd}}
\end{equation}    
\begin{equation}
  N^{KK}_{R} ~= ~~\begingroup\color{gray}\underbrace{\color{black}\lambda^{ \pi\pi\rightarrow KK}\cdot N^{\pi\pi}_{T}}_{\substack{\textrm{pion pairs reconstructed} \\ \textrm{as kaon pairs}}}\endgroup~~ + ~~\begingroup\color{gray}\underbrace{\color{black}\epsilon^{KK}\cdot N^{KK}_{T}}_{\textrm{true kaon pairs}}\endgroup~~ + ~~\begingroup\color{gray}\underbrace{\color{black}\lambda^{p\bar{p} \rightarrow KK} \cdot N^{p\bar{p}}_{T}}_{\substack{\textrm{proton pairs reconstructed} \\ \textrm{as kaon pairs}}}\endgroup~~ + ~~\textcolor{magenta}{N^{KK}_{bkgd}}
\end{equation}
\begin{equation}\hspace*{-25pt}
  N^{p\bar{p}}_{R}~~~= ~~\begingroup\color{gray}\underbrace{\color{black}\lambda^{\pi\pi \rightarrow p\bar{p}} \cdot N^{\pi\pi}_{T}}_{\substack{\textrm{pion pairs reconstructed} \\ \textrm{as proton pairs}}}\endgroup~~ + ~~\begingroup\color{gray}\underbrace{\color{black}\lambda^{ KK\rightarrow p\bar{p}}\cdot N^{KK}_{T}}_{\substack{\textrm{kaon pairs reconstructed} \\ \textrm{as proton pairs}}}\endgroup~~ + ~~\begingroup\color{gray}\underbrace{\color{black}\epsilon^{p\bar{p}}\cdot N^{p\bar{p}}_{T}}_{\textrm{true proton pairs}}\endgroup~~ + ~~\textcolor{magenta}{N^{p\bar{p}}_{bkgd}}
\end{equation}
\end{subequations}

Eqs.~\eqref{eq:misidentificationEqs} can be written in the matrix form, as shown in Eq.~\eqref{eq:misidentificationMatrix}, from which it is straightforward to obtain final formula for unfolded number of events of given ID, Eq.~\eqref{eq:pidUnfoldingMatrix}:

\begin{tabulary}{\textwidth}{LCL}
\begin{equation}\label{eq:misidentificationMatrix}\hspace*{-15pt}
\Spvek{N^{\pi\pi}_{R}-\textcolor{magenta}{N^{\pi\pi}_{bkgd}};~;N^{KK}_{R}-\textcolor{magenta}{N^{KK}_{bkgd}};~;N^{p\bar{p}}_{R}-\textcolor{magenta}{N^{p\bar{p}}_{bkgd}}} =  \underbrace{\left[ \begin{array}{ccc}
\epsilon^{\pi\pi} & \lambda^{ KK\rightarrow \pi\pi} & \lambda^{p\bar{p} \rightarrow \pi\pi} \\
~ & ~ & ~\\
\lambda^{\pi\pi\rightarrow KK} & \epsilon^{KK} & \lambda^{ p\bar{p} \rightarrow KK}\\
~ & ~ & ~\\
\lambda^{\pi\pi\rightarrow p\bar{p}} & \lambda^{ KK\rightarrow p\bar{p}} & \epsilon^{p\bar{p}}
\end{array} \right]}_{\text{``mixing matrix''}~\Lambda}\Spvek{N^{\pi\pi}_{T};~;N^{KK}_{T};~;N^{p\bar{p}}_{T}}
\end{equation}&%
\vspace{40pt}$\rightarrow$\hspace{20pt}&
\begin{equation}\label{eq:pidUnfoldingMatrix}
\Spvek{N^{\pi\pi}_{T};~;N^{KK}_{T};~;N^{p\bar{p}}_{T}} = \Lambda^{-1}\Spvek{N^{\pi\pi}_{R}-\textcolor{magenta}{N^{\pi\pi}_{bkgd}};~;N^{KK}_{R}-\textcolor{magenta}{N^{KK}_{bkgd}};~;N^{p\bar{p}}_{R}-\textcolor{magenta}{N^{p\bar{p}}_{bkgd}}}
\end{equation}
\end{tabulary}

\section{Non-exclusive background determination}

Determination of non-exclusive background makes use of the missing transverse momentum which is distributed differently for the signal and for aforementioned background. High statistics of the data allows to apply a data-driven method. What is more, not only integrated over observables but even differentially as a function of these.

It has already been demonstrated in Sec.~\ref{subsec:ptMiss} that $p_{\text{T}}^{\text{miss}}$ from exclusive events is much narrower compared to background, visible as a peak near the axis origin. Additional feature of 
$p_{\text{T}}^{\text{miss}}$ - probability density approaching to zero with $p_{\text{T}}^{\text{miss}}$ moving towards the axis origin, enables performing a polynomial fit to $p_{\text{T}}^{\text{miss}}$ distribution in the background-dominated range and extrapolation of this polynomial down to  $p_{\text{T}}^{\text{miss}} = 0$ under the peak of CEP signal. The procedure used to determine non-exclusive background content in the signal region is described below.



\section{Normalization of signal and background models}\label{sec:bkgdSignalNorm}


\begin{figure}[h]
\centering
\parbox{0.4725\textwidth}{
  \centering
  \begin{subfigure}[b]{\linewidth}
                \subcaptionbox{\label{fig:Ratio_DeltaZVtx_DeltaPhiBin_0}}{\includegraphics[width=1.05\linewidth,page=1]{graphics/backgrounds/dataVsMc/Ratio_DeltaZVtx_DeltaPhiBin_0.pdf}}
  \end{subfigure}\\
  \begin{subfigure}[b]{\linewidth}\addtocounter{subfigure}{1}
                \subcaptionbox{\label{fig:Ratio_Linear_DeltaZVtx_DeltaPhiBin_0}}{\includegraphics[width=1.05\linewidth,page=1]{graphics/backgrounds/dataVsMc/Ratio_Linear_DeltaZVtx_DeltaPhiBin_0.pdf}}
  \end{subfigure}
}%
\quad\quad%
\parbox{0.4725\textwidth}{
  \centering
  \begin{subfigure}[b]{\linewidth}
                \subcaptionbox{\label{fig:Ratio_DeltaZVtx_DeltaPhiBin_1}}{\includegraphics[width=1.05\linewidth,page=1]{graphics/backgrounds/dataVsMc/Ratio_DeltaZVtx_DeltaPhiBin_1.pdf}}
  \end{subfigure}\\
  \begin{subfigure}[b]{\linewidth}\addtocounter{subfigure}{1}
                \subcaptionbox{\label{fig:Ratio_Linear_DeltaZVtx_DeltaPhiBin_1}}{\includegraphics[width=1.05\linewidth,page=1]{graphics/backgrounds/dataVsMc/Ratio_Linear_DeltaZVtx_DeltaPhiBin_1.pdf}}
  \end{subfigure} 
}\caption[Comparison of $\Delta z_{\text{vtx}}$ for CEP $\pi^{+}\pi^{-}$ events in two ranges of $\Delta\varphi$ between data and embedded MC.]{Comparison of $\Delta z_{\text{vtx}}$ for CEP $\pi^{+}\pi^{-}$ events in two ranges of $\Delta\varphi$ (left: $\Delta\varphi<90^{\circ}$, right: $\Delta\varphi>90^{\circ}$) between data and embedded MC. Plots in top and bottom row differ only in the $y$-axis (top: logarithmic, bottom: linear). Data are represented by black points, while stacked MC predictions are drawn as histograms of different colors. Histogram from each MC process has been normalized according to prescription in the text. Vertical error bars represent statistical uncertainties, horizontal bars represent bin sizes.}\label{fig:Ratio_DeltaZVtx_DeltaPhiBins}%
\end{figure}
%--------------------------- 


\begin{figure}[h]
\centering
\parbox{0.4725\textwidth}{
  \centering
  \begin{subfigure}[b]{\linewidth}
                \subcaptionbox{\label{fig:Ratio_DeltaZVtx}}{\includegraphics[width=1.05\linewidth,page=1]{graphics/backgrounds/dataVsMc/Ratio_DeltaZVtx.pdf}}
  \end{subfigure}
}%
\quad\quad%
\parbox{0.4725\textwidth}{
  \centering
  \begin{subfigure}[b]{\linewidth}
                \subcaptionbox{\label{fig:Ratio_Linear_DeltaZVtx}}{\includegraphics[width=1.05\linewidth,page=1]{graphics/backgrounds/dataVsMc/Ratio_Linear_DeltaZVtx.pdf}}
  \end{subfigure}
}\caption[Comparison of $\Delta z_{\text{vtx}}$ for CEP $\pi^{+}\pi^{-}$ events between data and embedded MC.]{Comparison of $\Delta z_{\text{vtx}}$ for CEP $\pi^{+}\pi^{-}$ events between data and embedded MC. Left and right plot differ only in the $y$-axis (left: logarithmic, right: linear). Data are represented by black points, while stacked MC predictions are drawn as histograms of different colors. Histogram from each MC process has been normalized according to prescription in the text. Vertical error bars represent statistical uncertainties, horizontal bars represent bin sizes.}\label{fig:Ratio_DeltaZVtx}%
\end{figure}
%---------------------------







%---------------------------
\begin{figure}[ht!]
\centering%
\parbox{0.4725\textwidth}{%
  \centering%
  \includegraphics[width=\linewidth]{graphics/backgrounds/dataVsMc/Ratio_Collinearity.pdf}
}%
\quad%
\parbox{0.4725\textwidth}{%
    \caption[Comparison of collinearity $\Delta\theta$ for CEP $\pi^{+}\pi^{-}$ events with $\Delta\varphi>90^{\circ}$, between data and embedded MC.]{Comparison of coliinearity $\Delta\theta$ for CEP $\pi^{+}\pi^{-}$ events with $\Delta\varphi>90^{\circ}$ between data and embedded MC. Data are represented by black points, while stacked MC predictions are drawn as histograms of different colors. Histogram from each MC process has been normalized according to prescription in the text. Vertical error bars represent statistical uncertainties, horizontal bars represent bin sizes.}\label{fig:Ratio_Collinearity}%  
}
\end{figure}
%---------------------------






\begin{figure}[h]
\centering
\parbox{0.4725\textwidth}{
  \centering
  \begin{subfigure}[b]{\linewidth}
                \subcaptionbox{\label{fig:Ratio_MissingPt_OppositeAndSameSign_DeltaPhiBin_0}}{\includegraphics[width=1.05\linewidth,page=1]{graphics/backgrounds/dataVsMc/Ratio_MissingPt_OppositeAndSameSign_DeltaPhiBin_0.pdf}}
  \end{subfigure}\\
  \begin{subfigure}[b]{\linewidth}\addtocounter{subfigure}{1}
                \subcaptionbox{\label{fig:Ratio_LogY_MissingPt_OppositeAndSameSign_DeltaPhiBin_0}}{\includegraphics[width=1.05\linewidth,page=1]{graphics/backgrounds/dataVsMc/Ratio_LogY_MissingPt_OppositeAndSameSign_DeltaPhiBin_0.pdf}}
  \end{subfigure} 
}%
\quad\quad%
\parbox{0.4725\textwidth}{
  \centering
  \begin{subfigure}[b]{\linewidth}
                \subcaptionbox{\label{fig:Ratio_MissingPt_OppositeAndSameSign_DeltaPhiBin_1}}{\includegraphics[width=1.05\linewidth,page=1]{graphics/backgrounds/dataVsMc/Ratio_MissingPt_OppositeAndSameSign_DeltaPhiBin_1.pdf}}
  \end{subfigure}\\
  \begin{subfigure}[b]{\linewidth}\addtocounter{subfigure}{1}
                \subcaptionbox{\label{fig:Ratio_LogY_MissingPt_OppositeAndSameSign_DeltaPhiBin_1}}{\includegraphics[width=1.05\linewidth,page=1]{graphics/backgrounds/dataVsMc/Ratio_LogY_MissingPt_OppositeAndSameSign_DeltaPhiBin_1.pdf}}
  \end{subfigure} 
}\caption[Comparison of $p_{T}^{\text{miss}}$ for CEP $\pi^{+}\pi^{-}$ events in two ranges of $\Delta\varphi$ between data and embedded MC.]{Comparison of $p_{T}^{\text{miss}}$ for CEP $\pi^{+}\pi^{-}$ events in two ranges of $\Delta\varphi$ (left: $\Delta\varphi<90^{\circ}$, right: $\Delta\varphi>90^{\circ}$) between data and embedded MC. Plots in top and bottom row differ only in the $y$-axis (top: linear, bottom: logarithmic). In addition to signal channel (opposite-sign particles) also control background channel (same-sign particles) is contained in the plots. Data are represented by black (opposite-sign) or red (same-sign) points, while stacked MC predictions are drawn as filled (opposite-sign) or hatched (same-sign) histograms of different colors. Histogram from each MC process has been normalized according to prescription in the text. Vertical error bars represent statistical uncertainties, horizontal bars represent bin sizes.}\label{fig:Ratio_MissingPt_OppositeAndSameSign_DeltaPhiBins}%
\end{figure}
%--------------------------- 



\begin{figure}[h]
\centering
\parbox{0.4725\textwidth}{
  \centering
  \begin{subfigure}[b]{\linewidth}
                \subcaptionbox{\label{fig:Ratio_MissingPt_OppositeAndSameSign}}{\includegraphics[width=1.05\linewidth,page=1]{graphics/backgrounds/dataVsMc/Ratio_MissingPt_OppositeAndSameSign.pdf}}
  \end{subfigure}\\
  \begin{subfigure}[b]{\linewidth}\addtocounter{subfigure}{1}
                \subcaptionbox{\label{fig:Ratio_MissingPt_OppositeAndSameSign_NeutralsNotSubtracted}}{\includegraphics[width=1.05\linewidth,page=1]{graphics/backgrounds/dataVsMc/Ratio_MissingPt_OppositeAndSameSign_NeutralsNotSubtracted.pdf}}
  \end{subfigure} 
}%
\quad\quad%
\parbox{0.4725\textwidth}{
  \centering
  \begin{subfigure}[b]{\linewidth}\vspace*{-45pt}
                \subcaptionbox{\label{fig:Ratio_LogY_MissingPt_OppositeAndSameSign}}{\includegraphics[width=1.05\linewidth,page=1]{graphics/backgrounds/dataVsMc/Ratio_LogY_MissingPt_OppositeAndSameSign.pdf}}
  \end{subfigure}\\
    \begin{minipage}[t][1.042\linewidth][t]{\linewidth}\vspace{20pt}
    \caption[Comparison of $p_{T}^{\text{miss}}$ for CEP $\pi^{+}\pi^{-}$ events between data and embedded MC.]{Comparison of $p_{T}^{\text{miss}}$ for CEP $\pi^{+}\pi^{-}$ events between data and embedded MC. Top left and top right plot differ only in the $y$-axis (left: linear, right: logarithmic). In addition to signal channel (opposite-sign particles) also control background channel (same-sign particles) is contained in the plots. Data are represented by black (opposite-sign) or red (same-sign) points, while stacked MC predictions are drawn as filled (opposite-sign) or hatched (same-sign) histograms of different colors. Histogram from each MC process has been normalized according to prescription in the text. Vertical error bars represent statistical uncertainties, horizontal bars represent bin sizes.\\Left bottom plot differ from the top plots in the content of Pythia MCs - in the bottom plot events with $\pi^{\pm}\pi^{\mp}$+neutrals in the central state were preserved, demonstrating significant inconsitency between data and MC in the ratio of opposite-sign to same-sign events if such events are not rejected.}\label{fig:Ratio_MissingPt_OppositeAndSameSign}%
  \end{minipage}
}%
\end{figure}
%--------------------------- 

