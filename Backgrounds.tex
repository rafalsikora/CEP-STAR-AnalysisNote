%%===========================================================%%
%%                                                           %%
%%                        BACKGROUNDS                        %%
%%                                                           %%
%%===========================================================%%

\chapter{Backgrounds}\label{chap:backgrounds}

\section{Sources of background}
\subsection{Non-exclusive background}\label{sec:nonExclBkgd}

The main background present in the final exclusive $\pi^{+}\pi^{-}/K^{+}K^{-}/p\bar{p}$ sample is the non-exclusive background. There are several classes of events which mimic topology of $h^{+}h^{-}$ CEP: two forward protons, two opposite sign central tracks and rapidity gaps. Below we list the most probable cases:
  \begin{itemize}
  \item Single physics processes:
  \begin{itemize}
  \item Central Diffraction (Fig.~\ref{fig:bkgdSources_cd}) - this process differs from CEP of $h^{+}h^{-}$ only by the number of particles produced in the mid-rapidity; protons originate from the same vertex as the central tracks, hence correlation of reconstructed vertex position from RPs and TPC is still observed.
  \end{itemize}
  \item Coincidences (pile-up):
  \begin{itemize}
  \item inelastic + elastic interaction (Fig.~\ref{fig:bkgdSources_mb_el}) - there may be overlap of protons from elastic scattering interaction and activity in the central detector from another (inelastic) interaction; it should be supressed by the rapidity gap veto in BBC-small (online) and BBC-large (offline); easy to identify through protons collinearity and lack of correlation of $z$-vertex from RPs and TPC.
  \end{itemize}\vspace*{-17pt}
  
  
  \hspace*{-7pt}$\left.
\begin{tabular}{p{.9\textwidth}}
  \begin{itemize}
  \item Single Diffraction + beam halo - there may be overlap of proton from SD on one side and beam halo proton on the opposite side, and activity in the central detector from diffractive state; it should be supressed by the rapidity gap vetos and (low) beam halo rate;
  \item  2$\times$beam halo + inelastic interaction.
  \end{itemize}
  \end{tabular}
\right\}_{\rotatebox{90}{~~~~negligibly low\hspace*{-25pt}}}$
  
 \end{itemize}

 These backgrounds are graphically presented in Fig.~\ref{fig:bkgdSources}.

%---------------------------
\begin{figure}[h]
\centering%
\parbox{0.315\textwidth}{%
  \centering%
  \begin{subfigure}[b]{0.9\linewidth}{
                \subcaptionbox{\label{fig:bkgdSources_cep}}{\includegraphics[width=\linewidth]{graphics/backgrounds/cep.pdf}}}
  \end{subfigure}
}%
\quad%
\parbox{0.315\textwidth}{%
  \centering%
  \begin{subfigure}[b]{0.9\linewidth}{
                \subcaptionbox{\label{fig:bkgdSources_cd}}{\includegraphics[width=\linewidth]{graphics/backgrounds/cd.pdf}}}
  \end{subfigure}
}%
\quad%
\parbox{0.315\textwidth}{%
  \centering%
  \begin{subfigure}[b]{0.9\linewidth}{
                \subcaptionbox{\label{fig:bkgdSources_mb_el}}{\includegraphics[width=\linewidth]{graphics/backgrounds/mb_el.pdf}}}
  \end{subfigure} 
}%
\caption[Sketches of main processes with CEP event topology.]{Sketches of main processes exhibiting $h^{+}h^{-}$ CEP event topology: the exclusive $h^{+}h^{-}$ signal itself (\subref{fig:bkgdSources_cep}), central diffraction event with some particles not detected (\subref{fig:bkgdSources_cd}) and elastic proton-proton scattering event with pile-up inelastic interaction in the central region (\subref{fig:bkgdSources_mb_el}). Particles represented by arrows are: forward scattered protons (blue), detected mid-rapidity particles (green) and undetected particles (dashed gray). Black dots mark primary interaction vertices.}\label{fig:bkgdSources}
\end{figure}
%---------------------------
 
 

% %---------------------------
% \begin{figure}[h]
% \centering%
% \includegraphics[width=0.65\linewidth,page=1]{graphics/backgrounds/Raw_MissingPtPid.pdf}%
% \caption{Missing pT.}\label{fig:missingPtBkgd}%
% \end{figure}
% %---------------------------
\newpage
\subsection{Exclusive background (particle misidentification)}\label{sec:exclBkgd}

Another source of background which is connected with finite particle identification power is the exclusive background from the particle species other than species under study.

%---------------------------
\begin{figure}[h]
\centering%
\parbox{0.4725\textwidth}{%
  \centering%
  \includegraphics[width=\linewidth]{graphics/backgrounds/pid-crop2.pdf}\label{fig:misidentificationGraph}
}%
\quad%
\parbox{0.4725\textwidth}{%
    \caption[Graph illustrating the misidentification problem.]{Graph illustrating the misidentification problem - the origin of exclusive background in selected samples. Gray arrows represent event rejection due to failed PID selection (\ref{enum:CutPid}). Magenta arrows indicate non-exclusive backgrounds described in Sec.~\ref{sec:nonExclBkgd}. Solid black arrows represent successful identification, whereas dashed black lines show misidentification paths.}
}%

\end{figure}
%---------------------------


\begin{subequations}\label{eq:misidentificationEqs}
\begin{equation}
  N^{\pi\pi}_{R}~~=~~\begingroup\color{gray}\underbrace{\color{black}\epsilon^{\pi\pi}\cdot N^{\pi\pi}_{T}}_{\textrm{true pion pairs}}\endgroup~~ + ~~\begingroup\color{gray}\underbrace{\color{black}\lambda^{ KK\rightarrow \pi\pi}\cdot N^{KK}_{T}}_{\substack{\textrm{kaon pairs reconstructed} \\ \textrm{as pion pairs}}}\endgroup~~ + ~~\begingroup\color{gray}\underbrace{\color{black}\lambda^{p\bar{p} \rightarrow \pi\pi} \cdot N^{p\bar{p}}_{T}}_{\substack{\textrm{proton pairs reconstructed} \\ \textrm{as pion pairs}}}\endgroup~~ + ~~\textcolor{magenta}{N^{\pi\pi}_{bkgd}}
\end{equation}    
\begin{equation}
  N^{KK}_{R} ~= ~~\begingroup\color{gray}\underbrace{\color{black}\lambda^{ \pi\pi\rightarrow KK}\cdot N^{\pi\pi}_{T}}_{\substack{\textrm{pion pairs reconstructed} \\ \textrm{as kaon pairs}}}\endgroup~~ + ~~\begingroup\color{gray}\underbrace{\color{black}\epsilon^{KK}\cdot N^{KK}_{T}}_{\textrm{true kaon pairs}}\endgroup~~ + ~~\begingroup\color{gray}\underbrace{\color{black}\lambda^{p\bar{p} \rightarrow KK} \cdot N^{p\bar{p}}_{T}}_{\substack{\textrm{proton pairs reconstructed} \\ \textrm{as kaon pairs}}}\endgroup~~ + ~~\textcolor{magenta}{N^{KK}_{bkgd}}
\end{equation}
\begin{equation}\hspace*{-25pt}
  N^{p\bar{p}}_{R}~~~= ~~\begingroup\color{gray}\underbrace{\color{black}\lambda^{\pi\pi \rightarrow p\bar{p}} \cdot N^{\pi\pi}_{T}}_{\substack{\textrm{pion pairs reconstructed} \\ \textrm{as proton pairs}}}\endgroup~~ + ~~\begingroup\color{gray}\underbrace{\color{black}\lambda^{ KK\rightarrow p\bar{p}}\cdot N^{KK}_{T}}_{\substack{\textrm{kaon pairs reconstructed} \\ \textrm{as proton pairs}}}\endgroup~~ + ~~\begingroup\color{gray}\underbrace{\color{black}\epsilon^{p\bar{p}}\cdot N^{p\bar{p}}_{T}}_{\textrm{true proton pairs}}\endgroup~~ + ~~\textcolor{magenta}{N^{p\bar{p}}_{bkgd}}
\end{equation}
\end{subequations}

Eqs.~\eqref{eq:misidentificationEqs} can be written in the matrix form, as shown in Eq.~\eqref{eq:misidentificationMatrix}, from which it is straightforward to obtain final formula for unfolded number of events of given ID, Eq.~\eqref{eq:pidUnfoldingMatrix}:

\begin{tabulary}{\textwidth}{LCL}
\begin{equation}\label{eq:misidentificationMatrix}\hspace*{-15pt}
\Spvek{N^{\pi\pi}_{R}-\textcolor{magenta}{N^{\pi\pi}_{bkgd}};~;N^{KK}_{R}-\textcolor{magenta}{N^{KK}_{bkgd}};~;N^{p\bar{p}}_{R}-\textcolor{magenta}{N^{p\bar{p}}_{bkgd}}} =  \underbrace{\left[ \begin{array}{ccc}
\epsilon^{\pi\pi} & \lambda^{ KK\rightarrow \pi\pi} & \lambda^{p\bar{p} \rightarrow \pi\pi} \\
~ & ~ & ~\\
\lambda^{\pi\pi\rightarrow KK} & \epsilon^{KK} & \lambda^{ p\bar{p} \rightarrow KK}\\
~ & ~ & ~\\
\lambda^{\pi\pi\rightarrow p\bar{p}} & \lambda^{ KK\rightarrow p\bar{p}} & \epsilon^{p\bar{p}}
\end{array} \right]}_{\text{``mixing matrix''}~\Lambda}\Spvek{N^{\pi\pi}_{T};~;N^{KK}_{T};~;N^{p\bar{p}}_{T}}
\end{equation}&%
\vspace{40pt}$\rightarrow$\hspace{20pt}&
\begin{equation}\label{eq:pidUnfoldingMatrix}
\Spvek{N^{\pi\pi}_{T};~;N^{KK}_{T};~;N^{p\bar{p}}_{T}} = \Lambda^{-1}\Spvek{N^{\pi\pi}_{R}-\textcolor{magenta}{N^{\pi\pi}_{bkgd}};~;N^{KK}_{R}-\textcolor{magenta}{N^{KK}_{bkgd}};~;N^{p\bar{p}}_{R}-\textcolor{magenta}{N^{p\bar{p}}_{bkgd}}}
\end{equation}
\end{tabulary}

\section{Non-exclusive background determination}

Determination of non-exclusive background makes use of the missing transverse momentum which is distributed differently for the signal and for aforementioned background. High statistics of the data allows to apply a data-driven method. What is more, not only integrated over observables but even differentially as a function of these.

It has already been demonstrated in Sec.~\ref{subsec:ptMiss} that $p_{\text{T}}^{\text{miss}}$ from exclusive events is much narrower compared to background, visible as a peak near the axis origin. Additional feature of 
$p_{\text{T}}^{\text{miss}}$ - probability density approaching to zero with $p_{\text{T}}^{\text{miss}}$ moving towards the axis origin, enables performing a polynomial fit to $p_{\text{T}}^{\text{miss}}$ distribution in the background-dominated range and extrapolation of this polynomial down to  $p_{\text{T}}^{\text{miss}} = 0$ under the peak of CEP signal. The procedure used to determine non-exclusive background content in the signal region is described below.



\section{Normalization of signal and background models}\label{sec:bkgdSignalNorm}


Consistency between data and MC, valuable to demonstrate good understanding of the backgrounds and data themselves, has been tested for exclusive $\pi^{+}\pi^{-}$ channel\footnote{Other channels - $K^{+}K^{-}$ and $p\bar{p}$ - were not subjected to similar study because of poor MC statistics. However, structure of backgrounds and level of agreement with MC is expected to be very similar to that presented for $\pi^{+}\pi^{-}$.}. We considered here only non-exclusive backgrounds because of negligible contribution from misidentifications.

The following MC samples were used in this study:
\begin{itemize}
 \item Exclusive $\pi^{+}\pi^{-}$ (signal) - events from GenEx\cite{GenEx} generator passed through Geant3 simulation of the STAR detector (STARsim) and Geant4 simulation of the RP Phase II* detectors, fully embedded into zero-bias data,
 \item CD (background) - Central Diffraction events from Pythia~8.1 generator, filtered at generation to ensure lack of signal in BBC-large, passed through Geant3 simulation of the STAR detector (STARsim) and Geant4 simulation of the RP Phase II* detectors, partially embedded into zero-bias data (only the simulated RP response embedded),
 \item MB+elastic, (background) - Minimum Bias events from Pythia~8.1 generator filtered at generation to ensure lack of signal in BBC-large, passed through Geant3 simulation of the STAR detector (STARsim) and Geant4 simulation of the RP Phase II* detectors, partially embedded into elastic trigger (RP\_ET) data (only the simulated RP response embedded).
\end{itemize}

Listed background samples were not fully embedded since an enormous CPU time would be required to obtain satisfactory statistics. It was also found unnesessary to embed TPC tracks into zero-bias data to obtain reliable agreement between distributions of desired quantities presented below.

MC samples from Pythia generator were additionally filtered before passing through Geant to increase generation efficiency, as well as overcome difficulty arising from missing simulation of the BBC-large in STARsim. For each event, all charged particles were analytically propagated through the magnetic field of TPC with the helical paths resulting from their hadron-level momenta. If any of these particles crossed the volume of BBC-large detector, event was dropped from generation.

Normalization of backgrounds was done separately for two ranges of $\Delta\varphi$. First, MB+elastic background was normalized. By definition this was done only for $\Delta\varphi$ bin representing elastic-like configuration of forward protons ($\Delta\varphi>90^{\circ}$). The MB+elastic MC was scaled to have the same integral as the data in range $|\Delta z_{\text{vtx}}|>100$~cm. In this range we assumed sole presence of this type of background, which is characterized by very wide distribution of $\Delta z_{\text{vtx}}$ because of TPC and RP vertices being independent. Comparison plots are contained in Figs.~\ref{fig:Ratio_DeltaZVtx_DeltaPhiBins} and~\ref{fig:Ratio_DeltaZVtx}. An important cross-check for correctness of this assumption is shown in Fig.~\ref{fig:Ratio_Collinearity}, where the data vs. MC collinearity $\Delta\theta$ is presented, defined as
%
\begin{equation}\label{eq:collinearity}
 \Delta\theta = \sqrt{\left(\Delta\theta_{x}\right)^{2} + \left(\Delta\theta_{y}\right)^{2}} = \sqrt{\left(\theta_{x}^{W}+\theta_{x}^{E}\right)^{2} + \left(\theta_{y}^{W}+\theta_{y}^{E}\right)^{2}}.
\end{equation}
%
One can notice part of distribution close to 0, with nearly perfectly collinear protons. The data is well described by MC, which would unlikely be the case without contribution from the red histogram representing MB+elastic background. An interesting observation related to this background contribution is that almost all MB+elastic events in the final plots originate from the Central Diffraction process, with the forward protons outside of RP acceptance - non-diffractive events do not pass tight CEP event selection.

In the second step the CD MC was normalized. It was scaled to have the same integral as the data (minus MB+elastic MC in $\Delta\varphi>90^{\circ}$ sub-sample) in range $p_{T}^{\text{miss}}>150$~MeV, where no exclusive signal is expected.

In the last step the exclusive $\pi^{+}\pi^{-}$ MC was normalized. It was scaled to have the same integral as the data (minus all considered non-exclusive backgrounds) in range $p_{T}^{\text{miss}}<75$~MeV, where exclusive signal is dominant. The result of this procedure for the distribution of $p_{T}^{\text{miss}}$ is given in Figs.\ref{fig:Ratio_MissingPt_OppositeAndSameSign_DeltaPhiBins} and~\ref{fig:Ratio_MissingPt}. Joint distribution for each quantity - without differentiation with respect to $\Delta\varphi$ - was obtained by adding corresponding event counts from two $\Delta\varphi$ ranges.

As can be observed in the comparison plots, presented data are generally well described by MC. In case of $\Delta z_{\text{vtx}}$ some imperfectness in the position and width of the simulated signal peak can be noticed, most probably arising from slightly underestimated timing resolution of the RP trigger counters in the simulation. Distribution of $p_{T}^{\text{miss}}$ is quite well described, for both signal and control channel. The ratio of number of opposite-sign pairs to same-sign pairs is compatible between data and MC, which was possible to achieve by rejecting in Pythia contributions from events with the central state consisting from two opposite-sign pions and at least one neutral particle. These events should be suppressed by the \DPE\ condition \eqref{eq:DPE_IGJPC}, which seem to be not taken into account in Pythia at the hadronization level. We demonstrate in Fig.~\ref{fig:Ratio_MissingPt_OppositeAndSameSign_NeutralsNotSubtracted} that if these events are preserved, Pythia MC cannot describe data in the background-dominating region (large $p_{T}^{\text{miss}}$).

\begin{figure}[h]
\centering
\parbox{0.4725\textwidth}{
  \centering
  \begin{subfigure}[b]{\linewidth}
                \subcaptionbox{\label{fig:Ratio_DeltaZVtx_DeltaPhiBin_0}}{\includegraphics[width=1.05\linewidth,page=1]{graphics/backgrounds/dataVsMc/Ratio_DeltaZVtx_DeltaPhiBin_0.pdf}}
  \end{subfigure}\\
  \begin{subfigure}[b]{\linewidth}\addtocounter{subfigure}{1}
                \subcaptionbox{\label{fig:Ratio_Linear_DeltaZVtx_DeltaPhiBin_0}}{\includegraphics[width=1.05\linewidth,page=1]{graphics/backgrounds/dataVsMc/Ratio_Linear_DeltaZVtx_DeltaPhiBin_0.pdf}}
  \end{subfigure}
}%
\quad\quad%
\parbox{0.4725\textwidth}{
  \centering
  \begin{subfigure}[b]{\linewidth}
                \subcaptionbox{\label{fig:Ratio_DeltaZVtx_DeltaPhiBin_1}}{\includegraphics[width=1.05\linewidth,page=1]{graphics/backgrounds/dataVsMc/Ratio_DeltaZVtx_DeltaPhiBin_1.pdf}}
  \end{subfigure}\\
  \begin{subfigure}[b]{\linewidth}\addtocounter{subfigure}{1}
                \subcaptionbox{\label{fig:Ratio_Linear_DeltaZVtx_DeltaPhiBin_1}}{\includegraphics[width=1.05\linewidth,page=1]{graphics/backgrounds/dataVsMc/Ratio_Linear_DeltaZVtx_DeltaPhiBin_1.pdf}}
  \end{subfigure} 
}\caption[Comparison of $\Delta z_{\text{vtx}}$ for CEP $\pi^{+}\pi^{-}$ events in two ranges of $\Delta\varphi$ between data and embedded MC.]{Comparison of $\Delta z_{\text{vtx}}$ for CEP $\pi^{+}\pi^{-}$ events in two ranges of $\Delta\varphi$ (left: $\Delta\varphi<90^{\circ}$, right: $\Delta\varphi>90^{\circ}$) between data and embedded MC after full selection (except cut on presented quantity). Plots in top and bottom row differ only in the $y$-axis (top: logarithmic, bottom: linear). Data are represented by black points, while stacked MC predictions are drawn as histograms of different colors. Histogram from each MC process has been normalized according to prescription in the text. Vertical error bars represent statistical uncertainties, horizontal bars represent bin sizes.}\label{fig:Ratio_DeltaZVtx_DeltaPhiBins}%
\end{figure}
%--------------------------- 


\begin{figure}[h]
\centering
\parbox{0.4725\textwidth}{
  \centering
  \begin{subfigure}[b]{\linewidth}
                \subcaptionbox{\label{fig:Ratio_DZVtx}}{\includegraphics[width=1.05\linewidth,page=1]{graphics/backgrounds/dataVsMc/Ratio_DeltaZVtx.pdf}}
  \end{subfigure}
}%
\quad\quad%
\parbox{0.4725\textwidth}{
  \centering
  \begin{subfigure}[b]{\linewidth}
                \subcaptionbox{\label{fig:Ratio_Linear_DeltaZVtx}}{\includegraphics[width=1.05\linewidth,page=1]{graphics/backgrounds/dataVsMc/Ratio_Linear_DeltaZVtx.pdf}}
  \end{subfigure}
}\caption[Comparison of $\Delta z_{\text{vtx}}$ for CEP $\pi^{+}\pi^{-}$ events between data and embedded MC.]{Comparison of $\Delta z_{\text{vtx}}$ for CEP $\pi^{+}\pi^{-}$ events between data and embedded MC after full selection (except cut on presented quantity). Left and right plot differ only in the $y$-axis (left: logarithmic, right: linear). Data are represented by black points, while stacked MC predictions are drawn as histograms of different colors. Histogram from each MC process has been normalized according to prescription in the text. Vertical error bars represent statistical uncertainties, horizontal bars represent bin sizes.}\label{fig:Ratio_DeltaZVtx}%
\end{figure}
%---------------------------







%---------------------------
\begin{figure}[ht!]
\centering%
\parbox{0.4725\textwidth}{%
  \centering%
  \includegraphics[width=\linewidth]{graphics/backgrounds/dataVsMc/Ratio_Collinearity.pdf}
}%
\quad%
\parbox{0.4725\textwidth}{%
    \caption[Comparison of collinearity $\Delta\theta$ for CEP $\pi^{+}\pi^{-}$ events with $\Delta\varphi>90^{\circ}$, between data and embedded MC.]{Comparison of coliinearity $\Delta\theta$ for CEP $\pi^{+}\pi^{-}$ events with $\Delta\varphi>90^{\circ}$ between data and embedded MC after full selection. Data are represented by black points, while stacked MC predictions are drawn as histograms of different colors. Histogram from each MC process has been normalized according to prescription in the text. Vertical error bars represent statistical uncertainties, horizontal bars represent bin sizes.}\label{fig:Ratio_Collinearity}%  
}
\end{figure}
%---------------------------






\begin{figure}[h]
\centering
\parbox{0.4725\textwidth}{
  \centering
  \begin{subfigure}[b]{\linewidth}
                \subcaptionbox{\label{fig:Ratio_MissingPt_OppositeAndSameSign_DeltaPhiBin_0}}{\includegraphics[width=1.05\linewidth,page=1]{graphics/backgrounds/dataVsMc/Ratio_MissingPt_OppositeAndSameSign_DeltaPhiBin_0.pdf}}
  \end{subfigure}\\
  \begin{subfigure}[b]{\linewidth}\addtocounter{subfigure}{1}
                \subcaptionbox{\label{fig:Ratio_LogY_MissingPt_OppositeAndSameSign_DeltaPhiBin_0}}{\includegraphics[width=1.05\linewidth,page=1]{graphics/backgrounds/dataVsMc/Ratio_LogY_MissingPt_OppositeAndSameSign_DeltaPhiBin_0.pdf}}
  \end{subfigure} 
}%
\quad\quad%
\parbox{0.4725\textwidth}{
  \centering
  \begin{subfigure}[b]{\linewidth}
                \subcaptionbox{\label{fig:Ratio_MissingPt_OppositeAndSameSign_DeltaPhiBin_1}}{\includegraphics[width=1.05\linewidth,page=1]{graphics/backgrounds/dataVsMc/Ratio_MissingPt_OppositeAndSameSign_DeltaPhiBin_1.pdf}}
  \end{subfigure}\\
  \begin{subfigure}[b]{\linewidth}\addtocounter{subfigure}{1}
                \subcaptionbox{\label{fig:Ratio_LogY_MissingPt_OppositeAndSameSign_DeltaPhiBin_1}}{\includegraphics[width=1.05\linewidth,page=1]{graphics/backgrounds/dataVsMc/Ratio_LogY_MissingPt_OppositeAndSameSign_DeltaPhiBin_1.pdf}}
  \end{subfigure} 
}\caption[Comparison of $p_{T}^{\text{miss}}$ for CEP $\pi^{+}\pi^{-}$ events in two ranges of $\Delta\varphi$ between data and embedded MC.]{Comparison of $p_{T}^{\text{miss}}$ for CEP $\pi^{+}\pi^{-}$ events in two ranges of $\Delta\varphi$ (left: $\Delta\varphi<90^{\circ}$, right: $\Delta\varphi>90^{\circ}$) between data and embedded MC after full selection (except cut on presented quantity). Plots in top and bottom row differ only in the $y$-axis (top: linear, bottom: logarithmic). In addition to signal channel (opposite-sign particles) also control background channel (same-sign particles) is contained in the plots. Data are represented by black (opposite-sign) or red (same-sign) points, while stacked MC predictions are drawn as filled (opposite-sign) or hatched (same-sign) histograms of different colors. Histogram from each MC process has been normalized according to prescription in the text. Vertical error bars represent statistical uncertainties, horizontal bars represent bin sizes.}\label{fig:Ratio_MissingPt_OppositeAndSameSign_DeltaPhiBins}%
\end{figure}
%--------------------------- 



\begin{figure}[h]
\centering
\parbox{0.4725\textwidth}{
  \centering
  \begin{subfigure}[b]{\linewidth}
                \subcaptionbox{\label{fig:Ratio_MissingPt_OppositeAndSameSign}}{\includegraphics[width=1.05\linewidth,page=1]{graphics/backgrounds/dataVsMc/Ratio_MissingPt_OppositeAndSameSign.pdf}}
  \end{subfigure}\\
  \begin{subfigure}[b]{\linewidth}\addtocounter{subfigure}{1}
                \subcaptionbox{\label{fig:Ratio_MissingPt_OppositeAndSameSign_NeutralsNotSubtracted}}{\includegraphics[width=1.05\linewidth,page=1]{graphics/backgrounds/dataVsMc/Ratio_MissingPt_OppositeAndSameSign_NeutralsNotSubtracted.pdf}}
  \end{subfigure} 
}%
\quad\quad%
\parbox{0.4725\textwidth}{
  \centering
  \begin{subfigure}[b]{\linewidth}\vspace*{-45pt}
                \subcaptionbox{\label{fig:Ratio_LogY_MissingPt_OppositeAndSameSign}}{\includegraphics[width=1.05\linewidth,page=1]{graphics/backgrounds/dataVsMc/Ratio_LogY_MissingPt_OppositeAndSameSign.pdf}}
  \end{subfigure}\\
    \begin{minipage}[t][1.042\linewidth][t]{\linewidth}\vspace{20pt}
    \caption[Comparison of $p_{T}^{\text{miss}}$ for CEP $\pi^{+}\pi^{-}$ events between data and embedded MC.]{Comparison of $p_{T}^{\text{miss}}$ for CEP $\pi^{+}\pi^{-}$ events between data and embedded MC after full selection (except cut on presented quantity). Top left and top right plot differ only in the $y$-axis (left: linear, right: logarithmic). In addition to signal channel (opposite-sign particles) also control background channel (same-sign particles) is contained in the plots. Data are represented by black (opposite-sign) or red (same-sign) points, while stacked MC predictions are drawn as filled (opposite-sign) or hatched (same-sign) histograms of different colors. Histogram from each MC process has been normalized according to prescription in the text. Vertical error bars represent statistical uncertainties, horizontal bars represent bin sizes.\\Left bottom plot differ from the top plots in the content of Pythia MCs - in the bottom plot events with $\pi^{\pm}\pi^{\mp}$+neutrals in the central state were preserved, demonstrating significant inconsitency between data and MC in the ratio of opposite-sign to same-sign events if such events are not rejected.}\label{fig:Ratio_MissingPt}%
  \end{minipage}
}%
\end{figure}
%--------------------------- 

