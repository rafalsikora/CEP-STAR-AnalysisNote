\documentclass[a4paper,10pt,notitlepage]{report}
\usepackage[utf8]{inputenc}
\usepackage{authblk}
\usepackage{geometry}
\usepackage{graphicx}
\usepackage{grffile}
\usepackage{tabulary}
\usepackage{amsmath}
\usepackage{amssymb}
\usepackage{mathtools}
\usepackage{float}
\usepackage{cite}
\usepackage{color}
\usepackage{caption}
\usepackage{subcaption}
\usepackage[bottom]{footmisc}
% \usepackage[pdfborder={0 0 0.6}]{hyperref}
\usepackage[linktocpage=true]{hyperref}
\usepackage{listliketab}
\usepackage{enumitem}
\usepackage{multirow}
\usepackage{multicol}
\usepackage{makecell}
\usepackage{bm}
\usepackage{wrapfig}
\usepackage{empheq}
\usepackage[titletoc]{appendix}
\usepackage{cleveref}
\usepackage{blindtext}
\usepackage{colortbl}
\usepackage{array}
\usepackage[dvipsnames]{xcolor}
\usepackage[maxfloats=80]{morefloats}
\usepackage{cancel}

% \usepackage[hang, flushmargin]{footmisc}
% \usepackage[colorlinks=true]{hyperref}
% \usepackage{footnotebackref}


%\usepackage{lineno}
%\linenumbers  
%%%%%%%%%%%%%%%%%%%%%%%%%%%%%%%%%%%%%%% making figures in the subsection where placed (according to https://tex.stackexchange.com/questions/279/how-do-i-ensure-that-figures-appear-in-the-section-theyre-associated-with/235312#235312 ) 
\usepackage{placeins}

\let\Oldsection\section
\renewcommand{\section}{\FloatBarrier\Oldsection}

\let\Oldsubsection\subsection
\renewcommand{\subsection}{\FloatBarrier\Oldsubsection}

\let\Oldsubsubsection\subsubsection
\renewcommand{\subsubsection}{\FloatBarrier\Oldsubsubsection}
%%%%%%%%%%%%%%%%%%%%%%%%%%%%%%%%%%%%%%%5
% 
% \newcolumntype{L}[1]{>{\raggedright\let\newline\\\arraybackslash\hspace{0pt}}m{#1}}
% \newcolumntype{C}[1]{>{\centering\let\newline\\\arraybackslash\hspace{0pt}}m{#1}}
% \newcolumntype{R}[1]{>{\raggedleft\let\newline\\\arraybackslash\hspace{0pt}}m{#1}}


%extra subfigure with label on top of it (a), b) ) 
\newcommand{\subfigimg}[3][,]{%
  \setbox1=\hbox{\includegraphics[#1]{#3}}% Store image in box
  \leavevmode\rlap{\usebox1}% Print image
  \rlap{\hspace*{10pt}\raisebox{\dimexpr\ht1-2\baselineskip}{#2}}% Print label
  \phantom{\usebox1}% Insert appropriate spcing
}


\pdfinfo{%
  /Title    ()
  /Author   (Rafał Sikora)
}


%colors
\definecolor{gray}{rgb}{0.5, 0.5, 0.5}

%additional symbols
\newcommand{\Pom}{\texorpdfstring{I$\!$P}{P}}             % gives pomeron symbol
\newcommand{\Reg}{\texorpdfstring{I$\!$R}{R}}               % gives pomeron symbol
\newcommand{\DPE}{D\Pom E}
\newcommand{\Pomeron}{\Pom omeron}
\newcommand{\Reggeon}{\Reg eggeon}
\newcommand{\bis}{\prime\prime}

\newcommand{\RPSIDE}{\text{RP}^{\text{side}}}
\newcommand{\RPE}{\text{RP}^{\text{E}}}
\newcommand{\RPW}{\text{RP}^{\text{W}}}
\newcommand{\TRE}{\text{TR}^{\text{E}}}
\newcommand{\TRSIDE}{\text{TR}^{\text{side}}}
\newcommand{\TRNSIDE}{\text{TR}^{\cancel{\text{side}}}}
\newcommand{\TRNE}{\text{TR}^{\cancel{\text{E}}}}
\newcommand{\TRW}{\text{TR}^{\text{W}}}
\newcommand{\TRNW}{\text{TR}^{\cancel{\text{W}}}}
\newcommand{\V}{\text{Veto}}
\newcommand{\Vpu}{\text{Veto}^{\text{PU}}}
\newcommand{\Vdm}{\text{Veto}^{\text{DM}}}

% ---- ---- additional commands ---- ----
\newcommand{\specialcell}[2][c]{%
  \begin{tabular}[#1]{@{}c@{}}#2\end{tabular}}
% \newcolumntype{L}[1]{>{\raggedright\let\newline\\\arraybackslash\hspace{0pt}}m{#1}}
% \newcolumntype{C}[1]{>{\centering\let\newline\\\arraybackslash\hspace{0pt}}m{#1}}
\newcolumntype{T}[1]{>{\raggedleft\let\newline\\\arraybackslash\hspace{0pt}}m{#1}}

\makeatletter
\def\tagform@#1{\maketag@@@{(\ignorespaces#1\unskip\@@italiccorr)}}
\renewcommand{\eqref}[1]{\textup{{\normalfont(\ref{#1}}\normalfont)}}
\makeatother
  
\makeatletter
\newcommand{\Spvek}[2][l]{%
  \gdef\@VORNE{1}
  \left[\hskip-\arraycolsep%
    \begin{array}{#1}\vekSp@lten{#2}\end{array}%
  \hskip-\arraycolsep\right]}
\def\vekSp@lten#1{\xvekSp@lten#1;vekL@stLine;}
\def\vekL@stLine{vekL@stLine}
\def\xvekSp@lten#1;{\def\temp{#1}%
  \ifx\temp\vekL@stLine
  \else
    \ifnum\@VORNE=1\gdef\@VORNE{0}
    \else\@arraycr\fi%
    #1%
    \expandafter\xvekSp@lten
  \fi}
\makeatother
% ---- ---- ---- ---- ---- ---- ---- ----

%chapter heading
\makeatletter
\renewcommand{\@makechapterhead}[1]{%
 \vspace*{-18\p@}%
  {\parindent \z@ \raggedright
%     \LARGE \bfseries \thechapter. #1\par\nobreak
%     \vskip 40\p@
      \Huge \bfseries \thechapter. #1\par\nobreak
      \vskip 20\p@
  }}
\makeatother

\Crefname{figure}{Fig.}{Figs.}

% Title Page
\title{%
\centering\hspace*{-0.08\linewidth}\begin{minipage}{1.16\linewidth}\centering%
\textbf{Measurement of diffractive Central Exclusive Production of $\bm{h^{+}h^{-}}$ pairs ($\bm{h=\pi,K,p}$) in proton-proton collisions at~$\bm{\sqrt{s}=}$~200~GeV with forward proton reconstruction\\in Roman Pot detectors}
\end{minipage}%
\vspace*{10pt}}
% \author[1]{Leszek Adamczyk}
% \author[1]{Łukasz Fulek}
% \author[2]{Włodek Guryn}
% \author[1]{Mariusz Przybycień}
% \author[1,$\dag$]{\underline{Rafał Sikora}}
% \affil[1]{AGH University of Science and Technology, FPACS, Kraków, Poland} 
% \affil[2]{Brookhaven National Laboratory, Upton, NY, USA}
% \affil[$\dag$]{e-mail:~\href{mailto:rafal.sikora@fis.agh.edu.pl}{rafal.sikora@fis.agh.edu.pl}}

\author[ ]{\href{mailto:leszek.adamczyk@agh.edu.pl}{Leszek Adamczyk}}
\author[ ]{\href{mailto:lukasz.fulek@fis.agh.edu.pl}{Łukasz Fulek}}
\author[ ]{\href{mailto:mariusz.przybycien@agh.edu.pl}{Mariusz Przybycień}}
\author[ ]{\href{mailto:rafal.sikora@fis.agh.edu.pl}{Rafał Sikora}}
\affil[ ]{AGH University of Science and Technology, FPACS, Kraków, Poland}

\setcounter{Maxaffil}{0}
\renewcommand\Affilfont{\itshape\small}
\renewcommand{\bibname}{References}

\begin{document}

\begin{center}
\begin{minipage}[c]{0.12\linewidth}%
\vspace{5.5pt}\textbf{\LARGE{of the}}
\end{minipage}
\begin{minipage}[c]{0.15\linewidth}%
\hspace*{-8pt}\includegraphics[width=\linewidth]{graphics/STAR_logo.pdf}
\end{minipage}~
\begin{minipage}[c]{0.24\linewidth}%
\vspace{9pt}\hspace*{-8pt}\textbf{\LARGE{Experiment}}
\end{minipage}\\[-50pt]
\textbf{\LARGE{Analysis Note}}

\vspace*{150pt}
\begin{minipage}{\linewidth}
\maketitle
\begin{abstract}
In this note we present analysis of diffractive Central Exclusive Production using 2015 data from proton-proton collisions at $\sqrt{s}=200$~GeV. This dataset was collected with newly installed Roman Pot detectors in Phase II* configuration which ensured efficient triggering and measuring diffractively scattered protons. We describe intermediate stages of analysis involving choice of selection cuts, comparison of data with Monte Carlo models folded into detector acceptance, and study of systematic uncertainties specific to the analysis. Finally, we show the physics outcome of the analysis. Parts of the analysis which are of more technical nature (calculation of efficiencies, derivation of corrections to efficiencies, adjustment of the STAR simulation, systematic uncertainty of efficiencies) are described in a supplementary analysis note~\cite{supplementaryNote}.
\end{abstract}
\thispagestyle{empty}
\end{minipage}

\vspace{50pt}

%  \Huge{\textbf{\textit{DRAFT}}}
% \vspace{10pt}
ver.~1.90
\end{center}


\clearpage
\thispagestyle{empty}
\newgeometry{hmargin={2cm, 2cm}, height=10.0in}

\setcounter{secnumdepth}{3}
\setcounter{tocdepth}{4}
\tableofcontents

\newpage
%% =====  LIST OF CONTRIBUTIONS,  CHANGE LOG ====
\section*{\LARGE List of contributions\footnote{See also list of contributions in Ref.~\cite{supplementaryNote}, since presented document utilizes all fruits of work described in referenced note. Reference~\cite{supplementaryNote} should be treated as a note supplementary to the current document.}}%\\[10pt]
\addcontentsline{toc}{chapter}{List of contributions}%
   \rule{\textwidth}{1.0pt}\\[5pt]%
   %
   %
   %
      \begin{tabular}{>{\raggedright}p{0.25\linewidth}p{0.7\linewidth}}
		Leszek Adamczyk & Analysis coordination/supervision, production of picoDST, production of embedded MC samples\\
		~&~\\
		Łukasz Fulek & Analysis support\\
		~&~\\
		Mariusz Przybycień & Analysis supervision\\
		~&~\\
        Rafał Sikora$^{\star}$  & Main analyzer, write-up author\\
      \end{tabular}\newline
   \rule{\textwidth}{1.0pt}\\[10pt]%
   $^{\star}$ - contact editor
   \\[50pt]%\\[1pt]% 
  %
  %
\section*{\LARGE Change log}%
\addcontentsline{toc}{chapter}{Change log}%
  \rule{\textwidth}{1.0pt}\\[5pt]%
  %
  %
  %
  \begin{tabular}{>{\raggedright}p{0.15\linewidth}p{0.1\linewidth}p{0.7\linewidth}}
  	13 Nov 2019 & ver. 1.0 & Initial revision
  \end{tabular}\newline%
 \rule{\textwidth}{1.0pt}


%% =====  INTRODUCTION ====
%%===========================================================%%
%%                                                           %%
%%                       INTRODUCTION                        %%
%%                                                           %%
%%===========================================================%%


\chapter{Introduction}\label{chap:introduction}
% \section{Physics motivation for the measurement}
\section{Central Exclusive Production}
The Central Exclusive Production (CEP) takes place when interacting particles form in the mid-rapidity region a state (``central production'') whose all constituents/decay products are measured in the detector (``exclusive''). The initial state particles can either dissociate, excite or stay intact. The latter case of CEP in proton-proton collisions can be written as
\begin{equation}\label{eq:cep}%
%  p~~+~~p~~~\rightarrow~~~p~~\stackrel{\Delta\eta_{1}}{\oplus}~~X~~\stackrel{\Delta\eta_{2}}{\oplus}~~p
p~~+~~p~~~\rightarrow~~~p~~+~~X~~+~~p
\end{equation}
and depicted as in Fig.~\ref{fig:eta_phi}. Mass and rapidity of state $X$ is given by
\begin{equation}\label{eq:mass_X}
M_{X} = \sqrt{s\Big(\xi_{1}\xi_{2}\sin^{2}{(\alpha/2)}-(1-\xi_{1}-\xi_{2})\cos^{2}{(\alpha/2)}\Big)} \stackrel{\alpha=\pi}{=} \sqrt{s\xi_{1}\xi_{2}},
\end{equation}\vspace{-10pt}
\begin{equation}\label{eq:rapidity_X}
y_{X} = \frac{1}{2}\ln{\frac{\xi_{1}}{\xi_{2}}},
\end{equation}
where $\alpha$ is angle between scattered protons and $\xi=(p_{0}-p)/p_{0}$ is the fractional momentum loss of proton.

\section{Double \Pomeron\  Exchange}
Reaction from Eq.~\ref{eq:cep} can exhibit purely electromagnetic ($\gamma$-$\gamma$), mixed ($\gamma$-$\mathcal{O}$) or purely strong nature ($\mathcal{O}$-$\mathcal{O}$). The last type is dominant at RHIC energies. It is characterized by the lack of hard scale (if protons are scattered at small angles), therefore perturbative QCD cannot be applied and Regge theory~\cite{IntroductionToRegge} is used instead. An object $\mathcal{O}$ does not have direct QCD representation - in Regge framework it is the so-called ``trajectory`` (\Reg eggeon, \Reg). \Reg eggeon with quantum numbers of vacuum is called ''\Pomeron`` (\Pom) and \Pom-\Pom\ reaction (Fig.~\ref{fig:DPE}) is called ''Double \Pomeron\  Exchange``. %

%---------------------------
\begin{figure}[b!]
\centering
\parbox{0.475\textwidth}{%
  \centering%
  \hspace*{-10pt}\includegraphics[width=1.1\linewidth]{graphics/introduction/eta_phi.pdf}\vspace*{-10pt}%
  \caption{Central Exclusive Production in $\eta$-$\phi$ space.\\}%
  \label{fig:eta_phi}%
}
\quad
\parbox{0.475\textwidth}{%
  \centering%
  \includegraphics[width=0.64\linewidth]{graphics/introduction/DPE.pdf}%
  \caption{Diagram of D\Pom E process.\\}%
  \label{fig:DPE}%
}\vspace*{-20pt}
\end{figure}
%---------------------------

Processes involing \Pomeron\  exchange are referred as diffraction due to cross-section in scattering angle resembling similar shape to instesity pattern of diffracted light. Diffractive events have specific property of the ''rapidity gap`` which is an angular region free of hadrons. In \DPE\ two such gaps are present, marked in Fig.~\ref{fig:eta_phi} as $\Delta\eta_{1}$ and $\Delta\eta_{2}$.%

\DPE\ is a spin-parity filter - from the fact that scattered particles have all quantum numbers unchanged after the interaction, central states must satisfy
\begin{equation}\label{eq:DPE_IGJPC}
 I^{G}J^{PC}=0^{+}\textrm{even}^{++}.
\end{equation}

The lowest order QCD representation of the \Pomeron\ is a pair of oppositely colored gluons. This fact makes the \DPE\ recognized as the gluon-rich environment process which should enhance production of the bound states of gluons (''glueballs``), whose existence has not yet been proven experimentally.

For detailed introduction to the topic of diffraction see \cite{pomeronAndQCD,barone}.\vspace*{-20pt}

\section{Physics motivation for the measurement}
aaa

%% =====  DATASET ====
%%===========================================================%%
%%                                                           %%
%%                          DATASET                          %%
%%                                                           %%
%%===========================================================%%


\chapter{Data set}\label{chap:dataset}

\section{Bad runs}
% \subsection{SSD detecting efficiency}
% \subsection{Reconstruction efficiency}
\label{sec:badRuns}

%---------------------------
\begin{figure}[hb]
\centering
\parbox{0.4\textwidth}{
  \centering
  \begin{subfigure}[b]{\linewidth}{
                \subcaptionbox{\label{fig:positionHistograms}}{\includegraphics[width=\linewidth]{graphics/badRuns/positionHistograms.pdf}}}
  \end{subfigure}
}
\quad
\parbox{0.545\textwidth}{
  \centering
  \begin{subfigure}[b]{\linewidth}{
                \subcaptionbox{\label{fig:positionVsRunGraph}}{\includegraphics[width=\linewidth]{graphics/badRuns/positionVsRunGraph.pdf}}}
  \end{subfigure}
}%
\caption[Hit map of elastic proton hits in .]{map of elastic proton hits in .}
% \label{fig:xy_recoEff}
\end{figure}
%---------------------------


%% =====  MONTE CARLO ====
%%===========================================================%%
%%                                                           %%
%%                       MONTE CARLO                         %%
%%                                                           %%
%%===========================================================%%


\chapter{Monte Carlo simulations}\label{chap:mc}

This chapter contains description of MC generators and MC samples used for determination of signal event reconstruction and selection efficiency (Sec.~\ref{sec:mcExclusiveSignal}), modelling of background contribution (Sec.~\ref{sec:mcBkgdContrib}), and comparison of hadron level cross sections with model predictions (Sec.~\ref{sec:mcModelPred}). Apart from described samples also single particles ($\pi$, $K$, $p$) embedded into zero-bias data were used for the purpose of the TPC and TOF reconstruction efficiency calculations, but their description is ommited here as related calculations were presented in Ref.~\cite{supplementaryNote}.

\section{Exclusive signal}\label{sec:mcExclusiveSignal}

\subsection{GenEx $\pi^{+}\pi^{-}$ embedded into zero-bias data}

Signal sample with exclusive $\pi^{+}\pi^{-}$ was prepared with GenEx\cite{GenEx} event generator. Each event has been passed independently through Geant3 simulation of the STAR detector (STARsim) and Geant4 simulation of the RP Phase II* detectors, merged at the end and fully embedded into the same zero-bias data event. Before passing through the STAR detector model, events were filtered in order to gain production efficiency. At the particle level pions were required to have $p_{T}>0.15$~GeV and $|\eta|<1$, while forward scattered protons (after added beam divergence) were required to fit within fiducial region envelope (Eq.~\eqref{eq:RpFiducial}) extended by 3 standard deviations of the angular beam divergence.

%moze napisac ile przypadkow

\subsection{Forward scattered protons embedded into zero-bias data}

Independent embedded MC sample was prepared especially for determination of the RP track reconstruction and selection efficiency. Large sample of forward scattered protons from GenEx was simulated in Geant4 and embedded into zero-bias data. The same runs were simulated from which data were used in the physics analysis. Independent forward proton MC embedded into zero-bias data had advantage in possibility of using entire sample of zero-bias triggers - it is different in case of TPC signal embedding, when only small fraction of data (``adc files'') carries enough information for MC and data overlay.


\subsection{Fast MC generator for particle identification studies}
Corrections reflecting identification efficiency and misidentification probability requires good modeling of detector response in terms of $dE/dx$ and TOF time ($\rightarrow m^{2}_{\text{TOF}}$) measurement, which were used for this purpose as described in Sec.~\ref{subsec:pidCuts}. In addition to this, significant number of simulated events is needed to reduce statistical uncertainties of efficiency. The former was provided by adjusting $dE/dx$ spectra from embedded MC to match the data, as elaborated in Chap.~7 of Ref.~\cite{supplementaryNote}. The latter, however, was not easy to achieve for exclusive $K^{+}K^{-}$ and $p\bar{p}$ whose identification is most challenging and information about identification efficiency is the most needed among studied species. Specially for study of particle (exlusive pair) identification a dedicated MC simulation was prepared.

This dedicated MC simulation was designed to work as follows (simulation of single CEP event of predefined pair ID is described):
\begin{enumerate}
 \item The position of $z_{\text{vtx}}$ was drawn from predefined distribution.
 \item Kinematics of central state particles was set: momentum (magnitude) $p$, pseudorapidity $\eta$ and azimuthal angle $\phi$ of positive and negative charge particles were drawn from predefined distributions.
 \item Both particles were tested if doubled radius of curvature $2R$ of associated track in the magnetic field of the TPC ($B=0.5$~T, $R \propto p_{\text{T}}/B$) is smaller than the radius of TOF detector barell (assumed 212~cm). If not then event was skipped and procedure was restarted (back to 1.).
 \item The particles were propagated from the vertex at $(0,0,z_{\text{vtx}})$ through the magnetic field of TPC using Newton's method with the time step (in the laboratory) equal 100~ps, corresponding to space step $<3$~cm.
 \item After step 4. the position of the TOF cell was known allowing to calculate the TOF path length $L$ between the vertex and position of the TOF hit. Also the TOF hit time $t$ was then known, further smeared by adding random number from normal distribution with mean at 0 and standard deviation $\sigma_{\text{TOF}}=60$~ps to account for the finite TOF time measurement resolution. In addition to this, reconstructed tracks' (transverse) momenta were defined as the true momenta smeared by 6~MeV if $p_{\text{T}}<0.3$~GeV or by $2.4~\text{MeV} + 1.2\%\times p_{\text{T}}$ if $p_{\text{T}}>0.3$~GeV, to account for finite TPC momentum resolution. At this stage it was possible to calculate $m^{2}_{\text{TOF}}$ using Eq.~\eqref{eq:mSquared}.
 \item The $dE/dx$ measurement was simulated. For each particle a $dE/dx$ was drawn from the distribution of the form given by Eq.~(7.6) and of parameters (for given particle ID and momentum) which were extracted from the data and tabulated in Tab.~7.1, all contained in Chap.~7 of Ref.~\cite{supplementaryNote}. This assured that the simulated $dE/dx$ exactly matched the data. Once $dE/dx$ for both particles (tracks) was obtained, value of the $dE/dx$ error (more strictly: uncertainty of $\ln(dE/dx~[\text{keV/cm}])$) and value of $\log_{2}(dx)$ was also set up. These quantities depend on the number of TPC hit points used in the reconstruction of $dE/dx$ (the more hits in tracks, the better resolution of $dE/dx$ and higher $dx$), which obviously is not accessible without full STAR simulation in Geant. This problem was solved by extracting dependence of $\sigma(\ln(dE/dx))$ and $\log_{2}(dx)$ on the TOF path length from the data (from CEP events, Fig.~\ref{fig:correlationsTofPathLength}). Since the length of the TOF path is very strongly correlated with the number of hits forming the track and thus number of hits used to reconstruct $dE/dx$, one is allowed to draw $\sigma(\ln(dE/dx))$ and $\log_{2}(dx)$ from their distributions for particular TOF path lengths calculated in 5. and use as measured ones. In this way the simulation preserves relevant connections between $dE/dx$-related quantities. After these steps are taken the $n\sigma_{X}$ ($X=\pi$, $K$, $p$) variables are calculated for each track using the definition (Eq.~\eqref{eq:nSigmaDef}), in exactly the same way as it is done during standard data reconstruction.
 \item Event information needed to study pair identification was stored in the ROOT tree: ID of particles forming a pair, their three-momenta, $m^{2}_{\text{TOF}}$, $n\sigma_{\pi}$, $n\sigma_{K}$ and $n\sigma_{p}$.
\end{enumerate}

%--------------------------- 
\begin{figure}%[ht!] 
\centering
\parbox{0.4725\textwidth}{
  \centering
  \begin{subfigure}[b]{\linewidth}{
                \subcaptionbox{\label{fig:dEdxErrorVsTofPathLength}}{\includegraphics[width=\linewidth]{graphics/corrections/DEdxErrorVsTofPathLength.pdf}}}
  \end{subfigure}
}
\quad
\parbox{0.4725\textwidth}{
  \centering
  \begin{subfigure}[b]{\linewidth}{
                \subcaptionbox{\label{fig:Log2dEdxVsTofPathLength}}{\includegraphics[width=\linewidth]{graphics/corrections/Log2dxVsTofPathLength.pdf}}}
  \end{subfigure}
}% 
\caption[$dE/dx$ error vs. TOF path length and $\log_{2}(dx)$ vs. TOF path length for exclusive event candidates.]{Correlation between uncertainty of the natural logarithm of $dE/dx / (1~\text{keV/cm})$ and track TOF path length (\ref{fig:dEdxErrorVsTofPathLength}) and correlation between base 2 logarithm of $dx$ and track TOF path length (\ref{fig:Log2dEdxVsTofPathLength}). The distributions were obtained for the exclusive event candidates after full selection, with all three types of particle pairs combined.} \label{fig:correlationsTofPathLength}
\end{figure} 


For the purpose of determination of pair identification efficiency in CEP analysis descibed in this note, parameters of vertex distribution were set to match the data: $\langle z_{\text{vtx}}\rangle=0$ and $\sigma(z_{\text{vtx}})=50$~cm, as well as $z_{\text{vtx}}$ was required to lie within the analysis limits (cut~\ref{enum:CutZVx}). Distribution of particle $\eta$ was set to flat and limited to analyzed range $|\eta|<0.7$, while particle $\phi$ was defined as uniformly distributed in full azimuth ($2\pi$~rad), both fairly agreeing with expectations from physics models and observations in data.

\section{Background modelling}\label{sec:mcBkgdContrib}

The following MC samples were used to study non-exclusive backgrounds:
\begin{itemize}
 \item Central Diffraction (CD) events from Pythia~8.1 generator with MBR model of \Pom omeron flux, filtered at generation to ensure lack of signal in BBC-large, passed through Geant3 simulation of the STAR detector (STARsim) and Geant4 simulation of the RP Phase II* detectors, partially embedded into zero-bias data (only the simulated RP response embedded),%\vspace*{-5pt}
 \item Minimum Bias events from Pythia~8.1 generator with MBR model of \Pom omeron flux, filtered at generation to ensure lack of signal in BBC-large, passed through Geant3 simulation of the STAR detector (STARsim) and Geant4 simulation of the RP Phase II* detectors, partially embedded into elastic trigger (RP\_ET) data (only the simulated RP response embedded).%\vspace*{-5pt}
\end{itemize}

Listed background samples were not fully embedded since large amount of CPU time would be required to obtain satisfactory statistics, while they were expected to only provide qualitative description of the backgrounds. It was also found unnesessary to embed TPC tracks into zero-bias data to obtain reliable agreement between distributions of desired quantities presented below.

MC samples from Pythia generator were additionally filtered before passing through Geant to increase generation efficiency, as well as overcome difficulty arising from missing simulation of the BBC-large in STARsim. For each event, all charged particles were analytically propagated through the magnetic field of TPC with the helical paths resulting from their hadron-level momenta. If any of these particles crossed the volume of BBC-large detector, event was dropped from generation.


%% %% %% %% %% %% %% %% %% %% %% %% %% %% %% %% %% %% %% %% %% %% %% %% %% %% %% %% %% %% %% %% %% %% %% %% 
\section{Model predictions}\label{sec:mcModelPred}

The following MC samples were produced for comparisons of the measured cross section with model predictions. The list contains all models which are currently available.

\subsection{GenEx}
The GenEx~\cite{GenEx} event generator is based on a simple phenomenological models~\cite{LSmodel, LSModelKK} of continuum production mechanism of $\pi^+\pi^-$ or $K^+K^-$ pairs. In the implemented model absorption corrections are not taken into account explicitly. A damping factor was estimated to be of the order of 2-5 ($\pi^+\pi^-$) and 2 ($K^+K^-$)~\cite{LSAbsorption}. To account for absorption the cross sections obtained from GenEx are scaled by 0.25 and 0.4 for $\pi^+\pi^-$ and $K^+K^-$, respectively, to fit DiMe predictions which include absorption effects (see below). Predictions are also sensitive to the choice of meson form factor. GenEx predictions are shown using exponential form factor with $\Lambda_{of\!f}^{2}=1.0~\textrm{GeV}^{2}$. Changes of $\Lambda_{of\!f}^{2}$ by 50\% lead to cross section changes up to a factor of 2.

\subsection{DiMe}
The DiMe~\cite{DurhamModel} event generator is also based on a simple phenomenological model~\cite{harland_lang_1} of continuum production mechanism of $\pi^+\pi^-$ or $K^+K^-$ pairs. However, DiMe in contrary to GenEx models absorption effects with four different models for absorption available. The prediction from "model 1" giving the best consistency with data is used in the comparisons. DiMe predictions are also sensitive to the choice of meson form factor. Three different parameterizations of meson form factor are implemented. We chose exponential form with the same slope as used for GenEx predictions. Therefore the differences between GenEx and DiMe are almost entirely due to the absorption.

\subsection{Pythia8 MBR}
The MBR model~\cite{mbr_pythia8} implemented in PYTHIA8~\cite{pythia8} was founded to describe inclusive 
central diffraction (CD, $p+p\rightarrow p+X+p$) cross section at CDF while exclusive $h^+h^-$ state occurs from fragmentation and hadronization of the central state based on the Lund string model. MBR model implemented in PYTHIA8.165 allows generation of the central state starting from the mass threshold of $2 m_h$. In later versions region below 1~GeV mass was excluded. Therefore PYTHIA8 expectations for very low masses are in question but are shown for completeness.


%% =====  EVENT SELECTION ====
%%===========================================================%%
%%                                                           %%
%%                     EVENT SELECTION                       %%
%%                                                           %%
%%===========================================================%%


\newcommand{\itemm}{\item\hspace*{-5pt}.\hspace*{-1pt}~}

\chapter{Event selection}\label{chap:eventSelection}

Complete list of analysis cuts used for signal extraction is presented in Sec.~\ref{sec:listOfCuts}. Detailed description of each cut can be found in Sec.~\ref{sec:descriptionOfCuts}. [For PDF readers: you can directly move to description of given cut by clicking on corresponding bold cut number \textbf{CX} at the start of line in the list of cuts.]

\section[List of cuts]{List of cuts\footnote{Some cuts (e.g.~\ref{enum:CutTpcTrks}) are decomposed to constituent sub-cuts. Cut is formed by the logical AND of all its sub-cuts. Events must pass all cuts to be identified as a signal.}}\label{sec:listOfCuts}
\begin{enumerate}[label=\textbf{\hyperref[sec:C\arabic*]{C\arabic*}},ref=C\arabic*]
 \itemm Exactly 1 primary vertex with TPC track(s) matched with hits in TOF.\label{enum:CutPrimVx}
 \itemm TPC vertex from~\ref{enum:CutPrimVx} is placed within $|z_{\text{vx}}|<80$~cm.\label{enum:CutZVx}
 \itemm Exactly 2 opposite-sign primary TPC tracks~(\ref{enum:TpcOppoSign}) of good quality~(\ref{enum:TpcQualityCuts}) matched with hits in TOF~(\ref{enum:TpcTofMatched}) and reconstructed within kinematic region of high TPC acceptance~(\ref{enum:TpcKinematicCuts}), with associated global tracks characterized by small distance of closest approach (DCA) to the primary vertex~(\ref{enum:TpcDcaCuts}) and high proximity to each other at the beamline~(\ref{enum:TpcDeltaZ0Cut}).\label{enum:CutTpcTrks}
    \begin{enumerate}[label=\textbf{\theenumi.\arabic*},ref=\theenumi.\arabic*]
      \itemm Exactly 2 TOF-matched (match flag $>0$) primary tracks and no additional primary tracks matched with BEMC clusters,\label{enum:TpcTofMatched}
      \itemm Tracks are of opposite signs,\label{enum:TpcOppoSign}
      \itemm Both tracks are contained within the kinematic range:\label{enum:TpcKinematicCuts}\hspace*{13pt}
      $|\eta|<0.7$,~~~~$p_{T}>0.2~\text{GeV}$,
      \itemm Associated global tracks satisfy quality criteria:\label{enum:TpcQualityCuts}\hspace*{44pt}
      $N_{\text{hits}}^{\text{fit}}\geq25$,~~~$N_{\text{hits}}^{\text{dE/dx}}\geq15$,~~~$|d_{0}|<1.5$~cm,
      \itemm Associated global tracks match well to the prim. vertex:\label{enum:TpcDcaCuts}\hspace*{3.5pt}
      $\text{DCA}(R)<1.5$~cm,~~~~$|\text{DCA}(z)|<1$~cm,
      \itemm Associated global tracks are close at the beamline:\label{enum:TpcDeltaZ0Cut}\hspace*{29pt}
      $|\Delta z_{0}|<2$~cm.
    \end{enumerate}
 \itemm Exactly 1 RP track on each side of STAR central detector~(\ref{enum:RpOneTrkPerSide}) of good quality~(\ref{enum:RpQualityCuts}), with local angles consistent with the IP being the track origin~(\ref{enum:RpLocalAngles}), lying within fiducial region of high geometrical acceptance~(\ref{enum:RpFiducial}).\label{enum:CutRpTrks}
      \begin{enumerate}[label=\textbf{\theenumi.\arabic*},ref=\theenumi.\arabic*]
      \itemm RP tracks contain only track-points with at least 3 (out of 4) planes used in reconstruction,\label{enum:RpQualityCuts}
      \itemm Local angles ($\theta_{x}^{\text{RP}}$, $\theta_{y}^{\text{RP}}$) consistent with expectation for protons originating from the IP\label{enum:RpLocalAngles}%
      \[-2~\text{mrad}<\theta_{x}^{\text{RP}}-x^{\text{RP}}/|z^{\text{RP}}|<4~\text{mrad},~~~~~-2~\text{mrad}<\theta_{y}^{\text{RP}}-y^{\text{RP}}/|z^{\text{RP}}|<2~\text{mrad},\]
      \itemm Exactly 1 track passing cuts \ref{enum:RpQualityCuts}-\ref{enum:RpLocalAngles} per side,\label{enum:RpOneTrkPerSide}
      \itemm Tracks passing cut~\ref{enum:RpOneTrkPerSide} lie within the fiducial $(p_{x},p_{y})$ region defined as\label{enum:RpFiducial}:\\
      \[0.2<|p_{y}|<0.4,~~~-0.2<p_{x},~~~(p_{x}+0.3)^{2}+p_{y}^{2}<0.5^{2}~~~(\text{all in GeV}).\]
    \end{enumerate}
 \itemm Vertex $z$-positions measured in TPC and reconstructed from the difference of proton detection time in west and east RPs are consistent with each other within the resolution (at $3.5\sigma_{\Delta z_{\text{vtx}}}$ level):
 \[|\Delta z_{\text{vtx}}| = |z_{\text{vx}}^{\text{TPC}}-z_{\text{vx}}^{\text{RP}}|<36~\text{cm}.\vspace{-17pt}\]\label{enum:CutDeltaZVx}
 \itemm No signal in any tile of BBC-large (east or west) with $\text{ADC}>\text{ADC}_{\text{thr}}$ and $100<\text{TDC}<2400$, where $\text{ADC}_{\text{thr}}$ is specific for each channel (see Tab.~\ref{tab:bbcLargeThresholds}).\label{enum:CutBbcLarge}%
 %
 \itemm Maximally 3 reconstructed TOF clusters $N^{\text{TOF}}_{\text{clstrs}}\leq 3$.\label{enum:CutTofClusters}%
 %
 \itemm Particle/pair identification (PID):\label{enum:CutPid}
 \begin{enumerate}[label=\textbf{\theenumi.\arabic*},ref=\theenumi.\arabic*]
      \itemm Identification of particle pairs based on $dE/dx$ ($\chi^{2}$ or $n\sigma$) and $m^{2}_{\text{TOF}}$ (def. in Sec.~\ref{sec:C8} and App.~\ref{appendix:squaredMass}):\label{enum:CutPidNoPtLimit}\\[3pt]
%         \textbf{~~if~~} $n\sigma_{\text{pion}}^{\text{pair}}>3$ \textbf{~~and~~} $n\sigma_{\text{kaon}}^{\text{pair}}>3$ \textbf{~~and~~} $n\sigma_{\text{proton}}^{\text{pair}}<3$ \textbf{~~and~~} $m^{2}_{\text{TOF}}>0.6~\text{GeV}~~\rightarrow~~p\bar{p}$\\%
%         %
%         \textbf{elif~~} $n\sigma_{\text{pion}}^{\text{pair}}>3$ \textbf{~~and~~} $n\sigma_{\text{kaon}}^{\text{pair}}<3$ \textbf{~~and~~} $n\sigma_{\text{proton}}^{\text{pair}}>3$ \textbf{~~and~~} $m^{2}_{\text{TOF}}>0.15~\text{GeV}~~\rightarrow~~K^{+}K^{-}$\\%
%         %
%         \textbf{elif~~} $|n\sigma_{\text{pion}}^{\text{trk1}}|<3$ \textbf{~~and~~} $|n\sigma_{\text{pion}}^{\text{trk2}}|<3~~\rightarrow~~\pi^{+}\pi^{-}$\\%
        \textbf{~~if~~~}\hspace*{4.5pt}$\chi^{2}(\pi\pi)>9$\textbf{~~and~~}$\chi^{2}(KK)>9$\textbf{~~and~~}$\chi^{2}(pp)<9$\textbf{~~and~~}$m^{2}_{\text{TOF}}>0.6~\text{GeV}~~\rightarrow~~p\bar{p}$\\[5pt]%
        %
        \textbf{elif~~}$\chi^{2}(\pi\pi)>9$\textbf{~~and~~}$\chi^{2}(KK)<9$\textbf{~~and~~}$\chi^{2}(pp)>9$\textbf{~~and~~}$m^{2}_{\text{TOF}}>0.15~\text{GeV}~~\rightarrow~~K^{+}K^{-}$\\[5pt]%
        %
        \textbf{elif~} $|n\sigma_{\text{pion}}^{\text{trk1}}|<3$ \textbf{~and~} $|n\sigma_{\text{pion}}^{\text{trk2}}|<3~~\rightarrow~~\pi^{+}\pi^{-}$.%~~~~~~~~~~~~~~~~~~~~~~~~~~~~~~~\textbf{otherwise~~} event rejected.
      \itemm Restricting fiducial cuts on $K^{+}K^{-}$ and $p\bar{p}$ (to reduce misidentifications and assure high PID eff.):\label{enum:CutPidPtLimits}\\[2pt]
      \textbf{~if~} $p\bar{p}$:~~~~~~~~~~~~\hspace*{1.7pt}$p_{T}>0.4~\text{GeV}$,~~~~$min(p_{T}^{+},p_{T}^{-})<1.1~\text{GeV}$\\%
      \textbf{~if~} $K^{+}K^{-}$:~~~~~~$p_{T}>0.3~\text{GeV}$,~~~~$min(p_{T}^{+},p_{T}^{-})<0.7~\text{GeV}$%
\end{enumerate}
\itemm Missing (total) momentum of TPC tracks and RP tracks $p_{T}^{\text{miss}}<75~\text{MeV}$.\label{enum:CutMissingPt}%
 %
 
\end{enumerate}
%
%
%
%
%
\section{Description of cuts}\label{sec:descriptionOfCuts}%
%
\subsection{(\ref{enum:CutPrimVx},\ref{enum:CutZVx})~Primary vertex and its \texorpdfstring{$z$}{z}-position}\label{sec:C1}\label{sec:C2}
As it was designed in the trigger logic, we aim to perform CEP analysis in a clean, pile-up-free environment, therefore we cut on primary vertex multiplicity~(Fig.~\ref{fig:NumberOfPrimaryVertices}) to reject events with more than one interaction per bunch crossing. We require exactly one primary vertex containing TPC tracks matched with hits in TOF (matching of the track with hit in TOF is identified with the TOF match flag being different from 0). Later in the text we refer to such events as a single ``TOF vertex`` events.

%---------------------------
\begin{figure}[ht!]%
\centering%
\begin{minipage}{.4725\textwidth}%
  \centering%
  \includegraphics[width=\linewidth]{graphics/eventSelection/NumberOfPrimaryVertices.pdf}%
  \caption{Primary vertex multiplicity. Red arrow marks bin with events with exactly one primary vertex (with track(s) matched with hit in TOF), which are used in physics analysis.}\label{fig:NumberOfPrimaryVertices}
\end{minipage}%
\quad\quad%
\begin{minipage}{.4725\textwidth}%
  \centering
  \includegraphics[width=\linewidth]{graphics/eventSelection/zVertex_oneTof.pdf}%
  \caption{\texorpdfstring{$z$}{z}-position of the primary vertex in single TOF vertex events. Red dashed line indicate range of longitudinal vertex position accepted in analysis.\newline}\label{fig:zVertexTpc}
\end{minipage}%
\end{figure}%
%---------------------------


The single TOF vertex is required to be placed within a range $(-80~\text{cm},~80~\text{cm})$ along the $z$-axis~(Fig.~\ref{fig:zVertexTpc}). Events with vertices away from the nominal IP have low acceptance both for the central tracks and the forward protons (comparing to events with vertices close to nominal IP), therefore we reject them as their inclusion to analysis would naturally introduce large systematic uncertainties. See Sec.~3.2.3 in Ref.~\cite{supplementaryNote}.








\subsection{(\ref{enum:CutTpcTrks})~TPC tracks}\label{sec:C3}

The TPC track selection starts from the selection of events with exactly two primary tracks matched with hit in TOF~(Fig.~\ref{fig:NumberOfTofTracksInSingleTofVertex}). Matching with TOF guarantee that analyzed tracks originate from the triggered bunch crossing (ensures that tracks are ''in-time``). It is in accordance with the trigger logic which required at least 2 L0 TOF hits, as well as it enables more accurate particle identification with merged time-of-flight and $dE/dx$ method, comparing to sole usage of $dE/dx$. Primary tracks not matched with hit in TOF, whose average multiplicity in single TOF vertex is $\sim$8, are hardly distinguished between real and fake (off-time) tracks, which is an additional reason for not analyzing events with only one TOF-matched primary TPC track (the other track might be unmatched due to TOF inefficiency).

%---------------------------
\begin{figure}[t!]%
\centering%
\begin{minipage}{.4725\textwidth}%
  \centering%
  \includegraphics[width=\linewidth]{graphics/eventSelection/TpcTracks/NumberOfTofTracksInSingleTofVertex.pdf}%
  \caption[Multiplicty of primary TPC tracks matched with hit in TOF for single TOF vertex events]{Multiplicty of primary TPC tracks matched with hit in TOF for single TOF vertex events. Red arrow marks bin with events with exactly two primary tracks matched with hit in TOF, which are used in physics analysis.\newline}\label{fig:NumberOfTofTracksInSingleTofVertex}
\end{minipage}%
\quad\quad%
\begin{minipage}{.4725\textwidth}%
  \centering
  \includegraphics[width=\linewidth]{graphics/eventSelection/TpcTracks/Rmin.pdf}%
  \caption[Distribution of a distance in $\eta-\phi$ space between the BEMC cluster closest to primary TPC track ($R_{\text{min}}$)]{Distribution of a distance in $\eta-\phi$ space between the BEMC cluster closest to primary TPC track matched (filled circle) or not matched (opened circle) with hit in TOF, for single TOF vertex events. Red dashed line indicate matching threshold $R^{\text{match}}_{\text{max}} = 0.05$.}\label{fig:Rmin} %which are expected to reack BEMC
\end{minipage}%
\end{figure}%
%---------------------------

Primary TPC tracks from the single TOF vertex which are matched with TOF are allowed to be also matched with BEMC clusters. Matching with BEMC cluster is claimed if the distance in $\eta-\phi$ space between the BEMC cluster position $(\eta_{\text{clus}},~\phi_{\text{clus}})$ and projected position of the track in BEMC $(\eta_{\text{proj}},~\phi_{\text{proj}})$, defined as
\begin{equation}\label{eq:etaPhiR}
 R=\sqrt{(\eta_{\text{clus}}-\eta_{\text{proj}})^{2} + (\phi_{\text{clus}}-\phi_{\text{proj}})^{2}},
\end{equation}
is less than $R^{\text{match}}_{\text{max}} = 0.05$. Distribution of the distance between the primary TPC track and the closest BEMC cluster is shown in~Fig.~\ref{fig:Rmin}.

However, if there are any primary TPC tracks matched with BEMC cluster and not matched with TOF in the single TOF vertex with two TOF-matched tracks, an event is rejected. Such configuration implies higher-than-2 multiplicity of the real tracks in the vertex, hence an event is unlikely a Central Exclusive Production of two particles.



%---------------------------
\begin{figure}[hb]
\centering
\parbox{0.4725\textwidth}{
  \centering
  \begin{subfigure}[b]{\linewidth}
                \subcaptionbox{\label{fig:NHitsFit}}{\includegraphics[width=\linewidth]{graphics/eventSelection/TpcTracks/NHitsFit.pdf}}
  \end{subfigure}\\
  \begin{subfigure}[b]{\linewidth}\addtocounter{subfigure}{1}
                \subcaptionbox{\label{fig:NHitsFit_to_NHitsPos}}{\includegraphics[width=\linewidth]{graphics/eventSelection/TpcTracks/NHitsFit_to_NHitsPos.pdf}}
  \end{subfigure}
}%
\quad\quad%
\parbox{0.4725\textwidth}{
  \centering
  \begin{subfigure}[b]{\linewidth}\addtocounter{subfigure}{-2}
                \subcaptionbox{\label{fig:NHits_dEdx}}{\includegraphics[width=\linewidth]{graphics/eventSelection/TpcTracks/NHits_dEdx.pdf}}
  \end{subfigure}\\
  \begin{minipage}[t][1.042\linewidth][t]{\linewidth}\vspace{10pt}
    \caption[Comparison of distribution of $N_{\text{hits}}^{\text{fit}}$,~$N_{\text{hits}}^{\text{dE/dx}}$ and $N_{\text{hits}}^{\text{fit}}/N_{\text{hits}}^{\text{poss}}$ in the data and embedded MC]
    {Comparison of distribution of the number of hits used in TPC track reconstruction $N_{\text{hits}}^{\text{fit}}$ (\ref{fig:NHitsFit}), number of hits used in specific energy loss reconstruction $N_{\text{hits}}^{\text{dE/dx}}$ (\ref{fig:NHits_dEdx}) and fraction of number of hits potentially generated by the track and finally used in the reconstruction $N_{\text{hits}}^{\text{fit}}/N_{\text{hits}}^{\text{poss}}$ (\ref{fig:NHitsFit_to_NHitsPos}) in the data and embedded MC. Normalizations of the signal and backgrounds were established from the comparison of $p_{T}^{\text{miss}}$ and $\Delta\theta$ distributions after full selection (without cut on the presented quantity and without exclusivity cut), as described in Sec.~\ref{sec:bkgdSignalNorm}. Red dashed line and red arrow indicate the range of each quantity which is accepted in analysis.}\label{fig:NHits}
  \end{minipage}
}%

\end{figure}
%---------------------------







% %---------------------------
% \begin{figure}[ht!]
% \centering
% \parbox{0.4725\textwidth}{
%   \centering
%   \begin{subfigure}[b]{\linewidth}{
%                 \subcaptionbox{\label{fig:d0}}{\includegraphics[width=\linewidth]{graphics/eventSelection/TpcTracks/d0.pdf}}}
%   \end{subfigure}
% }%
% \quad\quad%
% \parbox{0.4725\textwidth}{%
%   \centering
%   \begin{subfigure}[b]{\linewidth}{
%                 \subcaptionbox{\label{fig:z0}}{\includegraphics[width=\linewidth]{graphics/eventSelection/TpcTracks/z0.pdf}}}
%   \end{subfigure}
% }%
% \caption[Comparison of distribution of $d_{0}$ and $z_{0}$ in the data and embedded MC]
% {Comparison of distribution of the transverse impact parameter $d_{0}$ (\ref{fig:d0}) and the longitudinal impact parameter $z_{0}$ (\ref{fig:z0}) in the data and embedded MC. Normalizations of the signal and backgrounds were established from the comparison of $p_{T}^{\text{miss}}$ and $\Delta\theta$ distributions after full selection (without cut on the presented quantity and without exclusivity cut), as described in Sec.~\ref{sec:bkgdSignalNorm}. Red dashed lines and red arrows indicate the range of each quantity which is accepted in analysis.}\label{fig:d0z0}
% \end{figure}
% %---------------------------





% 
% %---------------------------
% \begin{figure}[ht!]
% \centering
% \parbox{0.4725\textwidth}{
%   \centering
%   \begin{subfigure}[b]{\linewidth}{
%                 \subcaptionbox{\label{fig:RadialDca}}{\includegraphics[width=\linewidth]{graphics/eventSelection/TpcTracks/RadialDCA.pdf}}}
%   \end{subfigure}
% }%
% \quad\quad%
% \parbox{0.4725\textwidth}{%
%   \centering
%   \begin{subfigure}[b]{\linewidth}{
%                 \subcaptionbox{\label{fig:LongitudinalDca}}{\includegraphics[width=\linewidth]{graphics/eventSelection/TpcTracks/LongitudinalDCA.pdf}}}
%   \end{subfigure}
% }%
% \caption[Comparison of distribution of $\text{DCA}(R)$ and $\text{DCA}(z)$ in the data and embedded MC]
% {Comparison of distribution of the distance of closest approach of the track to the vertex in the $xy$-plane $\text{DCA}(R)$ (\ref{fig:RadialDca}) and the the distance of closest approach of the track to the vertex along the $z$-axis $\text{DCA}(z)$ (\ref{fig:LongitudinalDca}) in the data and embedded MC. Normalizations of the signal and backgrounds were established from the comparison of $p_{T}^{\text{miss}}$ and $\Delta\theta$ distributions after full selection (without cut on the presented quantity and without exclusivity cut), as described in Sec.~\ref{sec:bkgdSignalNorm}. Red dashed lines and red arrows indicate the range of each quantity which is accepted in analysis.}\label{fig:DCA}
% \end{figure}
% %---------------------------



%---------------------------
\begin{figure}[ht!]
\centering
\parbox{0.4725\textwidth}{
  \centering
  \begin{subfigure}[b]{\linewidth}{
                \subcaptionbox{\label{fig:TrackEta}}{\includegraphics[width=\linewidth]{graphics/eventSelection/TpcTracks/TrackEta.pdf}}}
  \end{subfigure}
}%
\quad\quad%
\parbox{0.4725\textwidth}{%
  \centering
  \begin{subfigure}[b]{\linewidth}{
                \subcaptionbox{\label{fig:TrackPhi}}{\includegraphics[width=\linewidth]{graphics/eventSelection/TpcTracks/TrackPhi.pdf}}}
  \end{subfigure}
}%
\caption[Comparison of distribution of track $\eta$ and $\phi$ in the data and embedded MC]
{Comparison of the track pseudorapidity $\eta$ (\ref{fig:TrackEta}) and the track azimuthal angle $\phi$ (\ref{fig:TrackPhi}) in the data and embedded MC. Normalizations of the signal and backgrounds were established from the comparison of $p_{T}^{\text{miss}}$ and $\Delta\theta$ distributions after full selection (without cut on the presented quantity and without exclusivity cut), as described in Sec.~\ref{sec:bkgdSignalNorm}. Red dashed lines and red arrows indicate the range of each quantity which is accepted in analysis.}\label{fig:TrackEtaPhi}
\end{figure}
%---------------------------









\subsection{(\ref{enum:CutRpTrks})~RP tracks}\label{sec:C4}


% Clear band of primary proton tracks can be distinguished in the plots.

%---------------------------
\begin{figure}[h]
\centering
\parbox{0.4725\textwidth}{
  \centering
  \begin{subfigure}[b]{\linewidth}
                \subcaptionbox{\label{fig:localAngle2D_X}}{\includegraphics[width=\linewidth,page=1]{graphics/eventSelection/RpTracks/RpTrackCuts_2.pdf}\vspace*{-10pt}}
  \end{subfigure}\\
  \begin{subfigure}[b]{\linewidth}\addtocounter{subfigure}{1}
                \subcaptionbox{\label{fig:localAngle1D_X}}{\includegraphics[width=\linewidth,page=2]{graphics/eventSelection/RpTracks/RpTrackCuts_2.pdf}}
  \end{subfigure}
}%
\quad\quad%
\parbox{0.4725\textwidth}{
  \centering
  \begin{subfigure}[b]{\linewidth}\addtocounter{subfigure}{-2}
                \subcaptionbox{\label{fig:localAngle2D_Y}}{\includegraphics[width=\linewidth,page=3]{graphics/eventSelection/RpTracks/RpTrackCuts_2.pdf}\vspace*{-10pt}}
  \end{subfigure}\\
  \begin{subfigure}[b]{\linewidth}\addtocounter{subfigure}{1}
                \subcaptionbox{\label{fig:localAngle1D_Y}}{\includegraphics[width=\linewidth,page=4]{graphics/eventSelection/RpTracks/RpTrackCuts_2.pdf}}
  \end{subfigure}
}%
\label{fig:localAngleRp}%
\caption[Local angle vs. position of RP tracks matched with true level primary protons.]{Typical correlation between local angle ($y$-axis) and position ($x$-axis) of RP tracks matched with true level primary protons for $x$- (\subref{fig:localAngle2D_X}) and $y$-coordinate (\subref{fig:localAngle2D_Y}), here shown for branch WU. The same events are contained in \subref{fig:localAngle1D_X} and \subref{fig:localAngle1D_Y} for $x$- and $y$-coordinate respectively, where difference between reconstructed local angle and local angle expected from the elastic track is histogrammed. Red lines and arrows visualize cuts imposed on RP tracks for final selection (cuts~\ref{enum:RpLocalAngles}).}
\end{figure}
%---------------------------






\begin{figure}[h]
\centering
\includegraphics[width=.465\textwidth]{graphics/eventSelection/RpTracks/PxPyExclusiveAllMerged.pdf}
%\hfill
\includegraphics[width=.523\textwidth]{graphics/eventSelection/RpTracks/Paper_MandelstamT.pdf}
%
\caption{(left) Merged distributions of diffractively scattered protons momenta $p_y$ vs. $p_x$ in exclusive $h^{+}h^{-}$ events reconstructed with the East and West RP stations, together with the kinematic region used in the measurement marked with the black line. (right) Distributions of measured four momenta transfers at the proton vertices for exclusive $h^{+}h^{-}$ events with all particles in the fiducial phase space are shown for East and West stations with yellow and blue color, respectively.}
\label{fig:rp_hits}
\end{figure}





%%%%%%%%%%%%%%%%%%%%%%%%%%%%%%%%%%%%%%%%%%%%%%%%%%%%%%%%%%%%%%%%%%%%%%%%%%%%%%%%%%%%%%%%%%%%%%%%%%%%%%%%%%%%%%%%%%%%%%%%%%%%%%%%%%%%
\subsection{(\ref{enum:CutDeltaZVx})~TPC-RP \texorpdfstring{$z$}{z}-vertex matching}\label{sec:C5}

In CEP tracks in the central detector and tracks in Roman Pots originate from the same interaction vertex. Measurement of the time of detection of forward protons in RPs gives access to reconstruction of the position of the vertex
\begin{equation}
z_{\text{vtx}}^{\text{RP}} = c\cdot\frac{t^{\text{RP}}_{\text{W}} - t^{\text{RP}}_{\text{E}}}{2}
\end{equation}
independently from TPC, which allows their comparison and rejection of the background if the two values disagree. Time of detection of proton in RP is provided in StMuRpsTrack object - it is an average of all TAC values from PMTs in RPs used to form a track, corrected for the slewing effect and adjusted to have the best correlation with the $z$-position of the vertex measured in TPC, translated to unit of time (all these steps are done at the level of raw data reconstruction). In Fig.~\ref{fig:zVertexRpTpc} the comparisons of the $z_{\text{vtx}}^{\text{RP}}$ and $z_{\text{vtx}}^{\text{TPC}}$ are shown with some preselection cuts applied. A clear signal from the Central Diffraction (and thus CEP) process is manifesting in high correlation of the two values (diagonal in Fig.~\ref{fig:zVertexRpVsTpc}) or significant and relatively narrow peak centered at 0 for the difference of two values (Fig.~\ref{fig:zVertexRpMinusZVertexTpc}). %
%---------------------------
\begin{figure}[ht!]
\centering
\parbox{0.4\textwidth}{
  \centering
  \begin{subfigure}[b]{\linewidth}{
                \subcaptionbox{\label{fig:zVertexRpVsTpc}}{\includegraphics[width=\linewidth]{graphics/eventSelection/zVertexRpVsTpc.pdf}}}
  \end{subfigure}
}
\quad
\parbox{0.545\textwidth}{
  \centering
  \begin{subfigure}[b]{\linewidth}{
                \subcaptionbox{\label{fig:zVertexRpMinusZVertexTpc}}{\includegraphics[width=\linewidth]{graphics/eventSelection/zVertexRpMinusZVertexTpc.pdf}}}
  \end{subfigure}
}%
\caption[Correlation and difference of $z$-vertex position measured in Roman Pots and TPC.]{Correlation (Fig.~\ref{fig:zVertexRpVsTpc}) and difference (Fig.~\ref{fig:zVertexRpMinusZVertexTpc}) of $z$-vertex position measured in Roman Pots and TPC in RP\_CPT2 triggers, after preselection described in the plots.}\label{fig:zVertexRpTpc}
\end{figure}%
%---------------------------
The sum of two Gaussian distributions was fitted to data in Fig.~\ref{fig:zVertexRpMinusZVertexTpc} yielding good description of the distribution of $\Delta z_{\text{vtx}}$ with the width parameters equal $10.3$~cm (CD signal) and $73.9$~cm (pile-up). The first parameter reflects the time resolution of RPs (the $z_{\text{vtx}}^{\text{RP}}$ measurement), as the TPC resolution is much better ($\sim 1$~cm). Value of the second parameter, consistent with $\sqrt{2}\sigma_{z_{\text{vtx}}}\approx\sqrt{2}\cdot52~\text{cm}\approx 73.5$~cm, confirms that the wide distribution under the narrow signal peak is uncorrelated background, in other words forward protons originating from a different vertex than the central tracks. To reject this background without significant loss of the signal, we introduce $3.5\sigma_{\Delta z_{\text{vtx}}}$ cut on $\Delta z_{\text{vtx}}$.


% %---------------------------
% \begin{figure}[ht!]
% % \begin{wrapfigure}{l}{0.475\textwidth}%[ht!]
% \centering%
% \includegraphics[width=0.475\linewidth,page=1]{graphics/eventSelection/DeltaZVx.pdf}%
% % % \includegraphics[width=\linewidth,page=1]{graphics/eventSelection/DeltaZVx.pdf}%
% \caption{Delta z-vx.}\label{fig:DeltaZVx}%
% \end{figure}
% % \end{wrapfigure}
% %---------------------------
%%%%%%%%%%%%%%%%%%%%%%%%%%%%%%%%%%%%%%%%%%%%%%%%%%%%%%%%%%%%%%%%%%%%%%%%%%%%%%%%%%%%%%%%%%%%%%%%%%%%%%%%%%%%%%%%%%%%%%%%%%%%%%%%%%%%




%%%%%%%%%%%%%%%%%%%%%%%%%%%%%%%%%%%%%%%%%%%%%%%%%%%%%%%%%%%%%%%%%%%%%%%%%%%%%%%%%%%%%%%%%%%%%%%%%%%%%%%%%%%%%%%%%%%%%%%%%%%%%%%%%%%%
\subsection{(\ref{enum:CutBbcLarge})~BBC-large signal veto}\label{sec:C6}

At the trigger level a veto on signal in small BBC detectors was used. During offline analysis we found that the non-exclusive background can be reduced if an additional veto on signal in large BBC detectors is added. It is connected with the fact that vast majority of selected RP\_CPT2 triggers were from the central diffraction process to which CEP belongs. Many of central diffraction events have particles produced in the rapidity region outside the TPC and TOF acceptance, some hitting large BBC tiles. Presence of signal in large BBC is therefore a signature of background or a pile-up interaction.

The response of large BBC tiles is different from that of small BBC tiles, as shown in sample plots in Fig.~\ref{fig:sampleBbcResponse} (similar distributions for all channels can be found in Appendix~\ref{appendix:bbc}). Typically in small BBC tiles a peak visible in ADC distribution around $100-150$ (Figs.~\ref{fig:sampleBbcSmallAdcVsTac},\ref{fig:sampleBbcSmallAdc}), a signature of good separation of the electronics noise and signal from the ionizing particle. No such feature is observed in corresponding distribution for large BBC tile (Figs.~\ref{fig:sampleBbcLargeAdcVsTac},\ref{fig:sampleBbcLargeAdc}), which can be explained by the difference in geometry (in size) of small and large tiles. In large BBC tiles the path that scintillation light must travel to reach PMT is much longer in comparison to smal BBC tiles (multiple reflections on the main tile surface due to small thickness of the tile) therefore it is highly attenuated and extended in time. This is possible reason of lack of signal peak in the ADC distribution in large BBC tile spectrum (Fig.~\ref{fig:sampleBbcLargeAdc}), as well as the late-TAC (TAC$<\sim600$, ADC$<100$) tail in the ADC vs. TAC spectrum (slewing effect, Fig.~\ref{fig:sampleBbcLargeAdcVsTac}). Nevertheless, the above features of BBC-large response does not disqualify this detector from being used as a veto detector, as in this case lower efficiency of the detector only reduce the background rejection power.


%---------------------------
\begin{figure}[h]
\centering
\parbox{0.4725\textwidth}{
  \centering
  \begin{subfigure}[b]{\linewidth}
                \subcaptionbox{\label{fig:sampleBbcSmallAdcVsTac}}{\includegraphics[width=\linewidth,page=1]{graphics/eventSelection/bbc/Bbc_ADCvsTAC_collidingBunches.pdf}}
  \end{subfigure}\\
  \begin{subfigure}[b]{\linewidth}\addtocounter{subfigure}{1}
                \subcaptionbox{\label{fig:sampleBbcSmallAdc}}{\includegraphics[width=\linewidth,page=1]{graphics/eventSelection/bbc/Bbc_ADC.pdf}}
  \end{subfigure}
}%
\quad\quad%
\parbox{0.4725\textwidth}{
  \centering
  \begin{subfigure}[b]{\linewidth}\addtocounter{subfigure}{-2}
                \subcaptionbox{\label{fig:sampleBbcLargeAdcVsTac}}{\includegraphics[width=\linewidth,page=17]{graphics/eventSelection/bbc/Bbc_ADCvsTAC_collidingBunches.pdf}}
  \end{subfigure}\\
  \begin{subfigure}[b]{\linewidth}\addtocounter{subfigure}{1}
                \subcaptionbox{\label{fig:sampleBbcLargeAdc}}{\includegraphics[width=\linewidth,page=17]{graphics/eventSelection/bbc/Bbc_ADC.pdf}}
  \end{subfigure}
}%
\caption[Sample BBC-small and BBC-large response in zero-bias triggers.]{Sample BBC-small (left column) and BBC-large (right column) response in zero-bias data. Top row shows TAC vs. ADC distributions, bottom row shows projection of the corresponding two-dimensional ditribution on $x$-axis (ADC) in the TAC range quoted in the legend, for both abort gaps and colliding bunches. Red lines and arrows indicate thresholds for a signal in presented channels.}\label{fig:sampleBbcResponse}
\end{figure}
%---------------------------


Each channel of the BBC-large has different response to signal from ionizing particle, as well as different level of noise. We decided to set up a signal threshold for each channel based on a study of the noise in abort gaps (in zero-bias data). This noise, in principle, should be solely the electronics noise. We checked for each channel the probability to detect a signal with ADC above certain threshold and with TAC contained within 100 and 2400 (the same window is deafult for small BBC). The result is shown in Fig.~\ref{fig:bbcLargeThresholds}. Next, we established final ADC thresholds in each BBC-large channel by requiring that the noise in BBC-large would cause a veto in maximally $3.5\%$ of events. To transform it to $\text{ADC}_{thr}$ we first assumed that the noise is uncorrelated between the channels. With this assumption one can connect the probability of the veto in whole BBC-large detector (east and west) caused by noise $\mathcal{P}_{\text{veto}}^{\text{noise}}$ with the probability of the signal induced by noise in single BBC-large channel $\mathcal{P}_{i,\text{sig}}^{\text{noise}}$:
\begin{equation}\label{eq:bbcNoise1}
 \mathcal{P}_{\text{veto}}^{\text{noise}} = 1-\mathcal{P}_{!\text{veto}}^{\text{noise}} = 1-\left( 1-\mathcal{P}_{i,\text{sig}}^{\text{noise}} \right)^{N^{\text{BBC}}_{\text{ch}}}.
\end{equation}
In the equation above $N^{\text{BBC}}_{\text{ch}}$ denotes number of active channels in BBC-large. From plots contained in Appendix~\ref{appendix:bbc} one can read that there were 14 active channels in BBC-large. 2 dead channels were found on the west side (40 and 42). By transforming Eq.~\ref{eq:bbcNoise1} to the form presented below we can calculate the threshold probability for a single BBC-large channel:
\begin{equation}\label{eq:bbcNoise2}
 \mathcal{P}_{i,\text{sig}}^{\text{noise}} = 1-\sqrt[N^{\text{BBC}}_{\text{ch}}]{1-\mathcal{P}_{\text{veto}}^{\text{noise}}} = 1-\sqrt[14]{1-0.035} \approx 0.0025.
\end{equation}
In the last step we translated this number to ADC threshold for each channel of BBC-large. For this purpose we used Fig.~\ref{fig:bbcLargeThresholds}. The $x$-axis projection of the crossing point of each color line with the $y$-axis value of 0.0025 defines $\text{ADC}_{thr}$ for each particular channel. These numbers are listed in Tab.~\ref{tab:bbcLargeThresholds}. The event was dropped from analysis if any of the BBC-large channels registered signal of strength $\text{ADC}_{i}>\text{ADC}_{i,thr}$ and $100<\text{TAC}_{i}<2400$.



\begin{table}[h]
	\begin{minipage}{0.65\linewidth}
		\centering
		\includegraphics[width=\linewidth]{graphics/eventSelection/bbc/BbbLargeThreshold.pdf}
		\captionof{figure}[Probability of false BBC-large signal (noise-induced).]{Percentage of events in abort gaps from zero-bias triggers with the ADC counts larger than the ADC threshold given in the $x$-axis, for each BBC-large channel. Measured points with statistical uncertainties are connected with a smooth line of corresponding color for better visualization.}
		\label{fig:bbcLargeThresholds}
	\end{minipage}\hfill
	\begin{minipage}{0.3\linewidth}
		\centering
		\begin{tabular}{c|c||c|c}
			\multicolumn{2}{c||}{East} & \multicolumn{2}{c}{West} \\ \hline
			$i$  & $\text{ADC}_{\text{thr}}$ & $i$  & $\text{ADC}_{\text{thr}}$ \\ \hline
			16 & 27 & 40 & 0 \\
			17 & 30 & 41 & 31 \\
			18 & 26 & 42 & 0 \\
			19 & 37 & 43 & 14 \\
			20 & 25 & 44 & 29 \\
			21 & 55 & 45 & 30 \\
			22 & 43 & 46 & 33 \\
			23 & 27 & 47 & 22 \\
		\end{tabular}
		\caption[Offline ADC thresholds in BBC-large.]{Offline ADC thresholds in BBC-large.\newline\newline\newline\newline\newline\newline\newline\newline}
		\label{tab:bbcLargeThresholds}
	\end{minipage}

\end{table}

Observation of high purification of CEP sample with described BBC-large veto in the data from run 15 was helpful to improve the CEP trigger for run 17. The improved trigger called RP\_CPT2noBBCL was similar to RP\_CPT2 with an addition of BBC-large veto using ADC threshold of 50.

%%%%%%%%%%%%%%%%%%%%%%%%%%%%%%%%%%%%%%%%%%%%%%%%%%%%%%%%%%%%%%%%%%%%%%%%%%%%%%%%%%%%%%%%%%%%%%%%%%%%%%%%%%%%%%%%%%%%%%%%%%%%%%%%%%%%



%%%%%%%%%%%%%%%%%%%%%%%%%%%%%%%%%%%%%%%%%%%%%%%%%%%%%%%%%%%%%%%%%%%%%%%%%%%%%%%%%%%%%%%%%%%%%%%%%%%%%%%%%%%%%%%%%%%%%%%%%%%%%%%%%%%%
\subsection{(\ref{enum:CutTofClusters})~TOF clusters limit}\label{sec:C7}

The TOF is mainly used to distinguish real TPC tracks from the fakes, as well as it helps to identify particles. However, we also used it to reject non-CEP events in which the TPC tracks were not reconstructed or were not successfully matched to TOF hit. For this we introduced a concept of a TOF cluster - a group of offline TOF hits close in space and time. We expect that such cluster of hits is induced by the single primary particle, eventually associated with the secondaries (e.g. delta rays).

We define a TOF cluster as a group of reconstructed TOF hits with the ($\phi$, $\eta$) space distance $R$ to neighbouring hit (defined similarly to Eq.~\eqref{eq:etaPhiR} not larger than 0.1 and with the time distance to the same hit $\Delta t$ not larger than 1.5~ns. In other words, TOF clusters are formed by the offline hits that form at least one pair with the other hit in the cluster satisfying
\begin{equation}
 R<0.1,~~~~~~\Delta t<1.5~\text{ns}.
\end{equation}
Per event no more than 1 additional TOF cluster was allowed, thus in total the number of reconstructed TOF clusters $N^{\text{TOF}}_{\text{clstrs}}$ could not exceed 3.

%---------------------------
\begin{figure}[ht!]
% \begin{wrapfigure}{l}{0.475\textwidth}%[ht!]
\centering%
\includegraphics[width=0.475\linewidth,page=1]{graphics/eventSelection/NTofClusters.pdf}%
% % \includegraphics[width=\linewidth,page=1]{graphics/eventSelection/NTofClusters.pdf}% 
\caption{NTofClusters.}\label{fig:NTofClusters}%
\end{figure}
% \end{wrapfigure}
%---------------------------

\subsection{(\ref{enum:CutPid})~Particle identification}\label{subsec:pidCuts}\label{sec:C8}

Particles were identified using combined information from the TPC ($dE/dx$) and TOF (time of hit detection in the TOF subsystem). Merging informations from two sources led to reduction of misidentifications, as well as gave access to higher kaon and proton momentum range where $dE/dx$ of different species overlap.

Compatibility of track $dE/dx$ with that expected from particle of type $X$ was determined using the quantity $n\sigma_{X}$ widely used at STAR, defined as
\protect \begin{equation}\label{eq:nSigmaDef} n\sigma_{X} =  \ln{\left[(dE/dx)^\text{measured} / (dE/dx)_{X}^\text{theory}\right]}~~/~~\sigma_{dE/dx}, \end{equation}
%
where $(dE/dx)^\text{measured}$ is the ionization energy loss of the TPC track, $(dE/dx)_{X}^\text{theory}$ is the Bethe-Bloch~\cite{Bichsel} expectation for the given particle type ($X=\pi$, $K$, $p$) at reconstructed track momentum, and $\sigma_{dE/dx}$ is the statistical uncertainty of $\ln{(dE/dx)^\text{measured}}$. Quantity $n\sigma_{X}$ is in fact a pull: $(dE/dx)^\text{measured}$ is (in first order) an average over $\text{Landau}\otimes\text{normal}$-distributed $dE/dx$ of single TPC hits forming the track, hence according to the central limit theorem\footnote{Keeping in mind that it assumes finite mean and variance of the distribution that summed components follow.} the $(dE/dx)^\text{measured}$ is distributed log-normally and $\ln{(dE/dx)^\text{measured}}$ - normally. From $n\sigma_{X}$ of the two tracks the $\chi^{2}$ statistic for a $XX$ pair hypothesis was calculated:
%
\begin{equation}\label{eq:chiSqDef}\chi^{2}_{dE/dx}(XX) = \left(n\sigma_{X}^{\text{trk1}}\right)^{2} + \left(n\sigma_{X}^{\text{trk2}}\right)^{2}.\end{equation}
%
Sometimes we also quote $n\sigma^{\text{pair}}$ quantity (which is no longer a Gaussian pull) connected with $\chi^{2}$ through relation
%
\begin{equation}\label{eq:nSigmaPairDef}n\sigma^{\text{pair}}_{X} = \sqrt{\chi^{2}_{dE/dx}(XX)} = \sqrt{\left(n\sigma_{X}^{\text{trk1}}\right)^{2} + \left(n\sigma_{X}^{\text{trk2}}\right)^{2}}.\end{equation}
%
The time of detection of particle in the TOF system was used to reconstruct its squared mass $m^{2}_{\text{TOF}}$. For this purpose the time of primary interaction is typically used (''start time``), reconstructed by detecting fragments of dissociated beam particles in VPD detectors on both sides of the interaction point\footnote{Time measured from protons in the RP detectors cannot be used because RP readout runs on independent clock from that used by VPD and TOF.}. However, it is not accessible in CEP as the initial protons survive the interaction intact. We therefore assumed that both central tracks are of the same type which is natural consequence of quantum number conservation. With this assumption the time difference between TOF hits and measured tracks' momenta and lengths of helical paths between the primary vertex and TOF then allow to calculate $m^{2}_{\text{TOF}}$. The derivation of formula used to obtain $m^{2}_{\text{TOF}}$ is presented in Appendix~\ref{appendix:squaredMass}.

Particle identification involved a few steps. First, the $pp$ hypothesis was verified:
\begin{equation}\label{eq:pidPPbar}\lefteqn{\overbrace{\phantom{\chi^{2}_{dE/dx}(pp)<9\;\;\; \& \;\;\; m^{2}_{\text{TOF}} > 0.6~\mbox{GeV}^{2}}}^{\text{likely}~pp}}\chi^{2}_{dE/dx}(pp)<9\;\;\; \& \;\;\; \underbrace{m^{2}_{\text{TOF}} > 0.6~\mbox{GeV}^{2}\;\;\; \& \;\;\; \chi^{2}_{dE/dx}(\pi\pi)>9\;\;\; \& \;\;\; \chi^{2}_{dE/dx}(KK)>9}_{\text{unlikely}~\pi\pi~\text{or}~KK}.\end{equation}
If any of above was not satisfied, the pair was checked for compatibility with $KK$ hypothesis:
%
\begin{equation}\label{eq:pidKK}%
\lefteqn{\overbrace{\phantom{\chi^{2}_{dE/dx}(KK)<9\;\;\; \& \;\;\; m^{2}_{\text{TOF}} > 0.15~\mbox{GeV}^{2}}}^{\text{likely}~KK}}\chi^{2}_{dE/dx}(KK)<9\;\;\; \& \;\;\; \underbrace{m^{2}_{\text{TOF}} > 0.15~\mbox{GeV}^{2}\;\;\; \& \;\;\; \chi^{2}_{dE/dx}(\pi\pi)>9}_{\text{unlikely}~\pi\pi}\;\;\; \& \;\;\; \underbrace{\chi^{2}_{dE/dx}(pp)>9}_{\text{unlikely}~pp}.
\end{equation}
%
In case the pair was neither recognized as $p\bar{p}$ or $K^{+}K^{-}$, it was assumed to be a $\pi^{+}\pi^{-}$ pair if the $dE/dx$ of positive and negative charge track was consistent with pion hypothesis at $3\sigma$ level:
\begin{equation}\label{eq:pidPiPi}|n\sigma_{\pi}^{\text{trk1}}|<3\;\;\; \& \;\;\; |n\sigma_{\pi}^{\text{trk2}}|<3.\end{equation}




\begin{figure}[h]
\centering
\parbox{0.4725\textwidth}{
  \centering
  \begin{subfigure}[b]{\linewidth}
                \subcaptionbox{\label{fig:SqRootNSigma2D_a}}{\includegraphics[width=1.05\linewidth,page=1]{graphics/eventSelection/pid/PidSelector_SqRootNSigma2D.pdf}}
  \end{subfigure}\\
  \begin{subfigure}[b]{\linewidth}\addtocounter{subfigure}{1}
                \subcaptionbox{\label{fig:SqRootNSigma2D_c}}{\includegraphics[width=1.05\linewidth,page=3]{graphics/eventSelection/pid/PidSelector_SqRootNSigma2D.pdf}}
  \end{subfigure}
}%
\quad\quad%
\parbox{0.4725\textwidth}{
  \centering
  \begin{subfigure}[b]{\linewidth}\addtocounter{subfigure}{-2}\vspace*{-13pt}
                \subcaptionbox{\label{fig:SqRootNSigma2D_b}}{\includegraphics[width=1.05\linewidth,page=2]{graphics/eventSelection/pid/PidSelector_SqRootNSigma2D.pdf}}
  \end{subfigure}\\
  \begin{minipage}[t][1.042\linewidth][t]{\linewidth}\vspace{10pt}
    \caption[$n\sigma^{\text{pair}}_{X}$ vs. $n\sigma^{\text{pair}}_{Y}$.]{Two-dimensional distributions of $n\sigma^{\text{pair}}_{\pi}$ vs.~$n\sigma^{\text{pair}}_{K}$ (\subref{fig:SqRootNSigma2D_a}), $n\sigma^{\text{pair}}_{\pi}$ vs.~$n\sigma^{\text{pair}}_{p}$  (\subref{fig:SqRootNSigma2D_b}) and $n\sigma^{\text{pair}}_{K}$ vs.~$n\sigma^{\text{pair}}_{p}$  (\subref{fig:SqRootNSigma2D_c}) for exclusive event candidates after full event selection except PID cuts (except cuts~\ref{enum:CutPid}). Dashed lines indicate the value of $n\sigma^{\text{pair}}$ which is used in pair identification~\ref{enum:CutPid} ($n\sigma^{\text{pair}}_{X}=9$ which is equivalent to $\chi^{2}(XX)=9$).}\label{fig:SqRootNSigma2D}
  \end{minipage}
}%

\end{figure}
%--------------------------- 


In Fig.~\ref{fig:SqRootNSigma2D} we present two-dimensional distributions of $n\sigma^{\text{pair}}$ variables which help better undestand the behaviour and aim of $n\sigma^{\text{pair}}$ ($\chi^{2}$) cuts in Eqs.~\eqref{eq:pidPPbar}, \eqref{eq:pidKK}. Regions of enriched population of specific pair species are appropriately labeled. Similar connections between $n\sigma^{\text{pair}}$ and $m^{2}_{\text{TOF}}$ are shown in Fig.~\ref{fig:mSqVsNSigmaPair}.
 


% %--------------------------- 
% \begin{figure}[ht!]
% \centering 
% \parbox{0.315\textwidth}{
%   \centering
%   \begin{subfigure}[b]{\linewidth}{
%                 \subcaptionbox{\label{fig:SqRootNSigmaPionVsKaon}}{\includegraphics[width=\linewidth]{graphics/eventSelection/SqRootNSigmaPionVsKaon.pdf}}}
%   \end{subfigure}
% }
% \quad
% \parbox{0.315\textwidth}{
%   \centering
%   \begin{subfigure}[b]{\linewidth}{
%                 \subcaptionbox{\label{fig:SqRootNSigmaPionVsProton}}{\includegraphics[width=\linewidth]{graphics/eventSelection/SqRootNSigmaPionVsProton.pdf}}}
%   \end{subfigure}
% }
% \quad
% \parbox{0.315\textwidth}{
%   \centering
%   \begin{subfigure}[b]{\linewidth}{
%                 \subcaptionbox{\label{fig:SqRootNSigmaKaonVsProton}}{\includegraphics[width=\linewidth]{graphics/eventSelection/SqRootNSigmaKaonVsProton.pdf}}}
%   \end{subfigure}
% }%
% \caption{Correlation between $n\sigma^{\text{pair}}$ from TPC for $\pi^{+}\pi^{-}$, $K^{+}K^{-}$ and $p\bar{p}$.} 
% \end{figure}
% %---------------------------
% 
% 
 
\begin{figure}[ht!]
  \centering
  \begin{tabular}{@{}p{0.47\linewidth}@{\quad\quad}p{0.47\linewidth}@{}}
    \subfigimg[width=\linewidth,page=1]{~~~~~~~~~~~a)}{graphics/eventSelection/pid/PidSelector_SqMassTofVsSqRootNSigma_pion.pdf} &
    \subfigimg[width=\linewidth,page=1]{~~~~~~~~~~~c)}{graphics/eventSelection/pid/SqMassTofVsSqRootNSigma_pion.pdf} \\[-10pt]
    \subfigimg[width=\linewidth,page=1]{~~~~~~~~~~~d)}{graphics/eventSelection/pid/PidSelector_SqMassTofVsSqRootNSigma_kaon.pdf} &
    \subfigimg[width=\linewidth,page=1]{~~~~~~~~~~~f)}{graphics/eventSelection/pid/SqMassTofVsSqRootNSigma_kaon.pdf} \\[-10pt]
    \subfigimg[width=\linewidth,page=1]{~~~~~~~~~~~g)}{graphics/eventSelection/pid/PidSelector_SqMassTofVsSqRootNSigma_proton.pdf} &
    \subfigimg[width=\linewidth,page=1]{~~~~~~~~~~~i)}{graphics/eventSelection/pid/SqMassTofVsSqRootNSigma_proton.pdf}    
  \end{tabular}\vspace*{-5pt}
  \caption[$n\sigma^{\text{pair}}_{X}$ vs. $m^{2}_{\text{TOF}}$.]{Two-dimensional distributions of $n\sigma^{\text{pair}}_{\pi}$ (top row), $n\sigma^{\text{pair}}_{K}$ (middle row) and $n\sigma^{\text{pair}}_{p}$ (bottom row) vs. $m^{2}_{\text{TOF}}$. The left column contains all clean BBC-large events with single TOF vertex and two opposite sign TOF-matched tracks (passing cuts~\ref{enum:CutPrimVx}, \ref{enum:TpcTofMatched}, \ref{enum:TpcOppoSign} and~\ref{enum:CutBbcLarge}), which provides excellent statistics to see the signatures or pairs of specific ID. The right column cantains exclusive event candidates after full event selection except PID cuts (except cuts~\ref{enum:CutPid}). Dashed red line and arrow indicate the cut imposed on plotted quantities which are used to select exclusive pairs of given particle species (keep in mind that these are not the only cuts).}\label{fig:mSqVsNSigmaPair}
\end{figure}






%%%%%%%%


% \begin{figure}[tbp]
% \centering
% \includegraphics[width=.49\textwidth,page=1]{graphics/eventSelection/pid/Chi2NSigma_pion.pdf}
% \hfill
% \includegraphics[width=.49\textwidth,page=1]{graphics/eventSelection/pid/SqMassTof_pion.pdf}
% \newline
% \newline
% \includegraphics[width=.49\textwidth,page=1]{graphics/eventSelection/pid/Chi2NSigma_kaon.pdf}
% \hfill
% \includegraphics[width=.49\textwidth,page=1]{graphics/eventSelection/pid/SqMassTof_kaon.pdf}
% \newline
% \newline
% \includegraphics[width=.49\textwidth,page=1]{graphics/eventSelection/pid/Chi2NSigma_proton.pdf}
% \hfill
% \includegraphics[width=.49\textwidth,page=1]{graphics/eventSelection/pid/SqMassTof_proton.pdf}
% % 
% \caption{Raw distributions of $\chi^{2}_{dE/dx}$ (left column) and $m^{2}_{\text{TOF}}$ (right column) for exclusive $\pi^+\pi^-$ (top row), $K^+K^-$ (middle row) and $p\bar{p}$ (bottom row) candidates after full event selection. Dashed red line and arrow indicate the value of cut imposed on plotted quantity to select exclusive pairs of given particle species.}
% \label{fig:pid_plots} 
% \end{figure}



\begin{figure}[ht!]
  \centering
  \begin{tabular}{@{}p{0.49\linewidth}@{\quad}p{0.49\linewidth}@{}}
    \subfigimg[width=\linewidth,page=1]{~~~~~~~~~~~~~~~~~~~~~~~~~~~~~~~~~~~~~~~~~~~~~~~~~~~~~~~~~~~~~~a)}{graphics/eventSelection/pid/Chi2NSigma_pion.pdf} &
    \subfigimg[width=\linewidth,page=1]{~~~~~~~~~~~~~~~~~~~~~~~~~~~~~~~~~~~~~~~~~~~~~~~~~~~~~~~~~~~~~~c)}{graphics/eventSelection/pid/SqMassTof_pion.pdf} \\
    \subfigimg[width=\linewidth,page=1]{~~~~~~~~~~~~~~~~~~~~~~~~~~~~~~~~~~~~~~~~~~~~~~~~~~~~~~~~~~~~~~d)}{graphics/eventSelection/pid/Chi2NSigma_kaon.pdf} &
    \subfigimg[width=\linewidth,page=1]{~~~~~~~~~~~~~~~~~~~~~~~~~~~~~~~~~~~~~~~~~~~~~~~~~~~~~~~~~~~~~~f)}{graphics/eventSelection/pid/SqMassTof_kaon.pdf} \\
    \subfigimg[width=\linewidth,page=1]{~~~~~~~~~~~~~~~~~~~~~~~~~~~~~~~~~~~~~~~~~~~~~~~~~~~~~~~~~~~~~~g)}{graphics/eventSelection/pid/Chi2NSigma_proton.pdf} &
    \subfigimg[width=\linewidth,page=1]{~~~~~~~~~~~~~~~~~~~~~~~~~~~~~~~~~~~~~~~~~~~~~~~~~~~~~~~~~~~~~~i)}{graphics/eventSelection/pid/SqMassTof_proton.pdf}    
  \end{tabular}
  \caption[$\chi^{2}_{dE/dx}$ and $m^{2}_{\text{TOF}}$ for exclusive $\pi^+\pi^-$, $K^+K^-$ and $p\bar{p}$ candidates.]{Raw distributions of $\chi^{2}_{dE/dx}$ (left column) and $m^{2}_{\text{TOF}}$ (right column) for exclusive $\pi^+\pi^-$ (top row), $K^+K^-$ (middle row) and $p\bar{p}$ (bottom row) candidates after full event selection. Dashed red line and arrow indicate the value of cut imposed on plotted quantity to select exclusive pairs of given particle species. Presented distributions were obtained after all the cuts were applied, except the cut on presented quantity in the last step in PID algorithm used to select pairs of given species.}\label{fig:pid_plots}
\end{figure}

%---------------------------------------------------------------------------------------------------------------

\subsection{(\ref{enum:CutMissingPt})~Exclusivity cut (missing \texorpdfstring{$p_{\text{T}}$}{pT} cut)}\label{subsec:ptMiss}\label{sec:C9}%

The most important cut which is used in this analysis to select events of exclusively produced pairs of particles is the missing transverse momentum, or the total transverse momentum cut. It benefits from detection and reconstruction of the forward proton in RP detectors - a rare capability among high energy physics experiments which STAR provides. The observable $p_{\text{T}}^{\text{miss}}$ used to select exclusive event is defined as
\begin{equation}\label{eq:missingPt}
 p_{\text{T}}^{\text{miss}} = \Big( \vec{p}_{p'}^{\hspace*{2pt}\text{E}} + \vec{p}_{h^{+}} + \vec{p}_{h^{-}} + \vec{p}_{p'}^{\hspace*{2pt}\text{W}} \Big)_{\text{T}} = \sqrt{\Big(p_{x}^{\text{miss}}\Big)^{2} + \Big(p_{y}^{\text{miss}}\Big)^{2}},
\end{equation}
with the other total momentum components defined analogously:\\
\begin{tabulary}{\textwidth}{LL}
\begin{equation}\label{eq:missingPx}
 p_{x}^{\text{miss}} = \Big( \vec{p}_{p'}^{\hspace*{2pt}\text{E}} + \vec{p}_{h^{+}} + \vec{p}_{h^{-}} + \vec{p}_{p'}^{\hspace*{2pt}\text{W}} \Big)_{x},
\end{equation}~~~~~~~~~~~~~~~~~~~~~
&
\begin{equation}\label{eq:missingPy}
 p_{y}^{\text{miss}} = \Big( \vec{p}_{p'}^{\hspace*{2pt}\text{E}} + \vec{p}_{h^{+}} + \vec{p}_{h^{-}} + \vec{p}_{p'}^{\hspace*{2pt}\text{W}} \Big)_{y}.
\end{equation}~~~~~~~~~~~~~~~~~~~~~
\end{tabulary}


Figure~\ref{fig:PxPyCentralTrksVsProtons} visualize the (anti-)correlation between the momentum components of the forward system (sum of two forward protons momenta) and the central system (sum of two central tracks momenta). The enhanced band at anti-diagonal restricted by dashed lines contains events balanced in momentum, a signature of exclusivity. Events outside this band are the non exclusive backgrounds, in most cases Central Diffraction events with some particles undected (due to detector inefficiency or produced outside acceptance). Slight horizontal enhancement in all distributions around $[\vec{p}^{\hspace*{2pt}\text{W}}_{p'}+\vec{p}^{\hspace*{2pt}\text{E}}_{p'}]_{x} = [\vec{p}^{\hspace*{2pt}\text{W}}_{p'}+\vec{p}^{\hspace*{2pt}\text{E}}_{p'}]_{x} =0$ is a signature of the elastic proton-proton scattering background with some non-elastic pile-up interaction which mimics the CEP event. All these backgrounds are reasonably low after the exclusivity cut, as described in Sec.~\ref{sec:nonExclBkgd}.

The momentum balance is shown one-dimensionally in Fig.~\ref{fig:MissingPxPy}, with the sum of $x$- and $y$-components of momentum shown repectively in the left and right column for each analyzed particle species. The sum of signal and background (both assumed to be described a Gaussian) was fitted to $p_{x}^{\text{miss}}$ and $p_{y}^{\text{miss}}$ distributions. Results of the fit are given in each subfigure. One can notice that the widths of Gaussian functions representing the exclusive signal are consistent among species and amount $\sigma_{p_{x}^{\text{miss}}}=27.4$~MeV for the $x$-component of total momentum, and $\sigma_{p_{y}^{\text{miss}}}=28.1$~MeV for the $y$-component of total momentum, taking the values of the lowest statistical uncertainty - for $\pi^{+}\pi^{-}$. These values are measures of the total momentum resolution respectively for $p_{x}^{\text{miss}}$ and $p_{y}^{\text{miss}}$. Having these number it is possible to form an elliptical on the missing momentum:

\begin{equation}\label{eq:ptMissEllipse}%
\left(\frac{p_{x}^{\text{miss}}}{\sigma_{p_{x}^{\text{miss}}}}\right)^{2} + \left(\frac{p_{y}^{\text{miss}}}{\sigma_{p_{y}^{\text{miss}}}}\right)^{2} < n_{\text{cut}}^{2}
\end{equation}
%
where $n_{\text{cut}}$ is the parameter denoting radius of limiting ellipsis in units of standard deviations of distributions of total momentum components (resolutions). Since these resolutions are nearly identical ($\sigma_{p_{x}^{\text{miss}}} = \sigma_{p_{y}^{\text{miss}}} = \sigma_{p_{x,y}^{\text{miss}}}$) such cut can be reduced (multiplying Ineq.~\ref{eq:ptMissEllipse} by $\sigma_{p_{x,y}^{\text{miss}}}^{2}$) to one-dimensional cut on a single quantity:

\begin{equation}%
\left(p_{x}^{\text{miss}}\right)^{2} + \left(p_{y}^{\text{miss}}\right)^{2} < \Big(n_{\text{cut}}\cdot\sigma_{p_{x,y}^{\text{miss}}}\Big)^{2}~~~~~~~\xrightarrow[~]{\sqrt{~}}~~~~~~~p_{\text{T}}^{\text{miss}} < n_{\text{cut}}\cdot\sigma_{p_{x,y}^{\text{miss}}}~~~~
\end{equation}%
%
In current analysis the $n_{\text{cut}}$ was set to 2.5, which translates to threshold value $2.5\times 30~\text{MeV} = 75$~MeV.



\begin{figure}[ht!]\vspace*{-20pt}
  \centering
  \begin{tabular}{@{}p{0.47\linewidth}@{\quad\quad}p{0.47\linewidth}@{}}
    \subfigimg[width=\linewidth,page=1]{~~~~~~~~~~~~~~~~a)}{graphics/eventSelection/exclusivity/PxCentralTrksVsProtons_pion.pdf} &
    \subfigimg[width=\linewidth,page=1]{~~~~~~~~~~~~~~~~~~~~~~c)}{graphics/eventSelection/exclusivity/PyCentralTrksVsProtons_pion.pdf} \\[-10pt]
    \subfigimg[width=\linewidth,page=1]{~~~~~~~~~~~~~~~~d)}{graphics/eventSelection/exclusivity/PxCentralTrksVsProtons_kaon.pdf} &
    \subfigimg[width=\linewidth,page=1]{~~~~~~~~~~~~~~~~~~~~~~f)}{graphics/eventSelection/exclusivity/PyCentralTrksVsProtons_kaon.pdf} \\[-10pt]
    \subfigimg[width=\linewidth,page=1]{~~~~~~~~~~~~~~~~g)}{graphics/eventSelection/exclusivity/PxCentralTrksVsProtons_proton.pdf} &
    \subfigimg[width=\linewidth,page=1]{~~~~~~~~~~~~~~~~~~~~~~i)}{graphics/eventSelection/exclusivity/PyCentralTrksVsProtons_proton.pdf}    
  \end{tabular}\vspace*{-5pt}
    \caption[Two-dimensional distributions of sum of forward protons momenta and sum of central tracks momenta for exclusive $\pi^+\pi^-$ (top row), $K^+K^-$ (middle row) and $p\bar{p}$ (bottom row) event candidates.]{Two-dimensional distributions of sum of forward protons momenta ($x$-axis) and sum of central tracks momenta ($y$-axis) for exclusive $\pi^+\pi^-$ (top row), $K^+K^-$ (middle row) and $p\bar{p}$ (bottom row) event candidates after full event selection, except the exclusivity cut~\ref{enum:CutMissingPt}. Left and right column shows correlation of respectively $x$- and $y$-component of tracks' momenta. Anti-diagonal representing perfect momentum balance of the central and forward system is limited with dashed lines extending by $\pm2.5\sigma$  ($\sigma\approx 30$~MeV) around the anti-diagonal. Three distinct horizontal regions in plots on the right hand side correspond to different forward proton configurations: elastic-like (protons in branches EU\&WD or ED\&WU, $\left|[\vec{p}^{\hspace*{2pt}\text{W}}_{p'}+\vec{p}^{\hspace*{2pt}\text{E}}_{p'}]_{y}\right| < 0.2$~GeV) and anti-elastic configuration (protons in branches ED\&WD or EU\&WU, $\left|[\vec{p}^{\hspace*{2pt}\text{W}}_{p'}+\vec{p}^{\hspace*{2pt}\text{E}}_{p'}]_{y}\right| > 0.4$~GeV).}\label{fig:PxPyCentralTrksVsProtons}
\end{figure}




\begin{figure}[ht!]
  \centering
  \begin{tabular}{@{}p{0.49\linewidth}@{\quad\quad}p{0.49\linewidth}@{}}
    \subfigimg[width=\linewidth,page=1]{~~~~~~~~~~~~~~~~~~~~~~~~~~~~~~~~~~~~~~~~~~~~~~~~~~~~~~~~~~~~~a)}{graphics/eventSelection/exclusivity/MissingPx_pion.pdf} &
    \subfigimg[width=\linewidth,page=1]{~~~~~~~~~~~~~~~~~~~~~~~~~~~~~~~~~~~~~~~~~~~~~~~~~~~~~~~~~~~~~b)}{graphics/eventSelection/exclusivity/MissingPy_pion.pdf} \\
    \subfigimg[width=\linewidth,page=1]{~~~~~~~~~~~~~~~~~~~~~~~~~~~~~~~~~~~~~~~~~~~~~~~~~~~~~~~~~~~~~c)}{graphics/eventSelection/exclusivity/MissingPx_kaon.pdf} &
    \subfigimg[width=\linewidth,page=1]{~~~~~~~~~~~~~~~~~~~~~~~~~~~~~~~~~~~~~~~~~~~~~~~~~~~~~~~~~~~~~d)}{graphics/eventSelection/exclusivity/MissingPy_kaon.pdf} \\
    \subfigimg[width=\linewidth,page=1]{~~~~~~~~~~~~~~~~~~~~~~~~~~~~~~~~~~~~~~~~~~~~~~~~~~~~~~~~~~~~~e)}{graphics/eventSelection/exclusivity/MissingPx_proton.pdf} &
    \subfigimg[width=\linewidth,page=1]{~~~~~~~~~~~~~~~~~~~~~~~~~~~~~~~~~~~~~~~~~~~~~~~~~~~~~~~~~~~~~f)}{graphics/eventSelection/exclusivity/MissingPy_proton.pdf}    
  \end{tabular}\vspace*{-5pt}
    \caption[Raw distributions of $p_{x}^{\text{miss}}$ and $p_{y}^{\text{miss}}$ for exclusive $\pi^+\pi^-$, $K^+K^-$ and $p\bar{p}$candidates.]{%
    Raw distributions of $p_{x}^{\text{miss}}$ (left column) and $p_{y}^{\text{miss}}$ (right column) for exclusive $\pi^+\pi^-$ (top row), $K^+K^-$ (middle row) and $p\bar{p}$ (bottom row) candidates after full event selection, except exclusivity cut~\ref{enum:CutMissingPt}. Solid red line represents the fit of sum of two Gaussian functions representing the exclusive event signal (orange) and non-exclusive background (violet). Parameters of the total momentum resolution for signal events obtained from the fit (given in the plots) roughly agree between all species.
    }\label{fig:MissingPxPy}
\end{figure}



\begin{figure}[h]
\centering
\parbox{0.4725\textwidth}{
  \centering
  \begin{subfigure}[b]{\linewidth}
                \subcaptionbox{\label{fig:MissingPt_pion}}{\includegraphics[width=1.05\linewidth,page=1]{graphics/eventSelection/exclusivity/Paper_MissingPt_pion.pdf}}
  \end{subfigure}\\
  \begin{subfigure}[b]{\linewidth}\addtocounter{subfigure}{1}
                \subcaptionbox{\label{fig:MissingPt_kaon}}{\includegraphics[width=1.05\linewidth,page=1]{graphics/eventSelection/exclusivity/Paper_MissingPt_proton.pdf}}
  \end{subfigure} 
}%
\quad\quad%
\parbox{0.4725\textwidth}{
  \centering
  \begin{subfigure}[b]{\linewidth}\addtocounter{subfigure}{-2}\vspace*{17pt}
                \subcaptionbox{\label{fig:MissingPt_proton}}{\includegraphics[width=1.05\linewidth,page=1]{graphics/eventSelection/exclusivity/Paper_MissingPt_kaon.pdf}}
  \end{subfigure}\\
  \begin{minipage}[t][1.042\linewidth][t]{\linewidth}\vspace{20pt}
    \caption[....]{Uncorrected distributions of the CEP event candidates for missing transverse momentum $p_\mathrm{T}^\mathrm{\scriptscriptstyle miss}$ for $\pi^+\pi^-$ (top), $K^+K^-$ (middle) and $p\bar{p}$ (bottom) pairs. Distributions for opposite-sign and same-sign particle pairs are shown as black and red symbols, respectively. The vertical error bars represent statistical uncertainties. The horizontal bars represent bin sizes. Distribution for $\pi^+\pi^-$ channel with MC predictions for both signal and background can be found in Fig.~\ref{fig:Ratio_MissingPt_OppositeAndSameSign}.}\label{fig:MissingPt}
  \end{minipage}
}%
\end{figure}
%--------------------------- 

 



% \section{Signal per integrated luminosity}
% 
% \section{Cut flow}\label{sec:cutFlow}
% 
% %---------------------------
% \begin{figure}[ht!]
% \centering%
% \includegraphics[width=0.85\linewidth,page=1]{graphics/eventSelection/CutFlow.pdf}%
% \caption{Cut flow.}\label{fig:CutFlow}%
% \end{figure}
% %---------------------------


%% =====  BACKGROUNDS ====
%%===========================================================%%
%%                                                           %%
%%                        BACKGROUNDS                        %%
%%                                                           %%
%%===========================================================%%

\chapter{Backgrounds}\label{chap:backgrounds}

\section{Non-exclusive background}
\section{Exclusive background (particle misidentification)}

%---------------------------
\begin{figure}[ht!]
\centering%
\parbox{0.4725\textwidth}{%
  \centering%
  \includegraphics[width=\linewidth]{graphics/backgrounds/pid-crop.pdf}\label{fig:misidentificationGraph}
}%
\quad%
\parbox{0.4725\textwidth}{%
    \caption[Graph illustrating the misidentification problem.]{Graph illustrating the misidentification problem - the origin of exclusive background in selected samples. Gray arrows represent event rejection due to failed PID selection (\ref{enum:CutPid}). Solid black arrows represent successful identification, whereas dashed black lines show misidentification paths.}
}%

\end{figure}
%---------------------------


\begin{subequations}\label{eq:misidentificationEqs}
\begin{equation}
  N^{\pi\pi}_{R}~~=~~\begingroup\color{gray}\underbrace{\color{black}\epsilon^{\pi\pi}\cdot N^{\pi\pi}_{T}}_{\textrm{true pion pairs}}\endgroup~~ + ~~\begingroup\color{gray}\underbrace{\color{black}\lambda^{ KK\rightarrow \pi\pi}\cdot N^{KK}_{T}}_{\substack{\textrm{kaon pairs reconstructed} \\ \textrm{as pion pairs}}}\endgroup~~ + ~~\begingroup\color{gray}\underbrace{\color{black}\lambda^{p\bar{p} \rightarrow \pi\pi} \cdot N^{p\bar{p}}_{T}}_{\substack{\textrm{proton pairs reconstructed} \\ \textrm{as pion pairs}}}\endgroup~~ + ~~\textcolor{red}{N^{\pi\pi}_{bkgd}}
\end{equation}    
\begin{equation}
  N^{KK}_{R} ~= ~~\begingroup\color{gray}\underbrace{\color{black}\lambda^{ \pi\pi\rightarrow KK}\cdot N^{\pi\pi}_{T}}_{\substack{\textrm{pion pairs reconstructed} \\ \textrm{as kaon pairs}}}\endgroup~~ + ~~\begingroup\color{gray}\underbrace{\color{black}\epsilon^{KK}\cdot N^{KK}_{T}}_{\textrm{true kaon pairs}}\endgroup~~ + ~~\begingroup\color{gray}\underbrace{\color{black}\lambda^{p\bar{p} \rightarrow KK} \cdot N^{p\bar{p}}_{T}}_{\substack{\textrm{proton pairs reconstructed} \\ \textrm{as kaon pairs}}}\endgroup~~ + ~~\textcolor{red}{N^{KK}_{bkgd}}
\end{equation}
\begin{equation}\hspace*{-25pt}
  N^{p\bar{p}}_{R}~~~= ~~\begingroup\color{gray}\underbrace{\color{black}\lambda^{\pi\pi \rightarrow p\bar{p}} \cdot N^{\pi\pi}_{T}}_{\substack{\textrm{pion pairs reconstructed} \\ \textrm{as proton pairs}}}\endgroup~~ + ~~\begingroup\color{gray}\underbrace{\color{black}\lambda^{ KK\rightarrow p\bar{p}}\cdot N^{KK}_{T}}_{\substack{\textrm{kaon pairs reconstructed} \\ \textrm{as proton pairs}}}\endgroup~~ + ~~\begingroup\color{gray}\underbrace{\color{black}\epsilon^{p\bar{p}}\cdot N^{p\bar{p}}_{T}}_{\textrm{true proton pairs}}\endgroup~~ + ~~\textcolor{red}{N^{p\bar{p}}_{bkgd}}
\end{equation}
\end{subequations}

Eqs.~\eqref{eq:misidentificationEqs} can be written in the matrix form, as shown in Eq.~\eqref{eq:misidentificationMatrix}.
\begin{equation}\label{eq:misidentificationMatrix}
\Spvek{N^{\pi\pi}_{R}-\textcolor{red}{N^{\pi\pi}_{bkgd}};~;N^{KK}_{R}-\textcolor{red}{N^{KK}_{bkgd}};~;N^{p\bar{p}}_{R}-\textcolor{red}{N^{p\bar{p}}_{bkgd}}} =  \underbrace{\left[ \begin{array}{ccc}
\epsilon^{\pi\pi} & \lambda^{ KK\rightarrow \pi\pi} & \lambda^{p\bar{p} \rightarrow \pi\pi} \\
~ & ~ & ~\\
\lambda^{\pi\pi\rightarrow KK} & \epsilon^{KK} & \lambda^{ p\bar{p} \rightarrow KK}\\
~ & ~ & ~\\
\lambda^{\pi\pi\rightarrow p\bar{p}} & \lambda^{ KK\rightarrow p\bar{p}} & \epsilon^{p\bar{p}}
\end{array} \right]}_{\text{``mixing matrix''}~\Lambda}\Spvek{N^{\pi\pi}_{T};~;N^{KK}_{T};~;N^{p\bar{p}}_{T}}
\end{equation}

\begin{equation}
\Spvek{N^{\pi\pi}_{T};~;N^{KK}_{T};~;N^{p\bar{p}}_{T}} = \Lambda^{-1}\Spvek{N^{\pi\pi}_{R}-\textcolor{red}{N^{\pi\pi}_{bkgd}};~;N^{KK}_{R}-\textcolor{red}{N^{KK}_{bkgd}};~;N^{p\bar{p}}_{R}-\textcolor{red}{N^{p\bar{p}}_{bkgd}}}
\end{equation}


%% =====  CORRECTIONS ====
%%===========================================================%%
%%                                                           %%
%%                        CORRECTIONS                        %%
%%                                                           %%
%%===========================================================%%


\chapter{Corrections}\label{chap:corrections}

\section{Method of corrections application}
\begin{equation}
  \frac{d\sigma}{dq} = \frac{1}{\Delta q} \times \frac{1}{\varepsilon} \times \frac{N^{\mathit{w}}-N^{\mathit{w}}_\textrm{bkgd}}{\mathit{L}_{\textrm{int}}^{\textrm{eff}}}
\end{equation}

%remembed about accounting for RP trigger eff!!!
\begin{equation}\label{eq:effectiveLumi}
	\mathit{L}_{\textrm{int}}^{\textrm{eff}} = \sum\limits_{\textrm{run}}\mathit{L}_{\textrm{int}}^{\textrm{run}} \times \epsilon_{\textrm{veto}}(L^{\textrm{run}})
\end{equation}
% \left(\epsilon_{\textrm{veto}}^{\textrm{online}} \oplus \epsilon_{\textrm{veto}}^{\textrm{offline}}(L_{\textrm{run}}) \right)

\begin{equation}
	\varepsilon = \epsilon_{\textrm{\tiny ET/IT}} \times \epsilon_{\textrm{vrtx}}(q) \times \epsilon_{\ref{enum:CutZVx}} \times \epsilon_{\ref{enum:CutDeltaZVx}} \times \epsilon_{\ref{enum:CutMissingPt}} \times \epsilon_{\textrm{\tiny PID}}(q)
\end{equation}

\begin{equation}
	N^{\mathit{w}} = \sum\limits_{\textrm{event}}\mathit{w}_{\textrm{event}}
\end{equation}



\begin{equation}
	\mathit{w} = \left[\prod\limits_{\textrm{sign}} \epsilon_{\textrm{\tiny TOF}}(\textrm{sign}, \textrm{PID}, p_{T},z_{vx},\eta)  \times \prod\limits_{\textrm{sign}} \epsilon_{\textrm{\tiny TPC}}(\textrm{sign}, \textrm{PID}, p_{T},z_{vx},\eta) \times \prod\limits_{\textrm{side}}\epsilon_{\textrm{\tiny RP}}^{\textrm{side}}(p_{x},p_{y}) \right]^{-1},
\end{equation}
\[\textrm{sign}=\{+,-\},~~\textrm{side}=\{E,W\}\]
% ()

\section{Efficiencies and acceptances}
\subsection{Trigger efficiency}\label{sec:triggerEff}
\subsubsection{Online veto (BBC-small and ZDC veto)}
%---------------------------
\begin{figure}[ht!]
\centering%
\includegraphics[width=0.65\linewidth,page=1]{graphics/corrections/OnlineVetoEffVsInstLumi_graph.pdf}%
\caption{Overall efficiency of the online BBC-small and ZDC veto as a function of instantaneous luminosity.}\label{fig:onlineVetoEff}%
\end{figure}
%---------------------------
\subsubsection{RP triggering efficiency}
\subsubsection{Up and Down RP combination veto}
\subsection{Cuts efficiency}\label{sec:cutsEff}
\subsubsection{TPC \texorpdfstring{$z$}{z}-vertex cut~(\ref{enum:CutZVx})}
\subsubsection{TPC-RP \texorpdfstring{$z$}{z}-vertex matching~(\ref{enum:CutDeltaZVx})}
\subsubsection{Primary vertices limit~(\ref{enum:CutPrimVx}), BBC-large veto~(\ref{enum:CutBbcLarge}) and TOF clusters limit~(\ref{enum:CutTofClusters})}
%---------------------------
\begin{figure}[ht!]
\centering%
\includegraphics[width=0.65\linewidth,page=1]{graphics/corrections/OnlineAndOfflineVetoEffVsInstLumi_graph.pdf}%
\caption{Overall efficiency of the online BBC-small and ZDC veto, primary vertices limit~(\ref{enum:CutPrimVx}), BBC-large veto~(\ref{enum:CutBbcLarge}) and TOF clusters limit~(\ref{enum:CutTofClusters}) as a function of instantaneous luminosity.}\label{fig:onlineAndOfflineVetoEff}%
\end{figure}
%---------------------------
\subsubsection{Missing \texorpdfstring{$p_{T}$}{pT} cut~(\ref{enum:CutMissingPt})}
\subsubsection{Particle identification~(\ref{enum:CutPid})}

It is possible to transform dE/dx in MC to make it follow the shape of dE/dx in the data. 
We know that nSigmaX (where X=Pion, Kaon, Proton, ...) variable follows a gaussian distribution (for particle X)
 \[nSigmaX = \Big( \ln{\frac{dE/dx}{\langle dE/dx\rangle_{X}}} \Big) / \sigma_{dE/dx},~~~~~f(nSigmaX) = \mathcal{N}(nSigmaX; \mu=0,\sigma=1)\]
therefore $dE/dx$ itself follows log-normal distribution:
\[f(dE/dx) = \mathcal{L}og\mathcal{N}(dE/dx; \mu=\langle dE/dx\rangle,\sigma=\sigma_{dE/dx}) = \frac{1}{\sqrt{2\pi}\cdot \sigma\cdot dE/dx}e^{-\frac{\ln^{2}{\frac{dE/dx}{\langle dE/dx\rangle}}}{2\sigma^{2}}}\]
The transformation we want to apply should preserve the shape of $dE/dx$ (so that it is still described by $\mathcal{L}og\mathcal{N}$), however it should change $\mu$ and $\sigma$ so that these values are euqal to those seen in the data. The transformation that satisfies above postulate is
\[dE/dx' = c\cdot (dE/dx)^{a}\]
Parameters of the distribution $\mathcal{L}og\mathcal{N}(dE/dx')$ would be then
\[\mu' = c\cdot\mu^{a},~~~~\sigma' = a\cdot\sigma\]
From above we get formulae for parameters of the transformation:
\[a=\sigma'/\sigma,~~~~c = \frac{\mu'}{\mu^{a}}\]

AlternativeToCrystallBall~\cite{AlternativeToCrystallBall}~Eq.~\eqref{eq:expTail}

\begin{equation}\label{eq:expTail}
	f(dE/dx)=\left\{
                \begin{array}{ll}
                  \frac{A}{\sqrt{2\pi}\cdot \sigma\cdot dE/dx}\exp{\Bigg(-\frac{1}{2}\Big(\frac{\ln{\frac{dE/dx}{\langle dE/dx\rangle}}}{\sigma}\Big)^{2}\Bigg)} & \textrm{for}~\frac{\ln{\frac{dE/dx}{\langle dE/dx\rangle}}}{\sigma} \leq k \\
                  \frac{A}{\sqrt{2\pi}\cdot \sigma\cdot dE/dx}\exp{\Bigg(-k\cdot \frac{\ln{\frac{dE/dx}{\langle dE/dx\rangle}}}{\sigma} + \frac{1}{2}k^{2} - k^{-1}\left(\frac{\frac{\ln{\frac{dE/dx}{\langle dE/dx\rangle}}}{\sigma}}{k}-1\right)^{k} \Bigg)} & \textrm{for}~\frac{\ln{\frac{dE/dx}{\langle dE/dx\rangle}}}{\sigma} > k
                \end{array}
              \right.
\end{equation}


%---------------------------
\begin{figure}[hb]
\centering
\parbox{0.4725\textwidth}{
  \centering
  \begin{subfigure}[b]{\linewidth}{
                \subcaptionbox{\label{fig:dEdxMeanOffsetMC}}{\includegraphics[width=\linewidth]{graphics/corrections/dEdxMeanOffset_allPIDs.pdf}\vspace*{-10pt}}}
  \end{subfigure}\\
  \begin{subfigure}[b]{\linewidth}\addtocounter{subfigure}{1}{
                \subcaptionbox{\label{fig:dEdxMeanOffsetData}}{\includegraphics[width=\linewidth]{graphics/corrections/dEdxMeanOffset_allPIDs_data.pdf}\vspace*{-10pt}}}
  \end{subfigure}
}
\quad
\parbox{0.4725\textwidth}{
  \centering
  \begin{subfigure}[b]{\linewidth}\addtocounter{subfigure}{-2}{
                \subcaptionbox{\label{fig:dEdxWidthMC}}{\includegraphics[width=\linewidth]{graphics/corrections/dEdxWidth_allPIDs.pdf}\vspace*{-10pt}}}
  \end{subfigure}\\
  \begin{subfigure}[b]{\linewidth}\addtocounter{subfigure}{1}{
                \subcaptionbox{\label{fig:dEdxWidthData}}{\includegraphics[width=\linewidth]{graphics/corrections/dEdxWidth_allPIDs_data.pdf}\vspace*{-10pt}}}
  \end{subfigure}
}%
\caption[Parameters of track dE/dx as a function of reconstructed momentum for a few particle species.]{Difference between MPV of dE/dx predicted by Bichsel parametrization and obtained from the fit of Eq.~\eqref{eq:expTail} to dE/dx distribution in the data (\ref{fig:dEdxMeanOffsetData}) and MC sample (\ref{fig:dEdxMeanOffsetMC}) and dE/dx width parameter in data (\ref{fig:dEdxWidthData}) and MC (\ref{fig:dEdxWidthMC}) as a function of reconstructed particle momentum for a few particle species. Solid lines represent fits to points of corresponding color.}\label{fig:dEdxParametersMC}
\end{figure}
%---------------------------


\begin{equation}\label{eq:dEdxParametrization}
	g(p) = P_{1} + P_{2}\cdot \exp{\left(-P_{3}\cdot p\right)} + P_{4}\cdot \arctan{\big(P_{5}\cdot(p-P_{6})\big)}
\end{equation}



\begin{table}[ht!]\centering
\subcaptionbox{\label{fig:vvv}}{
 \begin{tabular}{r||c|c|c|c|c|c||c|c|c|c|c|c}%\hline
 \multirow{2}{*}{\textbf{PID}} &  \multicolumn{6}{c||}{\bm{$\langle dE/dx\rangle_{\textrm{\textbf{Bichsel}}} - \langle dE/dx\rangle_{\textrm{\textbf{MC}}}$}} & \multicolumn{6}{c}{\bm{$\sigma(dE/dx)_{\textrm{\textbf{MC}}}$}} \\ \cline{2-13}
  & $P_{1}$ & $P_{2}$ & $P_{3}$ & $P_{4}$ & $P_{5}$ & $P_{6}$ & $P_{1}$ & $P_{2}$ & $P_{3}$ & $P_{4}$ & $P_{5}$ & $P_{6}$ \\ \Xhline{2\arrayrulewidth}
 $\bm{\pi^{\pm}}$ & \scriptsize3.293e-8 & \scriptsize2.237e-8 & \scriptsize4.399 & \scriptsize& \scriptsize& \scriptsize& \scriptsize0.0668 & \scriptsize1.186 & \scriptsize28.683 & \scriptsize-1.5e-3 & \scriptsize7.887 & \scriptsize1.028\\ \hline
 $\bm{K^{\pm}}$ & \scriptsize3.068e-9 & \scriptsize-8.190e-6 & \scriptsize7.089 & \scriptsize& \scriptsize& \scriptsize& \scriptsize0.0664 & \scriptsize0.323 & \scriptsize-10.438 & \scriptsize& \scriptsize& \scriptsize\\ \hline
 $\bm{p,\bar{p}}$ & \scriptsize-3.061e-8 & \scriptsize-1.061e-5 & \scriptsize4.08 & \scriptsize& \scriptsize& \scriptsize& \scriptsize0.0674 & \scriptsize0.103 & \scriptsize-5.952 & \scriptsize& \scriptsize& \scriptsize\\ \hline
 $\bm{e^{\pm}}$ & \scriptsize-1.483e-7 & \scriptsize3.710e-7 & \scriptsize5.639 & \scriptsize& \scriptsize& \scriptsize& \scriptsize0.0590 & \scriptsize-0.009 & \scriptsize-194.42 & \scriptsize-1.81e-3 & \scriptsize26.392 & \scriptsize0.607 \\ \hline
 $\bm{d,\bar{d}}$ & \scriptsize-2.474e-6 & \scriptsize0.385 & \scriptsize21.719 & \scriptsize5.165e-7 & \scriptsize29.642 & \scriptsize0.781 & \scriptsize0.131 & \scriptsize-0.999 & \scriptsize4.749 & \scriptsize& \scriptsize& \scriptsize%\hline
\end{tabular}
}
\subcaptionbox{\label{fig:aaa}}{
\begin{tabular}{r||c|c|c|c|c|c||c|c|c|c|c|c}%\hline
 \multirow{2}{*}{\textbf{PID}} &  \multicolumn{6}{c||}{\bm{$\langle dE/dx\rangle_{\textrm{\textbf{Bichsel}}} - \langle dE/dx\rangle_{\textrm{\textbf{Data}}}$}} & \multicolumn{6}{c}{\bm{$\sigma(dE/dx)_{\textrm{\textbf{Data}}}$}} \\ \cline{2-13}
  & $P_{1}$ & $P_{2}$ & $P_{3}$ & $P_{4}$ & $P_{5}$ & $P_{6}$ & $P_{1}$ & $P_{2}$ & $P_{3}$ & $P_{4}$ & $P_{5}$ & $P_{6}$ \\ \Xhline{2\arrayrulewidth}
 $\bm{\pi^{\pm}}$ & \scriptsize-1.236e-8 & \scriptsize1.777e-7 & \scriptsize-9.938 & \scriptsize& \scriptsize& \scriptsize& \scriptsize0.0738 & \scriptsize16.86 & \scriptsize39.44 & \hspace*{-3pt}\scriptsize-1.704e-3\hspace*{-2pt} & \scriptsize~6.482~ & \scriptsize0.628\\ \hline
 $\bm{K^{\pm}}$ & \scriptsize5.49e-10 & \scriptsize-2.732e-6 & \scriptsize-7.712 & \scriptsize& \scriptsize& \scriptsize& \scriptsize0.0743 & \hspace*{-2pt}\scriptsize2.67e-5\hspace*{-2pt} & \scriptsize7.17089 & \scriptsize& \scriptsize& \scriptsize\\ \hline
 $\bm{p,\bar{p}}$ & \scriptsize-2.140e-7 & \scriptsize0.0421 & \scriptsize48.305 & \scriptsize7.512e-8 & \scriptsize15.544 & \scriptsize0.575 & \scriptsize0.0779 & \scriptsize1.822 & \scriptsize22.4277 & \scriptsize& \scriptsize& \scriptsize\\ \hline
 $\bm{e^{\pm}}$ & \scriptsize6.701e-8 & \scriptsize3.304e-7 & \scriptsize-7.845 & \scriptsize& \scriptsize& \scriptsize& \scriptsize0.0678 & \scriptsize468.9 & \scriptsize59.4001 & \scriptsize& \scriptsize& \scriptsize\\ \hline
 $\bm{d,\bar{d}}$ & \scriptsize-1.631e-7 & \scriptsize0.0818 & \scriptsize-18.91 & \scriptsize& \scriptsize& \scriptsize& \scriptsize0.1259 & \scriptsize-0.288 & \scriptsize3.28733 & \scriptsize& \scriptsize& \scriptsize%\\ \hline
\end{tabular}
}\caption[Parameters of functions from Fig.~\ref{fig:dEdxParametersMC} describing track dE/dx as a function of reconstructed momentum for a few particle species (STARsim MC).]{Parameters of functions from Fig.~\ref{fig:dEdxParametersMC} describing track dE/dx as a function of reconstructed momentum for a few particle species. Units of parameters $P_{i}$ are such that if one provides momentum in Eq.~\eqref{eq:dEdxParametrization} in GeV/$c$ the resultant offset of dE/dx MPV with respect to Bichsel parametrization is in GeV/cm, and the resultant $\sigma$ parameter is unitless.}\label{tab:dEdxParametersMC}
\end{table}



















\subsection{RP track acceptance and reconstruction efficiency}\label{sec:rpAccAndEff}
\subsection{TPC track acceptance and reconstruction efficiency}\label{sec:tpcAccAndEff}
\subsection{TOF acceptance, reconstruction and track-matching efficiency}\label{sec:tofAccAndEff}
\subsection{TPC vertex reconstruction efficiency}\label{sec:tpcVxRecoEff}
\section{Particle energy loss}\label{sec:energyLoss}
\section{Background subtraction}\label{sec:bkgdSubtraction}
\section{Unfolding}\label{sec:unfolding}

%% =====  SYSTEMATIC ERRORS ====
%%===========================================================%%
%%                                                           %%
%%                   SYSTEMATIC ERRORS                       %%
%%                                                           %% 
%%===========================================================%%


\chapter{Systematic uncertainties}\label{chap:systematicErrors} 

\section{Luminosity}\label{sec:lumiSyst}
Relative luminosity at STAR is determined by the coincidence rate in ZDC detectors at both beam directions. Absolute calibration is given by a special Van der Meer scan~\cite{vanderMeer}. For determination of the systematic uncertainty of the integrated luminosity we use the results of a dedicated study of data from Van der Meer scans~\cite{lumiNote} performed during fill \#18915. These scans were targeted on providing a precise calculation of the luminosity and an estimate of its uncertainty for the elastic proton-proton scattering measurement~\cite{elasticNote}. From this study we learn that the effective cross-secton visible in the ZDC detectors is equal to 0.294~mb with 4\% systematic uncertainty. The number is different from the effective ZDC cross-section used in the initial calculation of the luminosity at STAR, equal to 0.264~mb. The ratio of these numbers is 1.114. One concludes that the initial calculations of the luminosity based on the comparison of the coincidence rate of the East and West ZDC detectors, were overestimated by factor 1.114. Therefore, for the cross-section calculation we use a corrected integrated luminosity - a nominal integrated luminosity divided by aforementioned factor.

\indent
To account for possible fill-by-fill dependence in the luminosity measurement an additional 4\% uncertainty was assigned to the luminosity. It was determined by comparing variations of the effective cross-sections for elastic scattering process relative to the measurement done solely based on data collected during the fill \#18915~(Fig.~\ref{fig:lumiSyst_elastic}). This effective cross-section was determined from the elastic proton-proton scattering events reconstructed with the same proton track selection as in the CEP analysis. For this purpose fiducial region for elastically-scattered proton tracks was chosen to correspond to area matching the rectangular $(|t|, \varphi)$ window of 100\% geometrical acceptance. This window is determined by $0.04 < |t| < 0.1~\text{GeV}^{2}$ and $1.3<|\varphi|<1.9$.

\indent
The overall luminosity uncertainty of 6.0\% was estimated by the quadratic sum of the two uncertainty sources described above.

% Additional component of the systematic uncertainty on the integrated luminosity accounts for a consistency between integrated cross-section measured within reference fill \#18915, and measured with the entire dataset. The integrated cross-section measured using solely data from fill \#18915 is equal to
% \begin{equation}
%  \sigma_{\text{fid}}^{\text{CEP}} = 55.3\pm2.4~\text{mb},
% \end{equation}
% while using entire dataset we obtain
% \begin{equation}
%  \sigma_{\text{fid}}^{\text{CEP}} = 51.3\pm0.4~\text{mb}.
% \end{equation}
% The absolute and relative difference between the two is, respectively,
% \begin{equation}
%  \Delta\sigma_{\text{fid}}^{\text{CEP}} = 4.0\pm2.4~\text{mb},~~~~~~~~~\Delta\sigma_{\text{fid}}^{\text{CEP}}/\sigma_{\text{fid}}^{\text{CEP}} = \left(7.8\pm4.7\right)~\%.
% \end{equation}
% 
% Finally, the total systematic uncertainty of the integrated luminosity is a quadratic sum of generic uncertainty equal to 4\%, and uncertainty arising from the consistentsy between cross-section measured using data from fill \#18915 and entire dataset, equal to 7.8\%:
% 
% \begin{equation}
% \Delta\mathcal{L}/\mathcal{L} = 4.0\%\oplus7.8\% = 8.8\%. 
% \end{equation}
% 
% 
\begin{figure}[h]
\centering
\includegraphics[width=.48\textwidth,page=5]{graphics/systematics/sigmaVsRunNumber_elastic.pdf}~~~~
\includegraphics[width=.48\textwidth]{graphics/systematics/sigmaVsRunNumber_elastic.pdf}
\caption[Luminosity uncertainty systematics.]{Integrated fiducial elastic proton-proton scattering cross-section for runs from fill \#18915~(left) and for all runs~(right). Dashed red lines mark an average fiducial cross-section within displayed run range. Solid blue lines mark $\pm4\%$ uncertainty bands assigned to the luminosity to account to fill-by-fill variations.}
\label{fig:lumiSyst_elastic}
\end{figure}


\section{Trigger veto effect (due to dead material)}\label{sec:systTrigVeto}

Systematic uncertainty related to the trigger veto correction (Sec.~\ref{sec:rpDeadMat}) has been studied with elastic scattering events, in a way similar to analyses presented in Sec.~10.3 (and following) of Ref.~\cite{supplementaryNote}. 

Triggers dedicated for elastic scattering process (RP\_ET triggers) have been studied. The trigger required signal in at least one PMT (out of four) in two RP branches opposite to each other with respect to interaction region. In addition to trigger selection, vetoes were imposed offline on any activity in other STAR detectors, such as BBC (small and large), ZDC, TOF and VPD - it reduced probability of a non-elastic pile-up interaction. It has been demonstrated in Sec.~10.3.1 of Ref.~\cite{supplementaryNote} that once the single good quality RP track is required on one side, such sample consists only of elastic proton-proton scattering events.

Each side (branch) was analyzed independently; when single good quality RP track, in addition of $|\xi|<0.01$, was found on the east side, the systematics for west side was investigated (and vice versa). Two histograms were filled per event. The first histogram was filled with all selected events. The second histogram was filled only if there was no simultaneous signal in upper and lower RP on studied side (no ET and IT trigger bits fired at the same time) - just the same, as it was implemented in RP\_CPT2 trigger. However, the second histogram was filled with weight equal to inverse efficiency of the veto, $1/\mbox{\LARGE$\epsilon$}_{\text{DM~veto}}^{\text{side}}$. The ratio of the second to first histogram,
\begin{equation}
 R_{\text{DM~veto}} = \frac{\text{histogram of events with satisfied veto, filled with weight } 1/\mbox{\LARGE$\epsilon$}_{\text{DM~veto}}^{\text{side}}}{\text{histogram of all events, filled with unit weight}},
\end{equation}
has been presented in Fig.~\ref{fig:systDMveto} as a function of $p_{x}$ and $p_{y}$ of elastically scattered proton on studied side. If single good quality proton track was reconstructed in studied branch, parameters of that track were histogrammed. Otherwise, transverse components of momentum of unreconsructed elastic proton (e.g. due to induced shower) were estimated as $-p_{x}$ and $-p_{y}$ of elastic proton track on the opposite side.

\begin{figure}[h]
\centering
\includegraphics[width=.48\textwidth,page=1]{graphics/systematics/deadMatSyst.pdf}~~~~%
\includegraphics[width=.48\textwidth,page=2]{graphics/systematics/deadMatSyst.pdf}%
\caption[Estimated systematic uncertainty related to trigger veto induced by interaction with dead material.]{Ratio $R_{\text{DM~veto}}$ illustrating systematic uncertainty of the dead material trigger veto correction in east (left) and west (right) Roman Pots.}
\label{fig:systDMveto}
\end{figure}

Results presented in Fig.~\ref{fig:systDMveto} indicate imperfect description of the dead material of the elements surrounding RPs (DX-D0 chamber, RF shield), which has been also overved in studies of systematic uncertainties related to RP track reconstruction efficiency, presented in Ref.~\cite{supplementaryNote}. Based on current study, the correction and systematic uncertainty of the dead material veto efficiency $\mbox{\LARGE$\epsilon$}_{\text{DM~veto}}^{\text{side}}$ are assumed to have the following form: 
\begin{itemize}
\item multiplicative correction is applied to $\mbox{\LARGE$\epsilon$}_{\text{DM~veto}}^{\text{side}}$, equal to $1+\frac{1}{2}(R_{\text{DM~veto}}-1)$,
 \item systematic uncertainty is assumed to be equal to $1+\frac{1}{2}|R_{\text{DM~veto}}-1|$, therefore propagation of this systematic effect is done by variating the efficiency $\mbox{\LARGE$\epsilon$}_{\text{DM~veto}}^{\text{side}}$ by the multiplicative factor equal to $\pm\left[1+\frac{1}{2}|R_{\text{DM~veto}}-1|\right]$. 
\end{itemize} 


\section{TPC track quality cuts variation}\label{sec:systTpcQuaCuts}

We have tested senstivity of the fiducial cross-sections on the variation of the quality cuts used in TPC track selection. Any change of these cuts requires re-calculation of the TPC track reconstruction and TOF matching efficiencies (the latter is conditional w.r.t. the former). We have checked two working points, 'loose' and 'tight, with different set of cuts on $d_{0}$, $N_{\text{fit}}^{\text{hits}}$ and $N^{\text{hits}}_{\text{dE/dx}}$ compared to nominal working point. Definition of 'loose' and 'tight' cuts is provided in Sec.~3.2.4 of Ref.~\cite{supplementaryNote}, together with depiction of resulted changes of the efficiencies.

Observed ratios of the cross-sections obtained with modified cuts and accordingly redefined TPC and TOF efficiencies, to the nominal cross-section, are shown in Fig.~\ref{fig:tpcQuaVariation}. The upward and downward relative changes of the cross-sections were added in quadrature. The resulting systematic uncertainty of the cross-section related to TPC track quality cuts is estimated to be $\pm1.5\%$.
 
\begin{figure}[h]
\centering
\includegraphics[width=.8\textwidth,page=2]{graphics/systematics/TrackQualityCutVariation_InvMass.pdf}
%
\caption[Result of variation of the TPC track quality cuts on $d\sigma/dm(\pi^{+}\pi^{-})$.]{Result of variation of the TPC track quality cuts on the fiducial differential exclusive $\pi^{+}\pi^{-}$ production cross-section as a function of the invariant mass of $\pi^{+}\pi^{-}$. Dashed horizontal lines respresent an average ratio for a modified cut marked with the same color.}
\label{fig:tpcQuaVariation}
\end{figure}


%%%%%%%%%%%%%%%%%%%%%%%%%%%%%%%%%%%%%%%%%% 

\section{Discussion of systematic effects}\label{sec:systEffectsList}
The following contributions to the overall systematic uncertainty have been studied. Influence of each systematic effect on measured cross sections has been tested by changing amount of the quantity that the systematic effect refers to and comparing the result with that obtained using nominal values.

\begin{enumerate}
 \item \textbf{Representativeness of the embedding sample ($\bm{\Delta\varepsilon_{\text{TPC}}}$ (embed. stat.)).}\\
 Zero-bias data available for the MC embedding is only a fraction of all zero-bias triggers, therefore physics data may be not fully represented in MC events used for determination of the TPC track reconstruction efficiency. This effect was studied by comparing estimated average levels of pile-up in the data and embedded MC. The difference was found to be of the order of 1\%, which we establish as a symmetric systematic uncertainty on the TPC track reconstruction efficiency (per track). See the last paragraph of Sec.~10.1.1 of Ref.~\cite{supplementaryNote} for more details.
 %
 \item \textbf{Embedding procedure/off-time pile-up effect ($\bm{\Delta\varepsilon_{\text{TPC}}}$ (pile-up)).}\\
 Reliability and precision of the embedding technique was verified and quantitatively estimated in the procedure described in Sec.~10.1.1 of Ref.~\cite{supplementaryNote}. Embedded MC samples were divided into sub-samples representing different levels of off-time pile-up/densitity of hit points in TPC. With dedicated analysis it was possible to verify if the TPC track reconstruction efficiency is compatible between all sub-samples when the effect of pile-up (changing number of hits forming a track) is reduced. The average systematic uncertainty related to the embedding procedure is $<1\%$ (per track, Fig.~10.5 of Ref.~\cite{supplementaryNote}).
 %
 \item \textbf{Modelling of the dead material in front of the TPC ($\bm{\Delta\varepsilon_{\text{TPC}}}$ (dead mat.)).}\\
 Not all detector elements are fully modeled in the MC simulation, quite often some simplifications are used. This leads to inaccuracies in efiiciencies derived from the simulation. We estimated systematic uncertainty related to amount of simulated material between the primary vertex and STAR TPC to be 25\% which translates to $\approx 0.5\%$ of uncertainty of the TPC track reconstruction efficiency. See Chap.~9 of Ref.~\cite{supplementaryNote} for details.
 %
 \item \textbf{Modelling of TPC track quality parameters in embedded MC ($\bm{N^{\text{hits}}}$ and $\bm{d_{0}/\text{DCA}(R)}$).}\\
 We have checked the impact of variation of the track quality cuts on the obtained cross-sections. It reflects systematic uncertainty related to the quality of modeeling of the quantities used to select the primary TPC tracks. The estimated uncertainty on the fiducial cross-section amounts $^{+2.0}_{-1.5}\%$. See Sec.~\ref{sec:systTpcQuaCuts}.
  %
 \item \textbf{Vertexing ($\bm{\Delta\epsilon_{\text{vtx}}}$).}\\
 Vertexing efficiency has been obtained using data-driven method presented in Sec.~\ref{sec:tpcVxRecoEff}, thus systematic uncertainty related to this efficiency has been significantly reduced. Systematic uncertainty has been estimated as a difference between efficiency with and without subtracted background.
 %
 \item \textbf{Modelling of the TOF system and validity of derived efficiency corrections ($\bm{\Delta\varepsilon_{\text{TOF}}}$).}\\
 The efficiency of matching TOF hits with the TPC tracks has been extracted from embedded MC sample. It has been confronted with TOF efficiency extracted from the data using two independent techniques: tag\&probe (Sec.~4.1 of Ref.~\cite{supplementaryNote}) and HFT-tagging (Sec.~10.2.2 of Ref.~\cite{supplementaryNote}). Average difference between the data and MC efficiency has been used as a correction to MC efficiency, while half of the difference between data-extracted efficiencies has been treated as a systematic uncertainty (Fig.~10.16, Ref.~\cite{supplementaryNote}). This amounts 1\%-3\% (per track), depending on particle species.
%
 \item \textbf{Modelling of the RP system and validity of derived efficiency corrections ($\bm{\Delta\varepsilon_{\text{RP}}}$, $\bm{\Delta\varepsilon_{\text{RP}}^{\text{trig.}}}$ and $\bm{\Delta\varepsilon_{\text{RP}}^{\text{DM veto}}}$).}\\
 Reliability of RP simulation which was used to extract efficiency corrections with MC embedded into zero-bias data, has been verified and quantitatively estimated in the procedure described in Sec.~10.3 of Ref.~\cite{supplementaryNote}. For this purpose elastic proton-proton scattering events have been used. The same analysis has been performed on embedded elastic scattering MC and the data, leading to estimates of the RP acceptance and track reconstruction efficiency. The differences between two results has been considered as a measure of the systematic uncertainty that covers RP track reconstruction efficiency itself, detectors alignment, embedding technique. Similar studies have been performed to determine systematic uncertainty related to the trigger veto efficiency correction, as presented in Sec.~\ref{sec:systTrigVeto}.
 %
 \item \textbf{Pile-up veto correction ($\bm{\Delta\epsilon_{\text{veto}}}$).}\\
 Luminosity-dependent correction related to veto of pile-up interactions is derived from the zero-bias data on run by run basis. Residual systematic uncertainty has been estimated as a difference between the correction factor calculated for particular run, and correction factor obtained from the exponential fit to all points representing correction factors as a function of instantaneous luminosity.
 %
 \item \textbf{Longitudinal shape and position of the primary vertex distribution ($\bm{\Delta\langle z_{\text{vtx}}\rangle}$ and $\bm{\Delta\sigma( z_{\text{vtx}})}$).}\\
 Comparison of the $z_{\text{vtx}}$ distribution as seen in the TPC and in RPs leads to conservative estimate of the uncertainty of central position of the vertex equal 2~cm, and the spread (standard deviation) of vertex equal 3~cm.
 %
 \item \textbf{Non-exclusive background estimate ($\bm{\Delta N_{\text{bkgd}}^{\text{non-excl}}}$).}\\
 As explained in Sec.~\ref{sec:nonExclBkgdDetermination}, estimated systematic difference between real level of non-exclusive background, and level determined with a data-driven method, may be as high as 10\%. We apply such variation of the non-exclusive background and assign resulting differences of cross sections as their systematic uncertainty related to non-exclusive background determination method.
 %
 \item \textbf{Luminosity determination ($\bm{\Delta\mathcal{L}}$).}\\
 Uncertainty of the integrated luminosity has been estimated to 6\%, which is still subject to change.
\end{enumerate}

\section{Graphical representation of systematic uncertainties}
In this section we present relative contributions of effects listed in Sec.~\ref{sec:systEffectsList} to the total systematic uncertainties on differential fiducial cross sections presented in Chap.~\ref{chap:physicsResults}. Numbering of figures is preserved with respect to corresponding cross section results in the next chapter. The color code is the same in all figures, with the legend explaining the meaning of each color attached at the bottom of Fig.~\ref{systematics_01}.


\begin{figure}[h]
\centering
\includegraphics[width=.9\textwidth,page=1]{graphics/systematics/FinalResult_InvMass_pion_Systematics.pdf}
%
\caption{Systematic uncertainties of the differential cross sections for CEP of charged particle pairs $\pi^+\pi^-$ as a function of the invariant mass of the pair in the fiducial region explained on the plots.}
\label{systematics_01}
\end{figure}

% {
% \renewcommand{\arraystretch}{1.5}
% \begin{table}[]\centering
% \begin{tabular}{cc T{1.4cm}T{1.4cm}T{1.4cm} T{1.4cm}T{1.4cm}T{1.4cm}}
% \multicolumn{2}{c}{~}   & \multicolumn{3}{c}{$\bm{\Delta\varphi<90^{\circ}}$} &  \multicolumn{3}{c}{$\bm{\Delta\varphi>90^{\circ}}$}  \\
%     \multicolumn{2}{c}{~}  & \multicolumn{1}{c}{$\bm{\pi^{+}\pi^{-}}$} & \multicolumn{1}{c}{$\bm{K^{+}K^{-}}$} & \multicolumn{1}{c}{$\bm{p\bar{p}}$} & \multicolumn{1}{c}{$\bm{\pi^{+}\pi^{-}}$} & \multicolumn{1}{c}{$\bm{K^{+}K^{-}}$} & \multicolumn{1}{c}{$\bm{p\bar{p}}$} \\ \hline\hline
% \multirow{6}{*}{  \specialcell{ $\bm{ \frac{\delta_{\text{\bf{syst}}}}{\sigma_{\text{\bf{fid}}}} }$ \\ $\bm{[\text{\bf{\%}}]}$ }   } &  TOF & $\prescript{3.1}{-2.9}{~~~}$ & $\prescript{10.0}{-8.6}{~~~}$ & $\prescript{5.4}{-4.9}{~~~}$ & $\prescript{2.7}{-2.5}{~~~}$ & $\prescript{10.3}{-8.9}{~~~}$ & $\prescript{5.7}{-5.2}{~~~}$  \\
% &  TPC  & $\prescript{3.3}{-3.1}{~~~}$ & $\prescript{6.6}{-6.1}{~~~}$ & $\prescript{3.6}{-3.4}{~~~}$ & $\prescript{3.2}{-3.1}{~~~}$ & $\prescript{6.3}{-5.8}{~~~}$ & $\prescript{3.6}{-3.5}{~~~}$ \\
% &  RP  &  $\prescript{7.3}{-6.1}{~~~}$ & $\prescript{7.4}{-6.2}{~~~}$ & $\prescript{8.4}{-6.9}{~~~}$ & $\prescript{6.3}{-5.4}{~~~}$ & $\prescript{6.9}{-5.8}{~~~}$ & $\prescript{7.0}{-5.2}{~~~}$ \\
% &  Other  & $\prescript{2.6}{-2.4}{~~~}$ & $\prescript{2.7}{-2.4}{~~~}$ & $\prescript{2.9}{-2.4}{~~~}$ & $\prescript{2.6}{-2.4}{~~~}$ & $\prescript{2.6}{-2.4}{~~~}$ & $\prescript{3.2}{-2.4}{~~~}$ \\
% & Lumi. & \multicolumn{6}{c}{$7.0$} \\ \cline{2-8}
% &  Total & $\prescript{11.4}{-10.5}{~~~}$ & $\prescript{16.0}{-14.3}{~~~}$ & $\prescript{13.0}{-11.8}{~~~}$ & $\prescript{10.7}{-10.0}{~~~}$ & $\prescript{15.8}{-14.2}{~~~}$ & $\prescript{12.4}{-11.6}{~~~}$  \\ 
% \end{tabular}
% \caption{Systematic uncertainties of integrated fiducial cross sections for CEP of $\pi^{+}\pi^{-}$, $K^{+}K^{-}$ and $p\bar{p}$ pairs in two ranges of azimuthal angle difference $\Delta\varphi$ between forward scattered protons. Provided numbers are decomposed into major components. Single number in a cell indicates symmetric uncertainty, while positive and negative number denotes asymmetric uncertainty.}\label{tab:xSecSyst}
% \end{table}
% } 



{
\renewcommand{\arraystretch}{1.5}
\begin{table}[]\centering
\begin{tabular}{c ccccc|c}
 ~ & \multicolumn{6}{c}{  $\bm{ \delta_{\text{\bf{syst}}}/\sigma_{\text{\bf{fid}}}~[\text{\bf{\%}}]}$   } \\
 ~ & \bf{TOF} & \bf{TPC} & \bf{RP} & \bf{Other} & \bf{Lumi.} & \bf{Total} \\ \hline\hline
$\bm{\pi^{+}\pi^{-}}$ & $\prescript{3.1}{-2.9}{~~~}$ & $\prescript{3.3}{-3.1}{~~~}$ & $\prescript{7.3}{-6.1}{~~~}$ & $\prescript{2.6}{-2.4}{~~~}$ & \multirow{3}{*}{$\prescript{7.5}{-6.5}{~}$} & $\prescript{11.4}{-10.5}{~~~}$  \\
$\bm{K^{+}K^{-}}$  & $\prescript{10.0}{-8.6}{~~~}$ & $\prescript{6.6}{-6.1}{~~~}$ & $\prescript{7.4}{-6.2}{~~~}$ & $\prescript{2.7}{-2.4}{~~~}$ &  & $\prescript{16.0}{-14.3}{~~~}$ \\
$\bm{p\bar{p}}$  &  $\prescript{5.4}{-4.9}{~~~}$ & $\prescript{3.6}{-3.4}{~~~}$ & $\prescript{8.4}{-6.9}{~~~}$ & $\prescript{2.9}{-2.4}{~~~}$ &  & $\prescript{13.0}{-11.8}{~~~}$ \\
\end{tabular}
\caption{Typical systematic uncertainties of integrated fiducial cross sections for CEP of $\pi^{+}\pi^{-}$, $K^{+}K^{-}$ and $p\bar{p}$. Provided numbers are decomposed into major components. Single number in a cell indicates symmetric uncertainty, while positive and negative number denotes asymmetric uncertainty.}\label{tab:xSecSyst}
\end{table}
}


%
\begin{figure}[h]
\centering
\includegraphics[width=.48\textwidth,page=1]{graphics/systematics/FinalResult_InvMass_kaon_Systematics2.pdf}
\includegraphics[width=.48\textwidth,page=1]{graphics/systematics/FinalResult_InvMass_proton_Systematics2.pdf}
%
\caption{Systematic uncertainties of the differential cross sections for CEP of charged particle pairs $K^+K^-$ (left) and $p\bar{p}$ (right) as a function of the invariant mass of the pair in the fiducial region explained on the plots.}
\label{systematics_02}
\end{figure}
% 
\begin{figure}[h]
\centering
\includegraphics[width=.31\textwidth,page=1]{graphics/systematics/FinalResult_Rapidity_pion_Systematics2.pdf}
\hfill
\includegraphics[width=.31\textwidth,page=1]{graphics/systematics/FinalResult_Rapidity_kaon_Systematics2.pdf}
\hfill
\includegraphics[width=.31\textwidth,page=1]{graphics/systematics/FinalResult_Rapidity_proton_Systematics2.pdf}
%
\caption{Systematic uncertainties of the differential cross sections for CEP of charged particle pairs $\pi^+\pi^-$ (let), $K^+K^-$ (middle) and $p\bar{p}$ (right) as a function of the pair rapidity measured in the fiducial region explained on the plots.}
\label{systematics_1}
\end{figure}
%
% \indent
% Figure~\ref{systematics_2}(right column) shows the differential cross sections for CEP of different particle species pairs as a function of the sum of the squares of the four-momenta transfers at the proton vertices.
% %
% The shapes of measured cross sections are strongly affected by the fiducial cuts applied to the forward scattered protons.
% %
% The shapes of the differential cross sections for both $\pi^+\pi^-$ snd $K^+K^-$ pairs production are better described by the DiMe model than by GenEx and MBR models.
% In case of the cross section for $p\bar{p}$ pairs production the MBR model implemented in PYTHIA8 describes normalization of the data fairly well but predicts a steeper slope.\\
% %
% \indent
% Figure~\ref{systematics_3} shows the differential cross sections for CEP of different particle species pairs as a function of the pair invariant mass separately in two $\Delta\phi$ regions: $\Delta\phi<90$ degree (left column) and $\Delta\phi>90$ degree (right column).
% %
% Sharp drops of the measured cross sections at $m(\pi^+\pi^-) < 0.6$~GeV and at $m(K^+K^-) < 1.3$~GeV for the $\Delta\phi>$ 90 degree range are due to the fiducial cuts applied to the forward scattered protons. 
% %
% In case of the cross section for CEP of $\pi^+\pi^-$ pairs in $\Delta\phi<90$ degree range the peak around $f_2(1270)$ resonance in data is significantly suppressed while the peak at $f_0(980)$ is enhanced as well as possible resonances in the mass range $1.3-1.5$ MeV compared to the $\Delta\phi>90$ degrees range. 
%
\begin{figure}[h]
\centering
\includegraphics[width=.31\textwidth,page=1]{graphics/systematics/FinalResult_DeltaPhi_pion_Systematics2.pdf}
\hfill
\includegraphics[width=.31\textwidth,page=1]{graphics/systematics/FinalResult_DeltaPhi_kaon_Systematics2.pdf}
\hfill
\includegraphics[width=.31\textwidth,page=1]{graphics/systematics/FinalResult_DeltaPhi_proton_Systematics2.pdf}
\newline
\includegraphics[width=.31\textwidth,page=1]{graphics/systematics/FinalResult_MandelstamTSum_pion_Systematics2.pdf}
\hfill
\includegraphics[width=.31\textwidth,page=1]{graphics/systematics/FinalResult_MandelstamTSum_kaon_Systematics2.pdf}
\hfill
\includegraphics[width=.31\textwidth,page=1]{graphics/systematics/FinalResult_MandelstamTSum_proton_Systematics2.pdf}
%
\caption{Systematic uncertainties of the differential cross sections for CEP of charged particle pairs $\pi^+\pi^-$ (left column), $K^+K^-$ (middle column) and $p\bar{p}$ (right column) as a function of the difference of azimuthal angles of the forward scattered protons (top) and of the sum of the squares of the four-momenta losses in the proton vertices (bottom) measured in the fiducial region explained on the plots.}
\label{systematics_2}
\end{figure}
%
\begin{figure}[h]
\centering
\hspace*{5pt}
\includegraphics[width=.46\textwidth,page=1]{graphics/systematics/FinalResult_InvMass_DeltaPhiBin1_pion_Systematics2.pdf}
\hfill
\includegraphics[width=.46\textwidth,page=1]{graphics/systematics/FinalResult_InvMass_DeltaPhiBin2_pion_Systematics2.pdf}
\hspace*{5pt}
\newline
\hspace*{5pt}
\includegraphics[width=.46\textwidth,page=1]{graphics/systematics/FinalResult_InvMass_DeltaPhiBin1_kaon_Systematics2.pdf}
\hfill
\includegraphics[width=.46\textwidth,page=1]{graphics/systematics/FinalResult_InvMass_DeltaPhiBin2_kaon_Systematics2.pdf}
\hspace*{5pt}
\newline
\hspace*{5pt}
\includegraphics[width=.46\textwidth,page=1]{graphics/systematics/FinalResult_InvMass_DeltaPhiBin1_proton_Systematics2.pdf}
\hfill
\includegraphics[width=.46\textwidth,page=1]{graphics/systematics/FinalResult_InvMass_DeltaPhiBin2_proton_Systematics2.pdf}
\hspace*{5pt}
%
\caption{Systematic uncertainties of the differential cross sections for CEP of charged particle pairs $\pi^+\pi^-$ (top), $K^+K^-$ (middle) and $p\bar{p}$ (bottom) as a function of the invariant mass of the pair in two $\Delta\phi$ regions: $\Delta\phi<90$ degree (left column) and $\Delta\phi>90$ degree (right column) measured in the fiducial region explained on the plots.}
\label{systematics_3}
\end{figure}
% %
% \FloatBarrier
% %
% Such correlation between resonances seen in mass spectrum and azimuthal angle between outgoing protons indicates factorization breaking between the two proton vertices. In the range $\Delta\phi<$ 90 degrees the DiMe model well describes both normalization and shape of mass spectrum at $m(\pi^+\pi^-)<$ 0.5 GeV.
% %
% In case of the cross section for CEP of $K^+K^-$ pairs the data do not show any significant asymmetry except possible widening
% of the peak at $f_2^\prime(1520)$ in the region $\Delta\phi<90$ degrees which may indicate an enhancement of additional resonances around 1.7~GeV in this configuration.
% %
% In case of the cross section for CEP of $p\bar{p}$ pairs data do not show any significant asymmetry except possible enhancement in the $2.2-2.4$ mass range for the $\Delta\phi>90$ degrees region.\\
% %
% \indent
% Due to high statistics of the two-pion sample it is possible to study the CEP of $\pi^+\pi^-$ pairs in more detail.
% Figure~\ref{systematics_4} shows the differential cross sections for CEP of $\pi^+\pi^-$ pairs as a function of the pair rapidity (left column), $\Delta\phi$ (middle column) and $|t_1+t_2|$ (right column) in three characteristic ranges of the invariant mass of the pair: $m(\pi^+\pi^-)<1.0$ GeV (mainly non-resonant production), $1.0< m(\pi^+\pi^-) <1.5$ GeV ($f_2(1270)$ mass range) and $m(\pi^+\pi^-)>1.5$ GeV (higher invariant masses).\\
% %
% \noindent
% Figure~\ref{systematics_4} shows the differential cross sections for CEP of different particle species pairs as a function of the pair rapidity (left column), of the difference of forward protons azimuthal angles (middle column) and of the sum of squares of the four-momenta transfers at the proton vertices (right column), for the three invariant mass ranges. In the case of the cross section $d\sigma/dy$ all models agree in shape with data in all three mass ranges except for the GenEx and DiMe predictions in the highest mass range where predictions is narrower.
% 
% Strong suppression of the fiducial cross section close to $90^\circ$ is due to the STAR RP acceptance while asymmetry $0^\circ$ vs. $180^\circ$ in the lowest mass region is due to the STAR TPC acceptance. $\Delta\phi$ distribution is sensitive to absorption  which are treated fully differentialy in DiMe generator and only on average in GenEx. This is consistent with generally better agreement between data and DiMe expectations except $f_2(1270)$ mass region. MBR model predicts symmetric $\Delta\phi$ distributions in all mass ranges which is not supported by the data.
% 
% The slope of the cross section as the function of $|t_1+t_2|$ is less steep in the $f_2(1270)$ mass region compared to other mass regions. In the low mass region the DiMe prediction has steeper slope compared to data.\\
%
\begin{figure}[h]
\centering
\includegraphics[width=.31\textwidth,page=1]{graphics/systematics/FinalResult_Rapidity_pion_MassBin_1_Systematics2.pdf}
\hfill
\includegraphics[width=.31\textwidth,page=1]{graphics/systematics/FinalResult_DeltaPhi_pion_MassBin_1_Systematics2.pdf}
\hfill
\includegraphics[width=.31\textwidth,page=1]{graphics/systematics/FinalResult_MandelstamTSum_pion_MassBin_1_Systematics2.pdf}
\newline
\includegraphics[width=.31\textwidth,page=1]{graphics/systematics/FinalResult_Rapidity_pion_MassBin_2_Systematics2.pdf}
\hfill
\includegraphics[width=.31\textwidth,page=1]{graphics/systematics/FinalResult_DeltaPhi_pion_MassBin_2_Systematics2.pdf}
\hfill
\includegraphics[width=.31\textwidth,page=1]{graphics/systematics/FinalResult_MandelstamTSum_pion_MassBin_2_Systematics2.pdf}
\newline
\includegraphics[width=.31\textwidth,page=1]{graphics/systematics/FinalResult_Rapidity_pion_MassBin_3_Systematics2.pdf}
\hfill
\includegraphics[width=.31\textwidth,page=1]{graphics/systematics/FinalResult_DeltaPhi_pion_MassBin_3_Systematics2.pdf}
\hfill
\includegraphics[width=.31\textwidth,page=1]{graphics/systematics/FinalResult_MandelstamTSum_pion_MassBin_3_Systematics2.pdf}
%
\caption{Systematic uncertainties of the differential cross sections for CEP of $\pi^+\pi^-$ pairs as a function of the rapidity of the pair (left column) difference of azimuthal angles of the forward scattered protons (middle column) and of the sum of the squares of the four-momenta losses in the proton vertices (right column) measured in the fiducial region explained on the plots, separately for three ranges of the $\pi^+\pi^-$ pair invariant mass: $m<1$ GeV (top), $1<m<1.5$ GeV (middle) and $m>1.5$ GeV (bottom).}
\label{systematics_4}
\end{figure}
%
%\FloatBarrier
%
\begin{figure}[h]
\centering
\hspace*{5pt}
\includegraphics[width=.46\textwidth,page=1]{graphics/systematics/FinalResult_InvMass_pion_SmallDpt_DeltaPhiLessThan90_Systematics2.pdf}
\hfill
\includegraphics[width=.46\textwidth,page=1]{graphics/systematics/FinalResult_InvMass_pion_LargeDpt_DeltaPhiLessThan90_Systematics2.pdf}
\hspace*{5pt}
%
\caption{Systematic uncertainties of the differential cross sections $d\sigma/dm(\pi^+\pi^-)$ for CEP of $\pi^+\pi^-$ pairs in two $|\vec{p}_{1,T}^{\,\prime}-\vec{p}_{2,T}^{\,\prime}|$ regions: $|\vec{p}_{1,T}^{\,\prime}-\vec{p}_{2,T}^{\,\prime}|<0.12$ GeV (left) and $|\vec{p}_{1,T}^{\,\prime}-\vec{p}_{2,T}^{\,\prime}|>0.12$ GeV (right)  in the fiducial region and $\Delta\phi<90$ degree.}
\label{systematics_5}
\end{figure}


% We have also studied angular distributions of the charged particles produced in the final state. This can be done in various reference frames. However, for an easy comparison with theoretical predictions we use here the Collins-Soper \cite{cs_frame} reference frame also used e.g. in Ref.~\cite{lebiedowicz_3}. Collins-Soper frame is the centre-of-mass frame of the charged particles pair with the $z$-axis making equal angles with the beam protons momenta which in addition define the new $x-z$ plane. It can be reached from the laboratory frame (proton-proton c.m.s.) in two steps. First, boost along the $z$-axis to an intermediate frame in which the pair longitudinal momentum is equal to zero. In this frame the beam protons momenta remain parallel to the $z$-axis and the transverse momentum of the pair remains unchanged. Second, boost in the direction of the transverse momentum of the pair, to get to the pair c.m.s. frame. 
%
\begin{figure}[h]
\centering
\includegraphics[width=.31\textwidth,page=1]{graphics/systematics/FinalResult_CosThetaCS_pion_MassBin_1_Systematics2.pdf}
\hfill
\includegraphics[width=.31\textwidth,page=1]{graphics/systematics/FinalResult_CosThetaCS_pion_MassBin_2_Systematics2.pdf}
\hfill
\includegraphics[width=.31\textwidth,page=1]{graphics/systematics/FinalResult_CosThetaCS_pion_MassBin_3_Systematics2.pdf}
\newline
\includegraphics[width=.31\textwidth,page=1]{graphics/systematics/FinalResult_PhiCS_pion_MassBin_1_Systematics2.pdf}
\hfill
\includegraphics[width=.31\textwidth,page=1]{graphics/systematics/FinalResult_PhiCS_pion_MassBin_2_Systematics2.pdf}
\hfill
\includegraphics[width=.31\textwidth,page=1]{graphics/systematics/FinalResult_PhiCS_pion_MassBin_3_Systematics2.pdf}
%
\caption{Systematic uncertainties of the differential cross sections for CEP of $\pi^+\pi^-$ pairs as a function of $\cos{\theta^\mathrm{CS}}$ (top) and of $\phi^\mathrm{CS}$ (bottom)  measured in the fiducial region explained on the plots, separately for three ranges of the $\pi^+\pi^-$ pair invariant mass: $m<1$ GeV (left column), $1<m<1.5$ GeV (middle column) and $m>1.5$ GeV (right column).}
\label{systematics_7}
\end{figure}


%% =====  PHYSICS RESULTS ====
%%===========================================================%%
%%                                                           %%
%%                    PHYSICS RESULTS                        %%
%%                                                           %%
%%===========================================================%%


\chapter{Physics results}\label{chap:physicsResults}

\section{Differential cross sections}

\begin{figure}[h]
\centering
\includegraphics[width=.7\textwidth,page=1]{graphics/physicsResults/FinalResult_InvMass_pion.pdf}
%
\caption[Differential cross sections for CEP of charged particle pairs $\pi^+\pi^-$ as a function of the invariant mass of the pair in the fiducial region.]{Differential cross sections for CEP of charged particle pairs $\pi^+\pi^-$ as a function of the invariant mass of the pair in the fiducial region explained on the plots. Data are shown as solid points with error bars representing the statistical uncertainties. The typical systematic uncertainties are shown as gray boxes for only few data points as they are almost fully correlated between neighboring bins. Predictions from MC models GenEx, DiMe and MBR are shown as histograms. In the lower panels in the bottom plots the ratios of the MC predictions scaled to data and the data are shown.}
\label{results_01}
\end{figure}
%
\begin{figure}[h]
\centering
\includegraphics[width=.48\textwidth,page=1]{graphics/physicsResults/FinalResult_InvMass_kaon.pdf}
\includegraphics[width=.48\textwidth,page=1]{graphics/physicsResults/FinalResult_InvMass_proton.pdf}
%
\caption[Differential cross sections for CEP of charged particle pairs $K^+K^-$ and $p\bar{p}$ as a function of the invariant mass of the pair in the fiducial region.]{Differential cross sections for CEP of charged particle pairs $K^+K^-$ (left) and $p\bar{p}$ (right) as a function of the invariant mass of the pair in the fiducial region explained on the plots. Data are shown as solid points with error bars representing the statistical uncertainties. The typical systematic uncertainties are shown as gray boxes for only few data points as they are almost fully correlated between neighboring bins. Predictions from MC models GenEx, DiMe and MBR are shown as histograms. In the lower panels in the bottom plots the ratios of the MC predictions scaled to data and the data are shown.}
\label{results_02}
\end{figure}
% 
\begin{figure}[h]
\centering
\includegraphics[width=.31\textwidth,page=1]{graphics/physicsResults/Ratio_FinalResult_Rapidity_pion.pdf}
\hfill
\includegraphics[width=.31\textwidth,page=1]{graphics/physicsResults/Ratio_FinalResult_Rapidity_kaon.pdf}
\hfill
\includegraphics[width=.31\textwidth,page=1]{graphics/physicsResults/Ratio_FinalResult_Rapidity_proton.pdf}
%
\caption[Differential cross sections for CEP of charged particle pairs $\pi^+\pi^-$, $K^+K^-$ and $p\bar{p}$ as a function of the pair rapidity measured in the fiducial region.]{Differential cross sections for CEP of charged particle pairs $\pi^+\pi^-$ (left), $K^+K^-$ (middle) and $p\bar{p}$ (right) as a function of the pair rapidity measured in the fiducial region explained on the plots. Data are shown as solid points with error bars representing the statistical uncertainties. The typical systematic uncertainties are shown as gray boxes for only few data points as they are almost fully correlated between neighboring bins. Predictions from MC models GenEx, DiMe and MBR are shown as histograms. In the lower panels in the bottom plots the ratios of the MC predictions scaled to data and the data are shown.}
\label{results_1}
\end{figure}
%
% \indent
% Figure~\ref{results_2}(right column) shows the differential cross sections for CEP of different particle species pairs as a function of the sum of the squares of the four-momenta transfers at the proton vertices.
% %
% The shapes of measured cross sections are strongly affected by the fiducial cuts applied to the forward scattered protons.
% %
% The shapes of the differential cross sections for both $\pi^+\pi^-$ snd $K^+K^-$ pairs production are better described by the DiMe model than by GenEx and MBR models.
% In case of the cross section for $p\bar{p}$ pairs production the MBR model implemented in PYTHIA8 describes normalization of the data fairly well but predicts a steeper slope.\\
% %
% \indent
% Figure~\ref{results_3} shows the differential cross sections for CEP of different particle species pairs as a function of the pair invariant mass separately in two $\Delta\phi$ regions: $\Delta\phi<90$ degree (left column) and $\Delta\phi>90$ degree (right column).
% %
% Sharp drops of the measured cross sections at $m(\pi^+\pi^-) < 0.6$~GeV and at $m(K^+K^-) < 1.3$~GeV for the $\Delta\phi>$ 90 degree range are due to the fiducial cuts applied to the forward scattered protons. 
% %
% In case of the cross section for CEP of $\pi^+\pi^-$ pairs in $\Delta\phi<90$ degree range the peak around $f_2(1270)$ resonance in data is significantly suppressed while the peak at $f_0(980)$ is enhanced as well as possible resonances in the mass range $1.3-1.5$ MeV compared to the $\Delta\phi>90$ degrees range. 
%
\begin{figure}[h]
\centering
\includegraphics[width=.31\textwidth,page=1]{graphics/physicsResults/Ratio_FinalResult_DeltaPhi_pion.pdf}
\hfill
\includegraphics[width=.31\textwidth,page=1]{graphics/physicsResults/Ratio_FinalResult_DeltaPhi_kaon.pdf}
\hfill
\includegraphics[width=.31\textwidth,page=1]{graphics/physicsResults/Ratio_FinalResult_DeltaPhi_proton.pdf}
\newline
\includegraphics[width=.31\textwidth,page=1]{graphics/physicsResults/Ratio_FinalResult_MandelstamTSum_pion.pdf}
\hfill
\includegraphics[width=.31\textwidth,page=1]{graphics/physicsResults/Ratio_FinalResult_MandelstamTSum_kaon.pdf}
\hfill
\includegraphics[width=.31\textwidth,page=1]{graphics/physicsResults/Ratio_FinalResult_MandelstamTSum_proton.pdf}
%
\caption[Differential cross sections for CEP of charged particle pairs $\pi^+\pi^-$, $K^+K^-$ and $p\bar{p}$ as a function of the difference of azimuthal angles of the forward scattered protons and of the sum of the squares of the four-momenta losses in the proton vertices measured in the fiducial region.]{Differential cross sections for CEP of charged particle pairs $\pi^+\pi^-$ (left column), $K^+K^-$ (middle column) and $p\bar{p}$ (right column) as a function of the difference of azimuthal angles of the forward scattered protons (top) and of the sum of the squares of the four-momenta losses in the proton vertices (bottom) measured in the fiducial region explained on the plots. Data are shown as solid points with error bars representing the statistical uncertainties. The typical systematic uncertainties are shown as gray boxes for only few data points as they are almost fully correlated between neighboring bins. Predictions from MC models GenEx, DiMe and MBR are shown as histograms. In the lower panels the ratios of the MC predictions scaled to data and the data are shown.}
\label{results_2}
\end{figure}
%
\begin{figure}[h]
\centering
\hspace*{5pt}
\includegraphics[width=.46\textwidth,page=1]{graphics/physicsResults/FinalResult_InvMass_DeltaPhiBin1_pion.pdf}
\hfill
\includegraphics[width=.46\textwidth,page=1]{graphics/physicsResults/FinalResult_InvMass_DeltaPhiBin2_pion.pdf}
\hspace*{5pt}
\newline
\hspace*{5pt}
\includegraphics[width=.46\textwidth,page=1]{graphics/physicsResults/FinalResult_InvMass_DeltaPhiBin1_kaon.pdf}
\hfill
\includegraphics[width=.46\textwidth,page=1]{graphics/physicsResults/FinalResult_InvMass_DeltaPhiBin2_kaon.pdf}
\hspace*{5pt}
\newline
\hspace*{5pt}
\includegraphics[width=.46\textwidth,page=1]{graphics/physicsResults/FinalResult_InvMass_DeltaPhiBin1_proton.pdf}
\hfill
\includegraphics[width=.46\textwidth,page=1]{graphics/physicsResults/FinalResult_InvMass_DeltaPhiBin2_proton.pdf}
\hspace*{5pt}
%
\caption[Differential cross sections for CEP of charged particle pairs $\pi^+\pi^-$, $K^+K^-$ and $p\bar{p}$ as a function of the invariant mass of the pair in two $\Delta\phi$ regions: $\Delta\phi<90$ degree and $\Delta\phi>90$ degree  measured in the fiducial region explained on the plots.]{Differential cross sections for CEP of charged particle pairs $\pi^+\pi^-$ (top), $K^+K^-$ (middle) and $p\bar{p}$ (bottom) as a function of the invariant mass of the pair in two $\Delta\phi$ regions: $\Delta\phi<90$ degree (left column) and $\Delta\phi>90$ degree (right column) measured in the fiducial region explained on the plots. Data are shown as solid points with error bars representing the statistical uncertainties. The typical systematic uncertainties are shown as gray boxes for only few data points as they are almost fully correlated between neighboring bins. Predictions from MC models GenEx, DiMe and MBR are shown as histograms.}
\label{results_3}
\end{figure}



{
\renewcommand{\arraystretch}{1.5}
\begin{table}[]\centering
\begin{tabular}{cc c c}
~ & ~ & \multicolumn{2}{c}{$\bm{ \sigma_{\text{\bf{fid}}} \pm \delta_{\text{\bf{stat}}} \pm \delta_{\text{\bf{syst}}}}$} \\
 \bf{PID} & \bf{unit} & $\bm{\Delta\varphi<90^{\circ}}$ & $\bm{\Delta\varphi>90^{\circ}}$ \\ \hline\hline
 $\bm{\pi^{+}\pi^{-}}$ & \bf{nb} & $38.1\pm0.2^{+4.3}_{-4.0}$ & $18.4\pm0.1^{+2.0}_{-1.8}$ \\ %\hline
 $\bm{K^{+}K^{-}}$ & \bf{pb} & $976\pm46^{+156}_{-140}$ & $533\pm33^{+84}_{-76}$ \\ %\hline
 $\bm{p\bar{p}}$ & \bf{pb} & $17.7\pm3.6^{+2.3}_{-2.1}$ & $31.5\pm5.4^{+3.9}_{-3.6}$\\ %\hline%\hline
\end{tabular}
\caption{Integrated fiducial cross sections for CEP of $\pi^{+}\pi^{-}$, $K^{+}K^{-}$ and $p\bar{p}$ pairs in two ranges of azimuthal angle difference $\Delta\varphi$ between forward scattered protons. Statistical and systematic uncertainties are provided for each cross section.}\label{tab:xSec}
\end{table}
}


% %
% \FloatBarrier
% %
% Such correlation between resonances seen in mass spectrum and azimuthal angle between outgoing protons indicates factorization breaking between the two proton vertices. In the range $\Delta\phi<$ 90 degrees the DiMe model well describes both normalization and shape of mass spectrum at $m(\pi^+\pi^-)<$ 0.5 GeV.
% %
% In case of the cross section for CEP of $K^+K^-$ pairs the data do not show any significant asymmetry except possible widening
% of the peak at $f_2^\prime(1520)$ in the region $\Delta\phi<90$ degrees which may indicate an enhancement of additional resonances around 1.7~GeV in this configuration.
% %
% In case of the cross section for CEP of $p\bar{p}$ pairs data do not show any significant asymmetry except possible enhancement in the $2.2-2.4$ mass range for the $\Delta\phi>90$ degrees region.\\
% %
% \indent
% Due to high statistics of the two-pion sample it is possible to study the CEP of $\pi^+\pi^-$ pairs in more detail.
% Figure~\ref{results_4} shows the differential cross sections for CEP of $\pi^+\pi^-$ pairs as a function of the pair rapidity (left column), $\Delta\phi$ (middle column) and $|t_1+t_2|$ (right column) in three characteristic ranges of the invariant mass of the pair: $m(\pi^+\pi^-)<1.0$ GeV (mainly non-resonant production), $1.0< m(\pi^+\pi^-) <1.5$ GeV ($f_2(1270)$ mass range) and $m(\pi^+\pi^-)>1.5$ GeV (higher invariant masses).\\
% %
% \noindent
% Figure~\ref{results_4} shows the differential cross sections for CEP of different particle species pairs as a function of the pair rapidity (left column), of the difference of forward protons azimuthal angles (middle column) and of the sum of squares of the four-momenta transfers at the proton vertices (right column), for the three invariant mass ranges. In the case of the cross section $d\sigma/dy$ all models agree in shape with data in all three mass ranges except for the GenEx and DiMe predictions in the highest mass range where predictions is narrower.
% 
% Strong suppression of the fiducial cross section close to $90^\circ$ is due to the STAR RP acceptance while asymmetry $0^\circ$ vs. $180^\circ$ in the lowest mass region is due to the STAR TPC acceptance. $\Delta\phi$ distribution is sensitive to absorption  which are treated fully differentialy in DiMe generator and only on average in GenEx. This is consistent with generally better agreement between data and DiMe expectations except $f_2(1270)$ mass region. MBR model predicts symmetric $\Delta\phi$ distributions in all mass ranges which is not supported by the data.
% 
% The slope of the cross section as the function of $|t_1+t_2|$ is less steep in the $f_2(1270)$ mass region compared to other mass regions. In the low mass region the DiMe prediction has steeper slope compared to data.\\
%
\begin{figure}[h]
\centering
\includegraphics[width=.31\textwidth,page=1]{graphics/physicsResults/Ratio_FinalResult_Rapidity_pion_MassBin_1.pdf}
\hfill
\includegraphics[width=.31\textwidth,page=1]{graphics/physicsResults/Ratio_FinalResult_DeltaPhi_pion_MassBin_1.pdf}
\hfill
\includegraphics[width=.31\textwidth,page=1]{graphics/physicsResults/Ratio_FinalResult_MandelstamTSum_pion_MassBin_1.pdf}
\newline
\includegraphics[width=.31\textwidth,page=1]{graphics/physicsResults/Ratio_FinalResult_Rapidity_pion_MassBin_2.pdf}
\hfill
\includegraphics[width=.31\textwidth,page=1]{graphics/physicsResults/Ratio_FinalResult_DeltaPhi_pion_MassBin_2.pdf}
\hfill
\includegraphics[width=.31\textwidth,page=1]{graphics/physicsResults/Ratio_FinalResult_MandelstamTSum_pion_MassBin_2.pdf}
\newline
\includegraphics[width=.31\textwidth,page=1]{graphics/physicsResults/Ratio_FinalResult_Rapidity_pion_MassBin_3.pdf}
\hfill
\includegraphics[width=.31\textwidth,page=1]{graphics/physicsResults/Ratio_FinalResult_DeltaPhi_pion_MassBin_3.pdf}
\hfill
\includegraphics[width=.31\textwidth,page=1]{graphics/physicsResults/Ratio_FinalResult_MandelstamTSum_pion_MassBin_3.pdf}
%
\caption[Differential cross sections for CEP of $\pi^+\pi^-$ pairs as a function of the rapidity of the pair, difference of azimuthal angles of the forward scattered protons and of the sum of the squares of the four-momenta losses in the proton vertices measured in the fiducial region explained on the plots, separately for three ranges of the $\pi^+\pi^-$ pair invariant mass: $m<1$ GeV, $1<m<1.5$ GeV and $m>1.5$ GeV.]{Differential cross sections for CEP of $\pi^+\pi^-$ pairs as a function of the rapidity of the pair (left column) difference of azimuthal angles of the forward scattered protons (middle column) and of the sum of the squares of the four-momenta losses in the proton vertices (right column) measured in the fiducial region explained on the plots, separately for three ranges of the $\pi^+\pi^-$ pair invariant mass: $m<1$ GeV (top), $1<m<1.5$ GeV (middle) and $m>1.5$ GeV (bottom). Data are shown as solid points with error bars representing the statistical uncertainties. The typical systematic uncertainties are shown as gray boxes for only few data points as they are almost fully correlated between neighboring bins. Predictions from MC models GenEx, DiMe and MBR are shown as histograms. In the lower panels the ratios of the MC predictions scaled to data and the data are shown.}
\label{results_4}
\end{figure}
%
%\FloatBarrier
%
\begin{figure}[h]
\centering
\hspace*{5pt}
\includegraphics[width=.46\textwidth,page=1]{graphics/physicsResults/FinalResult_InvMass_pion_SmallDpt_DeltaPhiLessThan90.pdf}
\hfill
\includegraphics[width=.46\textwidth,page=1]{graphics/physicsResults/FinalResult_InvMass_pion_LargeDpt_DeltaPhiLessThan90.pdf}
\hspace*{5pt}
%
\caption[Differential cross sections $d\sigma/dm(\pi^+\pi^-)$ for CEP of $\pi^+\pi^-$ pairs in two $|\vec{p}_{1,T}^{\,\prime}-\vec{p}_{2,T}^{\,\prime}|$ regions: $|\vec{p}_{1,T}^{\,\prime}-\vec{p}_{2,T}^{\,\prime}|<0.12$ GeV and $|\vec{p}_{1,T}^{\,\prime}-\vec{p}_{2,T}^{\,\prime}|>0.12$ GeV in the fiducial region and $\Delta\phi<90$ degree.]{Differential cross sections $d\sigma/dm(\pi^+\pi^-)$ for CEP of $\pi^+\pi^-$ pairs in two $|\vec{p}_{1,T}^{\,\prime}-\vec{p}_{2,T}^{\,\prime}|$ regions: $|\vec{p}_{1,T}^{\,\prime}-\vec{p}_{2,T}^{\,\prime}|<0.12$ GeV (left) and $|\vec{p}_{1,T}^{\,\prime}-\vec{p}_{2,T}^{\,\prime}|>0.12$ GeV (right)  in the fiducial region and $\Delta\phi<90$ degree. There is no difference for two $|\vec{p}_{1,T}^{\,\prime}-\vec{p}_{2,T}^{\,\prime}|$ regions. Data are shown as solid points with error bars representing the statistical uncertainties. The typical systematic uncertainties are shown as gray boxes for only few data points as they are almost fully correlated between neighboring bins. Predictions from MC models GenEx, DiMe and MBR are shown as histograms.}
\label{results_5}
\end{figure}


% We have also studied angular distributions of the charged particles produced in the final state. This can be done in various reference frames. However, for an easy comparison with theoretical predictions we use here the Collins-Soper \cite{cs_frame} reference frame also used e.g. in Ref.~\cite{lebiedowicz_3}. Collins-Soper frame is the centre-of-mass frame of the charged particles pair with the $z$-axis making equal angles with the beam protons momenta which in addition define the new $x-z$ plane. It can be reached from the laboratory frame (proton-proton c.m.s.) in two steps. First, boost along the $z$-axis to an intermediate frame in which the pair longitudinal momentum is equal to zero. In this frame the beam protons momenta remain parallel to the $z$-axis and the transverse momentum of the pair remains unchanged. Second, boost in the direction of the transverse momentum of the pair, to get to the pair c.m.s. frame.
%
\begin{figure}[h]
\centering
\includegraphics[width=.31\textwidth,page=1]{graphics/physicsResults/Ratio_FinalResult_CosThetaCS_pion_MassBin_1.pdf}
\hfill
\includegraphics[width=.31\textwidth,page=1]{graphics/physicsResults/Ratio_FinalResult_CosThetaCS_pion_MassBin_2.pdf}
\hfill
\includegraphics[width=.31\textwidth,page=1]{graphics/physicsResults/Ratio_FinalResult_CosThetaCS_pion_MassBin_3.pdf}
\newline
\includegraphics[width=.31\textwidth,page=1]{graphics/physicsResults/Ratio_FinalResult_PhiCS_pion_MassBin_1.pdf}
\hfill
\includegraphics[width=.31\textwidth,page=1]{graphics/physicsResults/Ratio_FinalResult_PhiCS_pion_MassBin_2.pdf}
\hfill
\includegraphics[width=.31\textwidth,page=1]{graphics/physicsResults/Ratio_FinalResult_PhiCS_pion_MassBin_3.pdf}
%
\caption[Differential cross sections for CEP of $\pi^+\pi^-$ pairs as a function of $\cos{\theta^\mathrm{CS}}$ and of $\phi^\mathrm{CS}$ measured in the fiducial region explained on the plots, separately for three ranges of the $\pi^+\pi^-$ pair invariant mass: $m<1$ GeV, $1<m<1.5$ GeV and $m>1.5$ GeV.]{Differential cross sections for CEP of $\pi^+\pi^-$ pairs as a function of $\cos{\theta^\mathrm{CS}}$ (top) and of $\phi^\mathrm{CS}$ (bottom)  measured in the fiducial region explained on the plots, separately for three ranges of the $\pi^+\pi^-$ pair invariant mass: $m<1$ GeV (left column), $1<m<1.5$ GeV (middle column) and $m>1.5$ GeV (right column). Data are shown as solid points with error bars representing the statistical uncertainties. The typical systematic uncertainties are shown as gray boxes for only few data points as they are almost fully correlated between neighboring bins. Predictions from MC models GenEx, DiMe and MBR are shown as histograms. In the lower panels the ratios of the MC predictions scaled to data and the data are shown.}
\label{results_7}
\end{figure}


\section{Invariant mass spectrum modelling}\label{sec:InvMassFit}

We make an attempt to fit extrapolated differential cross-section with a simplified model of the $\pi^{+}\pi^{-}$ invariant mass spectrum. In this model we assume contributions from the direct pair production and three resonances in the mass range of $0.6-1.7$ GeV: $f_0(980)$, $f_2(1270)$ and $f_0(1500)$. It is important to note, that only $f_{2}(1270)$ has in the fit fixed mass and width and thus is explicitely assumed. For the other two resonances the masses and widths are left free, however their fitted masses and widths are compatible with $f_0(980)$ and $f_0(1500)$, therefore we use such labeling from the very beginning.

The total amplitude for the exclusive $\pi^{+}\pi^{-}$ production is given by:
%
\begin{equation}
\label{eq:amplitude}
\begin{aligned}
A(m) = & \;A_{\textrm{cont}}\times f_{\textrm{cont}(m)}+ \\
        & \;A_{\textrm{f}_0(980)} \times \mathcal{R}_{\textrm{F}}\left(m;M_{f_0(980)},\Gamma_{f_0(980)}\right)+ \\
        & \;A_{\textrm{f}_2(1270)} \times \mathcal{R}_{\textrm{BW}}\left(m;M_{f_2(1270)},\Gamma_{f_2(1270)}\right) +\\
        & \;A_{\textrm{f}_0(1500)} \times \mathcal{R}_{\textrm{BW}}\left(m;M_{f_0(1500)},\Gamma_{f_0(1500)}\right),\\
\end{aligned}
\end{equation}
%
thus all states are added coherently (interfere with each other). The amplitude for continuum production is chosen to be real while multiplicative amplitude factors for resonances are allowed to be complex:
\begin{equation}A_{\textrm{cont}}\in\mathbb{R},~~~A_{\textrm{f}_0(980)},A_{\textrm{f}_2(1270)},A_{\textrm{f}_0(1500)}\in\mathbb{C}~~~~\rightarrow~~~~A_{\textrm{f}}=|A_{\textrm{f}}|e^{i\phi_{\textrm{f}}}.\end{equation}
%
The shape of the continuum amplitude is assumed to have the form
\begin{equation}f_{\textrm{cont}}(m) = \sqrt{\frac{q}{m}}\times e^{-\frac{B}{2}\cdot m}\end{equation}
with the break-up momentum $q$ equal to
\begin{equation}\label{eq:breakupMom}
q(m) = \frac{1}{2}\sqrt{m^{2}-4m_{\pi}^{2}}.
\end{equation}
For $f_2(1270)$ and $f_0(1500)$ resonances we use relativistic Breit-Wigner form of the production amplitude with mass-dependent width:
\begin{equation}\label{eq:BW}\mathcal{R}_{\textrm{BW}}(m;M,\Gamma_{0}) = \frac{M\sqrt{\Gamma_{0}}\sqrt{\Gamma(m)}}{M^{2}-m^{2}-i M\Gamma(m)},~~\Gamma(m) = \Gamma_{0}\frac{q}{m}\frac{M}{q_{0}}\left(\frac{B_{J}(q^{2}R^{2})}{B_{J}(q_{0}^{2}R^{2})}\right)^{2}.\end{equation}
The centrifugal effects in Eq.~\eqref{eq:BW} are accounted through the Blatt-Weisskopf barrier factors $B_{J}$~\cite{BarrierFactors} with the empirical interaction radius $R$ set to 1~fm and $q_{0} = q(M)$. Naturally $J=2$ and $J=0$ is used for $f_2(1270)$ and $f_0(1500)$, respectively.

Meson $f_0(980)$ requires different treatment due to large branching ratio to $K\bar{K}$ channel which opens in the vicinity of the mass peak. This changes the resonance shape and is accounted for in the parametrisation of the amplitude via the Flatt\'e formula~\cite{Flatte}:
\begin{equation}\label{eq:Flatte}\mathcal{R}_{\textrm{F}}(m;M,\Gamma_{0}) = \frac{M\sqrt{\Gamma_{0}}\sqrt{\Gamma_{\pi}(m)}}{M^{2}-m^{2}-i M\left(\Gamma_{\pi}(m)+\Gamma_{K}(m)\right)}\end{equation}
%
with the partial width $\Gamma_{j}$ ($j=\pi, K$) described by the product of the coupling parameter $g_{j}$ and the break-up momentum $q$ (Eq.~\eqref{eq:breakupMom}) for particle $j$:
\begin{equation}
    \Gamma_{j}(m) = g_{j}q_{j}(m) = \frac{g_{j}}{2}\sqrt{m^{2}-4m_{j}^{2}}.
\end{equation} and the partial width in $\pi^{+}\pi^{-}$ channel at the resonance mass equal to
\begin{equation}
    \Gamma_{0} = g_{\pi}q_{\pi}(M).
\end{equation}
In the fit the ratio $g_{K}/g_{\pi}$ is fixed to 4.21, the value well constrained experimentally through the measurement of $J/\psi$ decay into $\phi$ and $\pi^{+}\pi^{-}$/$K^{+}K^{-}$~\cite{BES_JPsi}.
%

Squared amplitude from Eq.~\eqref{eq:amplitude}, $|A|^{2}$, is convoluted for the purpose of the fit with the normal distribution $\mathcal{N}(m; \sigma_{\text{res}})$ representing finite resolution of reconstructed invariant mass of the pion pair. The resolution parameter $\sigma_{\text{res}}$ is provided to the fitting algorithm; it is set to grow linearly with increasing invariant mass according to MC simulation of the STAR TPC detector. The $m(\pi^{+}\pi^{-})$ resolution at the lower and upper limit of the fit range is equal to 4~MeV and 13~MeV, respectively. The final form of function fitted to extrapolated $d\sigma/dm(\pi^{+}\pi^{-})$ is given by
\begin{equation}
    \mathcal{F}(m) = \int\limits_{m-3.5\cdot\sigma_{\text{res}}(m)}^{m+3.5\cdot\sigma_{\text{res}}(m)}dm'\mathcal{N}\left(m'-m; \sigma_{\text{res}}(m)\right)|A(m')|^{2}.
\end{equation}

The fitting is performed using Minuit2 toolkit~\cite{Minuit2} within the ROOT analysis software~\cite{ROOT}. The standard-defined $\chi^{2}$ is minimized simultaneously in two $\Delta\varphi$ ranges with the masses and widths of both $f_0$'s forced to be equal, while phases and absolute values of $f_2$ and $f_0$'s amplitudes left independent in the two $\Delta\varphi$ subsets. The mass and width of $f_2(1270)$ resonance is fixed to the well known Particle Data Group values~\cite{pdg}.
%

Systematic uncertainties of the model parameters are estimated through the independent fits to extrapolated $d\sigma/dm(\pi^{+}\pi^{-})$ with applied each of the systematic variations described in Sec.~\ref{sec:systematics}. In addition to this we take into consideration uncertainties related to the geometrical acceptance correction - we check the effect of extrapolation to full kinematic region of the pions determined with the pure $D_{0}$-wave instead of nominally used $S_{0}$-wave. At the end, systematic uncertainty on the parameter is calculated as a quadratic sum of the differences between the nominal fit result and the result of the fit to $d\sigma/dm(\pi^{+}\pi^{-})$ with each systematic effect considered.

\begin{figure}%[t]
\centering
\includegraphics[width=\textwidth,page=1]{graphics/physicsResults/InvMassFit/RatioAndInterference_PiPiInvMass_Fit.pdf}
%
\caption[Differential cross-section $d\sigma/dm(\pi^{+}\pi^{-})$ extrapolated from the fiducial region to the Lorentz invariant phase space given by the central state rapidity $|y(\pi^{+}\pi^{-})|<0.4$ and squared four-momentum transferred in forward proton vertices $0.05~\text{GeV}^{2} < -t_{1}, -t_{2} < 0.16~\text{GeV}^{2}$]{Differential cross-section $d\sigma/dm(\pi^{+}\pi^{-})$ extrapolated from the fiducial region to the Lorentz invariant phase space given by the central state rapidity $|y(\pi^{+}\pi^{-})|<0.4$ and squared four-momentum transferred in forward proton vertices $0.05~\text{GeV}^{2} < -t_{1}, -t_{2} < 0.16~\text{GeV}^{2}$. Left and right parts of the figure show cross-sections for $\Delta\varphi<45^\circ$ and $\Delta\varphi>135^\circ$, respectively. The data are shown as black points with error bars representing statistical uncertainties. Result of the fit $\mathcal{F}(m)$ is drawn with solid black line. The squared amplitudes for continuum and resonance production are drawn with lines of different colors, as explained in the legend. The most significant interference terms are plotted in the middle panels, while the relative difference between each data point and fitted model is drawn in the bottom panel.}
\label{invMassFit}
\end{figure}

The extrapolated cross-sections are shown in Fig.~\ref{invMassFit} together with the result of the fit described above. The model parameters providing minimum $\chi^{2}$ are listed in Tab.~\ref{tab:fitRes}. The fit with the total of 20 free parameters gives $\chi^2$/ndf = 289/200 which shows that the data and model are in reasonable agreement in the fit region. Three resonances are required in the mass range of $0.6-1.7$~GeV: $f_0(980)$, $f_2(1270)$ and $f_0(1500)$. The deviation of the fitted model from the extrapolated data is the most significant at the end part of the fitted mass range, above $1.5-1.55$~GeV. This might result from the presence of $f_{0}(1710)$ and other wide resonances above the fit limits, which are not included in the model.

{
\renewcommand{\arraystretch}{1.5}
\begin{table}[]\centering
\begin{tabular}{ccc c c} ~ & ~ & ~ &\multicolumn{2}{c}{$\bm{ p \pm \delta_{\text{\bf{stat}}} \pm \delta_{\text{\bf{syst}}} \pm \delta_{\text{\bf{model}}}}$} \\ ~ & \bf{parameter} & \bf{unit} & $\bm{\Delta\varphi<45^{\circ}}$ & $\bm{\Delta\varphi>135^{\circ}}$ \\ \hline\hline \multirow{2}{*}{\bf{Continuum}} & $\bm{A}$ & $\bm{\left(\text{\bf{nb/GeV}}\right)^{\frac{1}{2}}}$ & $63.9 \pm 3.8 ^{+4.5}_{-8.1}$ & $35.8 \pm 1.8 ^{+3.0}_{-2.9}$ \\ %\hline
& $\bm{B}$ & $\bm{\text{\bf{GeV}}^{-1}}$ & $6.3 \pm 0.4 ^{+0.1}_{-0.2}$ & $4.7 \pm 0.3 ^{+0.2}_{-0.2}$ \\ \hline
\multirow{5}{*}{$\bm{f_{0}(980)}$} & $\bm{|A|}$ & $\bm{\left(\text{\bf{nb/GeV}}\right)^{\frac{1}{2}}}$ & $20.9 \pm 0.5 ^{+1.1}_{-1.4}$ & $7.8 \pm 0.4 ^{+0.4}_{-0.5}$ \\ %\hline
& $\bm{\phi}$ & \bf{rad} & $0.68 \pm 0.07 ^{+0.02}_{-0.03}$ & $0.54 \pm 0.08 ^{+0.01}_{-0.02}$ \\ %\hline
& $\bm{M}$ & \bf{MeV} & \multicolumn{2}{c}{$955.6 \pm 5.6 ^{+0.9}_{-3.3}$} \\ %\hline
& $\bm{\Gamma_{0}}$ & \bf{MeV} & \multicolumn{2}{c}{$161.6 \pm 21.1 ^{+4.4}_{-3.8}$} \\ %\hline
& $\bm{\sigma}$ & \bf{nb} & $38.7 \pm 3.5 ^{+3.9}_{-4.8}$ & $5.4 \pm 0.7 ^{+0.5}_{-0.6}$ \\ \hline
\multirow{3}{*}{$\bm{f_{2}(1270)}$} & $\bm{|A|}$ & $\bm{\left(\text{\bf{nb/GeV}}\right)^{\frac{1}{2}}}$ & $4.0 \pm 0.4 ^{+0.3}_{-0.3}$ & $7.9 \pm 0.3 ^{+0.4}_{-0.5}$ \\ %\hline
& $\bm{\phi}$ & \bf{rad} & $-1.83 \pm 0.10 ^{+0.03}_{-0.01}$ & $-0.86 \pm 0.04 ^{+0.03}_{-0.03}$ \\ %\hline
& $\bm{\sigma}$ & \bf{nb} & $4.4 \pm 0.9 ^{+0.6}_{-0.6}$ & $16.9 \pm 1.3 ^{+1.9}_{-2.1}$ \\ \hline
\multirow{5}{*}{$\bm{f_{0}(1500)}$} & $\bm{|A|}$ & $\bm{\left(\text{\bf{nb/GeV}}\right)^{\frac{1}{2}}}$ & $4.2 \pm 0.3 ^{+0.3}_{-0.3}$ & $0.9 \pm 0.3 ^{+0.1}_{-0.1}$ \\ %\hline
& $\bm{\phi}$ & \bf{rad} & $0.31 \pm 0.11 ^{+0.09}_{-0.06}$ & $1.45 \pm 0.31 ^{+0.05}_{-0.03}$ \\ %\hline
& $\bm{M}$ & \bf{MeV} & \multicolumn{2}{c}{$1471.7 \pm 6.3 ^{+2.1}_{-1.3}$} \\ %\hline
& $\bm{\Gamma_{0}}$ & \bf{MeV} & \multicolumn{2}{c}{$83.8 \pm 11.0 ^{+2.2}_{-3.0}$} \\ %\hline
& $\bm{\sigma}$ & \bf{nb} & $2.3 \pm 0.4 ^{+0.3}_{-0.3}$ & $0.1 \pm 0.1 ^{+0.0}_{-0.0}$ \\ \hline
\end{tabular}
\caption{Results of the fit described in the text in two ranges of azimuthal angle difference $\Delta\varphi$ between forward scattered protons. Statistical and systematic uncertainties are provided for each parameter.}\label{tab:fitRes}\vspace{-5pt} %temporary
\end{table}
}
%
%

Since the masses and widths of $f_0(980)$ and $f_0(1500)$ are free parameters it is possible to confront their fitted values with the PDG data~\cite{pdg}. In the case of $f_0(980)$ the mass and width is found to be respectively $M_{f_0(980)}=956 \pm 6 (\text{stat.}) ^{+1}_{-3} (\text{syst.})$~MeV and $\Gamma_{0,f_0(980)} = 162 \pm 21 (\text{stat.}) \pm 4 (\text{syst.})$~MeV. Such numbers differ from the PDG estimates of mass ($990 \pm 20$~MeV) and width (from 10~MeV to 100~MeV), nonetheless PDG emphasizes strong dependence of the resonance parameters on the model of amplitude. Some measurements listed in Ref.~\cite{pdg} are in very good agreement with obtained numbers. In addition to this, mass and width of $f_0(980)$ resulting from the fit with the Breit-Wigner form of amplitude gives result $M_{f_0(980)}=972 \pm 1 (\text{stat.}) \pm 1 (\text{syst.})$~MeV and $\Gamma_{0,f_0(980)} = 65 \pm 3 (\text{stat.}) \pm 1 (\text{syst.})$~MeV, albeit with notably worse $\chi^{2}$/ndf of 361/200. These values are in excellent agreement with PDG estimates and $f_0(980)$ parameters from other measurements assuming Breit-Wigner resonance shape~\cite{pdg}. From this we conclude that the structures in the mass spectrum around 1~GeV are indisputably due to interference of $f_0(980)$ with the remaining states, dominantly with the $\pi^{+}\pi^{-}$ continuum.
%
%

For the $f_0(1500)$ we obtain from the fit $M_{f_0(1500)}=1472 \pm 6 (\text{stat.}) \pm 2 (\text{syst.})$~MeV and $\Gamma_{0,f_0(1500)} = 84 \pm 11 (\text{stat.}) \pm 3 (\text{syst.})$~MeV. These numbers also deviate from the PDG averages: $1505\pm 6$~MeV for the mass and $109\pm 7$~MeV for the width. However, numerous measurements on $f_{0}(1500)$ referenced in PDG (and not used for averages calculation) report mass below 1500~MeV and width below 100~MeV, which are consistent with our result. Removing $f_0(1500)$ from the fit doubles the $\chi^2$ to value 590, a 300 standard deviations effect. From above we infer that the shape of $d\sigma/dm(\pi^{+}\pi^{-})$ around 1.4-1.6~GeV - the high-mass part of the $f_{2}(1270)$ region, is determined by the presence of $f_0(1500)$ interfering mainly with $\pi^{+}\pi^{-}$ continuum and also with $f_{2}(1270)$.
%

\begin{figure}%[t]
\centering
\includegraphics[width=\textwidth,page=1]{graphics/physicsResults/InvMassFit/F0980_BREITWIGNER/RatioAndInterference_PiPiInvMass_Fit.pdf}
%
\caption{Extrapolated $d\sigma/dm(\pi^{+}\pi^{-})$ with the fit assuming Breit-Wigner amplitude for $f_{0}(980)$.}
\label{invMassFit_F0980_BREITWIGNER}
\end{figure}

\begin{figure}%[t]
\centering
\includegraphics[width=\textwidth,page=1]{graphics/physicsResults/InvMassFit/NO_F01500/RatioAndInterference_PiPiInvMass_Fit.pdf}
%
\caption{Extrapolated $d\sigma/dm(\pi^{+}\pi^{-})$ with the fit ignoring $f_{0}(1500)$ component.}
\label{invMassFit_NO_F01500}
\end{figure}

\begin{figure}%[t]
\centering
\includegraphics[width=\textwidth,page=1]{graphics/physicsResults/InvMassFit/EXTRA_RESONANCE/RatioAndInterference_PiPiInvMass_Fit.pdf}
%
\caption{Extrapolated $d\sigma/dm(\pi^{+}\pi^{-})$ with the fit accounting for an additional $f_{0}(X)$ component.}
\label{invMassFit_EXTRA_RESONANCE}
\end{figure}


\begin{figure}%[t]
\centering
\includegraphics[width=\textwidth,page=1]{graphics/physicsResults/InvMassFit/WITH_RHO/RatioAndInterference_PiPiInvMass_Fit.pdf}
%
\caption{Extrapolated $d\sigma/dm(\pi^{+}\pi^{-})$ with the fit accounting for $\rho_{0}(770)$ production.}
\label{invMassFit_WITH_RHO}
\end{figure}

%

We have tested possibility of existence of an additional resonance produced in CEP between 1.2 and 1.5 GeV. With component $A_{f_0} \times \mathcal{R}_{\textrm{BW}}\left(m;M_{f_0},\Gamma_{f_0}\right)$ added to the model from Eq.~\eqref{eq:amplitude} the best fit is achieved for $M_{f_0}=1326 \pm 9 (\text{stat.})$~MeV and $\Gamma_{0,f_0} = 22 \pm 11 (\text{stat.})$~MeV. In that case the $\chi^{2}$/ndf is improved and equals 270/194 - the dip in $d\sigma/dm(\pi^{+}\pi^{-})$ at $\Delta\varphi<45^{\circ}$ is much better described compared to the nominal fit. Other parameters in the fit change slightly and remain compatible with their original values. Noteworthy is widening of $f_{0}(1500)$ to $\Gamma_{0,f_0(1500)} = 98 \pm 15 (\text{stat.})$~MeV, which gets consistent with the PDG average. The fitted content of additional resonance $f_0$ at $\Delta\varphi<45^{\circ}$ is several times lower than extracted yield of $f_{0}(1500)$, while for $\Delta\varphi>135^{\circ}$ it is consistent with 0. The value of mass agrees with that of $f_{0}(1370)$ resonance, however the obtained width is much lower than PDG estimates of about 200-500~MeV. Given that there are no known resonances of the observed properties we abstain from making definite statements about existence of an additional resonance in the measured exclusive $\pi^{+}\pi^{-}$.
%

From fitted parameters of resonances it is finally possible to calculate integrated cross-section $\sigma$ on their production and decay in the $\pi^{+}\pi^{-}$ channel within earlier defined $|y|$ and $-t$ limits. It is obtained by integrating squared total amplitude on resonance production over entire invariant mass domain. Calculated values of $\sigma$ are provided for each resonance in Tab~\ref{tab:fitRes}. We observe significant dependence of the resonance production cross-section on the azimuthal separation of the forward scattered protons. Two scalar mesons $f_{0}(980)$ and $f_{0}(1500)$ are dominantly produced at $\Delta\varphi<45^{\circ}$, whereas tensor meson $f_{2}(1270)$ - at $\Delta\varphi>135^{\circ}$. This $\Delta\varphi$ dependence is consistent with the observation made by WA102 Collaboration~\cite{WA102}.




% \[A_{\textrm{S-wave}}(m)~~~=~~~A_{\textrm{cont}}\times\sqrt{f_{\textrm{cont}}(m)}~~+~~~~~~~~~~~~~~~~~~~~~~~~~~~~~~~~~~\]\vspace{-15pt}
% \[~~~~~+~~A_{\textrm{f}_0(980)}\times \textrm{BW}\left(m;M_{f_0(980)},\Gamma_{f_0(980)}\right)\]
% \[~~~~~~~~+~~A_{\textrm{f}_0(1500)}\times \textrm{BW}\left(m;M_{f_0(1500)},\Gamma_{f_0(1500)}\right)\]\vspace{-10pt}
% \[A_{\textrm{D-wave}}(m)~~~=~~~A_{\textrm{f}_2(1270)}\times \textrm{BW}\left(m;M_{f_2(1270)},\Gamma_{f_2(1270)}\right)~~~~~~~~~~~~\]\vspace*{-5pt}
% where\vspace{5pt}
% \[A_{\textrm{cont}},A_{\textrm{f}_2(1270)}\in\mathbb{R},~~~A_{\textrm{f}_0(980)},A_{\textrm{f}_0(1500)}\in\mathbb{C}~~~~\rightarrow~~~~A=|A|e^{i\phi}.\]
% 
% Continuum model:
%  \[f_{\textrm{cont}}(m) = (m-2m_{\pi})^{B}\times e^{-C\cdot m}\]
%  Breit-Wigner form (relativistic Breit-Wigner with mass-dependent width):\vspace{-5pt}
% \[\hspace*{-5pt}\textrm{BW}(m;M,\Gamma_{0}) = \frac{M\sqrt{\Gamma_{0}}\sqrt{\Gamma(m)}}{M^{2}-m^{2}-i M\Gamma(m)},~~\Gamma(m) = \Gamma_{0}\frac{q}{m}\frac{M}{q_{0}}\left(\frac{B_{J}((qR)^{2})}{B_{J}((q_{0}R)^{2})}\right)^{2},\]\vspace*{-7pt}
% \[q(m) = \frac{1}{2}\sqrt{m^{2}-4m_{\pi}^{2}},~~~~~~q_{0} = q(M),~~~~~~R=1~\textrm{fm}\]\vspace*{-15pt}
% \begin{itemize}
%  \item masses and widths of $f_{2}(1270)$ are well known therefore they were fixed in the fit (used PDG values)
%  \item fit was done simultaneously to $d\sigma/dm$ in $\Delta\phi<45^{\circ}$ and $\Delta\phi>135^{\circ}$ mass bins
%  \item phases and amplitudes were independent is two $\Delta\phi$ bins, while masses and widths were forced to be the same
%  \item no residual background was accounted since we already subtracted it (background determined using data-driven method, using missing $p_{T}$)
%  \item data were corrected using assumption of the $\pi\pi$ pair being S-wave
% \end{itemize}







% %% ===== DODATKI ===== 
\begin{appendices}
%%===========================================================%%
%%                                                           %%
%%                 WORKING POINT FOR CUTS                    %%
%%                                                           %%
%%===========================================================%%

\chapter{Working points optimization for cuts~\ref{enum:CutBbcLarge}, \ref{enum:CutTofClusters} and \ref{enum:CutMissingPt}}\label{appendix:workingPoint}

The described study has been done at an early stage of analysis with some of cuts and the fiducial region defined differently from that finally established, therefore it has not been contained in the main part of this note. However, we consider it helpful to justify the cut thresholds in three significant cuts given in the title of this appendix. For aforementioned reason final numbers (for nominal fiducial region and nominal cuts) slightly differ from these presented in Fig.~\ref{fig:workingPoint}, but the general picture remains unchanged.

We define significance, efficiency, and purity of the three cuts: \ref{enum:CutBbcLarge}, \ref{enum:CutTofClusters} and \ref{enum:CutMissingPt}, according to equations shown below.\\[-15pt]%
\begin{tabulary}{\textwidth}{LLL}
\begin{equation}\label{eq:significance}\hspace*{-10pt}
	\text{Significance} = \frac{N_{\text{signal}}^{\text{cut}}}{\sqrt{N_{\text{signal}}^{\text{cut}} + N_{\text{bkgd}}^{\text{cut}}}},
\end{equation}~~~~~~~~~~~~~~~~~&
\begin{equation}\label{eq:efficiency}\hspace*{-10pt}
	\text{Efficiency} = \frac{N_{\text{signal}}^{\text{cut}}}{N_{\text{signal}}^{\text{no~cut}}},
\end{equation}~~~~~~~~~~~~~~&
\begin{equation}\label{eq:purity}\hspace*{-9pt}
	\text{Purity} = \frac{N_{\text{signal}}^{\text{cut}}}{N_{\text{signal}}^{\text{cut}}+N_{\text{bkgd}}^{\text{cut}}},
\end{equation}~~~~~~~~~~~~~~~
\end{tabulary}%
%--------------------------
\begin{figure}[b!]
\centering
\parbox{0.4725\textwidth}{
  \centering
  \begin{subfigure}[b]{\linewidth}
                \subcaptionbox{\label{fig:SignificanceVsEff}}{\includegraphics[width=\linewidth]{graphics/eventSelection/SignificanceVsEfficiency_pTmiss.pdf}}
  \end{subfigure}\\
  \begin{subfigure}[b]{\linewidth}\addtocounter{subfigure}{1}
                \subcaptionbox{\label{fig:EffVsBkgdFrac}}{\includegraphics[width=\linewidth]{graphics/eventSelection/ROC_pTmiss.pdf}}
  \end{subfigure}
}%
\quad\quad%
\parbox{0.4725\textwidth}{
  \centering
  \begin{subfigure}[b]{\linewidth}\addtocounter{subfigure}{-2}
                \subcaptionbox{\label{fig:SignificanceVsBkgdFrac}}{\includegraphics[width=\linewidth]{graphics/eventSelection/BkgdFractionVsEfficiency_pTmiss.pdf}}
  \end{subfigure}\\
  \begin{minipage}[t][1.042\linewidth][t]{\linewidth}\vspace{10pt}
    \caption[Relation between $\pi^{+}\pi^{-}$ significance, efficiency and purity vs. thresholds in cuts~\ref{enum:CutBbcLarge}, \ref{enum:CutTofClusters} and \ref{enum:CutMissingPt}]{Relation between $\pi^{+}\pi^{-}$ signal significance and efficiency (\subref{fig:SignificanceVsEff}), significance and purity (\subref{fig:SignificanceVsBkgdFrac}), and efficiency and purity (\subref{fig:EffVsBkgdFrac}) as a function of cut thresholds (the same for all channels) in BBC-large veto (\ref{enum:CutBbcLarge}), TOF cluster limit (\ref{enum:CutTofClusters}) and exclusivity cut (\ref{enum:CutMissingPt}). Lines show forementioned relations with changing $p_{T}^\text{miss}$ cut whose some specific values are indicated with different markers. Color denotes ADC threshold in BBC-large veto (black, red or green). Style of line (solid or dashed) denotes $N^{\text{TOF}}_{\text{clstrs}}$ limit. Working point considered optimal is marked with opened blue circle.}\label{fig:workingPoint}
  \end{minipage}
}%

\end{figure}\\
%--------------------------- 
In these equations $N_{\text{signal}}^{\text{cut}}$ is number of signal events in finally selected CEP $\pi^{+}\pi^{-}$ events, $N_{\text{bkgd}}^{\text{cut}}$ is number of non-exclusive background events in selected sample, and $N_{\text{signal}}^{\text{no~cut}}$ is number of signal events in sample after all cuts except the three studied cuts. These numbers were estimated using method described in Sec.~\ref{sec:nonExclBkgdDetermination}.

Relations between defined quantities are shown in Fig.~\ref{fig:workingPoint}. The first important observation was that allowing 3 TOF clusters instead of 2 (at fixed cuts~\ref{enum:CutBbcLarge} and~\ref{enum:CutMissingPt}) gives increase to selection efficiency by about 0.2, with only 0.01-0.02 decrease of the purity. We therefore decided to use condition $N^{\text{TOF}}_{\text{clstrs}}\leq 3$ in cut~\ref{enum:CutTofClusters}. 

In the next step the value of maximum $p_{T}^{\text{miss}}$ was found. From Fig~\ref{fig:workingPoint} one can read that with $N^{\text{TOF}}_{\text{clstrs}}\leq 3$ (dashed lines) the best balance between efficiency and purity is found for $p_{T}^{\text{miss}}$ cut value ranging between 60~MeV (rectangle) and 80~MeV (cross). We considered optimal threshold value of total transverse momentum $p_{T}^{\text{miss}}$ at 75~MeV which corresponds to 2.5$\sigma$, as elaborated in section devoted to exclusivity cut (\ref{enum:CutMissingPt}, Sec.~\ref{sec:C9}).

With fixed maximum number of TOF clusters and maximum $p_{T}^{\text{miss}}$ the maximum ADC in BBC-large was established. Similarly to previous paragraph, the best balance between efficiency and purity was found for ADC threshold of 40.

Certainly, not only factors considered here should be studied to find optimum working point, also e.g. size of systematic uncertainties for each cut value should be checked. Nonetheless, systematics related to these cuts are minor comparing to leading systematic uncertainties, therefore it was safe to ommit it.

%%===========================================================%%
%%                                                           %%
%%                    TOF EFFICIENCY APPENDIX                %%
%%                                                           %%
%%===========================================================%%

\chapter{BBC response}\label{appendix:bbc}

%---------------------------
\begin{figure}[hb]
\caption[Distribution of ADC vs. TAC counts (2D) and ADC conunts (1D) per BBC-small channel in abort gaps and colliding bunches.]{Two-dimensional distribution of ADC vs. TAC counts per BBC-small channel in abort gaps (left) and colliding bunches (middle), and one-dimensional projection on $x$-axis (ADC) for selected ranges of TAC for colliding bunches and abort gaps (right). Each row represents single channel (small BBC tile). Red lines and arrows indicate thresholds for a signal in given channel.}\label{fig:bbcSmallAdcVsTac}
\centering
\parbox{0.327\textwidth}{
  \centering
  \includegraphics[width=\linewidth,page=1]{graphics/eventSelection/bbc/Bbc_ADCvsTAC_abortGaps.pdf}\\
  \includegraphics[width=\linewidth,page=2]{graphics/eventSelection/bbc/Bbc_ADCvsTAC_abortGaps.pdf}\\
  \includegraphics[width=\linewidth,page=3]{graphics/eventSelection/bbc/Bbc_ADCvsTAC_abortGaps.pdf}\\
  \includegraphics[width=\linewidth,page=4]{graphics/eventSelection/bbc/Bbc_ADCvsTAC_abortGaps.pdf}
}~
\parbox{0.327\textwidth}{
  \centering
  \includegraphics[width=\linewidth,page=1]{graphics/eventSelection/bbc/Bbc_ADCvsTAC_collidingBunches.pdf}\\
  \includegraphics[width=\linewidth,page=2]{graphics/eventSelection/bbc/Bbc_ADCvsTAC_collidingBunches.pdf}\\
  \includegraphics[width=\linewidth,page=3]{graphics/eventSelection/bbc/Bbc_ADCvsTAC_collidingBunches.pdf}\\
  \includegraphics[width=\linewidth,page=4]{graphics/eventSelection/bbc/Bbc_ADCvsTAC_collidingBunches.pdf}
}%
\parbox{0.327\textwidth}{
  \centering
  \includegraphics[width=\linewidth,page=1]{graphics/eventSelection/bbc/Bbc_ADC.pdf}\\
  \includegraphics[width=\linewidth,page=2]{graphics/eventSelection/bbc/Bbc_ADC.pdf}\\
  \includegraphics[width=\linewidth,page=3]{graphics/eventSelection/bbc/Bbc_ADC.pdf}\\
  \includegraphics[width=\linewidth,page=4]{graphics/eventSelection/bbc/Bbc_ADC.pdf}
}%
\end{figure}
\begin{figure}[hb]\ContinuedFloat
\centering
\parbox{0.327\textwidth}{
  \centering
  \includegraphics[width=\linewidth,page=5]{graphics/eventSelection/bbc/Bbc_ADCvsTAC_abortGaps.pdf}\\
  \includegraphics[width=\linewidth,page=6]{graphics/eventSelection/bbc/Bbc_ADCvsTAC_abortGaps.pdf}\\
  \includegraphics[width=\linewidth,page=7]{graphics/eventSelection/bbc/Bbc_ADCvsTAC_abortGaps.pdf}\\
  \includegraphics[width=\linewidth,page=8]{graphics/eventSelection/bbc/Bbc_ADCvsTAC_abortGaps.pdf}\\
  \includegraphics[width=\linewidth,page=9]{graphics/eventSelection/bbc/Bbc_ADCvsTAC_abortGaps.pdf}\\
  \includegraphics[width=\linewidth,page=10]{graphics/eventSelection/bbc/Bbc_ADCvsTAC_abortGaps.pdf}
}~
\parbox{0.327\textwidth}{
  \centering
  \includegraphics[width=\linewidth,page=5]{graphics/eventSelection/bbc/Bbc_ADCvsTAC_collidingBunches.pdf}\\
  \includegraphics[width=\linewidth,page=6]{graphics/eventSelection/bbc/Bbc_ADCvsTAC_collidingBunches.pdf}\\
  \includegraphics[width=\linewidth,page=7]{graphics/eventSelection/bbc/Bbc_ADCvsTAC_collidingBunches.pdf}\\
  \includegraphics[width=\linewidth,page=8]{graphics/eventSelection/bbc/Bbc_ADCvsTAC_collidingBunches.pdf}\\
  \includegraphics[width=\linewidth,page=9]{graphics/eventSelection/bbc/Bbc_ADCvsTAC_collidingBunches.pdf}\\
  \includegraphics[width=\linewidth,page=10]{graphics/eventSelection/bbc/Bbc_ADCvsTAC_collidingBunches.pdf}
}%
\parbox{0.327\textwidth}{
  \centering
  \includegraphics[width=\linewidth,page=5]{graphics/eventSelection/bbc/Bbc_ADC.pdf}\\
  \includegraphics[width=\linewidth,page=6]{graphics/eventSelection/bbc/Bbc_ADC.pdf}\\
  \includegraphics[width=\linewidth,page=7]{graphics/eventSelection/bbc/Bbc_ADC.pdf}\\
  \includegraphics[width=\linewidth,page=8]{graphics/eventSelection/bbc/Bbc_ADC.pdf}\\
  \includegraphics[width=\linewidth,page=9]{graphics/eventSelection/bbc/Bbc_ADC.pdf}\\
  \includegraphics[width=\linewidth,page=10]{graphics/eventSelection/bbc/Bbc_ADC.pdf}
}%
\end{figure}
\begin{figure}[hb]\ContinuedFloat
\centering
\parbox{0.327\textwidth}{
  \centering
  \includegraphics[width=\linewidth,page=11]{graphics/eventSelection/bbc/Bbc_ADCvsTAC_abortGaps.pdf}\\
  \includegraphics[width=\linewidth,page=12]{graphics/eventSelection/bbc/Bbc_ADCvsTAC_abortGaps.pdf}\\
  \includegraphics[width=\linewidth,page=13]{graphics/eventSelection/bbc/Bbc_ADCvsTAC_abortGaps.pdf}\\
  \includegraphics[width=\linewidth,page=14]{graphics/eventSelection/bbc/Bbc_ADCvsTAC_abortGaps.pdf}\\
  \includegraphics[width=\linewidth,page=15]{graphics/eventSelection/bbc/Bbc_ADCvsTAC_abortGaps.pdf}\\
  \includegraphics[width=\linewidth,page=16]{graphics/eventSelection/bbc/Bbc_ADCvsTAC_abortGaps.pdf}
}~
\parbox{0.327\textwidth}{
  \centering
  \includegraphics[width=\linewidth,page=11]{graphics/eventSelection/bbc/Bbc_ADCvsTAC_collidingBunches.pdf}\\
  \includegraphics[width=\linewidth,page=12]{graphics/eventSelection/bbc/Bbc_ADCvsTAC_collidingBunches.pdf}\\
  \includegraphics[width=\linewidth,page=13]{graphics/eventSelection/bbc/Bbc_ADCvsTAC_collidingBunches.pdf}\\
  \includegraphics[width=\linewidth,page=14]{graphics/eventSelection/bbc/Bbc_ADCvsTAC_collidingBunches.pdf}\\
  \includegraphics[width=\linewidth,page=15]{graphics/eventSelection/bbc/Bbc_ADCvsTAC_collidingBunches.pdf}\\
  \includegraphics[width=\linewidth,page=16]{graphics/eventSelection/bbc/Bbc_ADCvsTAC_collidingBunches.pdf}
}%
\parbox{0.327\textwidth}{
  \centering
  \includegraphics[width=\linewidth,page=11]{graphics/eventSelection/bbc/Bbc_ADC.pdf}\\
  \includegraphics[width=\linewidth,page=12]{graphics/eventSelection/bbc/Bbc_ADC.pdf}\\
  \includegraphics[width=\linewidth,page=13]{graphics/eventSelection/bbc/Bbc_ADC.pdf}\\
  \includegraphics[width=\linewidth,page=14]{graphics/eventSelection/bbc/Bbc_ADC.pdf}\\
  \includegraphics[width=\linewidth,page=15]{graphics/eventSelection/bbc/Bbc_ADC.pdf}\\
  \includegraphics[width=\linewidth,page=16]{graphics/eventSelection/bbc/Bbc_ADC.pdf}
}%
\end{figure}
\begin{figure}[hb]\ContinuedFloat
\centering
\parbox{0.327\textwidth}{
  \centering
  \includegraphics[width=\linewidth,page=25]{graphics/eventSelection/bbc/Bbc_ADCvsTAC_abortGaps.pdf}\\
  \includegraphics[width=\linewidth,page=26]{graphics/eventSelection/bbc/Bbc_ADCvsTAC_abortGaps.pdf}\\
  \includegraphics[width=\linewidth,page=27]{graphics/eventSelection/bbc/Bbc_ADCvsTAC_abortGaps.pdf}\\
  \includegraphics[width=\linewidth,page=28]{graphics/eventSelection/bbc/Bbc_ADCvsTAC_abortGaps.pdf}\\
  \includegraphics[width=\linewidth,page=29]{graphics/eventSelection/bbc/Bbc_ADCvsTAC_abortGaps.pdf}\\
  \includegraphics[width=\linewidth,page=30]{graphics/eventSelection/bbc/Bbc_ADCvsTAC_abortGaps.pdf}
}~
\parbox{0.327\textwidth}{
  \centering
  \includegraphics[width=\linewidth,page=25]{graphics/eventSelection/bbc/Bbc_ADCvsTAC_collidingBunches.pdf}\\
  \includegraphics[width=\linewidth,page=26]{graphics/eventSelection/bbc/Bbc_ADCvsTAC_collidingBunches.pdf}\\
  \includegraphics[width=\linewidth,page=27]{graphics/eventSelection/bbc/Bbc_ADCvsTAC_collidingBunches.pdf}\\
  \includegraphics[width=\linewidth,page=28]{graphics/eventSelection/bbc/Bbc_ADCvsTAC_collidingBunches.pdf}\\
  \includegraphics[width=\linewidth,page=29]{graphics/eventSelection/bbc/Bbc_ADCvsTAC_collidingBunches.pdf}\\
  \includegraphics[width=\linewidth,page=30]{graphics/eventSelection/bbc/Bbc_ADCvsTAC_collidingBunches.pdf}
}%
\parbox{0.327\textwidth}{
  \centering
  \includegraphics[width=\linewidth,page=25]{graphics/eventSelection/bbc/Bbc_ADC.pdf}\\
  \includegraphics[width=\linewidth,page=26]{graphics/eventSelection/bbc/Bbc_ADC.pdf}\\
  \includegraphics[width=\linewidth,page=27]{graphics/eventSelection/bbc/Bbc_ADC.pdf}\\
  \includegraphics[width=\linewidth,page=28]{graphics/eventSelection/bbc/Bbc_ADC.pdf}\\
  \includegraphics[width=\linewidth,page=29]{graphics/eventSelection/bbc/Bbc_ADC.pdf}\\
  \includegraphics[width=\linewidth,page=30]{graphics/eventSelection/bbc/Bbc_ADC.pdf}
}%
\end{figure}
\begin{figure}[hb]\ContinuedFloat
\centering
\parbox{0.327\textwidth}{
  \centering
  \includegraphics[width=\linewidth,page=31]{graphics/eventSelection/bbc/Bbc_ADCvsTAC_abortGaps.pdf}\\
  \includegraphics[width=\linewidth,page=32]{graphics/eventSelection/bbc/Bbc_ADCvsTAC_abortGaps.pdf}\\
  \includegraphics[width=\linewidth,page=33]{graphics/eventSelection/bbc/Bbc_ADCvsTAC_abortGaps.pdf}\\
  \includegraphics[width=\linewidth,page=34]{graphics/eventSelection/bbc/Bbc_ADCvsTAC_abortGaps.pdf}\\
  \includegraphics[width=\linewidth,page=35]{graphics/eventSelection/bbc/Bbc_ADCvsTAC_abortGaps.pdf}\\
  \includegraphics[width=\linewidth,page=36]{graphics/eventSelection/bbc/Bbc_ADCvsTAC_abortGaps.pdf}
}~
\parbox{0.327\textwidth}{
  \centering
  \includegraphics[width=\linewidth,page=31]{graphics/eventSelection/bbc/Bbc_ADCvsTAC_collidingBunches.pdf}\\
  \includegraphics[width=\linewidth,page=32]{graphics/eventSelection/bbc/Bbc_ADCvsTAC_collidingBunches.pdf}\\
  \includegraphics[width=\linewidth,page=33]{graphics/eventSelection/bbc/Bbc_ADCvsTAC_collidingBunches.pdf}\\
  \includegraphics[width=\linewidth,page=34]{graphics/eventSelection/bbc/Bbc_ADCvsTAC_collidingBunches.pdf}\\
  \includegraphics[width=\linewidth,page=35]{graphics/eventSelection/bbc/Bbc_ADCvsTAC_collidingBunches.pdf}\\
  \includegraphics[width=\linewidth,page=36]{graphics/eventSelection/bbc/Bbc_ADCvsTAC_collidingBunches.pdf}
}%
\parbox{0.327\textwidth}{
  \centering
  \includegraphics[width=\linewidth,page=31]{graphics/eventSelection/bbc/Bbc_ADC.pdf}\\
  \includegraphics[width=\linewidth,page=32]{graphics/eventSelection/bbc/Bbc_ADC.pdf}\\
  \includegraphics[width=\linewidth,page=33]{graphics/eventSelection/bbc/Bbc_ADC.pdf}\\
  \includegraphics[width=\linewidth,page=34]{graphics/eventSelection/bbc/Bbc_ADC.pdf}\\
  \includegraphics[width=\linewidth,page=35]{graphics/eventSelection/bbc/Bbc_ADC.pdf}\\
  \includegraphics[width=\linewidth,page=36]{graphics/eventSelection/bbc/Bbc_ADC.pdf}
}%
\end{figure}
\begin{figure}[hb]\ContinuedFloat
\centering
\parbox{0.327\textwidth}{
  \centering
  \includegraphics[width=\linewidth,page=37]{graphics/eventSelection/bbc/Bbc_ADCvsTAC_abortGaps.pdf}\\
  \includegraphics[width=\linewidth,page=38]{graphics/eventSelection/bbc/Bbc_ADCvsTAC_abortGaps.pdf}\\
  \includegraphics[width=\linewidth,page=39]{graphics/eventSelection/bbc/Bbc_ADCvsTAC_abortGaps.pdf}\\
  \includegraphics[width=\linewidth,page=40]{graphics/eventSelection/bbc/Bbc_ADCvsTAC_abortGaps.pdf}
}~
\parbox{0.327\textwidth}{
  \centering
  \includegraphics[width=\linewidth,page=37]{graphics/eventSelection/bbc/Bbc_ADCvsTAC_collidingBunches.pdf}\\
  \includegraphics[width=\linewidth,page=38]{graphics/eventSelection/bbc/Bbc_ADCvsTAC_collidingBunches.pdf}\\
  \includegraphics[width=\linewidth,page=39]{graphics/eventSelection/bbc/Bbc_ADCvsTAC_collidingBunches.pdf}\\
  \includegraphics[width=\linewidth,page=40]{graphics/eventSelection/bbc/Bbc_ADCvsTAC_collidingBunches.pdf}
}%
\parbox{0.327\textwidth}{
  \centering
  \includegraphics[width=\linewidth,page=37]{graphics/eventSelection/bbc/Bbc_ADC.pdf}\\
  \includegraphics[width=\linewidth,page=38]{graphics/eventSelection/bbc/Bbc_ADC.pdf}\\
  \includegraphics[width=\linewidth,page=39]{graphics/eventSelection/bbc/Bbc_ADC.pdf}\\
  \includegraphics[width=\linewidth,page=40]{graphics/eventSelection/bbc/Bbc_ADC.pdf}
}%
\end{figure}







\begin{figure}[hb]
\caption[Distribution of ADC vs. TAC counts (2D) and ADC conunts (1D) per BBC-large channel in abort gaps and colliding bunches.]{Two-dimensional distribution of ADC vs. TAC counts per BBC-large channel in abort gaps (left) and colliding bunches (middle), and one-dimensional projection on $x$-axis (ADC) for selected ranges of TAC for colliding bunches and abort gaps (right). Each row represents single channel (large BBC tile). Red lines and arrows indicate thresholds for a signal in given channel.}\label{fig:bbcLargeAdcVsTac}
\centering
\parbox{0.327\textwidth}{
  \centering
  \includegraphics[width=\linewidth,page=17]{graphics/eventSelection/bbc/Bbc_ADCvsTAC_abortGaps.pdf}\\
  \includegraphics[width=\linewidth,page=18]{graphics/eventSelection/bbc/Bbc_ADCvsTAC_abortGaps.pdf}\\
  \includegraphics[width=\linewidth,page=19]{graphics/eventSelection/bbc/Bbc_ADCvsTAC_abortGaps.pdf}\\
  \includegraphics[width=\linewidth,page=20]{graphics/eventSelection/bbc/Bbc_ADCvsTAC_abortGaps.pdf}\\
  \includegraphics[width=\linewidth,page=21]{graphics/eventSelection/bbc/Bbc_ADCvsTAC_abortGaps.pdf}
}~
\parbox{0.327\textwidth}{
  \centering
  \includegraphics[width=\linewidth,page=17]{graphics/eventSelection/bbc/Bbc_ADCvsTAC_collidingBunches.pdf}\\
  \includegraphics[width=\linewidth,page=18]{graphics/eventSelection/bbc/Bbc_ADCvsTAC_collidingBunches.pdf}\\
  \includegraphics[width=\linewidth,page=19]{graphics/eventSelection/bbc/Bbc_ADCvsTAC_collidingBunches.pdf}\\
  \includegraphics[width=\linewidth,page=20]{graphics/eventSelection/bbc/Bbc_ADCvsTAC_collidingBunches.pdf}\\
  \includegraphics[width=\linewidth,page=21]{graphics/eventSelection/bbc/Bbc_ADCvsTAC_collidingBunches.pdf}
}~
\parbox{0.327\textwidth}{
  \centering
  \includegraphics[width=\linewidth,page=17]{graphics/eventSelection/bbc/Bbc_ADC.pdf}\\
  \includegraphics[width=\linewidth,page=18]{graphics/eventSelection/bbc/Bbc_ADC.pdf}\\
  \includegraphics[width=\linewidth,page=19]{graphics/eventSelection/bbc/Bbc_ADC.pdf}\\
  \includegraphics[width=\linewidth,page=20]{graphics/eventSelection/bbc/Bbc_ADC.pdf}\\
  \includegraphics[width=\linewidth,page=21]{graphics/eventSelection/bbc/Bbc_ADC.pdf}
}%
\end{figure}
\begin{figure}[hb]\ContinuedFloat
\centering
\parbox{0.327\textwidth}{
  \centering
  \includegraphics[width=\linewidth,page=22]{graphics/eventSelection/bbc/Bbc_ADCvsTAC_abortGaps.pdf}\\
  \includegraphics[width=\linewidth,page=23]{graphics/eventSelection/bbc/Bbc_ADCvsTAC_abortGaps.pdf}\\
  \includegraphics[width=\linewidth,page=24]{graphics/eventSelection/bbc/Bbc_ADCvsTAC_abortGaps.pdf}\\
  \includegraphics[width=\linewidth,page=41]{graphics/eventSelection/bbc/Bbc_ADCvsTAC_abortGaps.pdf}\\
  \includegraphics[width=\linewidth,page=42]{graphics/eventSelection/bbc/Bbc_ADCvsTAC_abortGaps.pdf}\\
  \includegraphics[width=\linewidth,page=43]{graphics/eventSelection/bbc/Bbc_ADCvsTAC_abortGaps.pdf}
}~
\parbox{0.327\textwidth}{
  \centering
  \includegraphics[width=\linewidth,page=22]{graphics/eventSelection/bbc/Bbc_ADCvsTAC_collidingBunches.pdf}\\
  \includegraphics[width=\linewidth,page=23]{graphics/eventSelection/bbc/Bbc_ADCvsTAC_collidingBunches.pdf}\\
  \includegraphics[width=\linewidth,page=24]{graphics/eventSelection/bbc/Bbc_ADCvsTAC_collidingBunches.pdf}\\
  \includegraphics[width=\linewidth,page=41]{graphics/eventSelection/bbc/Bbc_ADCvsTAC_collidingBunches.pdf}\\
  \includegraphics[width=\linewidth,page=42]{graphics/eventSelection/bbc/Bbc_ADCvsTAC_collidingBunches.pdf}\\
  \includegraphics[width=\linewidth,page=43]{graphics/eventSelection/bbc/Bbc_ADCvsTAC_collidingBunches.pdf}
}~
\parbox{0.327\textwidth}{
  \centering
  \includegraphics[width=\linewidth,page=22]{graphics/eventSelection/bbc/Bbc_ADC.pdf}\\
  \includegraphics[width=\linewidth,page=23]{graphics/eventSelection/bbc/Bbc_ADC.pdf}\\
  \includegraphics[width=\linewidth,page=24]{graphics/eventSelection/bbc/Bbc_ADC.pdf}\\
  \includegraphics[width=\linewidth,page=41]{graphics/eventSelection/bbc/Bbc_ADC.pdf}\\
  \includegraphics[width=\linewidth,page=42]{graphics/eventSelection/bbc/Bbc_ADC.pdf}\\
  \includegraphics[width=\linewidth,page=43]{graphics/eventSelection/bbc/Bbc_ADC.pdf}
}%
\end{figure}
\begin{figure}[hb]\ContinuedFloat
\centering
\parbox{0.327\textwidth}{
  \centering
  \includegraphics[width=\linewidth,page=44]{graphics/eventSelection/bbc/Bbc_ADCvsTAC_abortGaps.pdf}\\
  \includegraphics[width=\linewidth,page=45]{graphics/eventSelection/bbc/Bbc_ADCvsTAC_abortGaps.pdf}\\
  \includegraphics[width=\linewidth,page=46]{graphics/eventSelection/bbc/Bbc_ADCvsTAC_abortGaps.pdf}\\
  \includegraphics[width=\linewidth,page=47]{graphics/eventSelection/bbc/Bbc_ADCvsTAC_abortGaps.pdf}\\
  \includegraphics[width=\linewidth,page=48]{graphics/eventSelection/bbc/Bbc_ADCvsTAC_abortGaps.pdf}
}~
\parbox{0.327\textwidth}{
  \centering
  \includegraphics[width=\linewidth,page=44]{graphics/eventSelection/bbc/Bbc_ADCvsTAC_collidingBunches.pdf}\\
  \includegraphics[width=\linewidth,page=45]{graphics/eventSelection/bbc/Bbc_ADCvsTAC_collidingBunches.pdf}\\
  \includegraphics[width=\linewidth,page=46]{graphics/eventSelection/bbc/Bbc_ADCvsTAC_collidingBunches.pdf}\\
  \includegraphics[width=\linewidth,page=47]{graphics/eventSelection/bbc/Bbc_ADCvsTAC_collidingBunches.pdf}\\
  \includegraphics[width=\linewidth,page=48]{graphics/eventSelection/bbc/Bbc_ADCvsTAC_collidingBunches.pdf}
}~
\parbox{0.327\textwidth}{
  \centering
  \includegraphics[width=\linewidth,page=44]{graphics/eventSelection/bbc/Bbc_ADC.pdf}\\
  \includegraphics[width=\linewidth,page=45]{graphics/eventSelection/bbc/Bbc_ADC.pdf}\\
  \includegraphics[width=\linewidth,page=46]{graphics/eventSelection/bbc/Bbc_ADC.pdf}\\
  \includegraphics[width=\linewidth,page=47]{graphics/eventSelection/bbc/Bbc_ADC.pdf}\\
  \includegraphics[width=\linewidth,page=48]{graphics/eventSelection/bbc/Bbc_ADC.pdf}
}%
\end{figure}

%%===========================================================%%
%%                                                           %%
%%   FORMULATION OF TOTAL RP EFFICIENCY CORRECTION APPENDIX  %%
%%                                                           %%
%%===========================================================%%

\chapter{Reconstruction of \texorpdfstring{$\bf{m}^{2}_{\text{\small TOF}}$}{mTOF^2}}\label{appendix:totalRpEffFormulation}


\textbf{Definitions:}\\[7pt]
\begin{tabular}{ll}
$t_{0}$ &- time of the primary $pp$ interaction\\
$t_{1,2}$ &- time of detection of the hit in TOF by particle 1(2)\\
$L_{1,2}$ & - helical path of the particle 1(2) from the interaction vertex to the TOF cell with reconstructed hit,\\
$p_{1,2}$ & - magnitude of momentum of particle 1(2),\\
$m_{1,2}$ & - mass of particle 1(2),\\
\end{tabular}\vspace{10pt}


%---------------------------
\begin{figure}[ht!]
\centering%
\parbox{0.29\textwidth}{%
  \centering%
  \includegraphics[width=\linewidth]{graphics/eventSelection/TofScheme.pdf}\label{fig:tofScheme}
}%
\quad\quad%
\parbox{0.655\textwidth}{%
    \caption[Scheme of two central tracks with common vertex, hitting cells in TOF detector.]{Scheme of two central tracks of lengths $L_{1}$ and $L_{2}$, produced in common vertex in moment $t_{0}$, hitting cells in TOF detector in moments $t_{1}$ and $t_{2}$.}
}%

\end{figure}
%---------------------------


From the simple algebra below which describes relation between track lengths, momenta and times of hit detection one can derive formula for the squared mass of two particles, assuming that their masses are equal (particles are of the same type).

Below we assume $c=1$. We can write a set of two equations connecting the time that it takes for each particle to reach the TOF, starting from the interaction vertex:
\begin{equation}
 \left\{\begin{array}{l}%
 t_{1}-t_{0} = L_{1}\sqrt{1+\frac{m_{1}^{2}}{p_{1}^{2}}}, \\[3pt]
 t_{2}-t_{0} = L_{2}\sqrt{1+\frac{m_{2}^{2}}{p_{2}^{2}}}.
\end{array}\right.%
\end{equation}
By adding the two equations above we get
\begin{equation}\label{eq:m2Temp}
 \Delta t = t_{1}-t_{2} = L_{1}\sqrt{1+\frac{m_{1}^{2}}{p_{1}^{2}}} - L_{2}\sqrt{1+\frac{m_{2}^{2}}{p_{2}^{2}}}.
\end{equation}
In CEP of two opposite-sign particles always the same species of particles are produced, therefore
\begin{equation}m_{1}=m_{2}=m.\end{equation}
If we substitute $m_{1}$ and $m_{2}$ with $m$ in Eq.~\eqref{eq:m2Temp} and transform the equation to remove the square roots we get a quadratic equation of the form 
\begin{equation}\label{eq:m2Quad}\mathcal{A}\times \left(m^{2}_{\text{\tiny TOF}}\right)^{2} + \mathcal{B}\times m^{2}_{\text{\tiny TOF}} + \mathcal{C} = 0.\end{equation}
Parameters of the Eq.~\eqref{eq:m2Quad} are given below:
\begin{equation}
\mathcal{A}= -2\frac{L^2_1L^2_2}{p^2_1p^2_2}+\frac{L^4_1}{p^4_1}+\frac{L^4_2}{p^4_2},
\end{equation}
\begin{equation}
\mathcal{B}=-2L^2_1L^2_2\left({\frac{1}{p^2_1}} + {\frac{1}{p^2_2}}\right)+\frac{2L^4_1}{p_1^2}+\frac{2L^4_2}{p_2^2}-2\left(\Delta t\right)^2\left(\frac{L^2_1}{p_1^2}+\frac{L^2_2}{p_2^2}\right),
\end{equation}
\begin{equation}
\mathcal{C}=\left(\Delta t\right)^4-2\left(\Delta t\right)^2\left(L^2_1+L^2_2\right)+L^4_1+L^4_2-2L^2_1L^2_2,
\end{equation}
together with the final formula for a physical root of the quadratic equation which is used in the $m^{2}_{\text{\tiny TOF}}$ reconstruction:
\begin{equation}
 \label{eq:mSquared}
m^{2}_{\text{\tiny TOF}} = \frac{-\mathcal{B}+\sqrt{\mathcal{B}^2-4\mathcal{A}\mathcal{C}}}{2\mathcal{A}}.
\end{equation}

%%===========================================================%%
%%                                                           %%
%%                   PID EFFICIENCY APPENDIX                 %%
%%                                                           %%
%%===========================================================%%

\chapter{Particle identification efficiency}\label{appendix:pidEff}

\begin{figure}[ht!]\label{fig:pidEffVsP}
  \centering
  \begin{tabular}{@{}p{0.315\linewidth}@{\quad}p{0.315\linewidth}@{\quad}p{0.315\linewidth}@{}}
    \subfigimg[width=\linewidth,page=1]{~~~~~~~~~~~~~~~~~~~~~~~a)}{graphics/corrections/EffVsP.pdf} &
    \subfigimg[width=\linewidth,page=2]{~~~~~~~~~~~~~~~~~~~~~~~b)}{graphics/corrections/EffVsP.pdf} &
    \subfigimg[width=\linewidth,page=3]{~~~~~~~~~~~~~~~~~~~~~~~c)}{graphics/corrections/EffVsP.pdf} \\
    \subfigimg[width=\linewidth,page=4]{~~~~~~~~~~~~~~~~~~~~~~~d)}{graphics/corrections/EffVsP.pdf} &
    \subfigimg[width=\linewidth,page=5]{~~~~~~~~~~~~~~~~~~~~~~~e)}{graphics/corrections/EffVsP.pdf} &
    \subfigimg[width=\linewidth,page=6]{~~~~~~~~~~~~~~~~~~~~~~~f)}{graphics/corrections/EffVsP.pdf} \\
    \subfigimg[width=\linewidth,page=7]{~~~~~~~~~~~~~~~~~~~~~~~g)}{graphics/corrections/EffVsP.pdf} &
    \subfigimg[width=\linewidth,page=8]{~~~~~~~~~~~~~~~~~~~~~~~h)}{graphics/corrections/EffVsP.pdf} &
    \subfigimg[width=\linewidth,page=9]{~~~~~~~~~~~~~~~~~~~~~~~i)}{graphics/corrections/EffVsP.pdf} 
  \end{tabular}
  \caption[Pair identification efficiency and misidentification probability as a function of tracks' $p$.]{Pair identification efficiency (diagonal) and misidentification probability (off-diagonal) as a function of tracks' $p$ for $\pi^{+}\pi^{-}$, $K^{+}K^{-}$ and $p\bar{p}$ pairs. The results were obtained from the dedicated MC simulation described in Sec.~\ref{sec:pidEff}.}
\end{figure}

%%===========================================================%%
%%                                                           %%
%%   FORMULATION OF TOTAL RP EFFICIENCY CORRECTION APPENDIX  %%
%%                                                           %%
%%===========================================================%%

\chapter{Formulation of total RP efficiency}\label{appendix:totalRpEffFormulation}


\textbf{Definitions:}\\[7pt]
\begin{tabular}{ll}
$\RPE$ &- single good quality track (satisfying cuts \ref{enum:RpQualityCuts}-\ref{enum:RpLocalAngles}) on the east side,\\
$\RPW$ & - single good quality track (satisfying cuts \ref{enum:RpQualityCuts}-\ref{enum:RpLocalAngles}) on the west side,\\
$\TRE$ & - trigger signal in the RP branch with single good track on the east side,\\
$\TRNE$ & - trigger signal in the RP branch othar than branch with single good track on the east side,\\
$\TRW$ & - trigger signal in the RP branch with single good track on the west side,\\
$\TRNW$ & - trigger signal in the RP branch othar than branch with single good track on the west side,\\
$\V$ & - trigger veto on the simultaneous trigger signal in Up and Down RPs (ET\&IT),\\
$\Vpu$ & - trigger veto on ET\&IT ($\V$) due to pile-up interactions,\\
$\Vdm$ & - trigger veto on ET\&IT ($\V$) due to forward proton interaction with dead material.\\ 
\end{tabular}\vspace{10pt}


The total efficiency related to both east and west forward protons in CEP event has the following form:
\begin{equation}
\mbox{\LARGE$\varepsilon$}\left(\RPE\land\RPW\land\TRE\land\TRW\land~!\V\right) = \mbox{\LARGE$\varepsilon$}\left(\RPE\land\RPW\left|\TRE\land\TRW\land~!\V\right.\right)\times\mbox{\LARGE$\varepsilon$}\left(\TRE\land\TRW\land~!\V\right),
\end{equation}
where the r.h.s. part of the equation is factorized using the rules of conditional probability to two components describing reconstruction and selection efficiency (first) and trigger efficiency (second).


The reconstruction and selection efficiency part can be represented as a product of single-proton reconstruction and selection efficiencies described in Sec.~\ref{sec:rpAccAndEff} with an additional component that accounts for the correlation between east and west efficiencies. This correlation is defined in Eq.~\eqref{eq:rpEffCorrelation}:

%\begin{equation}
%\begin{split}
%\mbox{\LARGE$\varepsilon$}\left(\RPE\left|\TRE\land~!\TRNE\right.\right) \times %\mbox{\LARGE$\varepsilon$}\left(\RPW\left|\TRW\land~!\TRNW\right.\right) =\\= %\mbox{\LARGE$\varepsilon$}\left(\RPE\land\RPW\left|\TRE\land\TRW\land~!\V\right.\right) \times %\Big(1-\mbox{\LARGE$\varepsilon$}\left(!\RPE\land~!\RPW\Big|\TRE\land\TRW\land~!\V\Big.\right)\Big)
%\end{split}
%\end{equation}

\begin{equation}\label{eq:rpEffCorrelation}
 \rho_{\text{EW}} = 
 \frac{\mbox{\LARGE$\varepsilon$}_{\text{EW}} - \mbox{\LARGE$\varepsilon$}_{\text{E}} \times \mbox{\LARGE$\varepsilon$}_{\text{W}} }
 { \sqrt{ \mbox{\LARGE$\varepsilon$}_{\text{E}} \times (1-\mbox{\LARGE$\varepsilon$}_{\text{E}}) \times \mbox{\LARGE$\varepsilon$}_{\text{W}} \times (1-\mbox{\LARGE$\varepsilon$}_{\text{W}}) } },
\end{equation}
where \[\mbox{\LARGE$\varepsilon$}_{\text{E}} = \mbox{\LARGE$\varepsilon$}\left(\RPE\left|\TRE\land~!\TRNE\right.\right),~~~~~~~~~\mbox{\LARGE$\varepsilon$}_{\text{W}} = \mbox{\LARGE$\varepsilon$}\left(\RPW\left|\TRW\land~!\TRNW\right.\right),\] \[\mbox{\LARGE$\varepsilon$}_{\text{EW}} = \mbox{\LARGE$\varepsilon$}\left(\RPE\land\RPW\left|\TRE\land\TRW\land~!\V\right.\right).\]%
%
From that we get%
\begin{equation}
\mbox{\LARGE$\varepsilon$}\left(\RPE\land\RPW\left|\TRE\land\TRW\land~!\V\right.\right)=\mbox{\LARGE$\varepsilon$}_{\text{E}} \times \mbox{\LARGE$\varepsilon$}_{\text{W}} + \rho_{\text{EW}}\times \sqrt{ \mbox{\LARGE$\varepsilon$}_{\text{E}} \times (1-\mbox{\LARGE$\varepsilon$}_{\text{E}}) \times \mbox{\LARGE$\varepsilon$}_{\text{W}} \times (1-\mbox{\LARGE$\varepsilon$}_{\text{W}}) }.
\end{equation}

The correlation coefficient $\rho_{\text{EW}}$ governs information about simultaneous unsuccessful reconstruction/selection of RP track on the east and west side in the same event. This can be a result of e.g. a pile-up interaction, typically of elastic proton-proton scattering, producing additional tracks/showers simultaneously in east and west RPs and thus introducing simultaneous east and west RP inefficiency. Also, some corruption of the data stream might lead to unsuccessful reconstruction of the entire event. With the above correlation taken into account we are able to properly reconstruct the yield of true-level events and shape of distributions from the measured events, as shown in the closure test in Sec.~\ref{subsec:closureTestRp}.

In the component of RP efficiency related to the trigger we can use again the conditional probability and factorize it to part connected with the trigger veto (first) and the efficiency of detecting a signal of both forward protons by the trigger system (second):
\begin{equation}
\mbox{\LARGE$\varepsilon$}\left(\TRE\land\TRW\land~!\V\right) = \mbox{\LARGE$\varepsilon$}\left(!\V\left|\TRE\land\TRW\right.\right)\times\mbox{\LARGE$\varepsilon$}\left(\TRE\land\TRW\right).
\end{equation}

Efficiency of the triggering $\mbox{\LARGE$\varepsilon$}\left(\TRE\land\TRW\right)$ is much above 99\% (see the Ref.~\cite{supplementaryNote}). Efficiency of the (lack of) veto if forward protons are triggering in east and west RPs can be decomposed to efficiency of the veto induced by the pile-up interaction in the same bunch crossing ($\Vpu$) and efficiency of the veto induced by the interaction of the CEP protons with the material of the accelerator and detectors ($\Vdm$):
\begin{equation}\label{eq:trigerEffFact}
\begin{split}
\mbox{\LARGE$\varepsilon$}\left(!\V\left|\TRE\land\TRW\right.\right)=\Bigg|\V=\Vpu\vee\Vdm\Bigg|=\mbox{\LARGE$\varepsilon$}\left(!\Vpu\land~!\Vdm\left|\TRE\land\TRW\right.\right) = \\ 
= \mbox{\LARGE$\varepsilon$}\left(!\Vdm\left|!\Vpu\land\TRE\land\TRW\right.\right) \times \mbox{\LARGE$\varepsilon$}\left(!\Vpu\left|\TRE\land\TRW\right.\right).
\end{split}
\end{equation}

The first term of the last part of Eq.~\eqref{eq:trigerEffFact} described in Sec.~\ref{sec:rpTrigEff} can be safely factorized as the probability of veto induced by the primary CEP proton on the east side is totally independent from the similar probability on the west side:

\begin{equation}
\begin{split}
\mbox{\LARGE$\varepsilon$}\left(!\Vdm\left|!\Vpu\land\TRE\land\TRW\right.\right) = ~~~~~~~~~~~~~~~~~~~~~~~~~~~~~~~~~~~~~~~~~~~~~~~~~~~~~~~~~~~~~~~~~~~~~~~~~~~~~~~~~~\\ 
=\mbox{\LARGE$\varepsilon$}^{\text{E}}\left(!\Vdm\left|!\Vpu\land\TRE\land\TRW\right.\right) \times \mbox{\LARGE$\varepsilon$}^{\text{W}}\left(!\Vdm\left|!\Vpu\land\TRE\land\TRW\right.\right)
\end{split}
\end{equation}

The second term of the last part of Eq.~\eqref{eq:trigerEffFact} related with the pile-up is incorporated to overall efficiency of online and offline vetoes described in Sec.~\ref{sec:onlineAndOfflineVetoEff} - this is required by the correlation of vetoes, the possibility that vetoes in independent subdetectors take place in the same bunch crossing. This could happen if e.g. single diffraction event occurs on top of the CEP event, yielding a BBC signal and RP signal, both vetoing RP\_CPT2 trigger.




%Wydajnosc zwiazana z materialem martwym bede liczyl tak samo jak wydajnosc rekonstrukcji - w funkcji %$z_{\text{vtx}}$, $p_{x}$ oraz $p_{y}$, natomiast wydajnosc zwiazana z pile-up'em chyba trzeba %potraktowac jako jedna liczbe (osobna dla 4 kombinacji galezi) ktora policze osobno dla każdego runu a %nastepnie dopasuje funkcje okreslajaca zależnosc tej poprawki od chwilowego lumi (dokladnie tak jak to %robie z poprawka na weto na BBC, ZDC i slady w TPC/TOF).\\
%Trzeba tutaj uważac żeby nie poprawiac weta przez pile-up dwukrotnie, tzn. może byc jednoczesne weto %trygera w BBC-small i w RP spowodowane jakims przypadkiem dyfrakcyjnym (np. pojedyncza dyfrakcja, %dyfrakcyjna dysocjacja). Mysle, że najlepiej bedzie policzyc laczna poprawke na weto w RP, BBC itd. z %danych zerobias (wlaczyc RP do dotychczasowej poprawki na weto BBC + ...).\\

%%===========================================================%%
%%                                                           %%
%%                    TOF EFFICIENCY APPENDIX                %%
%%                                                           %%
%%===========================================================%%

\chapter{RP efficiency}\label{appendix:rpEff}

%---------------------------
\begin{figure}[hb]
\caption[RP eff]{Rp eff.}\label{fig:rpEff}
\centering
\parbox{0.495\textwidth}{
  \centering
  \includegraphics[width=\linewidth,page=3]{graphics/corrections/mcEffPxPy.pdf}\\
  \includegraphics[width=\linewidth,page=5]{graphics/corrections/mcEffPxPy.pdf}\\
  \includegraphics[width=\linewidth,page=7]{graphics/corrections/mcEffPxPy.pdf}
}~
\parbox{0.495\textwidth}{
  \centering
  \includegraphics[width=\linewidth,page=4]{graphics/corrections/mcEffPxPy.pdf}\\
  \includegraphics[width=\linewidth,page=6]{graphics/corrections/mcEffPxPy.pdf}\\
  \includegraphics[width=\linewidth,page=8]{graphics/corrections/mcEffPxPy.pdf}
}%
\end{figure}
\begin{figure}[hb]\ContinuedFloat
\centering
\parbox{0.495\textwidth}{
  \centering
  \includegraphics[width=\linewidth,page=9]{graphics/corrections/mcEffPxPy.pdf}\\
  \includegraphics[width=\linewidth,page=11]{graphics/corrections/mcEffPxPy.pdf}\\
  \includegraphics[width=\linewidth,page=13]{graphics/corrections/mcEffPxPy.pdf}
}~
\parbox{0.495\textwidth}{
  \centering
  \includegraphics[width=\linewidth,page=10]{graphics/corrections/mcEffPxPy.pdf}\\
  \includegraphics[width=\linewidth,page=12]{graphics/corrections/mcEffPxPy.pdf}\\
  \includegraphics[width=\linewidth,page=14]{graphics/corrections/mcEffPxPy.pdf}
}%
\end{figure}
\begin{figure}[hb]\ContinuedFloat
\centering
\parbox{0.495\textwidth}{
  \centering
  \includegraphics[width=\linewidth,page=15]{graphics/corrections/mcEffPxPy.pdf}\\
  \includegraphics[width=\linewidth,page=17]{graphics/corrections/mcEffPxPy.pdf}
}~
\parbox{0.495\textwidth}{
  \centering
  \includegraphics[width=\linewidth,page=16]{graphics/corrections/mcEffPxPy.pdf}\\
  \includegraphics[width=\linewidth,page=18]{graphics/corrections/mcEffPxPy.pdf}
}%
\end{figure}



%---------------------------
\begin{figure}[hb]
\caption[RP dead mat prob]{Rp dead mat. prob.}\label{fig:rpDeadMatProb}
\centering
\parbox{0.495\textwidth}{
  \centering
  \includegraphics[width=\linewidth,page=3]{graphics/corrections/mcDeadMatProbPxPy.pdf}\\
  \includegraphics[width=\linewidth,page=5]{graphics/corrections/mcDeadMatProbPxPy.pdf}\\
  \includegraphics[width=\linewidth,page=7]{graphics/corrections/mcDeadMatProbPxPy.pdf}
}~
\parbox{0.495\textwidth}{
  \centering
  \includegraphics[width=\linewidth,page=4]{graphics/corrections/mcDeadMatProbPxPy.pdf}\\
  \includegraphics[width=\linewidth,page=6]{graphics/corrections/mcDeadMatProbPxPy.pdf}\\
  \includegraphics[width=\linewidth,page=8]{graphics/corrections/mcDeadMatProbPxPy.pdf}
}%
\end{figure}
\begin{figure}[hb]\ContinuedFloat
\centering
\parbox{0.495\textwidth}{
  \centering
  \includegraphics[width=\linewidth,page=9]{graphics/corrections/mcDeadMatProbPxPy.pdf}\\
  \includegraphics[width=\linewidth,page=11]{graphics/corrections/mcDeadMatProbPxPy.pdf}\\
  \includegraphics[width=\linewidth,page=13]{graphics/corrections/mcDeadMatProbPxPy.pdf}
}~
\parbox{0.495\textwidth}{
  \centering
  \includegraphics[width=\linewidth,page=10]{graphics/corrections/mcDeadMatProbPxPy.pdf}\\
  \includegraphics[width=\linewidth,page=12]{graphics/corrections/mcDeadMatProbPxPy.pdf}\\
  \includegraphics[width=\linewidth,page=14]{graphics/corrections/mcDeadMatProbPxPy.pdf}
}%
\end{figure}
\begin{figure}[hb]\ContinuedFloat
\centering
\parbox{0.495\textwidth}{
  \centering
  \includegraphics[width=\linewidth,page=15]{graphics/corrections/mcDeadMatProbPxPy.pdf}\\
  \includegraphics[width=\linewidth,page=17]{graphics/corrections/mcDeadMatProbPxPy.pdf}
}~
\parbox{0.495\textwidth}{
  \centering
  \includegraphics[width=\linewidth,page=16]{graphics/corrections/mcDeadMatProbPxPy.pdf}\\
  \includegraphics[width=\linewidth,page=18]{graphics/corrections/mcDeadMatProbPxPy.pdf}
}%
\end{figure}


%%===========================================================%%
%%                                                           %%
%%                 INV MASS FITS SYSTEMATIS                  %%
%%                                                           %%
%%===========================================================%%

\chapter{Fits to extrapolated $d\sigma/dm(\pi^{+}\pi^{-})$}\label{appendix:invMassSystFits}

%--------------------------- 
\begin{figure}[hb]
\caption{Fits to $d\sigma/dm(\pi^{+}\pi^{-})$ modified by systematic and model variations.}\label{fig:invMassSystFits}
\centering
\parbox{0.495\textwidth}{
  \centering
  \includegraphics[width=\linewidth,page=1]{graphics/physicsResults/InvMassFit/PiPiInvMass_Fit_Systematics.pdf}\\
  \includegraphics[width=\linewidth,page=3]{graphics/physicsResults/InvMassFit/PiPiInvMass_Fit_Systematics.pdf}\\
  \includegraphics[width=\linewidth,page=5]{graphics/physicsResults/InvMassFit/PiPiInvMass_Fit_Systematics.pdf}\\
  \includegraphics[width=\linewidth,page=7]{graphics/physicsResults/InvMassFit/PiPiInvMass_Fit_Systematics.pdf}\\
  \includegraphics[width=\linewidth,page=9]{graphics/physicsResults/InvMassFit/PiPiInvMass_Fit_Systematics.pdf}
}~
\parbox{0.495\textwidth}{
  \centering
  \includegraphics[width=\linewidth,page=2]{graphics/physicsResults/InvMassFit/PiPiInvMass_Fit_Systematics.pdf}\\
  \includegraphics[width=\linewidth,page=4]{graphics/physicsResults/InvMassFit/PiPiInvMass_Fit_Systematics.pdf}\\
  \includegraphics[width=\linewidth,page=6]{graphics/physicsResults/InvMassFit/PiPiInvMass_Fit_Systematics.pdf}\\
  \includegraphics[width=\linewidth,page=8]{graphics/physicsResults/InvMassFit/PiPiInvMass_Fit_Systematics.pdf}\\
  \includegraphics[width=\linewidth,page=10]{graphics/physicsResults/InvMassFit/PiPiInvMass_Fit_Systematics.pdf}
}%
\end{figure}%
\begin{figure}[hb]%
\ContinuedFloat
\centering
\parbox{0.495\textwidth}{
  \centering
  \includegraphics[width=\linewidth,page=11]{graphics/physicsResults/InvMassFit/PiPiInvMass_Fit_Systematics.pdf}\\
  \includegraphics[width=\linewidth,page=13]{graphics/physicsResults/InvMassFit/PiPiInvMass_Fit_Systematics.pdf}\\
  \includegraphics[width=\linewidth,page=15]{graphics/physicsResults/InvMassFit/PiPiInvMass_Fit_Systematics.pdf}\\
  \includegraphics[width=\linewidth,page=17]{graphics/physicsResults/InvMassFit/PiPiInvMass_Fit_Systematics.pdf}\\
  \includegraphics[width=\linewidth,page=19]{graphics/physicsResults/InvMassFit/PiPiInvMass_Fit_Systematics.pdf}\\
  \includegraphics[width=\linewidth,page=21]{graphics/physicsResults/InvMassFit/PiPiInvMass_Fit_Systematics.pdf}
}~
\parbox{0.495\textwidth}{
  \centering
  \includegraphics[width=\linewidth,page=12]{graphics/physicsResults/InvMassFit/PiPiInvMass_Fit_Systematics.pdf}\\
  \includegraphics[width=\linewidth,page=14]{graphics/physicsResults/InvMassFit/PiPiInvMass_Fit_Systematics.pdf}\\
  \includegraphics[width=\linewidth,page=16]{graphics/physicsResults/InvMassFit/PiPiInvMass_Fit_Systematics.pdf}\\
  \includegraphics[width=\linewidth,page=18]{graphics/physicsResults/InvMassFit/PiPiInvMass_Fit_Systematics.pdf}\\
  \includegraphics[width=\linewidth,page=20]{graphics/physicsResults/InvMassFit/PiPiInvMass_Fit_Systematics.pdf}\\
  \includegraphics[width=\linewidth,page=22]{graphics/physicsResults/InvMassFit/PiPiInvMass_Fit_Systematics.pdf}
}%
\end{figure}%
\begin{figure}[hb]%
\ContinuedFloat
\centering
\parbox{0.495\textwidth}{
  \centering
  \includegraphics[width=\linewidth,page=23]{graphics/physicsResults/InvMassFit/PiPiInvMass_Fit_Systematics.pdf}\\
  \includegraphics[width=\linewidth,page=25]{graphics/physicsResults/InvMassFit/PiPiInvMass_Fit_Systematics.pdf}\\
  \includegraphics[width=\linewidth,page=27]{graphics/physicsResults/InvMassFit/PiPiInvMass_Fit_Systematics.pdf}\\
  \includegraphics[width=\linewidth,page=29]{graphics/physicsResults/InvMassFit/PiPiInvMass_Fit_Systematics.pdf}\\
  \includegraphics[width=\linewidth,page=31]{graphics/physicsResults/InvMassFit/PiPiInvMass_Fit_Systematics.pdf}\\
  \includegraphics[width=\linewidth,page=33]{graphics/physicsResults/InvMassFit/PiPiInvMass_Fit_Systematics.pdf}
}~
\parbox{0.495\textwidth}{
  \centering
  \includegraphics[width=\linewidth,page=24]{graphics/physicsResults/InvMassFit/PiPiInvMass_Fit_Systematics.pdf}\\
  \includegraphics[width=\linewidth,page=26]{graphics/physicsResults/InvMassFit/PiPiInvMass_Fit_Systematics.pdf}\\
  \includegraphics[width=\linewidth,page=28]{graphics/physicsResults/InvMassFit/PiPiInvMass_Fit_Systematics.pdf}\\
  \includegraphics[width=\linewidth,page=30]{graphics/physicsResults/InvMassFit/PiPiInvMass_Fit_Systematics.pdf}\\
  \includegraphics[width=\linewidth,page=32]{graphics/physicsResults/InvMassFit/PiPiInvMass_Fit_Systematics.pdf}\\[103pt]
%   \includegraphics[width=\linewidth,page=34]{graphics/physicsResults/InvMassFit/PiPiInvMass_Fit_Systematics.pdf}
}%
\end{figure}%
% \begin{figure}[hb]\ContinuedFloat
% \centering
% \parbox{0.495\textwidth}{
%   \centering
%   \includegraphics[width=\linewidth,page=7]{graphics/physicsResults/InvMassFit/PiPiInvMass_Fit_Systematics.pdf}\\
%   \includegraphics[width=\linewidth,page=9]{graphics/physicsResults/InvMassFit/PiPiInvMass_Fit_Systematics.pdf}\\
%   \includegraphics[width=\linewidth,page=11]{graphics/physicsResults/InvMassFit/PiPiInvMass_Fit_Systematics.pdf}
% }~
% \parbox{0.495\textwidth}{
%   \centering
%   \includegraphics[width=\linewidth,page=8]{graphics/physicsResults/InvMassFit/PiPiInvMass_Fit_Systematics.pdf}\\
%   \includegraphics[width=\linewidth,page=10]{graphics/physicsResults/InvMassFit/PiPiInvMass_Fit_Systematics.pdf}\\
%   \includegraphics[width=\linewidth,page=12]{graphics/physicsResults/InvMassFit/PiPiInvMass_Fit_Systematics.pdf}
% }%
% \end{figure}
% \begin{figure}[hb]\ContinuedFloat
% \centering
% \parbox{0.495\textwidth}{
%   \centering
%   \includegraphics[width=\linewidth,page=13]{graphics/physicsResults/InvMassFit/PiPiInvMass_Fit_Systematics.pdf}\\
%   \includegraphics[width=\linewidth,page=15]{graphics/physicsResults/InvMassFit/PiPiInvMass_Fit_Systematics.pdf}\\
%   \includegraphics[width=\linewidth,page=17]{graphics/physicsResults/InvMassFit/PiPiInvMass_Fit_Systematics.pdf}
% }~
% \parbox{0.495\textwidth}{
%   \centering
%   \includegraphics[width=\linewidth,page=14]{graphics/physicsResults/InvMassFit/PiPiInvMass_Fit_Systematics.pdf}\\
%   \includegraphics[width=\linewidth,page=16]{graphics/physicsResults/InvMassFit/PiPiInvMass_Fit_Systematics.pdf}\\
%   \includegraphics[width=\linewidth,page=18]{graphics/physicsResults/InvMassFit/PiPiInvMass_Fit_Systematics.pdf}
% }%
% \end{figure}

\end{appendices}

\listoffigures
\addcontentsline{toc}{chapter}{List of Figures}
\begingroup
\let\clearpage\relax
\listoftables
\addcontentsline{toc}{chapter}{List of Tables}
\endgroup

\bibliography{references.bib}{}
\bibliographystyle{utphys}
\addcontentsline{toc}{chapter}{References}

\end{document}          
